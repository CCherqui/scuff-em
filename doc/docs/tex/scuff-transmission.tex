\documentclass[letterpaper]{article}

../../../../tex/scufftex.tex

\graphicspath{{figures/}}

%------------------------------------------------------------
%------------------------------------------------------------
%- Special commands for this document -----------------------
%------------------------------------------------------------
%------------------------------------------------------------
\newcommand{\vbwhat}[1]{\vb{\widehat{#1}}}

%------------------------------------------------------------
%------------------------------------------------------------
%- Document header  -----------------------------------------
%------------------------------------------------------------
%------------------------------------------------------------
\title {Calculation of Reflection and Transmission Coefficients
        in {\sc scuff-transmission}}
\author {Homer Reid}
\date {May 9, 2015}

%------------------------------------------------------------
%------------------------------------------------------------
%- Start of actual document
%------------------------------------------------------------
%------------------------------------------------------------

\begin{document}
\pagestyle{myheadings}
\markright{Homer Reid: {\sc scuff-transmission}}
\maketitle

\tableofcontents

%%%%%%%%%%%%%%%%%%%%%%%%%%%%%%%%%%%%%%%%%%%%%%%%%%%%%%%%%%%%%%%%%%%%%%
%%%%%%%%%%%%%%%%%%%%%%%%%%%%%%%%%%%%%%%%%%%%%%%%%%%%%%%%%%%%%%%%%%%%%%
%%%%%%%%%%%%%%%%%%%%%%%%%%%%%%%%%%%%%%%%%%%%%%%%%%%%%%%%%%%%%%%%%%%%%%
\newpage
\section{The Setup}

{\sc scuff-transmission} considers geometries with 2D periodicity,
i.e. the structure consists of a unit-cell geometry of finite
extent in the $z$ direction that is infinitely periodically 
replicated in both the $x$ and $y$ directions. The structure is 
illuminated either from below (the default) or from above 
by a plane wave with propagation vector $\vb k$ confined to the 
$xz$ plane. 

Working at angular frequency $\omega$, let the free-space
wavelength be $k_0=\frac{\omega}{c}$, and let the relative 
permittivity and permeability of the lowermost and uppermost 
regions in the geometry be $\epsilon\subt{L,U}$ and $\mu\subt{L,U}$.
The wavenumber, refractive index, and relative wave impedance in the 
uppermost and lowermost regions are
%====================================================================%
\begin{align*}
  k\subt{L}&=n\subt{L} \cdot k_0,
   \qquad n\subt{L}\equiv \sqrt{\epsilon\subt{L} \mu\subt{L}},
   \qquad   
  Z\subt{L}=\sqrt\frac{\mu\subt{L}}{\epsilon\subt{L}}
\\
  k\subt{U}&=n\subt{U} \cdot k_0,
   \qquad n\subt{U}\equiv \sqrt{\epsilon\subt{U} \mu\subt{U}},
   \qquad   
  Z\subt{U}=\sqrt\frac{\mu\subt{U}}{\epsilon\subt{U}}.
\end{align*}
%====================================================================%
I will refer to region from which the planewave 
originates (either the uppermost or lowermost homogeneous
region in the {\sc scuff-em} geometry) as the ``incident'' region.
The region into which the planewave eventually emanates
is the ``transmitted'' region.
I will use the sub/superscripts $I,T$ to denote these quantities;
thus the wavenumber and relative wave impedance in the incident and
transmitted regions are
%====================================================================%
\begin{align*}
  \Big\{k\subt{I}, Z\subt{I}, k\subt{T}, Z\subt{T}\Big\} 
 &=\begin{cases}
     \big\{k\subt{L}, Z\subt{L}, k\subt{U}, Z\subt{U}\big\},
      \qquad &\text{wave incident from below}
      \\[5pt]
     \big\{k\subt{U}, Z\subt{U}, k\subt{L}, Z\subt{L}\big\},
      \qquad &\text{wave incident from above}
   \end{cases}
\end{align*}
%====================================================================%
In what follows, I will use the symbols
$\vb k\supt{I}, \vb k\supt{R}, \vb k\supt{T}$ respectively to denote the
propagation vectors of the incident, reflected, and transmitted waves.
I will take these vectors always to live in the $xz$ plane (i.e. $\vb k$
has no $y$ component, $k_y=0$). I will let 
$\theta\subt{I}$ and $\theta\subt{T}$
be the angles of incidence and transmission
(the angles between the incident and transmitted wavevector and the $z$ axis).
%%%%%%%%%%%%%%%%%%%%%%%%%%%%%%%%%%%%%%%%%%%%%%%%%%%%%%%%%%%%%%%%%%%%%%
\begin{align*}
 \text{wave incident from below:} 
  \qquad &\vb k\supt{I}= k\subt{L}\Big[ \sin\theta\subt{I}\,\vbhat{x} + \cos\theta\subt{I}\,\vbhat{z}\Big]
\\
 \text{wave incident from above:} 
  \qquad &\vb k\supt{I}= k\subt{U}\Big[ \sin\theta\subt{I}\,\vbhat{x} - \cos\theta\subt{I}\,\vbhat{z}\Big]
\end{align*}
%%%%%%%%%%%%%%%%%%%%%%%%%%%%%%%%%%%%%%%%%%%%%%%%%%%%%%%%%%%%%%%%%%%%%%
The reflected wavevector is 
%%%%%%%%%%%%%%%%%%%%%%%%%%%%%%%%%%%%%%%%%%%%%%%%%%%%%%%%%%%%%%%%%%%%%%
\begin{align*}
 \text{wave incident from below:} 
  \qquad &\vb k\supt{R}= k\subt{L}\Big[ \sin\theta\subt{I}\,\vbhat{x} - \cos\theta\subt{I}\,\vbhat{z}\Big]
\\
 \text{wave incident from above:} 
  \qquad &\vb k\supt{R}= k\subt{L}\Big[ \sin\theta\subt{I}\,\vbhat{x} + \cos\theta\subt{I}\,\vbhat{z}\Big]
\end{align*}
%%%%%%%%%%%%%%%%%%%%%%%%%%%%%%%%%%%%%%%%%%%%%%%%%%%%%%%%%%%%%%%%%%%%%%
The transmitted wavevector is 
%%%%%%%%%%%%%%%%%%%%%%%%%%%%%%%%%%%%%%%%%%%%%%%%%%%%%%%%%%%%%%%%%%%%%%
\begin{align*}
 \text{wave incident from below:}
  \qquad &\vb k\supt{T}= k\subt{U}\Big[ \sin\theta\subt{T}\,\vbhat{x} + \cos\theta\subt{T}\,\vbhat{z}\Big]
\\
 \text{wave incident from above:} 
  \qquad &\vb k\supt{T}= k\subt{L}\Big[ \sin\theta\subt{T}\,\vbhat{x} - \cos\theta\subt{T}\,\vbhat{z}\Big]
\end{align*}
%%%%%%%%%%%%%%%%%%%%%%%%%%%%%%%%%%%%%%%%%%%%%%%%%%%%%%%%%%%%%%%%%%%%%%
The incident and transmitted angles are related by 
%%%%%%%%%%%%%%%%%%%%%%%%%%%%%%%%%%%%%%%%%%%%%%%%%%%%%%%%%%%%%%%%%%%%%%
$$ n\subt{I}\sin\theta\subt{I} = n\subt{T}\sin\theta\subt{T}.$$
%%%%%%%%%%%%%%%%%%%%%%%%%%%%%%%%%%%%%%%%%%%%%%%%%%%%%%%%%%%%%%%%%%%%%%
For a general vector $\vb v$, I will define $\vbwhat{v}=\vb v/|\vb v|$ 
to be a unit vector in the direction of $\vb v$.

\subsection*{Definition of scattering coefficients}

The incident, reflected, and transmitted fields
may be written in the form
%%%%%%%%%%%%%%%%%%%%%%%%%%%%%%%%%%%%%%%%%%%%%%%%%%%%%%%%%%%%%%%%%%%%%%
\numeq{trDef}
{ \begin{array}{lclclcl}
%--------------------------------------------------------------------%
  \displaystyle{ \vb E\supt{I}(\vb x) }
& =
& \displaystyle{ E_0 \vbEps\supt{I} e^{ i \vb k\supt{I}\cdot \vb x}}
& \qquad 
& \displaystyle{ \vb H\supt{I}(\vb x) }
& =
& \displaystyle { H_0 \overline{\vbEps}\supt{I} e^{ i\vb k\supt{I}\cdot \vb x}}l
\\[8pt]
%--------------------------------------------------------------------%
  \displaystyle{ \vb E\supt{R}(\vb x) }
& =
& \displaystyle{ r E_0 \vbEps\supt{R} e^{ i \vb k\supt{R}\cdot \vb x}}
& \qquad 
& \displaystyle{ \vb H\supt{R}(\vb x) }
& =
& \displaystyle{r H_0 \overline{\vbEps}\supt{R} e^{ i \vb k \supt{R}\cdot \vb x}}
\\[8pt]
%--------------------------------------------------------------------%
  \displaystyle{ \vb E\supt{T}(\vb x) }
& =
& \displaystyle{ tE_0\vbEps\supt{T} e^{ i \vb k\supt{T}\cdot \vb x}}
& \qquad 
& \displaystyle{ \vb H\supt{T}(\vb x) }
& =
& \displaystyle{ tH_0^\prime\, \overline{\vbEps}\supt{T}, e^{i\vb k\supt{T}\cdot \vb x}}
\
\end{array}}
%%%%%%%%%%%%%%%%%%%%%%%%%%%%%%%%%%%%%%%%%%%%%%%%%%%%%%%%%%%%%%%%%%%%%%
where $E_0$ is the incident field magnitude,
$\vbEps\supt{I,R,T}$ are $\vb E$-field polarization vectors,
$\overline{\vbEps}\supt{I,R,T}$ are $\vb H$-field polarization vectors,
and we have
%%%%%%%%%%%%%%%%%%%%%%%%%%%%%%%%%%%%%%%%%%%%%%%%%%%%%%%%%%%%%%%%%%%%%%
$$ H_0\equiv \frac{i|\vb k|E_0}{Z\subt{I} Z_0},
   \qquad 
   H_0^\prime\equiv \frac{i|\vb k|E_0}{Z\subt{T} Z_0},
   \qquad 
   \overline{\vbEps} \equiv \vbwhat{k} \times \vbEps,
   \qquad 
   \vbEps = -\vbwhat{k} \times \overline{\vbEps}.
$$
%%%%%%%%%%%%%%%%%%%%%%%%%%%%%%%%%%%%%%%%%%%%%%%%%%%%%%%%%%%%%%%%%%%%%%
The polarization vectors are given by
%%%%%%%%%%%%%%%%%%%%%%%%%%%%%%%%%%%%%%%%%%%%%%%%%%%%%%%%%%%%%%%%%%%%%%
\begin{align*}
 \text{for the TE case:} \qquad
 &\vbEps\supt{I}\subt{TE}=
 \vbEps\supt{R}\subt{TE}=
 \vbEps\supt{T}\subt{TE}=\vbhat{y}, 
 \qquad 
 \overline{\vbEps}\supt{I,R,T}\subt{TE}=\wh{\vb k\supt{I,R,T}}\times \vbhat{y}
\\
 \text{for the TM case:} \qquad
 &\overline{\vbEps}\supt{I}\subt{TM}=
  \overline{\vbEps}\supt{R}\subt{TM}=
  \overline{\vbEps}\supt{T}\subt{TM}=\vbhat{y}
  \qquad 
  \vbEps\supt{I,R,T}\subt{TM}=-\wh{\vb k\supt{I,R,T}}\times \vbhat{y}
\end{align*}
%%%%%%%%%%%%%%%%%%%%%%%%%%%%%%%%%%%%%%%%%%%%%%%%%%%%%%%%%%%%%%%%%%%%%%

Equations (\ref{trDef})
define the reflection and transmission coefficients
$r$ and $t$ computed by {\sc scuff-transmission}.

%%%%%%%%%%%%%%%%%%%%%%%%%%%%%%%%%%%%%%%%%%%%%%%%%%%%%%%%%%%%%%%%%%%%%%
%%%%%%%%%%%%%%%%%%%%%%%%%%%%%%%%%%%%%%%%%%%%%%%%%%%%%%%%%%%%%%%%%%%%%%
%%%%%%%%%%%%%%%%%%%%%%%%%%%%%%%%%%%%%%%%%%%%%%%%%%%%%%%%%%%%%%%%%%%%%%
\newpage
\section{Scattering coefficients from surface currents}

Next we consider an extended structure described by Bloch-periodic
boundary conditions, i.e. all fields and currents satisfy 

\numeq{BlochPeriodicity1}
{\vb Q(\vb x+\vb L) = e^{i\vb k\subt{B} \cdot \vb L}\,\vb Q(\vb x)}
where $\vb Q$ is a field ($\vb E$ or $\vb H$) or a surface current
($\vb K$ or $\vb N$) and the Bloch wavevector is\footnote{Recall
our convention that the propagation vector lives
in the $xy$ plane.}
%%%%%%%%%%%%%%%%%%%%%%%%%%%%%%%%%%%%%%%%%%%%%%%%%%%%%%%%%%%%%%%%%%%%%%
$$\vb k\subt{B}=k\subt{I}\sin\theta\subt{I} \, \vbhat{x}
               =k\subt{T}\sin\theta\subt{T} \, \vbhat{x}.
$$
%%%%%%%%%%%%%%%%%%%%%%%%%%%%%%%%%%%%%%%%%%%%%%%%%%%%%%%%%%%%%%%%%%%%%%
For plane waves like (\ref{trDef}), equation (\ref{BlochPeriodicity1})
actually holds for any arbitrary vector $\vb L$; for our
purposes we will only need to use it for certain special
vectors $\vb L$ determined by the structure of the lattice in
our PBC geometry.
We will derive expressions for the
plane-wave reflection and transmission coefficients in terms
of the surface-current distribution in the unit cell of the structure.


%%%%%%%%%%%%%%%%%%%%%%%%%%%%%%%%%%%%%%%%%%%%%%%%%%%%%%%%%%%%%%%%%%%%%%
%%%%%%%%%%%%%%%%%%%%%%%%%%%%%%%%%%%%%%%%%%%%%%%%%%%%%%%%%%%%%%%%%%%%%%
%%%%%%%%%%%%%%%%%%%%%%%%%%%%%%%%%%%%%%%%%%%%%%%%%%%%%%%%%%%%%%%%%%%%%%
\subsection*{Fields from surface currents}

On the other hand, the scattered $\vb E$ fields in the incident
and transmitted regions may be obtained in the usual way from the
surface-current distributions on the surfaces bounding those
regions. For example, at points in the transmitted medium, 
the scattered (that is, transmitted) $\vb E$ field is given by
%%%%%%%%%%%%%%%%%%%%%%%%%%%%%%%%%%%%%%%%%%%%%%%%%%%%%%%%%%%%%%%%%%%%%%
\begin{align}
 \vb E\supt{T}(\vb x)
&=\oint_{\mc S\subt{T}} \Big\{ 
   \BG\supt{EE;T}(\vb x,\vb x^\prime)\cdot \vb K(\vb x^\prime)
  +\BG\supt{EM;T}(\vb x,\vb x^\prime)\cdot \vb N(\vb x^\prime)
                \Big\}d\vb x^\prime
\nn
&=ik\subt{T} \oint_{\mc S} \Big\{ 
   Z_0 Z\subt{T}\vb G(k\subt{T}; \vb x,\vb x^\prime)
                    \cdot \vb K(\vb x^\prime)
               +\vb C(k\subt{T}; \vb x,\vb x^\prime)
                    \cdot \vb N(\vb x^\prime) \Big\}d\vb x^\prime
\label{ETransPBC}
\end{align}
%%%%%%%%%%%%%%%%%%%%%%%%%%%%%%%%%%%%%%%%%%%%%%%%%%%%%%%%%%%%%%%%%%%%%%
where $\mc S\subt{T}$ is the surface bounding the transmitted region and 
$\vb G, \vb C$ are the homogeneous dyadic GFs for that region. Using the
Bloch periodicity of the surface currents, i.e.
%%%%%%%%%%%%%%%%%%%%%%%%%%%%%%%%%%%%%%%%%%%%%%%%%%%%%%%%%%%%%%%%%%%%%%
$$ \left\{ \begin{array}{c} 
   \vb K(\vb x+\vb L) \\[5pt] \vb N(\vb x+\vb L)
   \end{array}\right\}
   =
   e^{i\vb k\subt{B} \cdot \vb L}
   \left\{ \begin{array}{c} 
   \vb K(\vb x) \\[5pt] \vb N(\vb x)
   \end{array}\right\}
$$
%%%%%%%%%%%%%%%%%%%%%%%%%%%%%%%%%%%%%%%%%%%%%%%%%%%%%%%%%%%%%%%%%%%%%%
we can restrict the surface integral in (\ref{ETransPBC}) to 
just the lattice unit cell:
%%%%%%%%%%%%%%%%%%%%%%%%%%%%%%%%%%%%%%%%%%%%%%%%%%%%%%%%%%%%%%%%%%%%%%
\numeq{EIntegralUnitCell}
{
 \vb E\supt{T}(\vb x)
=ik\subt{T} \int\subt{UC}
  \Big\{
  Z_0 Z\subt{T} 
   \overline{\vb G}(k\subt{T}; \vb x, \vb x^\prime) \cdot \vb K(\vb x^\prime)
  +\overline{\vb C}(k\subt{T}; \vb x, \vb x^\prime) \cdot \vb N(\vb x^\prime)
  \Big\}\,d\vb x^\prime
}
%%%%%%%%%%%%%%%%%%%%%%%%%%%%%%%%%%%%%%%%%%%%%%%%%%%%%%%%%%%%%%%%%%%%%%
where the periodic Green's functions are
\numeq{PeriodicGFs}
{
   \left\{ \begin{array}{c} 
   \overline{\vb G}(\vb x, \vb x^\prime) \\[5pt]
   \overline{\vb C}(\vb x, \vb x^\prime)
   \end{array}\right\}
   \equiv 
   \sum_{\vb L} e^{i\vb k\subt{B} \cdot \vb L}
   \left\{ \begin{array}{c}
   \vb G(\vb x, \vb x^\prime+\vb L) \\[5pt]
   \vb C(\vb x, \vb x^\prime+\vb L)
   \end{array}\right\}
}
%%%%%%%%%%%%%%%%%%%%%%%%%%%%%%%%%%%%%%%%%%%%%%%%%%%%%%%%%%%%%%%%%%%%%%
I now invoke the following representation of the dyadic
Green's functions (Chew, 1995): for $z>z^\prime$,
%%%%%%%%%%%%%%%%%%%%%%%%%%%%%%%%%%%%%%%%%%%%%%%%%%%%%%%%%%%%%%%%%%%%%%
\begin{subequations}
\begin{align}
 \vb G(k; \vbrho, z; \vbrho^\prime, z^\prime)
&= \int \frac{d\vb q}{(2\pi)^2}
   \,
   \vb{\wt G}^\pm (k; \vb q)
      e^{i\vb q\cdot (\vbrho-\vbrho^\prime)}
     e^{\pm iq_z(z-z^\prime)}
\\
 \vb C(k; \vbrho, z; \vbrho^\prime, z^\prime)
&=\int \frac{d\vb q}{(2\pi)^2}
  \, 
  \vb{\wt C}^\pm (k; \vb q) e^{i\vb q\cdot (\vbrho-\vbrho^\prime)}
                   e^{\pm iq_z(z-z^\prime)}
\end{align}
\label{SpectralGFs}%
\end{subequations}
%%%%%%%%%%%%%%%%%%%%%%%%%%%%%%%%%%%%%%%%%%%%%%%%%%%%%%%%%%%%%%%%%%%%%%
where $\vb q=(q_x,q_y)$ is a two-dimensional Fourier wavevector,
$d\vb q=dq_x dq_y$,
$q_z=\sqrt{k^2-|\vb q|^2}$, $\pm = \text{sign}(z-z^\prime)$,
and
%%%%%%%%%%%%%%%%%%%%%%%%%%%%%%%%%%%%%%%%%%%%%%%%%%%%%%%%%%%%%%%%%%%%%%
\begin{align*}
 \vb{\wt G}^\pm(k; \vb q)
   &=\frac{i}{2 q_z}
     \left[
     \left(\begin{array}{ccc}
      1 & 0 & 0 \\ 
      0 & 1 & 0 \\ 
      0 & 0 & 1
     \end{array}\right)
     -\frac{1}{k^2}
     \left(\begin{array}{ccc}
      q_x^2       & q_x q_y     & \pm q_x q_z \\
      q_yq_x      & q_y^2       & \pm q_y q_z \\
      \pm q_z q_x & \pm q_z q_y & q_z^2
     \end{array}\right)\right]
\\
 \vb{\wt C}^\pm (k; \vb q)
   &=\frac{i}{2q_z k}
     \left(\begin{array}{ccc}
           0 & \pm q_z & -q_y \\
     \mp q_z &       0 &  q_x \\
         q_y &    -q_x &    0
     \end{array}\right).
\end{align*}
%%%%%%%%%%%%%%%%%%%%%%%%%%%%%%%%%%%%%%%%%%%%%%%%%%%%%%%%%%%%%%%%%%%%%%
Inserting (\ref{SpectralGFs}) into (\ref{PeriodicGFs}), I 
obtain, for the periodic version of e.g. the $\vb G$ kernel,
%%%%%%%%%%%%%%%%%%%%%%%%%%%%%%%%%%%%%%%%%%%%%%%%%%%%%%%%%%%%%%%%%%%%%%
\begin{align*}
\overline{\vb G}(k; \vbrho, z; \vbrho^\prime, z)
&=\int \frac{d\vb q}{(2\pi)^2}
  \vb{\wt G}^\pm(k; \vb q) e^{i\vb q \cdot (\vbrho - \vbrho^\prime)}
   e^{\pm iq_z(z-z^\prime)}
  \underbrace{\sum_{\vb L} e^{i(\vb k\subt{B} - \vb q) \cdot \vb L}}
             _{\mc V\subt{BZ} \sum_{\vbGamma} \delta(\vb q-\vb k-\vbGamma)}
\\
&=\mc V\subt{UC}^{-1}\sum_{\vb q=\vb k\subt{B} + \vbGamma}
  \vb{\wt G}^\pm(k; \vb q)
   e^{i\vb q \cdot (\vbrho - \vbrho^\prime)} e^{\pm iq_z(z-z^\prime)}
\intertext{where the sum over $\vbGamma$ runs over reciprocal lattice vectors;
           the prefactor $\mc V\subt{BZ}$, the volume of the Brillouin
           zone, is related to the unit-cell volume by 
           $\mc V\subt{BZ}=4\pi^2/V\subt{UC}$ for a 2D square lattice.
           Similarly, we find}
\overline{\vb C}(k; \vbrho, z; \vbrho^\prime, z)
&=\mc V\subt{UC}^{-1}\sum_{\vb q=\vb k\subt{B} + \vbGamma}
   \vb{\wt C}^\pm(k; \vb q)
   e^{i\vb q \cdot (\vbrho - \vbrho^\prime)} e^{\pm iq_z(z-z^\prime)}.
\end{align*}
%%%%%%%%%%%%%%%%%%%%%%%%%%%%%%%%%%%%%%%%%%%%%%%%%%%%%%%%%%%%%%%%%%%%%%a
Keeping only the $\vbGamma=0$ term in these sums, the scattered 
$\vb E$-fields in the uppermost and lowermost regions are thus
%====================================================================%
\begin{align}
\vb E\sups{upper}(\vb x) 
&= e^{i(k\subt{Ux} x + k\subt{Uz}z)}
   \Big[ ik\subt{U} Z_0 Z\subt{U} \vb{\wt G}^+(k\subt{U}; \vb k\subt{B})
         \wt{\vb K}\subt{U}(\vb k\subt{B})
        +ik\subt{U} \vb{\wt C}^+(k\subt{U}; \vb k\subt{B})
         \wt{\vb N}(\vb k\subt{B})
   \Big]
\\
\vb E\sups{lower}(\vb x) 
&= e^{i(k\subt{Lx} x - k\subt{Lz}z)}
   \Big[ ik\subt{L} Z_0 Z\subt{U} \vb{\wt G}^-(k\subt{L}; \vb k\subt{B})
         \wt{\vb K}(\vb k\subt{B})
        +ik\subt{L} \vb{\wt C}^-(k\subt{L}; \vb k\subt{B})
         \wt{\vb N}(\vb k\subt{B})
   \Big]
\end{align}
%====================================================================%
where e.g. $\wt{\vb K}\subt{U}$ is something like
the two-dimensional Fourier transform of the surface currents
on the boundary of the uppermost region $\mc R\subt{U}$:
%====================================================================%
$$
 \wt{\vb K}\subt{U}(\vb k\subt{B})
   \equiv \frac{1}{\mc V\subt{UC}}
     \int_{\partial \mc R\subt{U}}
     \vb K(\vbrho^\prime, z^\prime)
         e^{-i\vb k\subt{B}\cdot \vbrho^\prime
            -iq_z |z^\prime|}
         d\vb x^\prime,
 \qquad 
 q_z^2=k\subt{U}^2-|\vb k\subt{B}|^2.
$$
%====================================================================%
with $\wt{\vb K}\subt{L}$ and 
$\wt{\vb N}\subt{U,L}$ defined similarly.

Comparing this to (\ref{trDef}c), we see that the transmission and
reflection coefficients for the polarization defined by polarization
vector $\vbEps$ are given by 
%====================================================================%
\begin{align}
 \left\{\begin{array}{c}t \\ r\end{array}\right\}
&=
         ik\subt{U} Z_0 Z\subt{U} 
         \vbEps^\dagger \vb{\wt G}^+(k\subt{U}, \vb k\subt{B})
         \wt{\vb K}\subt{U}(\vb k\subt{B})
        +ik\subt{U} \vbEps^\dagger \vb{\wt C}^+(k\subt{U}, \vb k\subt{B})
         \wt{\vb N}(\vb k\subt{B})
\\
 \left\{\begin{array}{c}r \\ t\end{array}\right\}
&= ik\subt{L} Z_0 Z\subt{L} 
         \vbEps^\dagger \vb{\wt G}^-(k\subt{L}, \vb k\subt{B})
         \wt{\vb K}\subt{L}(\vb k\subt{B})
        +ik\subt{L} \vbEps^\dagger \vb{\wt C}^-(k\subt{L}, \vb k\subt{B})
         \wt{\vb N}(\vb k\subt{B})
\end{align}
%====================================================================%
The expressions on the RHS compute the upper (lower) quantities
on the LHS for the case in which the plane wave is incident
from below (above).

The $\wt{\vb K}$ and $\wt{\vb N}$ quantities are given by
sums of contributions from individual basis functions; for
example, 
%====================================================================%
$$
 \wt{\vb K}\subt{U}(\vb q)
  =\frac{1}{\mc V\subt{UC}}
    \sum_{\vb b_\alpha \in \partial \mc R\subt{U}}
     k_\alpha \wt{\vb b_\alpha}(\vb q),
\qquad
 \wt{\vb N}\subt{U}(\vb q)
  =-\frac{Z_0}{\mc V\subt{UC}} \sum_{\vb b_\alpha \in \partial \mc R\subt{U}}
     n_\alpha \wt{\vb b_\alpha}(\vb q)
$$
%====================================================================%
where the sums are over all RWG basis functions that live on
surfaces bounding the uppermost medium and $\{k_\alpha,n_\alpha\}$
are the surface-current coefficients obtained as the 
solution to the {\sc scuff-em} scattering problem.

%%%%%%%%%%%%%%%%%%%%%%%%%%%%%%%%%%%%%%%%%%%%%%%%%%%%%%%%%%%%%%%%%%%%%%
%%%%%%%%%%%%%%%%%%%%%%%%%%%%%%%%%%%%%%%%%%%%%%%%%%%%%%%%%%%%%%%%%%%%%%
%%%%%%%%%%%%%%%%%%%%%%%%%%%%%%%%%%%%%%%%%%%%%%%%%%%%%%%%%%%%%%%%%%%%%%
\subsection{Computation of $\wt{\vb b}(\vb q)$}

For RWG functions the quantity $\wt{\vb b}(\vb q)$
may be evaluated in closed form:
%%%%%%%%%%%%%%%%%%%%%%%%%%%%%%%%%%%%%%%%%%%%%%%%%%%%%%%%%%%%%%%%%%%%%%
\begin{align*}
\wt{\vb b_\alpha}(\vb q)
&\equiv
\int_{\text{sup } \vb b_\alpha}
  \vb b_\alpha(\vb x)e^{-i\vb q \cdot \vb x} \, d\vb x
\\
&=\ell_\alpha
  \int_0^1 \, du \int_0^u \, dv \,
  \bigg\{
  e^{-i\vb q\cdot [\vb Q^+ + u \vb A^+ + v\vb B]}
  \Big(u\vb A^+ + v\vb B\Big)
\\ 
  &\hspace{1.5in}
  - 
  e^{-i\vb q\cdot [\vb Q^- + u \vb A^- + v\vb B]}
  \Big(u\vb A^- + v\vb B\Big)
  \bigg\}
\\
&=\ell_\alpha\bigg\{ 
  e^{-i\vb q \cdot \vb Q^+} 
  \Big[ f_1\big(\vb q\cdot \vb A^+, \vb q\cdot \vb B\Big) 
           \vb A^+
       +f_2\big(\vb q\cdot \vb A^+, \vb q\cdot \vb B\Big)
           \vb B
  \Big]
\\
&\qquad\qquad
  -
  e^{-i\vb q \cdot \vb Q^-} 
  \Big[ f_1\big(\vb q\cdot \vb A^-, \vb q\cdot \vb B\Big)
           \vb A^-
       +f_2\big(\vb q\cdot \vb A^-, \vb q\cdot \vb B\Big)
           \vb B
  \Big]\bigg\}
\end{align*}
%%%%%%%%%%%%%%%%%%%%%%%%%%%%%%%%%%%%%%%%%%%%%%%%%%%%%%%%%%%%%%%%%%%%%%
where
%%%%%%%%%%%%%%%%%%%%%%%%%%%%%%%%%%%%%%%%%%%%%%%%%%%%%%%%%%%%%%%%%%%%%%
\begin{align*}
 f_1(x, y)&=\int_0^1 \, \int_0^u \, u e^{-i (ux + vy)} \,dv\,du  
\\
 f_2(x, y)&=\int_0^1 \, \int_0^u \, v e^{-i (ux + vy)} \,dv\,du.
\end{align*}
%%%%%%%%%%%%%%%%%%%%%%%%%%%%%%%%%%%%%%%%%%%%%%%%%%%%%%%%%%%%%%%%%%%%%%

\end{document} 

%====================================================================%
Comparing this to (\ref{trDef}c), we see that the quantity in
curly braces is to be identified as $t E_0 \vbEps_0\supt{T},$
and thus we can extract the transmission coefficient $t$ in the form
%%%%%%%%%%%%%%%%%%%%%%%%%%%%%%%%%%%%%%%%%%%%%%%%%%%%%%%%%%%%%%%%%%%%%%
\numeq{tExpression}
{
 t
 =-\frac{k_0^\prime}{2 \mc V k_z^\prime E_0}
   \int_U e^{-i(\vb k \cdot \vbrho^\prime + k_z z^\prime)}
   \underbrace{
        \left[ Z_0 Z\subt{T} 
               (\vbEps_0\supt{T})^\dagger \cdot 
               \vb{\wt G}(\vb k) \cdot \vb K(\vb x^\prime)
               +(\vbEps_0\supt{T})^\dagger \cdot 
               \vb{\wt C}(\vb k)\cdot \vb N(\vb x^\prime) 
        \right] \, d\vb x^\prime
}
%%%%%%%%%%%%%%%%%%%%%%%%%%%%%%%%%%%%%%%%%%%%%%%%%%%%%%%%%%%%%%%%%%%%%%

%%%%%%%%%%%%%%%%%%%%%%%%%%%%%%%%%%%%%%%%%%%%%%%%%%%%%%%%%%%%%%%%%%%%%%
%%%%%%%%%%%%%%%%%%%%%%%%%%%%%%%%%%%%%%%%%%%%%%%%%%%%%%%%%%%%%%%%%%%%%%
%%%%%%%%%%%%%%%%%%%%%%%%%%%%%%%%%%%%%%%%%%%%%%%%%%%%%%%%%%%%%%%%%%%%%%
\subsection*{Transmission coefficient for TE polarization}

In this case we have 
%====================================================================%
\begin{align}
&\hspace{-0.05in}
 (\vbEps_0\supt{T})^\dagger \cdot
 \vb{\wt G}(\vb k) \cdot \vb K(\vb x^\prime)
\nn
&=
 \left(\begin{array}{c}0 \\ 1 \\ 0\end{array}\right)^\dagger
 \left[ \left(\begin{array}{ccc}
               1 & 0 & 0 \\ 
               0 & 1 & 0 \\ 
               0 & 0 & 1
              \end{array}\right)
       -\frac{1}{k_0^{\prime 2}}
        \left(\begin{array}{ccc}
               k_x^2   & k_x k_y & k_x k_z \\
               k_yk_x  & k_y^2   & k_y k_z \\
               k_z k_x & k_z k_y & k_z^2
              \end{array}
        \right)
 \right]
 \left(\begin{array}{c}K_x \\ K_y \\ K_z\end{array}\right)
\nn
&=
 \frac{1}{k_0^{\prime 2}}
 \left(\begin{array}{c}0 \\ 1 \\ 0\end{array}\right)^\dagger
 \left(\begin{array}{c} -k_y k_x K_x \\ (k_0^{\prime 2}-k_y^2)K_y \\ -k_y k_z K_z
       \end{array}\right)
\nn
&= K_y
\label{ECN1}
\end{align}
%====================================================================%
since, in our setup, the $y$ component of the Bloch 
wavevector is $k_y=0.$

Similarly, 
%====================================================================%
\begin{align}
&\hspace{-0.05in}
 (\vbEps_0\supt{T})^\dagger \cdot
 \vb{\wt C}(\vb k) \cdot \vb N(\vb x^\prime)
\nn
&=
 \left(\begin{array}{c}0 \\ 1 \\ 0\end{array}\right)^\dagger
 \left[ \frac{1}{k_0^{\prime}}
        \left(\begin{array}{ccc}
               0       & k_z     & -k_y    \\
               -k_z    & 0       & k_x     \\
               k_y     & -k_x    & 0
              \end{array}
        \right)
 \right]
 \left(\begin{array}{c}N_x \\ N_y \\ N_z\end{array}\right)
\nn
&=
 \frac{1}{k_0^\prime}
 \left(\begin{array}{c}0 \\ 1 \\ 0\end{array}\right)^\dagger
 \left(\begin{array}{c} \cdots \\ -k_z N_x + k_x N_z \\ \cdots
       \end{array}\right)
\nn
&=-\cos\theta^\prime N_x + \sin\theta^\prime N_z.
\label{ECN2}
\end{align}
%====================================================================%
Comparing (\ref{ECN1}) and (\ref{ECN2}) to (\ref{TEPolVectors}),
we see that the 
vector-matrix-vector sandwiches that enter the definition of 
the transmission coefficient (\ref{tExpression}) wind up 
computing simply the inner products of the $\vb K$ and $\vb N$
surface currents with the field polarization vectors: 
%%%%%%%%%%%%%%%%%%%%%%%%%%%%%%%%%%%%%%%%%%%%%%%%%%%%%%%%%%%%%%%%%%%%%%
\numeq{EGKAndECN}
{
(\vbEps_0\supt{T})^\dagger
    \cdot
    \vb{\wt G}(\vb k)
    \cdot
    \vb K(\vb x^\prime)
   = \vbEps_0\supt{T} \cdot \vb K, 
   \qquad
   (\vbEps_0\supt{T})^\dagger
    \cdot
    \vb{\wt C}(\vb k)
    \cdot
    \vb N(\vb x^\prime)
   = \overline{\vbEps_0}\supt{T} \cdot \vb N
}
%%%%%%%%%%%%%%%%%%%%%%%%%%%%%%%%%%%%%%%%%%%%%%%%%%%%%%%%%%%%%%%%%%%%%%

%%%%%%%%%%%%%%%%%%%%%%%%%%%%%%%%%%%%%%%%%%%%%%%%%%%%%%%%%%%%%%%%%%%%%%
Putting it all together, 
%%%%%%%%%%%%%%%%%%%%%%%%%%%%%%%%%%%%%%%%%%%%%%%%%%%%%%%%%%%%%%%%%%%%%%
\begin{align}
 t
&=-\frac{k_0^\prime}{2\mc V k_z^\prime E_0}
   \int_U e^{-i(\vb k \cdot \vbrho^\prime + k_z^\prime z^\prime)}
        \Big[ Z_0 Z\subt{T} \vbEps_0\supt{T} \cdot \vb K(\vb x^\prime)
                +\overline{\vbEps}_0\supt{T} \cdot \vb N(\vb x^\prime)
        \Big] \, d\vb x^\prime
\\
&=-\frac{Z_0 Z\subt{T}}{2E_0 \cos \theta^\prime}
   \cdot \frac{1}{\mc V}
   \int_U \Big(  \vb E^{\text{\scriptsize{ref}}*}\cdot \vb K
                +\vb H^{\text{\scriptsize{ref}}*}\cdot \vb N
          \Big) \, d\vb x^\prime.
\label{TFromKN}
\end{align}
%%%%%%%%%%%%%%%%%%%%%%%%%%%%%%%%%%%%%%%%%%%%%%%%%%%%%%%%%%%%%%%%%%%%%%
Here $\{\vb E,\vb H\}\sups{ref}$ are the fields of a 
``reference'' plane wave with $E$-field magnitude $\vb E_0$
traveling in the upper medium. If the upper medium is the
same as the lower medium (for example, in a thin-film geometry
where both the upper and lower media are vacuum) then
the reference fields are just the same as the incident
fields. 
If the upper medium is a different medium than the lower medium
(for example, in a Fresnel scattering problem) then the 
reference fields are similar to the incident fields
but with propagation and polarization vectors modified 
appropriately.

%%%%%%%%%%%%%%%%%%%%%%%%%%%%%%%%%%%%%%%%%%%%%%%%%%%%%%%%%%%%%%%%%%%%%%

\subsection*{Transmission coefficient for TM polarization}

%%%%%%%%%%%%%%%%%%%%%%%%%%%%%%%%%%%%%%%%%%%%%%%%%%%%%%%%%%%%%%%%%%%%%%
\begin{align*}
 t\supt{TM}
&=-\frac{k_0^\prime}{2 \mc V k_z^\prime}
   \int_U e^{-i(\vb k \cdot \vbrho^\prime + k_z z^\prime)}
        \left[ Z_0 Z\subt{T} 
               (\vbEps_0\supt{T})^\dagger \cdot 
               \vb{\wt G}(\vb k) \cdot \vb K(\vb x^\prime)
               +(\vbEps_0\supt{T})^\dagger \cdot 
               \vb{\wt C}(\vb k)\cdot \vb N(\vb x^\prime) 
        \right] \, d\vb x^\prime
\end{align*}
%%%%%%%%%%%%%%%%%%%%%%%%%%%%%%%%%%%%%%%%%%%%%%%%%%%%%%%%%%%%%%%%%%%%%%
We have 
%%%%%%%%%%%%%%%%%%%%%%%%%%%%%%%%%%%%%%%%%%%%%%%%%%%%%%%%%%%%%%%%%%%%%%
\begin{align}
&\hspace{-0.05in}
 (\vbEps_0\supt{T})^\dagger \cdot
 \vb{\wt G}(\vb k) \cdot \vb K(\vb x^\prime)
\nn
&=
 \left(\begin{array}{c}\cos\theta^\prime \\ 0 \\-\sin\theta^\prime \end{array}\right)^\dagger
 \left[ \left(\begin{array}{ccc}
               1 & 0 & 0 \\ 
               0 & 1 & 0 \\ 
               0 & 0 & 1
              \end{array}\right)
       -\frac{1}{k_0^{\prime 2}}
        \left(\begin{array}{ccc}
               k_x^2   & k_x k_y & k_x k_z \\
               k_yk_x  & k_y^2   & k_y k_z \\
               k_z k_x & k_z k_y & k_z^2
              \end{array}
        \right)
 \right]
 \left(\begin{array}{c}K_x \\ K_y \\ K_z\end{array}\right)
\intertext{Use $\cos\theta^\prime=k_z/k_0^\prime$, 
               $\sin\theta^\prime=k_x/k_0^\prime$:}
&=
 \cos\theta^\prime K_x - \sin\theta^\prime K_z
\label{EGK2}
\end{align}
%%%%%%%%%%%%%%%%%%%%%%%%%%%%%%%%%%%%%%%%%%%%%%%%%%%%%%%%%%%%%%%%%%%%%%
\begin{align}
&\hspace{-0.05in}
 (\vbEps_0\supt{T})^\dagger \cdot
 \vb{\wt C}(\vb k) \cdot \vb N(\vb x^\prime)
\nn
&=
 \left(\begin{array}{c}\cos\theta^\prime \\ 0 \\-\sin\theta^\prime \end{array}\right)^\dagger
 \left[ \frac{1}{k_0^{\prime}}
        \left(\begin{array}{ccc}
               0       & k_z     & -k_y    \\
               -k_z    & 0       & k_x     \\
               k_y     & -k_x    & 0
              \end{array}
        \right)
 \right]
 \left(\begin{array}{c}N_x \\ N_y \\ N_z\end{array}\right)
\nn
&=
 N_y.
\label{ECN2}
\end{align}
%%%%%%%%%%%%%%%%%%%%%%%%%%%%%%%%%%%%%%%%%%%%%%%%%%%%%%%%%%%%%%%%%%%%%%
Comparing (\ref{EGN2}) and (\ref{ECN2}) to (\ref{TMPolVectors})
we see that once again equation (\ref{EGKandECN}) holds
and we simply recover equation (\ref{TFromKN}) for the 
TM polarization as well.

\newpage
\subsection*{Transmission coefficients from surface-current coefficients} 
%%%%%%%%%%%%%%%%%%%%%%%%%%%%%%%%%%%%%%%%%%%%%%%%%%%%%%%%%%%%%%%%%%%%%%
$$ \vb K(\vb x)=\sum k_\alpha \vb b_\alpha(\vb x), \qquad 
   \vb N(\vb x)=-Z_0 \sum n_\alpha \vb b_\alpha(\vb x)
$$
%%%%%%%%%%%%%%%%%%%%%%%%%%%%%%%%%%%%%%%%%%%%%%%%%%%%%%%%%%%%%%%%%%%%%%
$$ t=-\frac{Z_0 Z\subt{T}}{2E_0 \cos \theta^\prime \mc V}
      \sum_\alpha 
        \bigg( k_\alpha \INP{\vb E\sups{ref}}{\vb b_\alpha}
             -Z_0 n_\alpha \INP{\vb H\sups{ref}}{\vb b_\alpha}
        \bigg)
$$
%%%%%%%%%%%%%%%%%%%%%%%%%%%%%%%%%%%%%%%%%%%%%%%%%%%%%%%%%%%%%%%%%%%%%%
\begin{align*}
 \INP{\vb E\sups{ref}}{\vb b_\alpha}
&=\int_{\text{sup } \vb b_\alpha}
  \vbEps \cdot \vb b_\alpha(\vb x)e^{-i\vb q \cdot \vb x}
  \, d\vb x
\end{align*}
%%%%%%%%%%%%%%%%%%%%%%%%%%%%%%%%%%%%%%%%%%%%%%%%%%%%%%%%%%%%%%%%%%%%%%
