\documentclass[letterpaper]{article}

../../../../tex/scufftex.tex

\newcommand\supsstar[1]{^{\hbox{\scriptsize{#1}}*}}
\newcommand\suptstar[1]{^{\hbox{\scriptsize{#1}}*}}

\graphicspath{{figures/}}

%------------------------------------------------------------
%------------------------------------------------------------
%- Special commands for this document -----------------------
%------------------------------------------------------------
%------------------------------------------------------------

%------------------------------------------------------------
%------------------------------------------------------------
%- Document header  -----------------------------------------
%------------------------------------------------------------
%------------------------------------------------------------
\title {PFT and Non-Equilibrium Formulas in {\sc scuff-em}}
\author {Homer Reid}
\date {March 18, 2012}

%------------------------------------------------------------
%------------------------------------------------------------
%- Start of actual document
%------------------------------------------------------------
%------------------------------------------------------------

\begin{document}
\pagestyle{myheadings}
\markright{Homer Reid: PFT and NEQ formulas in {\sc scuff-em}}
\maketitle

%\tableofcontents

%%%%%%%%%%%%%%%%%%%%%%%%%%%%%%%%%%%%%%%%%%%%%%%%%%%%%%%%%%%%%%%%%%%%%%
%%%%%%%%%%%%%%%%%%%%%%%%%%%%%%%%%%%%%%%%%%%%%%%%%%%%%%%%%%%%%%%%%%%%%%
%%%%%%%%%%%%%%%%%%%%%%%%%%%%%%%%%%%%%%%%%%%%%%%%%%%%%%%%%%%%%%%%%%%%%%
\section{The {\sc scuff-em} versions of the BEM system}

The usual PMCHW equation reads, in continuous form, 
\numeq{PMCHWContinuous}
{  \left(\begin{array}{cc} 
      \BG\supt{EE} & \BG\supt{EM} \\
      \BG\supt{ME} & \BG\supt{MM}
    \end{array}\right)
   *
  \left(\begin{array}{c} \vb K \\ \vb N \end{array}\right)
  =
  -\left(\begin{array}{c} \vb E \\ \vb H \end{array}\right)\sups{inc}_{\parallel}
}
or, in discretized form,
%====================================================================%
\numeq{PMCHWDiscrete}
{
  \underbrace{
    \left(\begin{array}{cc} 
      \vb M\supt{EE} & \vb M\supt{EM} \\
      \vb M\supt{ME} & \vb M\supt{MM}
    \end{array}\right)
             }_{\vb M}
  \underbrace{
  \left(\begin{array}{c} \vb k \\ \vb n \end{array}\right)
             }_{\vb c}
  =
  -
  \underbrace{
   \left(\begin{array}{c} \vb v\supt{E} \\ \vb v\supt{H} \end{array}\right)
             }_{\vb v}
}
%====================================================================%
where e.g.
%====================================================================%
$$ M\supt{EE}_{mn} = \VMV{\vb b_m}{\BG\supt{EE}}{\vb b_n}.$$
%====================================================================%

\noindent
On the other hand, the {\sc scuff-em} version of (\ref{PMCHWDiscrete})
reads 
%====================================================================%

\numeq{SCUFFDiscrete}
{
  \underbrace{
    \left(\begin{array}{cc} 
      \displaystyle{ \frac{1}{Z_0} \vb M\supt{EE}} 
     &\displaystyle{ -\vb M\supt{EM}}
    \\[10pt]
      \displaystyle{ \vb M\supt{ME} }
     &\displaystyle{ \vphantom{\frac{1}{Z_0}} -Z_0 \vb M\supt{MM}}
    \end{array}\right)
             }_{\vb{\hat M}}
  \,\,
  \underbrace{
  \left(\begin{array}{c}
     \displaystyle{\vphantom{\frac{1}{Z_0}} \vb k} \\[10pt]
     \displaystyle{-\frac{1}{Z_0}} \vb n 
  \end{array}\right)
             }_{\vb{\hat  c}}
  =
  -
  \underbrace{
   \left(\begin{array}{c} 
     \displaystyle{\frac{1}{Z_0} \vb v\supt{E}}  \\[10pt] 
     \displaystyle{\vphantom{\frac{1}{Z_0}} \vb v\supt{H} }
         \end{array}\right)
             }_{\vb{\hat  v}}
}
%====================================================================%

\noindent The various vectors and matrices in (\ref{SCUFFDiscrete}) are 
related to their counterparts in (\ref{PMCHWDiscrete}) according to
\begin{align*}
\hat{\vb M}
&=
    \left(\begin{array}{cc} 
      \displaystyle{\frac{1}{Z_0} \vb M\supt{EE}} &  -\vb M\supt{EM} \\[10pt]
                                   \vb M\supt{ME} & -Z_0 \vb M\supt{MM}
    \end{array}\right)
\\[10pt]
&= \underbrace{
    \left(\begin{array}{cc} 
                    \frac{1}{Z_0} & 0 \\[5pt] 
                                 0 & 1 
    \end{array}\right)
              }_{\equiv \vb A} 
%  
    \,\,
%  
    \underbrace{
    \left(\begin{array}{cc} 
      \vb M\supt{EE} &  \vb M\supt{EM} \\[5pt]
      \vb M\supt{ME} &  \vb M\supt{MM}
    \end{array}\right)
               }_{\vb M}
%  
    \,\,
%  
    \underbrace{
    \left(\begin{array}{cc} 
      1 & 0 \\[5pt] 
      0 & -Z_0
    \end{array}\right)
               }_{\equiv \vb B}
\end{align*}
and similarly 
$$ \hat{\vb c}=\vb B^{-1} \vb c, \qquad 
   \hat{\vb v}=\vb A      \vb v.
$$
Thus (\ref{SCUFFDiscrete}) is obtained from 
(\ref{PMCHWDiscrete}) by \textbf{(i)} left-multiplying
each side by $\vb A$, and \textbf{(ii)} inserting
$\vb B \vb B^{-1}$:
$$ \underbrace{\vb A \vb M \vb B}_{\hat{\vb M}}
   \underbrace{\vb B^{-1} \vb c}_{\hat{\vb c}}
   =
   \underbrace{\vb A      \vb v}_{\hat{\vb v}}.
$$
Since the correspondence between BEM matrices in the ordinary and 
{\sc scuff-em} formulations is 
$$ \hat{\vb M} = \vb A \vb M \vb B $$
the correspondence between \textit{inverse} BEM matrices reads
$$ \hat{\vb W} = \vb B^{-1} \vb W \vb A^{-1}. $$

\section{Transformations of compact PFT Formulas}

The compact expressions for power, force, and torque in 
BEM scattering computations generally take the form
%
$$ Q = \vb c^\dagger \vb O \vb c $$ 
%
where $Q$ is a power, force, or torque, and $\vb O$ is an
overlap matrix.

Using $\vb c=\vb B \hat{\vb c}$, we can express $Q$ 
in terms of the {\sc scuff-em} surface-current vector
as follows:
%====================================================================%
\begin{align*}
 Q &= \hat{\vb c}^\dagger \Big(\vb B^\dagger \vb O \vb B\Big) \hat{\vb c }
\end{align*}
%====================================================================%
The matrix that enters this equation is 
\begin{align*}
 \vb B^\dagger \vb O \vb B
&=
 \left(\begin{array}{cc}
   1 & 0 \\
   0 & -Z_0
 \end{array}\right)
 \left(\begin{array}{cc}
   \vb O\supt{EE} & \vb O\supt{EM} \\
   \vb O\supt{ME} & \vb O\supt{MM} \\
 \end{array}\right)
 \left(\begin{array}{cc}
   1 & 0 \\
   0 & -Z_0
 \end{array}\right)
\\[5pt]
%--------------------------------------------------------------------%
&=
 \left(\begin{array}{rr}
        \vb O\supt{EE}      & -Z_0 \vb O\supt{EM} \\[5pt]
   -Z_0 \vb O\supt{ME} & Z_0^2 \vb O\supt{MM} \\
 \end{array}\right).
\end{align*}
 

\end{document}
