%%%%%%%%%%%%%%%%%%%%%%%%%%%%%%%%%%%%%%%%%%%%%%%%%%%%%%%%%%%%%%%%%%%%%%
%%%%%%%%%%%%%%%%%%%%%%%%%%%%%%%%%%%%%%%%%%%%%%%%%%%%%%%%%%%%%%%%%%%%%%
%%%%%%%%%%%%%%%%%%%%%%%%%%%%%%%%%%%%%%%%%%%%%%%%%%%%%%%%%%%%%%%%%%%%%%
\newpage
\section{Displaced surface-integral PFT (DSIPFT)}

Conceptually the simplest way to compute PFTs is simply to
integrate the total time-average flux of energy (Poynting vector) 
or momentum (Maxwell stress tensor) over a bounding surface 
containing the body. In this section we review the surface-current 
bilinear
product formula obtained in this approach.

%%%%%%%%%%%%%%%%%%%%%%%%%%%%%%%%%%%%%%%%%%%%%%%%%%%%%%%%%%%%%%%%%%%%%%
%%%%%%%%%%%%%%%%%%%%%%%%%%%%%%%%%%%%%%%%%%%%%%%%%%%%%%%%%%%%%%%%%%%%%%
%%%%%%%%%%%%%%%%%%%%%%%%%%%%%%%%%%%%%%%%%%%%%%%%%%%%%%%%%%%%%%%%%%%%%%
\subsection*{Energy and momentum flux from field bilinears}

The Poynting flux and Maxwell stress tensor are quadratic functions
of the field components and may be conveniently written in the form
of 6-dimensional vector-matrix-vector products. 
In particular, the time-average power flux in the direction of a 
unit vector $\vbhat{n}$ is
%====================================================================%
\begin{align}
 \vb P(\vb x) \cdot \vbhat{n}
   &=\frac{1}{2}\text{ Re }\varepsilon_{ijk}\vbhat{n}_i E^*_j(\vb x) H_k(\vb x)
\nn
   &=\frac{1}{4}\bmc{F}^\dagger(\vb x) \, \bmc{N}\supt{P}(\vbhat{n}) \, \bmc{F}(\vb x)
\label{PVMVP}
\end{align}
%====================================================================%
with
%====================================================================%
$$
   \bmc N\supt{P}=
   \left(\begin{array}{cc}
   0       & \vb N\supt{P}   \\ [3pt]
  -\vb N\supt{P} & 0
   \end{array}\right), 
\qquad 
   \vb N\supt{P}
   =
   \left(\begin{array}{ccc}
   0          &  \hat{n}_z & -\hat{n_y} \\
  -\hat{n}_z  &  0         & +\hat{n_x} \\
   \hat{n}_y  & -\hat{n}_x & 0
   \end{array}\right).
$$
%====================================================================%
Similarly, the flux of $i$-directed linear momentum is
%--------------------------------------------------------------------%
\begin{align}
 \vb T_{i}(\vb x) \cdot \vbhat{n} 
&=\frac{1}{2}\text{ Re }
  \left[ \epsilon E^*_i(\vb x) E_j(\vb x) 
             +\mu H^*_i(\vb x) H_j(\vb x) 
       -\frac{\delta_{ij}}{2}
         \Big( \epsilon |\vb E|^2
              +\mu      |\vb H|^2
         \Big)
 \right] \hat {n}_j
\nn
&= \frac{1}{4}\bmc{F}^\dagger(\vb x) 
   \, \bmc{N}\IF \, \bmc{F}(\vb x)
\label{IFVMVP}
\end{align}
with
%--------------------------------------------------------------------%
$$
   \bmc N\IF=
   \left(\begin{array}{cc}
   \epsilon \vb N\IF & 0 \\
            0        & \mu \vb N\IF 
   \end{array}\right)
$$
%--------------------------------------------------------------------%
where the $3\times 3$ matrix $N\IF$ has entries
%--------------------------------------------------------------------%
$$ N\IF_{ab} = 
   \delta_{ai} \hat{n}_b + \delta_{bi} \hat{n}_a
  - \hat{n}_i \delta_{ab}.
$$
%====================================================================%
For example, if we are computing the $x$-force ($i=x$)
we have 
%====================================================================%
$$ \vb N^{x\text{\tiny F}}=\left(\begin{array}{ccc}
   \hat{n}_x & \hat n_y   & \hat n_z \\
   \hat{n}_y & -\hat{n}_x & 0 \\
   \hat{n}_z & 0          & -\hat{n}_x
  \end{array}\right).
$$
The flux of $i$-directed \textit{angular} momentum, useful
for computations of torque about an origin $\vb x_0$, is
%====================================================================%
\begin{align}
 \vb t_{i}(\vb x) \cdot \vbhat{n}
&=\frac{1}{2}
  \text{ Re }
  \varepsilon_{ijk}(\vb x-\vb x_0)_j T_{k\ell}(\vb x) \hat{n}_\ell
\\
&= \frac{1}{4}\bmc{F}^\dagger(\vb x) 
   \, \bmc{N}\IT \, \bmc{F}(\vb x)
\label{ITVMVP}
\end{align}
%====================================================================%
with 
%--------------------------------------------------------------------%
$$
   \bmc N\IT=
   \left(\begin{array}{cc}
   \epsilon \vb N\IT & 0 \\
            0        & \mu \vb N\IT 
   \end{array}\right)
$$
%--------------------------------------------------------------------%
where the $3\times 3$ matrix $N\IT$ has entries ($\vb D=\vb x-\vb x_0$)
%--------------------------------------------------------------------%
$$ N\IT_{ab}=
   \varepsilon_{ija}D_j \hat{n}_b
  +\varepsilon_{ijb}D_j \hat{n}_a
  -\delta_{ab} \varepsilon_{ijk} D_j \hat{n}_k.
$$

%%%%%%%%%%%%%%%%%%%%%%%%%%%%%%%%%%%%%%%%%%%%%%%%%%%%%%%%%%%%%%%%%%%%%%
%%%%%%%%%%%%%%%%%%%%%%%%%%%%%%%%%%%%%%%%%%%%%%%%%%%%%%%%%%%%%%%%%%%%%%
%%%%%%%%%%%%%%%%%%%%%%%%%%%%%%%%%%%%%%%%%%%%%%%%%%%%%%%%%%%%%%%%%%%%%%
\subsection*{Energy and momentum flux from surface-current bilinears}

Equations (\ref{PVMVP}), (\ref{IFVMVP}), and (\ref{ITVMVP}) express
the flux of power or momentum as bilinear products of the field
six-vectors $\bmc F$. Using (\ref{FFromC}), we can turn these into
bilinear products of the surface-current coefficient vectors
$\vb c$. For example, the power flux (\ref{PVMVP}) becomes 
%--------------------------------------------------------------------%
\begin{align}
 \vb P(\vb x) \cdot \vbhat{n}
&=\frac{1}{4}\bmc F^\dagger(\vb x) \bmc N\supt{P}(\vbhat{n}) \bmc F(\vb x)
\nn
&=\frac{1}{4}\sum_{\alpha \beta} 
  c_\alpha^* 
  \Big[ \bmc F^\dagger_\alpha(\vb x)
        \,
        \bmc N\supt{P}(\vbhat{n})
        \,
        \bmc F_\beta(\vb x)
  \Big]
  c_\beta 
\nn
&= \vb c^\dagger \vb M\supt{PFLUX}(\vb x, \vbhat{n}) \vb c 
\label{PFluxVMVP}
\end{align}
%--------------------------------------------------------------------%
where $\vb M\supt{PFLUX}(\vb x,\vbhat{n})$ is a matrix 
appropriate for $\vbhat{n}$-directed power flux in at $\vb x$.
The fluxes of $i$-directed linear and angular momentum read
similarly
%--------------------------------------------------------------------%
\begin{align}
\vb T_i(\vb x) \cdot \vbhat{n}
&= \vb c^\dagger \vb M\IFFlux(\vb x, \vbhat{n}) \vb c 
\label{IFFluxVMVP}
\\
\vb t_i(\vb x) \cdot \vbhat{n}
&= \vb c^\dagger \vb M\ITFlux(\vb x, \vbhat{n}) \vb c 
\label{ITFluxVMVP}
\end{align}
%--------------------------------------------------------------------%
The $\vb M$ matrices in (\ref{PFluxVMVP}), (\ref{IFFluxVMVP}), and 
(\ref{ITFluxVMVP}) are $N\subt{B}\times N\subt{B}$ matrices
whose entries are themselves 6-dimensional matrix-vector products:
\begin{align*}
M_{\alpha\beta}\supt{PFLUX}(\vb x, \vbhat{n}) 
 &= \frac{1}{4} 
    \bmc F^\dagger_\alpha(\vb x) 
    \bmc N\supt{P}(\vbhat{n})
    \bmc F_\beta(\vb x) 
\\
M_{\alpha\beta}\IFFlux(\vb x, \vbhat{n})
 &= \frac{1}{4} 
    \bmc F^\dagger_\alpha(\vb x) 
    \bmc N\supt{iF}(\vbhat{n})
    \bmc F_\beta(\vb x) 
\\
M_{\alpha\beta}\ITFlux(\vb x, \vbhat{n})
 &= \frac{1}{4} 
    \bmc F^\dagger_\alpha(\vb x) 
    \bmc N\supt{iT}(\vbhat{n})
    \bmc F_\beta(\vb x).
\end{align*}

%%%%%%%%%%%%%%%%%%%%%%%%%%%%%%%%%%%%%%%%%%%%%%%%%%%%%%%%%%%%%%%%%%%%%%
%%%%%%%%%%%%%%%%%%%%%%%%%%%%%%%%%%%%%%%%%%%%%%%%%%%%%%%%%%%%%%%%%%%%%%
%%%%%%%%%%%%%%%%%%%%%%%%%%%%%%%%%%%%%%%%%%%%%%%%%%%%%%%%%%%%%%%%%%%%%%
\subsection*{Power, force, and torque from surface-current bilinears}

The simplest way to obtain surface-current bilinears for the
total PFT on one or more bodies is simply to integrate the 
spatially-resolved fluxes of the previous section over a closed
bounding surface $\mc S$ containing the bodies.
Indeed, noting that the $\vb x$ and $\vbhat{n}$ dependence of
the flux expressions (\ref{PFluxVMVP}), (\ref{IFFluxVMVP}),
and (\ref{ITFluxVMVP}) is entirely contained in the $\vb M$
matrices, it is easy to integrate those expressions over
$\mc S$, then pull the surface-current vectors $\vb c$
outside the integral to identify what we shall call
the \textit{displaced surface-integral PFT} (DSIPFT) matrices. 

For example, the total power absorbed by material bodies
contained within a closed surface $\mc S$
is given by integrating the LHS of (\ref{PFluxVMVP})
over $\mc S$:
%====================================================================%
\begin{align}
  P_{\mc S}\sups{abs}&=\int_{\mc S} \vb P(\vb x) \cdot \vbhat{n} \, d\vb x
\intertext{with $\vbhat{n}$ taken to be the inward-pointing surface
           normal. Insert the RHS of (\ref{PFluxVMVP}) and pull 
           $\vb c^\dagger, \vb c$ outside the integral:}
             &=\vb c^\dagger \vb M\PS \vb c
\label{PTotVMVP}
\end{align}
where the elements of 
$\vb M\PS$
involve integrals over $\mc S$:
%====================================================================%
\numeq{MPSEntries}
{ M_{\alpha\beta}\PS
  = \frac{1}{4} \int_{\mc S} 
    \bmc F^\dagger_\alpha(\vb x) 
    \bmc N\supt{P}(\vbhat{n})
    \bmc F_\beta(\vb x) 
     \, d\vb x.
}
%====================================================================%
Similarly, the time-average $i$-directed force and torque on 
material bodies contained in $\mc S$ are
%====================================================================%
\begin{align}
 F_{i\mc S} &=\vb c^\dagger \vb M\IFS \vb c
\label{FTotVMVP}
\\
 \mc T_{i\mc S} &=\vb c^\dagger \vb M\ITS \vb c
\label{TTotVMVP}
\end{align}
%====================================================================%
where the entries of $\vb M\IFS$ and $\vb M\ITS$ are similar
to (\ref{MPSEntries}) with $\bmc N\supt{P} \to \bmc N\IF, \bmc N\IT.$
