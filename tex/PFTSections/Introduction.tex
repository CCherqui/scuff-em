%%%%%%%%%%%%%%%%%%%%%%%%%%%%%%%%%%%%%%%%%%%%%%%%%%%%%%%%%%%%%%%%%%%%%%
%%%%%%%%%%%%%%%%%%%%%%%%%%%%%%%%%%%%%%%%%%%%%%%%%%%%%%%%%%%%%%%%%%%%%%
%%%%%%%%%%%%%%%%%%%%%%%%%%%%%%%%%%%%%%%%%%%%%%%%%%%%%%%%%%%%%%%%%%%%%%
\newpage
\section{Overview}

The time-average power absorbed and scattered by, and the
time-average force or torque on, material bodies in the presence
of harmonically-varying electromagnetic radiation
are common quantities of interest in computational electromagnetism.
These quantities may be computed from knowledge of the
electromagnetic fields $\bmc F={\vb E \choose \vb H}$ in the 
presence of the bodies; more specifically, powers, forces,
and torques (PFTs) depend quadratically (bilinearly) on $\bmc F$.
On the other hand, in the surface-integral-equation (SIE) approach 
to electromagnetic scattering, one solves first for equivalent electric and 
magnetic \textit{surface currents} $\bmc C={\vb K \choose \vb N}$
flowing on the body surfaces; the fields are then obtained from
the currents via linear convolution, $\bmc F=\bmc G \star \bmc C$
(with $\bmc G$ a Green's dyadic). Since the PFTs are quadratic 
in the fields, and the fields are linear in the currents, it follows 
that the PFTs are quadratic in the currents, and thus that
we may bypass the field-computation step and compute PFTs directly 
as quadratic (bilinear) functions of the surface currents.

There are (at least) three distinct ways to express these
bilinear relationships, which originate from distinct
starting points for evaluating PFTs from fields:

%====================================================================%
\begin{description}

\item[1. DSIPFT:] By integrating the Poynting vector or Maxwell stress
tensor over a bounding surface that encloses (but does not
coincide with) the surface of the object. I will refer to this
as the ``displaced surface-integral PFT'' (DSIPFT) approach, where the term
``displaced'' indicates that that surface over which we integrate 
does not coincide with the object surface. This method is conceptually
straightforward but computationally expensive.

\item[2. OPFT: ] By integrating the Poynting vector or Maxwell stress
tensor over the surface of the object. I refer to this
as the ``overlap PFT'' approach. A key distinction compared to
the DSIPFT and EPPFT is that the matrix of the bilinear form it
produces is \textit{sparse.}

\item[3. EPPFT: ] By considering the work done on, and the momentum
transferred to, the equivalent surface currents by the fields.
I refer to this as the ``equivalence-principle PFT'' approach.
In contrast to the other two methods, this approach has the 
feature of being able to distinguish surface-current distributions
induced by soures inside and outside the body.

\end{description}
%====================================================================%

All three of these methods are implemented in {\sc scuff-em}. In this
note I first derive the three methods and discuss their implementation
in the specific context of an SIE solver based on the PMCHWT formulation
with RWG basis functions. Then I compare and contrast their
relative strengths and weaknesses.
