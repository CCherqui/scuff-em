\documentclass[letterpaper]{article}

../../../../tex/scufftex.tex
%\usepackage{hyperref}

\graphicspath{{figures/}}

%------------------------------------------------------------
%------------------------------------------------------------
%- Special commands for this document -----------------------
%------------------------------------------------------------
%------------------------------------------------------------
\newcommand{\GB}{\overline{G}}
\newcommand{\vbGB}{\overline{\vb G}}
%\newcommand{\BG}{\boldsymbol{\Gamma}}
\newcommand{\erfc}{\text{Erfc }}

%------------------------------------------------------------
%------------------------------------------------------------
%- Document header  -----------------------------------------
%------------------------------------------------------------
%------------------------------------------------------------
\title {Implementation of Ewald Summation in {\sc scuff-em}}
\author {Homer Reid}
\date {June 12, 2012}


%------------------------------------------------------------
%------------------------------------------------------------
%- Start of actual document
%------------------------------------------------------------
%------------------------------------------------------------

\begin{document}
\pagestyle{myheadings}
\markright{Homer Reid: Ewald Summation in {\sc scuff-em}}
\maketitle

\tableofcontents

%%%%%%%%%%%%%%%%%%%%%%%%%%%%%%%%%%%%%%%%%%%%%%%%%%%%%%%%%%%%%%%%%%%%%%
%%%%%%%%%%%%%%%%%%%%%%%%%%%%%%%%%%%%%%%%%%%%%%%%%%%%%%%%%%%%%%%%%%%%%%
%%%%%%%%%%%%%%%%%%%%%%%%%%%%%%%%%%%%%%%%%%%%%%%%%%%%%%%%%%%%%%%%%%%%%%
\newpage

\section{The Periodic Dyadic Green's Function}

Consider a two-dimensional lattice consisting of a set of 
lattice vectors $\{\vb L=(L_x, L_y)\}.$
We use the symbol $\vb p=(p_x, p_y)$ to denote a two-dimensional
Bloch wavevector.

The Bloch-periodic version of the scalar Helmholtz Green's function is
%====================================================================%
\numeq{PGF}
{ \GB(k; \vb p; \vb x) 
  \equiv 
  \sum_{\vb L} e^{i\vb p \cdot \vb L} 
               \frac{ e^{ i k |\vb x + \vb L|}}
                    { 4\pi|\vb x + \vb L|}
}
%====================================================================%

\section{Evaluation by Ewald Summation: The JRS Method}

To evaluate the sum in (\ref{PGF}) efficiently, we use a method 
outlined by Jordan, Richter, and Sheng (JRS)\footnote{K. E. Jordan, G. R. Richter, 
and P. Sheng, "An Efficient Numerical Evaluation of the Green's 
Function for the Helmholtz Operator on Periodic Structures," 
\textit{Journal of Computational Physics} \textbf{63} 222 (1986).
\texttt{http://dx.doi.org/10.1016/0021-9991(86)90093-8}}
based on the original ideas of Ewald\footnote{}. For completeness,
I will briefly recapitulate the key ideas of this approach.

The starting point is the identity
$$ \frac{e^{ikr}}{4\pi r} 
   \equiv 
   \frac{1}{4\pi} \cdot \frac{2}{\sqrt{\pi}}
   \int_{\mathcal C} e^{ -z^2 r^2 + \frac{k^2}{4z^2}} dz
$$ 
where $\mathcal{C}$ is a contour in the complex plane running from
the origin to $\infty$ and satisfying certain conditions depending
on $k$. One choice of $\mathcal{C}$ 
that suffices for $k$ in the upper-right quadrant is a straight line
running from the origin to infinity at an angle of $-\pi/4$ from
the positive real axis, i.e. 
$$ z=\zeta s, \qquad 0\le s \le \infty, \qquad \zeta=e^{-i\pi/4}$$
in terms of which our identity becomes
$$ \frac{e^{ikr}}{4\pi r} 
   \equiv 
   \frac{1}{4\pi} \cdot \frac{2}{\sqrt{\pi}}
   \cdot \zeta \cdot 
   \int_{0}^\infty e^{ + i s^2 r^2 + \frac{ik^2}{4s^2}} ds.
$$ 

\subsection*{Split the sum into direct-lattice-local and reciprocal-lattice-local
             contributions}

\begin{align}
\GB(k; \vb p; \vb x) 
 &= \GB^{(1)}(k; \vb p; \vb x) 
   +\GB^{(2)}(k; \vb p; \vb x) 
\label{GBDecomposition}
\\
\GB^{(1)}(k; \vb p; \vb x) 
&= \frac{1}{4\pi} \cdot \frac{2}{\sqrt{\pi}}\cdot \zeta\cdot
   \sum_{\vb L} e^{i\vb p \cdot \vb L} 
                \int_0^E e^{is^2 |\vb x+\vb L|^2 + \frac{ik^2}{4s^2}} ds
\label{GB1Sum}
\\
\GB^{(2)}(k; \vb p; \vb x) 
&= \frac{1}{4\pi} \cdot \frac{2}{\sqrt{\pi}}\cdot\zeta\cdot
   \sum_{\vb L} e^{i\vb p \cdot \vb L} 
                \int_E^\infty e^{is^2 |\vb x + \vb L|^2 + \frac{ik^2}{4s^2}} ds
\label{GB2Sum}
\end{align}

\subsection*{Evaluate $\GB^{(2)}$ in real space} 

The first step is to tackle the sum in (\ref{GB2Sum}) head-on. 
The integral in that equation may be evaluated in closed form in terms of
the complex error function:
\numeq{ErfInt}
{
 \frac{2}{\sqrt\pi} \cdot \zeta \int_E^\infty e^{is^2 R^2 + \frac{ik^2}{4s^2}} ds
 = \frac{1}{2R}\bigg\{ e^{ikR} \,\erfc\Big[ RE\zeta + \frac{ik}{2E\zeta}\Big]
                     +e^{-ikR} \,\erfc\Big[ RE\zeta - \frac{ik}{2E\zeta}\Big]
               \bigg\}
}
Plugging into (\ref{GB2Sum}), the real-space sum is 
\begin{align*}
\GB^{(2)}(k; \vb p; \vb x) 
&= \frac{1}{8\pi}
   \sum_{\vb L} 
   \frac{e^{i\vb p \cdot \vb L}}{|\vb x + \vb L|}
    \bigg\{ e^{ik|\vb x + \vb L|}
            \erfc\Big[R|\vb x + \vb L| + \frac{ik}{2E} \Big]
\\
&\hspace{1.5in}
 +          e^{-ik|\vb x + \vb L|}
    \erfc\Big[R|\vb x + \vb L| - \frac{ik}{2E} \Big]
    \bigg\}
\end{align*}
This sum converges very rapidly in real space. Indeed, as soon as 
$|\vb r + \vb L|$ is greater than a few times $E$, the $\erfc$\!\!
factors begin to bite down \textit{doubly exponentially} on the
summand, and the contribution of distant lattice sites becomes 
utterly negligible. In practice, for a 2D lattice we achieve 8 
or more digits of accuracy by retaining only $\approx$ terms in the sum.
Thus we consider this term done, and move on to $\GB^{(1)}.$ 

\subsection*{Evaluate $\GB^{(1)}$ in reciprocal space} 

We would next like to do for $\GB^{(1)}$ what we just did for $\GB^{(2)}$,
but this requires a little more work. 

First, the integral in (\ref{GB2Sum}) involves the lower range of the 
$s$ integral (from $s=0$ to $E$), but the nice closed-form
expression we have, equation (\ref{ErfInt}), is for the upper range 
of the integral ($s=E$ to $\infty$). This difficulty was circumvented 
by Ewald in an ingenious way by appealing to the properties of the 
classical \textit{Jacobi theta function}, which is defined as 
\numeq{ThetaFunction}{\theta(s)=\sum_{n=-\infty}^\infty e^{-n^2 \pi s}}
and which satisfies the amazing and bizarre functional identity 
\numeq{ThetaIdentity}
{ \theta(s)=\sqrt\frac{1}{s} \cdot \theta\left(\frac{1}{s}\right).}
We will need a slightly more complicated version of this identity,
obtained by generalizing (\ref{ThetaFunction}) in two ways: 
\textbf{(a)} we consider a more general summand in which the exponent
contains a term linear in $n$ besides the quadratic term 
in (\ref{ThetaFunction}), and 
\textbf{(b)} instead of summing over all integers $n$ (which we may 
think of as a sum over a one-dimensional lattice) we instead sum 
over a \textit{two-dimensional} lattice.
The result is that the sum over lattice vectors $\vb L$ that enters into
our equation (\ref{GB1Sum}) is converted into a sum over
\textit{reciprocal lattice} vectors $\BG$:
\begin{align}
&\hspace{-0.5in}
 \frac{1}{4\pi} \cdot \frac{2}{\sqrt\pi} \sum_{\vb L} e^{i \vb p \cdot \vb L}
 e^{-s^2|\vb x + \vb L|^2 + \frac{k^2}{4s^2}}
\nn
&=\frac{1}{s^2} \cdot \frac{A_{\BG}}{8\pi^{5/2}}
 \sum_{\BG} e^{i(\BG-\vb p) \cdot \vb x} 
            e^{-\frac{1}{4s^2}[|\BG - \vb p|^2 - k^2] - z^2 s^2}
\label{DirectToReciprocal}
\end{align}
Here the sum is over all vectors $\BG$ that satisfy 
$\BG \cdot \vb L=2\pi$ for some lattice vector $\vb L,$
and $A_{\BG}$ is the area of Brillouin zone of the 
reciprocal lattice.\footnote{For a simple rectangular 
lattice with basis vectors 
$$\vb L_1=L_x\vbhat{x}, \qquad \vb L_2=L_y\vbhat{y},$$
a basis for the reciprocal lattice is 
$$\BG_1=\frac{2\pi}{L_x}\vbhat{x},\qquad \BG_2=\frac{2\pi}{L_y}\vbhat{y}$$
and we have $A_{\BG}=\frac{4\pi^2}{L_x L_y}$.\\
For a hexagonal lattice with basis vectors 
$$ \vb L_1=L\vbhat{x}, \qquad 
   \vb L_2=\frac{L}{2}\vbhat{x} + \frac{L\sqrt{3}}{2}\vbhat{y}
$$
a basis for the reciprocal lattice is 
$$ \BG_1=\frac{2\pi}{L}\vbhat{x} + \frac{2\pi}{L\sqrt{3}}\vbhat{y},
   \qquad 
   \BG_2=\frac{4\pi}{\sqrt{3} L}\vbhat{y}
$$
and we have $A_{\BG}=\frac{8\pi^2}{L^2 \sqrt{3}}$.
}
Inserting (\ref{DirectToReciprocal}) in (\ref{GB1Sum}), we have 
\begin{align}
\GB^{(1)}(k; \vb p; \vb x) 
&=\frac{\zeta A_{\BG}}{8\pi^{5/2}} \sum_{\BG} 
    e^{i(\BG-\vb p) \cdot \vb x} 
    \int_{0}^E 
    \frac{ds}{s^2} 
    e^{\frac{i}{4s^2}[|\BG - \vb p|^2 - k^2] + iz^2 s^2}
\nonumber
\intertext{Change integration variables according to $s\to\frac{1}{2s}$:}
&=\frac{\zeta A_{\BG}}{4\pi^{5/2}} \sum_{\BG} 
    e^{i(\BG-\vb p) \cdot \vb x} 
    \int_{\frac{1}{2E}}^\infty
    ds
    e^{is^2[|\BG - \vb p|^2 - k^2] + \frac{iz^2}{4s^2}}
\nonumber
\intertext{Evaluate the integral using (\ref{ErfInt}):}
&=\frac{A_{\BG}}{16\pi^2} \sum_{\BG}
   \frac{e^{i(\BG-\vb p) \cdot \vb x}}{Q(\BG, \vb p, k)} 
   \bigg\{  e^{+Qz}\erfc\Big[ \frac{Q(\BG, \vb p, k)}{2E\zeta} + E\zeta z \Big]
\label{GB1Sum2}\\
&\hspace{2in}
          +e^{-Qz}\erfc\Big[ \frac{Q(\BG, \vb p, k)}{2E\zeta} - E\zeta z \Big]
   \bigg\}
\nonumber
\end{align}
where we put
$$
Q(\BG, \vb p, k)  = \sqrt{ (\BG-\vb p)^2 - k^2 }.
$$
\end{document}
