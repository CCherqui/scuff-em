\documentclass{article} 
../../../../tex/scufftex.tex

\graphicspath{{figures/}}

%%%%%%%%%%%%%%%%%%%%%%%%%%%%%%%%%%%%%%%%%%%%%%%%%%%%%%%%%%%%%%%%%%%%%%
% special commands for this document %%%%%%%%%%%%%%%%%%%%%%%%%%%%%%%%% 
%%%%%%%%%%%%%%%%%%%%%%%%%%%%%%%%%%%%%%%%%%%%%%%%%%%%%%%%%%%%%%%%%%%%%%

%\renewcommand{\inp}[2]{ \big\langle #1 \big| #2 \big\rangle}
\newcommand{\INP}[2]{ \Big\langle #1 \Big| #2 \Big\rangle}
\newcommand{\vmv}[3]{ \big\langle #1 \big| #2 \big| #3 \Big\rangle}
\newcommand{\VMV}[3]{ \Big\langle #1 \Big| #2 \Big| #3 \Big\rangle}

%**************************************************
%* Document header info ***************************
%**************************************************
\title{Surface Impedance Boundary Conditions in {\sc scuff-em}}
\author {Homer Reid}
\date {May 9, 2012}

%**************************************************
%* Start of actual document ***********************
%**************************************************

\begin{document}
\maketitle

\pagestyle{myheadings}
\markright{Homer Reid: Surface Impedance Boundary Conditions in {\sc scuff-em} }

\tableofcontents 

%%%%%%%%%%%%%%%%%%%%%%%%%%%%%%%%%%%%%%%%%%%%%%%%%%%%%%%%%%%%%%%%%%%%%%
%%%%%%%%%%%%%%%%%%%%%%%%%%%%%%%%%%%%%%%%%%%%%%%%%%%%%%%%%%%%%%%%%%%%%%
%%%%%%%%%%%%%%%%%%%%%%%%%%%%%%%%%%%%%%%%%%%%%%%%%%%%%%%%%%%%%%%%%%%%%%
\newpage
\section{The Basic Formulation}

\subsubsection*{The usual boundary condition for PEC surfaces}

In SIE/BEM formulations, the usual boundary condition imposed at the 
surface of a perfectly electrically conducting (PEC) scatterer is that 
the total tangential electric field vanish:
\numeq{PECBC1}{ \vb E_{\parallel}\sups{tot}(\vb x) = 0. }

\subsubsection*{The usual EFIE procedure for PEC surfaces}
 
The usual electric-field-integral-equation (EFIE) procedure for 
handling scattering from zero-thickness PEC surfaces is the
the following:
%
\begin{enumerate}
 \item We introduce an electric surface current $\vb K(\vb x)$ on 
       the scatterer surface. This current is related to 
       the total tangential $\vb H$-field according to 
       $$\vb K(\vb x) = \vbhat{n}\times \vb H\sups{tot}(\vb x).$$
 \item We do \textit{not} need to introduce a magnetic surface 
       current; such a current would be proportional to the total
       tangential $\vb E$ field, but this vanishes in view of 
       the boundary condition (\ref{PECBC1}):
       $$\vb N(\vb x) = -\vbhat{n}\times \vb E\sups{tot}(\vb x)\equiv0.$$
 \item $\vb K$ gives rise to scattered $\vb E$ and $\vb H$ fields according to
       \begin{align*}
        \vb E\sups{scat} &= \int \BG_{\parallel}\supt{EE}(\vb x, \vb x^\prime) 
                            \cdot \vb K(\vb x^\prime) d\vb x^\prime
        \\
        \vb H\sups{scat} &= \int \BG_{\parallel}\supt{ME}(\vb x, \vb x^\prime) 
                            \cdot \vb K(\vb x^\prime) d\vb x^\prime.
       \end{align*}
 \item We solve for $\vb K$ by demanding that the scattered field 
       to which it gives rise satisfy the boundary condition
       (\ref{PECBC1}):
       \begin{align*}
                \vb E_{\parallel}\sups{scat}(\vb x) 
            &= -\vb E_{\parallel}\sups{inc}(\vb x)
        \intertext{or} 
                \int \BG_{\parallel}\supt{EE}(\vb x, \vb x^\prime) 
                     \cdot \vb K(\vb x^\prime) d\vb x^\prime 
            &= - \vb E\sups{inc}(\vb x).
        \end{align*}
\end{enumerate}
%
\subsubsection*{The modified boundary condition for IPEC surfaces}

At the surface of an \textit{imperfectly} electrically conducting 
(IPEC) scatterer with surface impedance $Z_s$, boundary condition 
(\ref{PECBC1}) is modified to read 
\numeq{IPECBC1}
{
\vb E_{\parallel}\sups{tot}(\vb x) 
= Z_s \vbhat{n}\times \vb H_{\parallel}\sups{tot}(\vb x)
}

\subsubsection*{The modified EFIE procedure for IPEC surfaces}
 
For IPEC surfaces, I modify the EFIE procedure as follows:
%
\begin{enumerate}
 \item As above, I introduce an electric surface current $\vb K(\vb x)$,
       which is again related to the total tangential $\vb H$-field 
       according to 
       $$\vb K(\vb x) = \vbhat{n}\times \vb H\sups{tot}(\vb x).$$
 \item In contrast to the above, I now have a nonvanishing magnetic 
       current, which is not an independent unknown but is instead
       determined by $\vb K$ via equation (\ref{IPECBC1}):
       the boundary condition (\ref{PECBC1}):
       $$\vb N(\vb x) = -\vbhat{n}\times \vb E\sups{tot}(\vb x)
                      = -Z_s \vbhat{n}\times \vb K(\vb x).$$
 \item $\vb K$ and $\vb N\equiv \vb N[\vb K]$ give rise to scattered $\vb E$ and 
       $\vb H$ fields according to
       \begin{align*}
         \vb E\sups{scat} &= 
           \int \left\{       \BG_{\parallel}\supt{EE}(\vb x, \vb x^\prime) 
                        \cdot \vb K(\vb x^\prime)
                        \,-\, 
                           Z_s\BG_{\parallel}\supt{ME}(\vb x, \vb x^\prime) 
                        \cdot \Big[\vbhat{n}\times \vb K(\vb x^\prime)\Big] 
                \right\}\,d\vb x^\prime
         \\
         \vb H\sups{scat} &= 
           \int \left\{       \BG_{\parallel}\supt{ME}(\vb x, \vb x^\prime) 
                        \cdot \vb K(\vb x^\prime)
                        \,-\, 
                           Z_s\BG_{\parallel}\supt{MM}(\vb x, \vb x^\prime) 
                        \cdot \Big[\vbhat{n}\times \vb K(\vb x^\prime)\Big] 
                \right\}\,d\vb x^\prime.
       \end{align*}
%       which I could write alternatively as 
%       \begin{align*}
%         \vb E\sups{scat} &= 
%           \int \left\{ \BG_{\parallel}\supt{EE}(\vb x, \vb x^\prime) 
%                        +Z_s \Big[\vbhat{n}\times \BG_{\parallel}\supt{ME}(\vb x, \vb x^\prime)\Big]
%                \right\}
%                        \cdot \vb K(\vb x^\prime) \,d\vb x^\prime
%         \\
%         \vb H\sups{scat} &= 
%           \int \left\{ \BG_{\parallel}\supt{ME}(\vb x, \vb x^\prime) 
%                        +Z_s \Big[\vbhat{n}\times \BG_{\parallel}\supt{MM}(\vb x, \vb x^\prime)\Big]
%                \right\}
%                        \cdot \vb K(\vb x^\prime) \,d\vb x^\prime
%       \end{align*}
 \item I solve for $\vb K$ by demanding that the scattered fields 
       to which it -- now in concert with $\vb N$ -- gives rise 
       satisfy the boundary condition (\ref{IPECBC1}):
       $$
           \vb E_{\parallel}\sups{scat}(\vb x) 
            -Z_s \vbhat{n}\times \vb H_{\parallel}\sups{scat}(\vb x)
       = 
          -\vb E_{\parallel}\sups{inc}(\vb x) 
            +Z_s \vbhat{n}\times \vb H_{\parallel}\sups{inc}(\vb x).
       $$
       or
       \begin{align*}
       \int\bigg\{ &\BG_{\parallel}\supt{EE}(\vb x, \vb x^\prime) 
                   \cdot \vb K(\vb x^\prime)
                   \\
                   &- Z_s\BG_{\parallel}\supt{ME}(\vb x, \vb x^\prime)
                   \cdot \Big[\vbhat{n}\times \vb K(\vb x^\prime)\Big] 
                   \\
                   &-Z_s\vbhat{n}\times \BG_{\parallel}\supt{EM}(\vb x, \vb x^\prime) 
                   \cdot \vb K(\vb x^\prime)
                   \\
                   &+Z_s^2
                   \vbhat{n}\times\BG_{\parallel}\supt{MM}(\vb x, \vb x^\prime)
                   \cdot \Big[\vbhat{n}\times \vb K(\vb x^\prime)\Big]
       \bigg\}
       = -\vb E_{\parallel}\sups{inc}(\vb x) 
            +Z_s \vbhat{n}\times \vb H_{\parallel}\sups{inc}(\vb x).
       \end{align*}
\end{enumerate}

%%%%%%%%%%%%%%%%%%%%%%%%%%%%%%%%%%%%%%%%%%%%%%%%%%%%%%%%%%%%%%%%%%%%%%
%%%%%%%%%%%%%%%%%%%%%%%%%%%%%%%%%%%%%%%%%%%%%%%%%%%%%%%%%%%%%%%%%%%%%%
%%%%%%%%%%%%%%%%%%%%%%%%%%%%%%%%%%%%%%%%%%%%%%%%%%%%%%%%%%%%%%%%%%%%%%
\newpage
\section{How it works: A simple analytically-solvable model}

\end{document}
