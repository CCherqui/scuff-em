\newpage
\section{Notation for Spherical Helmholtz Solutions}
\label{SphericalHelmoltzAppendix}

The spherical-multipole computations of 
Section \ref{SphericalMultipoleMatrixElementSection}
make reference to standard results in the theory of 
scattering from spherical geometries.\footnote{As discussed 
at length in Morse and Feschbach, Jackson, and other 
standard texts.} Here is a summary of the notation 
and conventions I use for this material.

%%%%%%%%%%%%%%%%%%%%%%%%%%%%%%%%%%%%%%%%%%%%%%%%%%%%%%%%%%%%%%%%%%%%%%
%%%%%%%%%%%%%%%%%%%%%%%%%%%%%%%%%%%%%%%%%%%%%%%%%%%%%%%%%%%%%%%%%%%%%%
%%%%%%%%%%%%%%%%%%%%%%%%%%%%%%%%%%%%%%%%%%%%%%%%%%%%%%%%%%%%%%%%%%%%%%
\subsection{Vector Helmholtz Solutions}

I use the symbols $\vb M(\vb r)$, $\vb N(\vb r)$ to denote 
divergenceless vector-valued functions whose components each 
separately satisfy the Helmholtz equation:
$$
 \nabla \cdot 
 \left\{ \begin{array}{c} \vb M \\ \vb N \end{array} \right\} 
  = 0 
$$
and
\begin{align*}
%%%%%%%%%%%%%%%%%%%%%%%%%
 \Big[\nabla^2 + k^2 \Big] 
 \left\{ \begin{array}{c} \vb M \\ \vb N \end{array} \right\} 
 &= 0 
 \qquad \text{real frequency}
  \\[8pt]
%%%%%%%%%%%%%%%%%%%%%%%%%
%%%%%%%%%%%%%%%%%%%%%%%%%
 \Big[\nabla^2 - \kappa^2 \Big] 
 \left\{ \begin{array}{c} \vb M \\ \vb N \end{array} \right\} 
 &= 0 
 \qquad \text{imaginary frequency}.
%%%%%%%%%%%%%%%%%%%%%%%%%%
\end{align*}
%
\subsubsection*{Vector Helmholtz Solutions for Spherical Geometries}
For spherical geometries we take $\vb M$ to have the form
%
\numeq{Mlm} {\vb M_{lm}(r,\theta,\phi) = R_l(r) \vb X_{lm}(\theta,\phi) }
%
where\footnote{Note that my $\vb X$ function is $-i$ times Jackson's $\vb X$
function.}
$$
 \vb X_{lm}(\theta, \phi)
=-(\vb r \times \nabla) Y_{lm}(\theta,\phi)
\equiv \frac{1}{\sqrt{l(l+1)}}
   \Big[ \frac{im}{\sin\theta} Y_{lm} \hat{\boldsymbol{\theta}}
         -\pard{Y_{lm}}{\theta} \hat{\boldsymbol{\varphi}}
   \Big]
$$
and where the radial function is a spherical Bessel function whose
precise form depends on whether we are at real or imaginary frequency and 
whether we want a solution that is well-behaved at the origin (the 
``interior'' solution) or well-behaved at infinity (the ``exterior'' 
solution):
%%%%%%%%%%%%%%%%%%%%%%%%%%%%%%%%%%%%%%%%%%%%%%%%%%
%%%%%%%%%%%%%%%%%%%%%%%%%%%%%%%%%%%%%%%%%%%%%%%%%%
%%%%%%%%%%%%%%%%%%%%%%%%%%%%%%%%%%%%%%%%%%%%%%%%%%
\begin{center}
\textbf{Forms of the radial function $R_l(r):$}

\medskip

\begin{tabular}{|c|c|c|}  \hline
                               & \textbf{Interior} & \textbf{Exterior}   \\ \hline
\textbf{Real frequency}        & $j_l(kr)$         & $j_l(kr) +iy_l(kr)$ \\ \hline
\textbf{Imaginary frequency}   & $i_l(\kappa r)$   & $k_l(\kappa r)    $ \\ \hline
\end{tabular}
\end{center}
\medskip

%%%%%%%%%%%%%%%%%%%%%%%%%%%%%%%%%%%%%%%%%%%%%%%%%%
%%%%%%%%%%%%%%%%%%%%%%%%%%%%%%%%%%%%%%%%%%%%%%%%%%
%%%%%%%%%%%%%%%%%%%%%%%%%%%%%%%%%%%%%%%%%%%%%%%%%%

\noindent Next, $\vb N$ is defined in terms of $\vb M$ according to
%
\numeq{NDef}
{\vb N_{lm}
   \equiv 
   \begin{cases} 
     %%%%%%%%%%%%%%%%%%%%%%%%%%%%%%%%%%%%%%%%%%%%%%%%%%
     \displaystyle{ -\frac{1}{ik} \nabla \times \vb M_{lm} }, 
      \qquad &\text{real frequency} \\[10pt]
     %%%%%%%%%%%%%%%%%%%%%%%%%%%%%%%%%%%%%%%%%%%%%%%%%%
     \displaystyle{ \quad \frac{1}{\kappa} \nabla \times \vb M_{lm} },
      \qquad &\text{imaginary frequency}.
     \end{cases}
}
%
This definition for $\vb N,$ together with the facts that $\vb M$ 
has vanishing divergence and satisfies the vector Helmholtz equation,
lead to the reciprocal curl identities:
%
$$\nabla \times \vb N_{lm}
   \equiv 
   \begin{cases} 
     %%%%%%%%%%%%%%%%%%%%%%%%%%%%%%%%%%%%%%%%%%%%%%%%%%
     \displaystyle{ +ik \vb M_{lm} }, 
      \qquad &\text{real frequency} \\[10pt]
     %%%%%%%%%%%%%%%%%%%%%%%%%%%%%%%%%%%%%%%%%%%%%%%%%%
     \displaystyle{ -\kappa \vb M_{lm} },
      \qquad &\text{imaginary frequency}.
     \end{cases}
$$

%%%%%%%%%%%%%%%%%%%%%%%%%%%%%%%%%%%%%%%%%%%%%%%%%%%%%%%%%%%%%%%%%%%%%%
%%%%%%%%%%%%%%%%%%%%%%%%%%%%%%%%%%%%%%%%%%%%%%%%%%%%%%%%%%%%%%%%%%%%%%
%%%%%%%%%%%%%%%%%%%%%%%%%%%%%%%%%%%%%%%%%%%%%%%%%%%%%%%%%%%%%%%%%%%%%%
\subsubsection*{Spherical Components of $\vb M$ and $\vb N$}
Explicit expression for the components of $\vb M$ and $\vb N$ are 
%====================================================================%
\begin{align}
%--------------------------------------------------------------------%
 \vb M_{lm}
&= \frac{1}{\sqrt{l(l+1)}}
   R_l\bigg[ +\frac{im}{\sin\theta} Y_{lm}\hat{\boldsymbol{\theta}} 
                   -\pard{Y_{lm}}{\theta}\hat{\boldsymbol{\varphi}}
         \bigg]
%--------------------------------------------------------------------%
\intertext{(valid at real or imaginary frequency)}
%--------------------------------------------------------------------%
 \vb N_{lm}
&= \frac{1}{-ik\sqrt{l(l+1)}}
   \bigg[\frac{l(l+1)}{r}R_l Y_{lm}\vbhat{r}
         +\Big(\frac{R_l}{r} + \pard{R_l}{r}\Big)
          \Big\{\pard{Y_{lm}}{\theta} \hat{\boldsymbol{\theta}}
              +\frac{im}{\sin\theta} Y_{lm} \hat{\boldsymbol{\varphi}}
          \Big\} 
   \bigg]
\intertext{(at real frequency)}
%--------------------------------------------------------------------%
 \vb N_{lm}
&= \frac{1}{\kappa\sqrt{l(l+1)}}
   \bigg[\frac{l(l+1)}{r}R_l Y_{lm}\vbhat{r}
         +\Big(\frac{R_l}{r} + \pard{R_l}{r}\Big)
          \Big\{\pard{Y_{lm}}{\theta} \hat{\boldsymbol{\theta}}
              +\frac{im}{\sin\theta} Y_{lm} \hat{\boldsymbol{\varphi}}
          \Big\} 
   \bigg]
%--------------------------------------------------------------------%
\end{align}
at imaginary frequency. Here $R_l$ is one of the functions in the table above.

%%%%%%%%%%%%%%%%%%%%%%%%%%%%%%%%%%%%%%%%%%%%%%%%%%%%%%%%%%%%%%%%%%%%%%
%%%%%%%%%%%%%%%%%%%%%%%%%%%%%%%%%%%%%%%%%%%%%%%%%%%%%%%%%%%%%%%%%%%%%%
%%%%%%%%%%%%%%%%%%%%%%%%%%%%%%%%%%%%%%%%%%%%%%%%%%%%%%%%%%%%%%%%%%%%%%
\subsubsection*{Hat/Wedge Notation}
%
I will use the notation $\MInt,\NInt$ and $\MExt,\NExt$ to refer 
respectively to the ``interior'' and ``exterior'' solutions. 
(Mnemonic: The $\vee$ symbol atop $\vb M$ or $\vb N$ suggests that 
the quantity in question is radiating \textit{outward} into space, 
as appropriate for an exterior solution.)
Thus
%====================================================================%
\begin{align}
%--------------------------------------------------------------------%
   \left\{\begin{array}{c} 
     \MInt_{lm}(r, \theta, \varphi) \\[5pt] 
     \MExt_{lm}(r, \theta, \varphi) 
   \end{array}\right\}
&=\frac{1}{\sqrt{l(l+1)}}
   \left\{\begin{array}{c} 
     j_l(kr) \\[5pt] 
     j_l(kr)+iy_l(kr)
   \end{array}\right\}
   \vb X_{lm}(\theta, \varphi)
  \label{MDefReal} 
\intertext{at real frequency, and}
%--------------------------------------------------------------------%
   \left\{\begin{array}{c} 
     \MInt_{lm}(r, \theta, \varphi) \\[5pt] 
     \MExt_{lm}(r, \theta, \varphi) 
   \end{array}\right\}
&=\frac{1}{\sqrt{l(l+1)}}
   \left\{\begin{array}{c} 
     i_l(\kappa r) \\[5pt] 
     k_l(\kappa r)
   \end{array}\right\}
   \vb X_{lm}(\theta, \varphi)
  \label{MDefImag}
%--------------------------------------------------------------------%
\end{align}
at imaginary frequency.

Notwithstanding the resemblance of my $\wedge$ and $\vee$ adornments to 
Slavic diacritic marks, there is no danger of ambiguity here, as in 
this particular document I will resist the temptation to write in either 
the Serbo-Croatian \textit{or} Bosnian dialects. (This means I will
have to forego all my best jokes, but whaddya gonna do.)
%====================================================================%

%%%%%%%%%%%%%%%%%%%%%%%%%%%%%%%%%%%%%%%%%%%%%%%%%%%%%%%%%%%%%%%%%%%%%%
%%%%%%%%%%%%%%%%%%%%%%%%%%%%%%%%%%%%%%%%%%%%%%%%%%%%%%%%%%%%%%%%%%%%%%
%%%%%%%%%%%%%%%%%%%%%%%%%%%%%%%%%%%%%%%%%%%%%%%%%%%%%%%%%%%%%%%%%%%%%%
\subsubsection*{Compound Indices}

I will frequently use $\alpha=(lm)$ and $\beta=(l^\prime m^\prime)$ 
as compound indices in summations over spherical multipole indices,
i.e. 
$$ \sum_{l=0}^{l\subt{max}} \sum_{m=-l}^{l}
   \sum_{l^\prime}^{l\subt{max}^\prime} \sum_{m=-l^\prime}^{l^\prime}
   \Big\{ \cdots \Big\}_{lm} 
   \Big\{ \cdots \Big\}_{l^\prime m^\prime} 
\qquad \longrightarrow \qquad 
   \sum_{\alpha=0}^{N_\alpha-1}
   \sum_{\beta=0}^{N_\beta-1}
   \Big\{ \cdots \Big\}_{\alpha} 
   \Big\{ \cdots \Big\}_{\beta}
$$
where 
$\alpha = l^2 + l + m$,
$N_\alpha=(l\subt{max}+1)^2.$ 

%%%%%%%%%%%%%%%%%%%%%%%%%%%%%%%%%%%%%%%%%%%%%%%%%%%%%%%%%%%%%%%%%%%%%%
%%%%%%%%%%%%%%%%%%%%%%%%%%%%%%%%%%%%%%%%%%%%%%%%%%%%%%%%%%%%%%%%%%%%%%
%%%%%%%%%%%%%%%%%%%%%%%%%%%%%%%%%%%%%%%%%%%%%%%%%%%%%%%%%%%%%%%%%%%%%%
\subsection{Translation Matrices}

The translation matrices relate exterior Helmholtz solutions 
about one origin to interior Helmholtz solutions about a 
different origin. Let the former origin be $\vb x_0$ and the latter
origin be $\vb x_0^\prime.$ Then the relevant relations read
%====================================================================%
\begin{subequations}
\begin{align*}
 \Big|\MExt_\alpha(\vb x - \vb x_0^\prime) \Big\rangle
&= \sum_{\beta} 
     A_{\alpha\beta} \Big|\MInt_\beta(\vb x-\vb x_0) \Big\rangle 
    +B_{\alpha\beta} \Big|\NInt_\beta(\vb x-\vb x_0) \Big\rangle
\\
%--------------------------------------------------------------------%
 \Big|\NExt_\alpha(\vb x - \vb x_0^\prime) \Big\rangle
&= \sum_{\beta} 
    -B_{\alpha\beta} \Big|\MInt_\beta(\vb x-\vb x_0) \Big\rangle 
    +A_{\alpha\beta} \Big|\NInt_\beta(\vb x-\vb x_0) \Big\rangle.
\end{align*}
\label{TranslationMatrixDefinition}
\end{subequations}
%====================================================================%
where the translation matrices $\{A_{\alpha\beta}, B_{\alpha\beta}\}$
depend on $k$ and on the separation
vector connecting the two origins.

The bra-ket notation in (\ref{TranslationMatrixDefinition})
obscures the need to rotate the spherical components of the
vector functions $\vb M$ and $\vb N$, which is actually fine  
as long as we only ever need to talk about inner products
of $\vb M$ and $\vb N$ with other vector-valued functions.
In particular, we have
\begin{align*}
 \big\langle \vb f \big| \MExt_\alpha \Big>
&= \sum_{\beta } \Big\{ 
   A_{\alpha\beta} \big\langle \vb f \big| \MInt_\beta \big>
  +B_{\alpha\beta} \big\langle \vb f \big| \NInt_\beta \big>
  \Big\} 
\\
 \big\langle \vb f \big| \NExt_\alpha \Big>
&= \sum_\beta \Big\{ 
   -B_{\alpha\beta} \big\langle \vb f \big| \MInt_\beta \big>
   +A_{\alpha\beta} \big\langle \vb f \big| \NInt_\beta \big>.
  \Big\}
\end{align*}
