\documentclass[letterpaper]{article}
\usepackage[square,sort,comma,numbers]{natbib}
\newcommand{\citeasnoun}[1]{Ref.~\citenum{#1}}

../../../../tex/scufftex.tex

\newcommand\supsstar[1]{^{\hbox{\scriptsize{#1}}*}}
\newcommand\suptstar[1]{^{\hbox{\scriptsize{#1}}*}}
\newcommand{\IF}{^{i\text{\scriptsize F}}}
\newcommand{\IFFlux}{^{i\text{\tiny FFLUX}}}
\newcommand{\IT}{^{i\text{\scriptsize T}}}
\newcommand{\ITFlux}{^{i\text{\tiny TFLUX}}}
\newcommand{\PS}{^{\text{\scriptsize P}\mc S}}
\newcommand{\IFS}{^{i\text{\scriptsize F}\mc S}}
\newcommand{\ITS}{^{i\text{\scriptsize T}\mc S}}
%\newcommand{\vbchi}{\boldsymbol{\chi}}


\graphicspath{{figures/}}

%------------------------------------------------------------
%------------------------------------------------------------
%- Special commands for this document -----------------------
%------------------------------------------------------------
%------------------------------------------------------------

%------------------------------------------------------------
%------------------------------------------------------------
%- Document header  -----------------------------------------
%------------------------------------------------------------
%------------------------------------------------------------
\title {Computation of Green's Functions and LDOS in {\sc scuff-em}}
\author {Homer Reid}
\date {September 27, 2014}

%------------------------------------------------------------
%------------------------------------------------------------
%- Start of actual document
%------------------------------------------------------------
%------------------------------------------------------------

\begin{document}
\pagestyle{myheadings}
\markright{Homer Reid: Periodic GF computations in {\sc scuff-em}}
\maketitle

\tableofcontents

%%%%%%%%%%%%%%%%%%%%%%%%%%%%%%%%%%%%%%%%%%%%%%%%%%%%%%%%%%%%%%%%%%%%%%
%%%%%%%%%%%%%%%%%%%%%%%%%%%%%%%%%%%%%%%%%%%%%%%%%%%%%%%%%%%%%%%%%%%%%%
%%%%%%%%%%%%%%%%%%%%%%%%%%%%%%%%%%%%%%%%%%%%%%%%%%%%%%%%%%%%%%%%%%%%%%
\section{Green's functions and LDOS in the non-periodic case}

The scattering parts of the dyadic Green's functions of a geometry
are
%%%%%%%%%%%%%%%%%%%%%%%%%%%%%%%%%%%%%%%%%%%%%%%%%%%%%%%%%%%%%%%%%%%%%%
%%%%%%%%%%%%%%%%%%%%%%%%%%%%%%%%%%%%%%%%%%%%%%%%%%%%%%%%%%%%%%%%%%%%%%
%%%%%%%%%%%%%%%%%%%%%%%%%%%%%%%%%%%%%%%%%%%%%%%%%%%%%%%%%%%%%%%%%%%%%%

In {\sc scuff-em} this may be computed easily by solving a scattering
problem in which the incident fields arise from a point dipole
radiator at $\vb x_0$:
%%%%%%%%%%%%%%%%%%%%%%%%%%%%%%%%%%%%%%%%%%%%%%%%%%%%%%%%%%%%%%%%%%%%%%
\begin{align*}
 \vb E\sups{inc}(\vb x) &= \vb E\supt{PD}(\vb p, \vb x_0; \vb x)
\\
 \vb H\sups{inc}(\vb x) &= \vb H\supt{PD}(\vb p, \vb x_0; \vb x)
\end{align*}
%%%%%%%%%%%%%%%%%%%%%%%%%%%%%%%%%%%%%%%%%%%%%%%%%%%%%%%%%%%%%%%%%%%%%%
where the fields of a point dipole are 
%%%%%%%%%%%%%%%%%%%%%%%%%%%%%%%%%%%%%%%%%%%%%%%%%%%%%%%%%%%%%%%%%%%%%%
\begin{subequations}
\begin{align*}
\end{align*}
\label{PDFields}
\end{subequations}
%%%%%%%%%%%%%%%%%%%%%%%%%%%%%%%%%%%%%%%%%%%%%%%%%%%%%%%%%%%%%%%%%%%%%%
(The ``PD'' superscript stands for ``point dipole'')

To compute

%%%%%%%%%%%%%%%%%%%%%%%%%%%%%%%%%%%%%%%%%%%%%%%%%%%%%%%%%%%%%%%%%%%%%%
%%%%%%%%%%%%%%%%%%%%%%%%%%%%%%%%%%%%%%%%%%%%%%%%%%%%%%%%%%%%%%%%%%%%%%
%%%%%%%%%%%%%%%%%%%%%%%%%%%%%%%%%%%%%%%%%%%%%%%%%%%%%%%%%%%%%%%%%%%%%%
\section{Extension to the periodic case}

In the Bloch-periodic module of {\sc scuff-em}, \textit{all}
fields and currents are assumed to be Bloch-periodic, i.e.
if $Q(\vb x)$ denotes any field or current component at $\vb x$,
then we have the built-in assumption
%====================================================================%
\numeq{BlochCondition}
{Q(\vb x + \vb L) = e^{i\vb k\subt{B} \cdot \vb L}Q(\vb x)}
%====================================================================%
where $\vb L$ is any lattice vector and 
$\vb k\subt{B}$ is the Bloch wavevector.

The fields of a point dipole, equation (\ref{PDFields}), do \textit{not}
satisfy (\ref{BlochCondition}), and hence may not be used in
Bloch-periodic {\sc scuff-em} calculations. Instead, what we can 
simulate in the periodic case are the fields of an infinite
phased \textit{array} of point dipoles,
%%%%%%%%%%%%%%%%%%%%%%%%%%%%%%%%%%%%%%%%%%%%%%%%%%%%%%%%%%%%%%%%%%%%%%
\begin{subequations}
\begin{align*}
 \vb E\supt{PDA}(\vb p, \vb x_0, \vb k\subt{B}; \vb x)
&=\sum_{\vb L} e^{i\vb k\subt{B}\cdot \vb L}
  \vb E\supt{PD}(\vb p, \vb x_0 + \vb L; \vb x)
\\
 \vb H\supt{PDA}(\vb p, \vb x_0, \vb k\subt{B}; \vb x)
&=\sum_{\vb L} e^{i\vb k\subt{B} \cdot \vb L}
  \vb H\supt{PD}(\vb p, \vb x_0  + \vb L; \vb x)
\end{align*}
\label{EHPDA}
\end{subequations}
%%%%%%%%%%%%%%%%%%%%%%%%%%%%%%%%%%%%%%%%%%%%%%%%%%%%%%%%%%%%%%%%%%%%%%
(where ``PDA'' stands for ``point dipole array''). The quantity
we can compute

The inversion of equations (\ref{EHPDA}) recovers the fields
of a point source as an integral over the Brillouin zone of the 
lattice:\footnote{To derive these equations, multiply both sides
of (\ref{EHPDA}) by $e^{-i\vb k\subt{B} \cdot \vb L^\prime}$,
integrate both sides over the Brillouin zone, and use the
condition 
$$\int\subt{BZ} e^{i\vb k\subt{B}\cdot (\vb L-\vb L^\prime)}\,d\vb k
  =\mc{V}\subt{BZ} \, \delta(\vb L,\vb L^\prime)
$$
where $\mc V\subt{BZ}$ is the Brillouin-zone volume [for example,
a square lattice with basis vectors
$\{\vb L_1, \vb L_2\}=\{L_x\vbhat{x}, L_y\vbhat{y}\}$ has
$\mc V\subt{BZ}=4\pi^2/(L_x L_y)$].
Setting $\vb L^\prime=0$ recovers (\ref{EHPDAInverse}).}
%%%%%%%%%%%%%%%%%%%%%%%%%%%%%%%%%%%%%%%%%%%%%%%%%%%%%%%%%%%%%%%%%%%%%%
\begin{subequations}
\begin{align}
  \vb E\supt{PD}(\vb p, \vb x_0; \vb x)
&=\frac{1}{\mc V\subt{BZ}} 
   \int\subt{BZ} \vb E\supt{PDA}(\vb p, \vb x_0, \vb k\subt{B}; \vb x) 
   d\vb k\subt{B}
\\
  \vb H\supt{PD}(\vb p, \vb x_0; \vb x)
&=\frac{1}{\mc V\subt{BZ}} 
   \int\subt{BZ} \vb H\supt{PDA}(\vb p, \vb x_0, \vb k\subt{B}; \vb x) 
   d\vb k\subt{B}
\end{align}
\label{EHPDAInverse}
\end{subequations}
%%%%%%%%%%%%%%%%%%%%%%%%%%%%%%%%%%%%%%%%%%%%%%%%%%%%%%%%%%%%%%%%%%%%%%

\end{document}
