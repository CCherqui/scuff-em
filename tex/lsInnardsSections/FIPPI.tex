%%%%%%%%%%%%%%%%%%%%%%%%%%%%%%%%%%%%%%%%%%%%%%%%%%%%%%%%%%%%%%%%%%%%%%
%%%%%%%%%%%%%%%%%%%%%%%%%%%%%%%%%%%%%%%%%%%%%%%%%%%%%%%%%%%%%%%%%%%%%%
%%%%%%%%%%%%%%%%%%%%%%%%%%%%%%%%%%%%%%%%%%%%%%%%%%%%%%%%%%%%%%%%%%%%%%
\newpage
\section{Evaluation of Frequency-Independent Panel-Panel Integrals}

The frequency-independent panel-panel integrals (FIPPIs) are
\numeq{FIPPIs}
{ \iint h_\bullet r^p,
  \qquad
  \iint h_\nabla r^p,
  \qquad
  \iint h_\times r^p
}
where $\iint$ denotes the four-dimensional integration over
the pair of triangles as in the previous section.

%The general strategy pursued by \lss is to evaluate the
%FIPPIs once for each pair of panels and then store and 
%reuse the results as often as possible. 

\subsubsection*{Evaluation of $\iint h_\bullet r^p$}

In terms of the $u,v,u^\prime,v^\prime$ variables of equation
(\ref{uvDefinition}), we have
\numeq{HBulletUVUPVP}
{ h_{\bullet}(u,v,u^\prime,v^\prime)
=\Big[ \big( \vb V_1-\vb Q\big) + u \vb A + v\vb B\Big]\cdot  
  \Big[ \big( \vb V_1^\prime-\vb Q^\prime\big) + u^\prime \vb A^\prime 
        + v^\prime\vb B^\prime\Big]
}
and 
$$
r(u,v,u^\prime,v^\prime)
 =\Big| \vb V_1 - \vb V_1^\prime
       + u\vb A+ v\vb B - u^\prime\vb A^\prime - v^\prime\vb B^\prime
  \Big|.
$$
Expanding the product (\ref{HBulletUVUPVP}) yields a sum of nine terms:
\begin{align}
\iint h_{\bullet} r^p 
 = \big (\vb V_1-\vb Q\big ) \cdot 
         \big (\vb V_1^\prime - \vb Q^\prime\big)
   \cdot &\iint r^p
\nn
 +\vb A \cdot \big (\vb V_1^\prime -\vb Q^\prime\big )
  &\iint u r^p
\nn
 +\vb B \cdot \big (\vb V_1^\prime -\vb Q^\prime\big )
  &\iint v r^p
\nn
  +  &\cdots
\nn
 +\vb B \cdot \vb B^\prime
  &\iint v v^\prime r^p. \label{HBulletExpansion}
\end{align}
%====================================================================%
It is important to notice that the integrals in (\ref{HBulletExpansion}) 
are independent of $\vb Q$ and $\vb Q^\prime$. This suggests that, to
compute $\iint h_\bullet r^p$,  I first compute the quantities
\numeq{QIFIPPIs}
{
   \iint \left\{ \begin{array}{c} 
   1 \\ u \\ v \\ u^\prime \\ uu^\prime \\ vu^\prime \\
   v^\prime \\ uv^\prime \\ vv^\prime
   \end{array}\right\} r^p
}
and then use (\ref{HBulletExpansion}) to compute $\iint h_\bullet r^p$.
The point of this step is that the quantities (\ref{QIFIPPIs}) need 
only be evaluated and stored once for each pair of panels, after which 
the results may be used in (\ref{HBulletExpansion}) to compute the 
nine separate quantities $\iint h_\bullet r^p$ that result 
different choices of the current source/sink vertices 
$\vb Q, \vb Q^\prime$.

The integrals (\ref{HBulletExpansion}) are 
known in \lss as the ``$\vb Q$-independent FIPPIs,''
while (\ref{FIPPIs}) are the ``$\vb Q$-dependent FIPPIs.''

\subsubsection*{Evaluation of $\iint h_\nabla r^p$}

Once we have evaluated the $\vb Q$-independent FIPPIs
for a pair of panels, we get $\iint h_\nabla r^p$ for 
free, since it is just the first entry in 
equation (\ref{QIFIPPIs}) times 4.

\subsubsection*{Evaluation of $\iint h_\times r^p$}

For $p>-3$, I proceed in analogy to equation (\ref{HBulletExpansion})
by writing $h_\times$ as a sum of nine terms,
$$ h_\times 
   = 
   \sum_{abcd} u^a v^b u^{\prime c} v^{\prime d} h_\times^{abcd},
$$
whereupon $\iint h_\times r^p$ may be reconstructed from 
the QIFIPPIs (\ref{QIFIPPIs}) according to
\numeq{HTimesRP}
{
 \iint h_\times r^p 
   = 
   \sum_{abcd} h_\times^{abcd} 
               \iint \Big[u^a v^b u^{\prime c} v^{\prime d}\Big] r^p.
}
The nonzero values of $h_\times^{abcd}$ are
%--------------------------------------------------------------------%
\begin{align*} 
h_\times^{0000} 
&=  (\vb Q_a\times \vb Q_b) \cdot (\vb V_0 - \vb V_0^\prime)
   +(\vb Q_a   -   \vb Q_b) \cdot (\vb V_0 \times\vb V_0^\prime) 
\\ 
%--------------------------------------------------------------------%
h_\times^{1000} 
&=   (\vb Q_a\times \vb Q_b) \cdot \vb A 
   + (\vb Q_a   -   \vb Q_b) \cdot (\vb A \times \vb V_0^\prime) 
\\
%--------------------------------------------------------------------%
h_\times^{0100} 
&= 
   (\vb Q_a\times \vb Q_b) \cdot \vb B 
  +(\vb Q_a   -   \vb Q_b) \cdot (\vb B \times \vb V_0^\prime) 
\\ 
%--------------------------------------------------------------------%
h_\times^{0010} 
&= -(\vb Q_a\times \vb Q_b) \cdot \vb A^\prime 
   +(\vb Q_a   -   \vb Q_b) \cdot (\vb V_0 \times \vb A^\prime) 
\\ 
%--------------------------------------------------------------------%
h_\times^{0001} 
&=  -(\vb Q_a\times \vb Q_b) \cdot \vb B^\prime 
    +(\vb Q_a   -   \vb Q_b) \cdot (\vb V_0 \times \vb B^\prime) 
\\ 
%--------------------------------------------------------------------%
h_\times^{1010} 
&=   (\vb Q_a   -   \vb Q_b) \cdot (\vb A \times \vb A^\prime)
\\
%--------------------------------------------------------------------%
h_\times^{1001}
&=   (\vb Q_a   -   \vb Q_b) \cdot (\vb A \times \vb B^\prime)
\\
%--------------------------------------------------------------------%
 h_\times^{0110}
&=   (\vb Q_a   -   \vb Q_b) \cdot (\vb B \times \vb A^\prime)
\\
%--------------------------------------------------------------------%
 h_\times^{0101}
&=   (\vb Q_a   -   \vb Q_b) \cdot (\vb B \times \vb B^\prime).
\end{align*}
%--------------------------------------------------------------------%
%
For the particular case $p=-3$, the above procedure is ill-behaved,
and instead I write
\numeq{HTimesRM3}
{
   \iint h_\times r^{-3}=
   (\vb Q_a \times \vb Q_b) 
   \cdot 
   \iint (\vb x_a - \vb x_b) r^{-3} 
   +
   (\vb Q_a -\vb Q_b) 
   \cdot 
   \iint (\vb x_a \times \vb x_b) r^{-3}
}
where now
\numeq{QIFIPPIsSupplement}
{
   \iint (\vb x_a-\vb x_b)r^{-3}, \qquad
    \iint (\vb x_a \times \vb x_b)r^{-3}
}
are to be included among the list of $\vb Q-$independent 
FIPPIs that must be calculated for each panel pair.

On first glance, it might seem that equations
(\ref{QIFIPPIs}) and (\ref{QIFIPPIsSupplement})
are redundant, since knowing the former should allow
reconstruction of the latter. This appearance is 
deceptive, because for panel pairs with common vertices 
some of the individual FIPPIs (\ref{QIFIPPIs}) are 
divergent for $p=-3$, while (\ref{QIFIPPIsSupplement}) 
are convergent. [For panel pairs with no common
vertices, the integrals (\ref{QIFIPPIs}) are convergent
but much more expensive to calculate than 
(\ref{QIFIPPIsSupplement}).] Of course, for $p=-3$
the $\vb Q-$independent FIPPIs defined by 
(\ref{QIFIPPIsSupplement}) are the only ones we 
\textit{need} 
(because for $p=-3$ the only $\vb Q$-dependent FIPPI 
we need is $\iint h_\times r^{p}$), while 
for $p>-3$ equation (\ref{QIFIPPIsSupplement}) 
does not suffice and we need instead the full
set (\ref{QIFIPPIs}). Thus a reasonable 
compromise seems to be to compute only 
(\ref{QIFIPPIsSupplement}) for the case $p=-3$, 
and to compute the full set (\ref{QIFIPPIs}) for
$p>-3$.

\subsubsection*{Evaluation of $\vb Q$-independent FIPPIs}

The $\vb Q$-independent FIPPIs, equation (\ref{QIFIPPIs}),
are nominally four-dimensional integrals, but we have the 
following simplifications:

\begin{enumerate}
  \item For $p=0$ the integral may be done analytically
        in closed form. (Actually, the same is true for $p=2$,
        but the result is too cumbersome to be useful.)
  \item If the panels have one or more common vertices, we may use
        the Taylor-Duffy method (Appendix \ref{TaylorDuffyAppendix})
        to perform one or more of the four integrals analytically.
  \item Even when the panels have no common vertices, we can 
        always evaluate one of the four integrals analytically,
        which accelerates numerical evaluation.
\end{enumerate}

We now address each of these points in turn. 

\paragraph{1. FIPPIs for $p=0$.} 

In this case all the FIPPIs are linear combinations of the 
basic integral
$$ \iint u^a v^b (u^\prime)^c (v^\prime)^d 
   = 
   \frac{1}{(2+a+b)(1+b)(2+c+d)(1+d)}.
$$
We have
$$
   \iint \left\{ \begin{array}{c} 
   1 \\ u \\ v \\ u^\prime \\ uu^\prime \\ vu^\prime \\
   v^\prime \\ uv^\prime \\ vv^\prime
   \end{array}\right\} r^0
   =
   \left\{ \begin{array}{c} 
           1/4 \\ 1/6 \\ 1/12 \\ 
           1/6 \\ 1/9 \\ 1/18 \\ 
           1/12 \\ 1/18 \\ 1/36
           \end{array}
   \right\}.
$$

\paragraph{2. FIPPIs for panels with common vertices.} 

As noted above, for panels with common vertices the Taylor-Duffy
method of Appendix (\ref{TaylorDuffyAppendix}) is available.

\paragraph{3. FIPPIs for panels with no common vertices.} 

As noted above, even when the panels have no common vertices
we can always evaluate one of the four integrals 
analytically to yield a three-dimensional integral.
We arbitrarily choose the integral we evaluate to be
the $v^\prime$ integral, and to facilitate its evaluation we 
write
\begin{align*}
r(u,v,u^\prime,v^\prime)
&=\Big| \vb V_1 - \vb V_1^\prime
       + u\vb A + v\vb B - u^\prime\vb A^\prime - v^\prime\vb B^\prime
  \Big|   
\\
&=a \sqrt{ (v^\prime + v_0^\prime)^2 + b^2 },
\end{align*}
%====================================================================%
$$
a = |\vb B^\prime|, 
\qquad
v_0^\prime = -\frac{1}{a^2} \vb B^\prime \cdot \vb Y,
\qquad
b^2=\frac{1}{a^2}|\vb Y|^2 - v_0^{\prime 2},
$$
$$
\vb Y=    \vb V_1 - \vb V_1^\prime  
       + u\vb A+ v\vb B - u^\prime\vb A^\prime.
$$
%====================================================================%
The integrals we need are now
%====================================================================%
\begin{subequations}
\begin{align}
 \int_{v_0^\prime}^{v_0^\prime + u^\prime} 
  \left\{ \begin{array}{c}  1 \\ v^\prime \end{array} \right\}
  \Big[v^{\prime 2} + b^2\Big]^{-3/2}d v^\prime
&= \frac{1}{a^3} 
   \left\{ \begin{array}{c}
    \frac{(u^\prime + v_0^\prime)}{b^2 S_2} - \frac{v_0^\prime}{b^2 S_1} 
    \\[5pt]
    \frac{1}{S_1} - \frac{1}{S_2}
   \end{array}\right\}
\\[5pt]
%--------------------------------------------------------------------%
 \int_{v_0^\prime}^{v_0^\prime + u^\prime} 
  \left\{ \begin{array}{c}  1 \\ v^\prime \end{array} \right\}
  \Big[v^{\prime 2} + b^2\Big]^{-1/2}d v^\prime
&= \frac{1}{a} 
   \left\{ \begin{array}{c}
    \log \frac{ S_2 + (u^\prime+v_0^\prime) }
              { S_1 + v_0^\prime }
    \\[5pt]
    S_2 - S_1
   \end{array}\right\}
\\[8pt]
%--------------------------------------------------------------------%
 \int_{v_0^\prime}^{v_0^\prime + u^\prime} 
  \left\{ \begin{array}{c}  1 \\ v^\prime \end{array} \right\}
  \Big[v^{\prime 2} + b^2\Big]^{1/2} d v^\prime
&= a
   \left\{ \begin{array}{c}
   \frac{b^2}{2} \log \frac{ S_2 + (u^\prime+v_0^\prime) }
                           { S_1 + v_0^\prime }
   +\frac{1}{2}\big[(u^\prime + v_0^\prime)S_2 - v_0^\prime S_1\big]
   \\[5pt]
   \frac{1}{3}(S_2^3 - S_1^3)
   \end{array}\right\}
\\[8pt]
%--------------------------------------------------------------------%
 \int_{v_0^\prime}^{v_0^\prime + u^\prime} 
  \left\{ \begin{array}{c}  1 \\ v^\prime \end{array} \right\}
  \Big[v^{\prime 2} + b^2\Big]d v^\prime
&= a^2
   \left\{ \begin{array}{c}
    \\[5pt]
   \end{array}\right\}
\end{align}
\label{vPrimeIntegrals}
\end{subequations}
$$ S_1\equiv\sqrt{b^2 + v_0^{\prime 2}}, 
   \qquad
   S_2\equiv\sqrt{b + (v_0^{\prime} + u^\prime)^2}, \qquad
$$
%====================================================================%
To make use of these results in evaluating the FIPPIs 
(\ref{QIFIPPIs}), we eliminate the $v^\prime$ integrals
and replace all factors of $\{1, v^\prime\}$ with
their counterparts on the RHS of (\ref{vPrimeIntegrals}),
leaving behind three-dimensional integrals that succumb
readily to numerical cubature.

%%%%%%%%%%%%%%%%%%%%%%%%%%%%%%%%%%%%%%%%%%%%%%%%%%%%%%%%%%%%%%%%%%%%%%
%%%%%%%%%%%%%%%%%%%%%%%%%%%%%%%%%%%%%%%%%%%%%%%%%%%%%%%%%%%%%%%%%%%%%%
%%%%%%%%%%%%%%%%%%%%%%%%%%%%%%%%%%%%%%%%%%%%%%%%%%%%%%%%%%%%%%%%%%%%%%
\newpage
