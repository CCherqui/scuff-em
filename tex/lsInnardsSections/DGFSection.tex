%%%%%%%%%%%%%%%%%%%%%%%%%%%%%%%%%%%%%%%%%%%%%%%%%%%%%%%%%%%%%%%%%%%%%%
%%%%%%%%%%%%%%%%%%%%%%%%%%%%%%%%%%%%%%%%%%%%%%%%%%%%%%%%%%%%%%%%%%%%%%
%%%%%%%%%%%%%%%%%%%%%%%%%%%%%%%%%%%%%%%%%%%%%%%%%%%%%%%%%%%%%%%%%%%%%%
\newpage
\section{Homogeneous Dyadic Green's Functions}
\label{DyadicGreensFunctionsSection}

Before proceeding, we must pause briefly to establish our 
conventions and notation for homogeneous dyadic Green's
functions.

Consider a spatial region characterized by spatially uniform 
relative permittivity and permeability $\epsilon^r$ and 
$\mu^r$. If we have
known distributions of electric and magnetic current
$\vb J(\vb x)$ and $\vb M(\vb x),$
we can compute the electric and magnetic fields in terms
of these currents and the properties of the medium, and  
the relevant convolution kernels in this procedure are
the dyadic Green's functions (DGFs):
%====================================================================%
\begin{align*}
 E_i(\vb x, \omega) 
&= 
   \int
    \Big\{
     \Gamma\supt{EE}_{ij}(\epsilon^r, \mu^r; \omega; \vb x, \vb x^\prime) 
     J_j(\vb x^\prime)
     +
     \Gamma\supt{EM}_{ij}(\epsilon^r, \mu^r; \omega; \vb x, \vb x^\prime) 
     M_j(\vb x^\prime)
    \Big\}
    \, d\vb x^\prime
\\
 H_i(\vb x, \omega) 
&= 
   \int
    \Big\{
     \Gamma\supt{ME}_{ij}(\epsilon^r, \mu^r; \omega; \vb x, \vb x^\prime) 
     J_j(\vb x^\prime)
     +
     \Gamma\supt{MM}_{ij}(\epsilon^r, \mu^r; \omega; \vb x, \vb x^\prime) 
     M_j(\vb x^\prime)
    \Big\}
    \, d\vb x^\prime.
\end{align*}
%====================================================================%

Explicit expressions for the DGFs are 
%====================================================================%
\begin{align*}
\BG\supt{EE}(\epsilon^r, \mu^r, \omega, \vb x, \vb x^\prime)
&=
 iZ_0 Z^r k^r \, \vb G(k^r, \vb x-\vb x^\prime)
\\[8pt]
%--------------------------------------------------------------------%
\BG\supt{ME}(\epsilon^r, \mu^r, \omega, \vb x, \vb x^\prime)
&=
 -ik^r \vb C(k^r, \vb x-\vb x^\prime)
\\[8pt]
%--------------------------------------------------------------------%
\BG\supt{EM}(\epsilon^r, \mu^r, \omega, \vb x, \vb x^\prime)
&=
 ik^r \, \vb C(k^r, \vb x-\vb x^\prime)
\\[8pt]
%--------------------------------------------------------------------%
\BG\supt{MM}(\epsilon^r, \mu^r, \omega, \vb x, \vb x^\prime)
&= \frac{ik^r}{Z_0 Z^r} \vb G(k^r, \vb x-\vb x^\prime)
\end{align*}
%====================================================================%
$$\bigg(Z_0=\sqrt\frac{\mu_0}{\epsilon_0},
        \qquad
        Z^r=\sqrt\frac{\mu^r}{\epsilon^r}, 
         \qquad
        k^r=\sqrt{\mu_0 \mu^r \epsilon_0 \epsilon^r}\cdot \omega
  \bigg)
$$
%====================================================================%
where $\vb G$, the ``photon Green's function,''
is the solution to the equation
\begin{equation}
 \Big[\nabla \times \nabla\times - \,\,k^2 \Big]\vb G(k; \vb r)
 =\delta(\vb r)\vb{1};
 \label{DyadicGFG}
\end{equation}
and $\vb C$ is defined by  
\begin{equation}
  \vb C(k, \vb r) = -\frac{1}{ik} \nabla \times \vb G(k, \vb r).
 \label{DyadicGFC}
\end{equation}
%
(Note that the $\BG$ dyadics depend separately on $\epsilon,\mu,$ 
and $\omega$, while $\vb G$ and $\vb C$ depend only on the 
combination $k=\sqrt{\epsilon\mu}\cdot \omega.$)

Explicit expressions for the components of $\vb G$ and $\vb C$ are 
\begin{align*}
G_{ij}(k,\vb r) 
 &= \frac{e^{ikr}}{4\pi (ik)^2 r^3}
    \bigg[ \Big(1-ikr + (ikr)^2 \Big)\delta_{ij}
          +\Big(-3 + 3ikr - (ikr)^2 \Big)\frac{r_i r_j}{r^2}
    \bigg]
\\
C_{ij}(k,\vb r) 
 &= \frac{e^{ikr}}{4\pi (ik) r^3} \varepsilon_{ijk} r_k 
    \Big(-1+ikr\Big)
\end{align*}
These may also be written in the form
\begin{equation}
G_{ij}(k,\vb r) =
  \Big[\delta_{ij} +\frac{1}{k^2} \partial_i \partial_j \Big]
  G_0(k, \vb r), \qquad
  C_{ij}(k, \vb r) = +\frac{1}{ik} \varepsilon_{ijl} 
                          \partial_l G_0(k, \vb r)
\label{GCfromG0}
\end{equation}
where $G_0$ is the scalar Green's function for the Helmholtz equation,
\numeq{ScalarGF}
{G_0(k; \vb r)
   =
  \frac{e^{ik|\vb r|}}{4\pi|\vb r|}
}
which satisfies
$$ \Big[ \nabla^2 + k^2 \Big] G_0(k; \vb r)
   =\delta(\vb r).
$$
With these expressions, we can verify that equation (\ref{DyadicGFC}) 
is actually just the first half of a pair of reciprocal curl identities 
relating $\vb G$ and $\vb C:$
\numeq{ReciprocalCurlIdentities}
{
  \frac{1}{ik} \nabla \times \vb G=-\vb C, 
\qquad
  \frac{1}{ik} \nabla \times \vb C=\vb G.
}
(As usual with tensors and dyadics, the vector notation here
is suggestive but vague; the precise meaning of (\ref{DyadicGFC}) is 
\numeq{ReciprocalCurlIdentities2}
{
   \frac{1}{ik} \varepsilon_{iAB} \partial_A G_{Bj} = -C_{ij},
   \qquad
   \frac{1}{ik} \varepsilon_{iAB} \partial_A C_{Bj} = G_{ij}.)
}


%%%%%%%%%%%%%%%%%%%%%%%%%%%%%%%%%%%%%%%%%%%%%%%%%%%%%%%%%%%%%%%%%%%%%%
%%%%%%%%%%%%%%%%%%%%%%%%%%%%%%%%%%%%%%%%%%%%%%%%%%%%%%%%%%%%%%%%%%%%%%
%%%%%%%%%%%%%%%%%%%%%%%%%%%%%%%%%%%%%%%%%%%%%%%%%%%%%%%%%%%%%%%%%%%%%%

\paragraph{Shorthand} In what follows, an equation like 

$$ E_i(\vb x, \omega) = 
   \int_{\mathcal S} 
     \Gamma\supt{EE}_{ij}(\epsilon^r, \mu^r; \omega; 
                          \vb x, \vb x^\prime) 
     K_j(\vb x^\prime)
    \, d\vb x^\prime
$$
will often be abbreviated to read 
$$ \vb E(\vb x) = 
    \int_{\mathcal S} 
      \BG\supt{EE}(\mathcal{R}^r; \vb x, \vb x^\prime) 
         \cdot \vb K(\vb x^\prime) 
    \, d\vb x^\prime
$$
(with $\omega$ arguments suppressed and the dependence 
on $\epsilon^r, \mu^r$ condensed into a dependence
on the region $\mathcal{R}^r$), or even abbreviated further
to read
$$ \vb E = \BG\supt{EE}(\mathcal{R}^r) \star \vb K $$
where $\star$ denotes a convolution operation.

