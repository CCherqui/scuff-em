\documentclass[letterpaper]{article}
\usepackage[square,sort,comma,numbers]{natbib}
\newcommand{\citeasnoun}[1]{Ref.~\citenum{#1}}

../../../../tex/scufftex.tex

\newcommand\supsstar[1]{^{\hbox{\scriptsize{#1}}*}}
\newcommand\suptstar[1]{^{\hbox{\scriptsize{#1}}*}}
\newcommand{\IF}{^{i\text{\scriptsize F}}}
\newcommand{\IFFlux}{^{i\text{\tiny FFLUX}}}
\newcommand{\IT}{^{i\text{\scriptsize T}}}
\newcommand{\ITFlux}{^{i\text{\tiny TFLUX}}}
\newcommand{\PS}{^{\text{\scriptsize P}\mc S}}
\newcommand{\IFS}{^{i\text{\scriptsize F}\mc S}}
\newcommand{\ITS}{^{i\text{\scriptsize T}\mc S}}
%\newcommand{\vbchi}{\boldsymbol{\chi}}

\graphicspath{{figures/}}

%------------------------------------------------------------
%------------------------------------------------------------
%- Special commands for this document -----------------------
%------------------------------------------------------------
%------------------------------------------------------------

%------------------------------------------------------------
%------------------------------------------------------------
%- Document header  -----------------------------------------
%------------------------------------------------------------
%------------------------------------------------------------
\title {{\sc scuff-neq} Technical Reference}
\author {Homer Reid}
\date {July 6, 2014}

%------------------------------------------------------------
%------------------------------------------------------------
%- Start of actual document
%------------------------------------------------------------
%------------------------------------------------------------

\begin{document}
\pagestyle{myheadings}
\markright{Homer Reid: {\sc scuff-neq} Technical Reference}
\maketitle

\tableofcontents

%%%%%%%%%%%%%%%%%%%%%%%%%%%%%%%%%%%%%%%%%%%%%%%%%%%%%%%%%%%%%%%%%%%%%%
%%%%%%%%%%%%%%%%%%%%%%%%%%%%%%%%%%%%%%%%%%%%%%%%%%%%%%%%%%%%%%%%%%%%%%
%%%%%%%%%%%%%%%%%%%%%%%%%%%%%%%%%%%%%%%%%%%%%%%%%%%%%%%%%%%%%%%%%%%%%%
\newpage
\section{Theoretical background}

{\sc scuff-neq} implements the ``fluctuating-surface-current''
(FSC) approach to computational fluctuation physics. The 
FSC method was originally developed for equilibrium Casimir
computations~\cite{Reid2009, Reid2011, Reid2013B} and later
extended to non-equilibrium phenomena in 
Refs.~\citenum{Rodriguez2012C} and \citenum{Rodriguez2013B}.
Here we present a quick summary of the key equations in this
approach.

%%%%%%%%%%%%%%%%%%%%%%%%%%%%%%%%%%%%%%%%%%%%%%%%%%%%%%%%%%%%%%%%%%%%%%
%%%%%%%%%%%%%%%%%%%%%%%%%%%%%%%%%%%%%%%%%%%%%%%%%%%%%%%%%%%%%%%%%%%%%%
%%%%%%%%%%%%%%%%%%%%%%%%%%%%%%%%%%%%%%%%%%%%%%%%%%%%%%%%%%%%%%%%%%%%%%
\subsection{Physical quantities from surface-current bilinears}
\label{PQsFromSCBsSection}

The FSC approach begins with the observation that, in a 
classical, deterministic scattering problem involving the 
fields of known external sources impinging upon a collection
of material bodies, many quantities of physical interest 
may be expressed as bilinear (quadratic) products of the 
\textit{surface currents} induced on the body surfaces by 
the external fields. The quantities that may be expressed
in this way include both \textbf{(a)} spatially-resolved 
fluxes of energy and momentum at individual points in space,
and \textbf{(b)} total (spatially-integrated) power-transfer 
rates, forces, and torques on individual bodies.

In this subsection we derive several such surface-current 
bilinear formulas in the context of (classical, deterministic)
scattering theory. In the following subsection we explain how 
the FSC approach converts these \textit{deterministic} 
expressions---describing energy and momentum transfer
from fixed external sources---into \textit{statistically averaged} 
expressions, describing energy and momentum transfer
due to thermal and quantum-mechanical fluctuations
in material bodies in the \textit{absence} of external
excitation.

%=================================================
%=================================================
%=================================================
\subsection*{Fields from surface currents}

Recall that in the surface-integral-equation (SIE) approach to
classical electromagnetism problems, we solve for tangential
\textit{surface currents} (both electric currents $\vb K$ 
and magnetic currents $\vb N$) flowing on the surfaces of 
homogeneous material bodies. In numerical solvers, we 
approximate these as finite expansions in a discrete set
of $N\subt{B}$ basis functions; using a convenient 6-vector notation 
in which $\bmc C\equiv {\vb K \choose \vb N}$, we put
%--------------------------------------------------------------------%
$$ \bmc C(\vb x)=\sum_{\alpha=1}^{N\subt{B}} 
   c_\alpha \bmc B_\alpha(\vb x) 
$$ 
%--------------------------------------------------------------------%
where $\{\bmc B_\alpha\}$ is a set of 6-vector basis 
functions\footnote{In {\sc scuff-neq} these take the form
$\bmc B_\alpha=$ 
$\vb b_\alpha \choose 0$ 
or 
$0 \choose \vb b_\alpha$
where $\vb b_\alpha$ is a three-vector RWG basis function.
However, the FSC formalism is not specific to this choice.}
and $\{c_\alpha\}$ are scalar expansion coefficients.
(We work at a fixed frequency $\omega$ and assume all fields
and currents vary in time like $e^{-i\omega t}$.)

The electric and magnetic fields produced by these currents
are linear functions of the $\{c_\alpha\}$ coefficients.
In 6-vector notation with $\bmc F\equiv {\vb E \choose \vb H}$, we have
%--------------------------------------------------------------------%
\numeq{FFromC}
{ \bmc F(\vb x)=
   \sum_{\alpha} c_\alpha \bmc F_\alpha(\vb x)
}
%--------------------------------------------------------------------%
where $\bmc F_\alpha={\vb E_\alpha \choose \vb H_\alpha}$ 
is the six-vector of fields produced by basis function $\bmc B_\alpha$ 
populated with unit strength. These may be calculated numerically
as convolutions over basis functions,
%--------------------------------------------------------------------%
$$ \bmc F_\alpha(\vb x)
   =\int_{\sup \bmc B_\alpha} \bmc{G}(\vb x, \vb x^\prime)
    \bmc B_\alpha(\vb x^\prime) \, d\vb x^\prime
$$
%--------------------------------------------------------------------%
where $\bmc{G}$ is an appropriate $6\times 6$ dyadic Green's
function.

%%%%%%%%%%%%%%%%%%%%%%%%%%%%%%%%%%%%%%%%%%%%%%%%%%%%%%%%%%%%%%%%%%%%%%
%%%%%%%%%%%%%%%%%%%%%%%%%%%%%%%%%%%%%%%%%%%%%%%%%%%%%%%%%%%%%%%%%%%%%%
%%%%%%%%%%%%%%%%%%%%%%%%%%%%%%%%%%%%%%%%%%%%%%%%%%%%%%%%%%%%%%%%%%%%%%
\subsection*{Energy and momentum flux from field bilinears}

The Poynting flux and Maxwell stress tensor are quadratic functions
of the field components and may be conveniently written in the form
of 6-dimensional vector-matrix-vector products. 
In particular, the power flux in the direction of a unit vector 
$\vbhat{n}$ is
%====================================================================%
\begin{align}
 \vb P(\vb x) \cdot \vbhat{n}
   &=\frac{1}{2}\text{ Re }\varepsilon_{ijk}\vbhat{n}_i E^*_j(\vb x) H_k(\vb x)
\nn
   &=\frac{1}{4}\bmc{F}^\dagger(\vb x) \, \bmc{N}\supt{P}(\vbhat{n}) \, \bmc{F}(\vb x)
\label{PVMVP}
\end{align}
%====================================================================%
with
%====================================================================%
$$
   \bmc N\supt{P}=
   \left(\begin{array}{cc}
   0       & \vb N\supt{P}   \\ [3pt]
  -\vb N\supt{P} & 0
   \end{array}\right), 
\qquad 
   \vb N\supt{P}
   =
   \left(\begin{array}{ccc}
   0          &  \hat{n}_z & -\hat{n_y} \\
  -\hat{n}_z  &  0         & +\hat{n_x} \\
   \hat{n}_y  & -\hat{n}_x & 0
   \end{array}\right).
$$
%====================================================================%
Similarly, the flux of $i$-directed linear momentum is
%--------------------------------------------------------------------%
\begin{align}
 \vb T_{i}(\vb x) \cdot \vbhat{n} 
&=\frac{1}{2}\text{ Re }
  \left[ \epsilon E^*_i(\vb x) E_j(\vb x) 
             +\mu H^*_i(\vb x) H_j(\vb x) 
       -\frac{\delta_{ij}}{2}
         \Big( \epsilon |\vb E|^2
              +\mu      |\vb H|^2
         \Big)
 \right] \hat {n}_j
\nn
&= \frac{1}{4}\bmc{F}^\dagger(\vb x) 
   \, \bmc{N}\IF \, \bmc{F}(\vb x)
\label{IFVMVP}
\end{align}
with
%--------------------------------------------------------------------%
$$
   \bmc N\IF=
   \left(\begin{array}{cc}
   \epsilon \vb N\IF & 0 \\
            0        & \mu \vb N\IF 
   \end{array}\right)
$$
%--------------------------------------------------------------------%
where the $3\times 3$ matrix $N\IF$ has entries
%--------------------------------------------------------------------%
$$ N\IF_{ab} = 
   \delta_{ai} \hat{n}_b + \delta_{bi} \hat{n}_a
  - \hat{n}_i \delta_{ab}.
$$
%====================================================================%
For example, if we are computing the $x$-force ($i=x$)
we have 
%====================================================================%
$$ \vb N^{x\text{\tiny F}}=\left(\begin{array}{ccc}
   \hat{n}_x & \hat n_y   & \hat n_z \\
   \hat{n}_y & -\hat{n}_x & 0 \\
   \hat{n}_z & 0          & -\hat{n}_x
  \end{array}\right).
$$
The flux of $i$-directed \textit{angular} momentum, useful
for computations of torque about an origin $\vb x_0$, is
%====================================================================%
\begin{align}
 \vb t_{i}(\vb x) \cdot \vbhat{n}
&=\frac{1}{2}
  \text{ Re }
  \varepsilon_{ijk}(\vb x-\vb x_0)_j T_{k\ell}(\vb x) \hat{n}_\ell
\\
&= \frac{1}{4}\bmc{F}^\dagger(\vb x) 
   \, \bmc{N}\IT \, \bmc{F}(\vb x)
\label{ITVMVP}
\end{align}
%====================================================================%
with 
%--------------------------------------------------------------------%
$$
   \bmc N\IT=
   \left(\begin{array}{cc}
   \epsilon \vb N\IT & 0 \\
            0        & \mu \vb N\IT 
   \end{array}\right)
$$
%--------------------------------------------------------------------%
where the $3\times 3$ matrix $N\IT$ has entries ($\vb D=\vb x-\vb x_0$)
%--------------------------------------------------------------------%
$$ N\IT_{ab}=
   \varepsilon_{ija}D_j \hat{n}_b
  +\varepsilon_{ijb}D_j \hat{n}_a
  -\delta_{ab} \varepsilon_{ijk} D_j \hat{n}_k.
$$

%%%%%%%%%%%%%%%%%%%%%%%%%%%%%%%%%%%%%%%%%%%%%%%%%%%%%%%%%%%%%%%%%%%%%%
%%%%%%%%%%%%%%%%%%%%%%%%%%%%%%%%%%%%%%%%%%%%%%%%%%%%%%%%%%%%%%%%%%%%%%
%%%%%%%%%%%%%%%%%%%%%%%%%%%%%%%%%%%%%%%%%%%%%%%%%%%%%%%%%%%%%%%%%%%%%%
\subsection*{Energy and momentum flux from surface-current bilinears}

Equations (\ref{PVMVP}), (\ref{IFVMVP}), and (\ref{ITVMVP}) express
the flux of power or momentum as bilinear products of the field
six-vectors $\bmc F$. Using (\ref{FFromC}), we can turn these into
bilinear products of the surface-current coefficient vectors
$\vb c$. For example, the power flux (\ref{PVMVP}) becomes 
%--------------------------------------------------------------------%
\begin{align}
 \vb P(\vb x) \cdot \vbhat{n}
&=\frac{1}{4}\bmc F^\dagger(\vb x) \bmc N\supt{P}(\vbhat{n}) \bmc F(\vb x)
\nn
&=\frac{1}{4}\sum_{\alpha \beta} 
  c_\alpha^* 
  \Big[ \bmc F^\dagger_\alpha(\vb x)
        \,
        \bmc N\supt{P}(\vbhat{n})
        \,
        \bmc F_\beta(\vb x)
  \Big]
  c_\beta 
\nn
&= \vb c^\dagger \vb M\supt{PFLUX}(\vb x, \vbhat{n}) \vb c 
\label{PFluxVMVP}
\end{align}
%--------------------------------------------------------------------%
where $\vb M\supt{PFLUX}(\vb x,\vbhat{n})$ is a matrix 
appropriate for $\vbhat{n}$-directed power flux in at $\vb x$.
The fluxes of $i$-directed linear and angular momentum read
similarly
%--------------------------------------------------------------------%
\begin{align}
\vb T_i(\vb x) \cdot \vbhat{n}
&= \vb c^\dagger \vb M\IFFlux(\vb x, \vbhat{n}) \vb c 
\label{IFFluxVMVP}
\\
\vb t_i(\vb x) \cdot \vbhat{n}
&= \vb c^\dagger \vb M\ITFlux(\vb x, \vbhat{n}) \vb c 
\label{ITFluxVMVP}
\end{align}
%--------------------------------------------------------------------%
The $\vb M$ matrices in (\ref{PFluxVMVP}), (\ref{IFFluxVMVP}), and 
(\ref{ITFluxVMVP}) are $N\subt{B}\times N\subt{B}$ matrices
whose entries are themselves 6-dimensional matrix-vector products:
\begin{align*}
M_{\alpha\beta}\supt{PFLUX}(\vb x, \vbhat{n}) 
 &= \frac{1}{4} 
    \bmc F^\dagger_\alpha(\vb x) 
    \bmc N\supt{P}(\vbhat{n})
    \bmc F_\beta(\vb x) 
\\
M_{\alpha\beta}\IFFlux(\vb x, \vbhat{n})
 &= \frac{1}{4} 
    \bmc F^\dagger_\alpha(\vb x) 
    \bmc N\supt{iF}(\vbhat{n})
    \bmc F_\beta(\vb x) 
\\
M_{\alpha\beta}\ITFlux(\vb x, \vbhat{n})
 &= \frac{1}{4} 
    \bmc F^\dagger_\alpha(\vb x) 
    \bmc N\supt{iT}(\vbhat{n})
    \bmc F_\beta(\vb x).
\end{align*}

%%%%%%%%%%%%%%%%%%%%%%%%%%%%%%%%%%%%%%%%%%%%%%%%%%%%%%%%%%%%%%%%%%%%%%
%%%%%%%%%%%%%%%%%%%%%%%%%%%%%%%%%%%%%%%%%%%%%%%%%%%%%%%%%%%%%%%%%%%%%%
%%%%%%%%%%%%%%%%%%%%%%%%%%%%%%%%%%%%%%%%%%%%%%%%%%%%%%%%%%%%%%%%%%%%%%
\subsection*{Power, force, and torque from surface-current bilinears}

The main quantities of interest in {\sc scuff-neq} are the
power, force, or torque (PFT) on individual bodies in a geometry.
There are actually three separate ways to derive surface-current 
bilinear expressions for these quantities, which we here consider 
in turn.

\subsubsection*{Dense surface-integral PFT bilinears}

The simplest way to obtain surface-current bilinears for the
total PFT on one or more bodies is simply to integrate the 
spatially-resolved fluxes of the previous section over a closed
bounding surface $\mc S$ containing the bodies.
Indeed, noting that the $\vb x$ and $\vbhat{n}$ dependence of
the flux expressions (\ref{PFluxVMVP}), (\ref{IFFluxVMVP}),
and (\ref{ITFluxVMVP}) is entirely contained in the $\vb M$
matrices, it is easy to integrate those expressions over
$\mc S$, then pull the surface-current vectors $\vb c$
outside the integral to identify what we shall call
the \textit{dense surface-integral PFT} (DSIPFT) matrices. 

For example, the total power absorbed by material bodies
contained within a closed surface $\mc S$
is given by integrating the LHS of (\ref{PFluxVMVP})
over $\mc S$:
%====================================================================%
\begin{align}
  P_{\mc S}\sups{abs}&=\int_{\mc S} \vb P(\vb x) \cdot \vbhat{n} \, d\vb x
\intertext{with $\vbhat{n}$ taken to be the inward-pointing surface
           normal. Insert the RHS of (\ref{PFluxVMVP}) and pull 
           $\vb c^\dagger, \vb c$ outside the integral:}
             &=\vb c^\dagger \vb M\PS \vb c
\label{PTotVMVP}
\end{align}
where the elements of 
$\vb M\PS$
involve integrals over $\mc S$:
%====================================================================%
\numeq{MPSEntries}
{ M_{\alpha\beta}\PS
  = \frac{1}{4} \int_{\mc S} 
    \bmc F^\dagger_\alpha(\vb x) 
    \bmc N\supt{P}(\vbhat{n})
    \bmc F_\beta(\vb x) 
     \, d\vb x.
}
%====================================================================%
Similarly, the time-average $i$-directed force and torque on 
material bodies contained in $\mc S$ are
%====================================================================%
\begin{align}
 F_{i\mc S} &=\vb c^\dagger \vb M\IFS \vb c
\label{FTotVMVP}
\\
 \mc T_{i\mc S} &=\vb c^\dagger \vb M\ITS \vb c
\label{TTotVMVP}
\end{align}
%====================================================================%
where the entries of $\vb M\IFS$ and $\vb M\ITS$ are similar
to (\ref{MPSEntries}) with $\bmc N\supt{P} \to \bmc N\IF, \bmc N\IT.$

%%%%%%%%%%%%%%%%%%%%%%%%%%%%%%%%%%%%%%%%%%%%%%%%%%%%%%%%%%%%%%%%%%%%%%
%%%%%%%%%%%%%%%%%%%%%%%%%%%%%%%%%%%%%%%%%%%%%%%%%%%%%%%%%%%%%%%%%%%%%%
%%%%%%%%%%%%%%%%%%%%%%%%%%%%%%%%%%%%%%%%%%%%%%%%%%%%%%%%%%%%%%%%%%%%%%
\subsubsection*{Sparse surface-integral PFT bilinears}

For some quantities of interest it is possible to write alternative
surface-current bilinears that compute the same quantities as 
(\ref{PTotVMVP}), (\ref{FTotVMVP}), and (\ref{TTotVMVP})
but which involve \textit{sparse} matrices instead of dense
matrices~\cite{Reid2013a}. In particular, the (classical, deterministic) 
power, force, or torque on a body may be written as a bilinear 
function of the surface currents on that body alone in the form
$$
 P = \frac{1}{4}\vb c \vb O\supt{P} \vb c, 
 \qquad
 F_i = \frac{1}{4}\vb c \vb O\IF \vb c
 \qquad
 \mc T_i = \frac{1}{4}\vb c \vb O\IT \vb c
$$
where the $\vb O$ matrices are sparse ``overlap'' matrices
with nonzero entries only for pairs of basis functions
with overlapping support.

%%%%%%%%%%%%%%%%%%%%%%%%%%%%%%%%%%%%%%%%%%%%%%%%%%%%%%%%%%%%%%%%%%%%%%
%%%%%%%%%%%%%%%%%%%%%%%%%%%%%%%%%%%%%%%%%%%%%%%%%%%%%%%%%%%%%%%%%%%%%%
%%%%%%%%%%%%%%%%%%%%%%%%%%%%%%%%%%%%%%%%%%%%%%%%%%%%%%%%%%%%%%%%%%%%%%
\subsubsection*{Volume-integral PFT bilinears}

\begin{align*}
P&=\text{Re }
  \int_{\mc V} \bmc F^*(\vb x) \cdot \bmc J(\vb x) \, d\vb x
\\
&=
  \int_{\mc V} \bmc F^*(\vb x) \vbchi(\vb x) \bmc F(\vb x)\, d\vb x
\\
&=\sum_{\alpha\beta} c_\alpha^*
  \underbrace{
  \Big[ \int_{\mc V} \bmc G_\alpha^*(\vb x) \vbchi(\vb x)
                     \bmc G_\beta(\vb x) d\vb x
  \Big] c_\beta
             }_{\text{sym } \vb G}
\\
&= \vb c^\dagger \Big[ {\text{sym }\vb G} \Big] \vb c
\intertext{Similarly,}
F_i&=\frac{1}{\omega} \text{Im }
  \int_{\mc V} \bmc F_i^*(\vb x) \cdot \bmc J(\vb x) \, d\vb x
\\
&= \vb c^\dagger 
   \Big[ \frac{1}{\omega}{\text{sym }\partial_i \vb G} \Big] 
   \vb c
\\
\intertext{Similarly,}
\tau_i &= \frac{1}{\omega} \text{Im }
  \int_{\mc V} \bmc F_\theta^*(\vb x) \cdot \bmc J(\vb x) \, d\vb x
\\
&= \vb c^\dagger 
   \Big[ \frac{1}{\omega}{\text{sym }\partial_\theta \vb G} \Big] 
   \vb c
\end{align*}

%%%%%%%%%%%%%%%%%%%%%%%%%%%%%%%%%%%%%%%%%%%%%%%%%%%%%%%%%%%%%%%%%%%%%%
%%%%%%%%%%%%%%%%%%%%%%%%%%%%%%%%%%%%%%%%%%%%%%%%%%%%%%%%%%%%%%%%%%%%%%
%%%%%%%%%%%%%%%%%%%%%%%%%%%%%%%%%%%%%%%%%%%%%%%%%%%%%%%%%%%%%%%%%%%%%%
\subsection{Statistical averages of surface-current bilinears}

The previous section discussed several examples of physical
quantities $Q$ which, in classical scattering problems with 
deterministic sources and deterministic induced surface 
currents, may be computed as bilinear functions of the surface 
currents, $ Q(\vb c)=\vb c^\dagger \vb Q \vb c$. 
The heart of the FSC 
approach to nonequilibrium phenomena is a concise formula for 
the \textit{statistical average} of such surface-current bilinears,
where the averaging is performed over thermal and quantum-mechanical
fluctuations of sources inside material bodies at fixed temperatures.
A precise statement of the FSC equivalence is as follows.

%%%%%%%%%%%%%%%%%%%%%%%%%%%%%%%%%%%%%%%%%%%%%%%%%%%%%%%%%%%%%%%%%%%%%%
\begin{itemize}

\item
If $Q$ is a physical quantity which (in a classical, deterministic 
setting) is expressed as a bilinear function of the surface currents 
involving a matrix $\vb Q$, i.e.
%====================================================================%
\begin{subequations}
\begin{equation}
 \text{if} \qquad Q(\vb c)=\vb c^\dagger \vb Q \vb c
\end{equation}
%====================================================================%

\item
then the statistical \textit{average} of the quantity $Q$---where
the averaging is over quantum and thermal fluctuations---is given by
a sum over the contributions of sources in all bodies:
%--------------------------------------------------------------------%
\begin{equation}
 \text{then} \qquad \big\langle Q\big\rangle
  = \int_0^\infty \, \sum_s \, \Theta_s(\omega) \Phi_s(\omega)\,d\omega 
\end{equation}
where 
$\Theta_s(\omega) = \frac{\hbar\omega}{e^{\hbar \omega/kT_s} - 1}$
is the Bose-Einstein factor for the temperature $T_s$ of 
body $s$, and where the ``generalized flux'' $\Phi_s$
is the trace of a four-matrix product involving the $\vb Q$ matrix:
%====================================================================%
\begin{equation}
\Phi_b
  = \frac{2}{\pi} 
    \Tr\left\{ \vb Q 
               \underbrace{\vb W \overline{\vb G}_s \vb W^\dagger}
                         _{\vb R_s}
       \right\}.
\end{equation}
%====================================================================%
\label{FSCEquivalence}
\end{subequations}
%--------------------------------------------------------------------%
Here $\vb W$ is the inverse of the BEM matrix for the 
\textit{entire} collection of bodies and $\overline{\vb G}_s$ 
is a symmetrized version of a portion of the BEM matrix for body 
$s$ alone.

\end{itemize}
%%%%%%%%%%%%%%%%%%%%%%%%%%%%%%%%%%%%%%%%%%%%%%%%%%%%%%%%%%%%%%%%%%%%%%
The matrix $\vb R_s$ in (\ref{FSCEquivalence})---the ``Rytov''
matrix--furnishes a concise description of the fluctuating sources
inside body $s$. More specifically, the eigenvectors of $\vb R_s$ 
may be interpreted as ``normal modes'' of the fluctuating source 
distribution inside body $s$; these are surface-current vectors 
whose contributions to time-average PFT quantities add without
mixing.

\subsection*{Structure of $\vb R_s$}

%%%%%%%%%%%%%%%%%%%%%%%%%%%%%%%%%%%%%%%%%%%%%%%%%%%%%%%%%%%%%%%%%%%%%%
{\scriptsize
\begin{align*}
 \vb R_s &= \vb W \overline{\vb G}_s \vb W^\dagger
\\
%--------------------------------------------------------------------%
&=\left(\begin{array}{cccc}
  \vphantom{\vb W^\dagger_11} 
  \vb W_{11} & \vb W_{12} & \cdots & \vb W_{1N} 
\\ 
  \vphantom{\vb W^\dagger_11} 
  \vb W_{21} & \vb W_{22} & \cdots & \vb W_{2N} 
\\ 
  \vphantom{\vb W^\dagger_11} 
  \vdots     & \vdots     & \ddots & \vdots     
\\
  \vphantom{\vb W^\dagger_11} 
  \vb W_{N1} & \vb W_{N2} & \cdots & \vb W_{NN}
  \end{array}\right)
%--------------------------------------------------------------------%
 \left(\begin{array}{cccc}
  \vphantom{\vb W^\dagger_11} 
  0          & 0          & \cdots &         0          
  \\ 
  \vphantom{\vb W^\dagger_11} 
  0          & 0          & \cdots &         0          
  \\ 
  \vphantom{\vb W^\dagger_11} 
  \vdots     & \vdots     & \overline{\vb G}_{s} &  \vdots     
  \\
  \vphantom{\vb W^\dagger_11} 
  0          & 0          & \cdots &         0
  \end{array}\right)
%--------------------------------------------------------------------%
  \left(\begin{array}{cccc}
  \vb W_{11}^\dagger & \vb W_{21}^\dagger & \cdots & \vb W_{N1}^\dagger \\ 
  \vb W_{12}^\dagger & \vb W_{22}^\dagger & \cdots & \vb W_{N2}^\dagger \\ 
  \vdots     & \vdots     & \ddots & \vdots     \\
  \vb W_{1N}^\dagger & \vb W_{2N}^\dagger & \cdots & \vb W_{NN}^\dagger
  \end{array}\right)
\end{align*}
}
%%%%%%%%%%%%%%%%%%%%%%%%%%%%%%%%%%%%%%%%%%%%%%%%%%%%%%%%%%%%%%%%%%%%%%$
The $a,b$ subblock of this matrix is
%%%%%%%%%%%%%%%%%%%%%%%%%%%%%%%%%%%%%%%%%%%%%%%%%%%%%%%%%%%%%%%%%%%%%%$
$$ \vb R_{s; ab} = \vb W_{as} \overline{\vb G}_s \vb W_{bs}^\dagger. $$
%%%%%%%%%%%%%%%%%%%%%%%%%%%%%%%%%%%%%%%%%%%%%%%%%%%%%%%%%%%%%%%%%%%%%%$

%%%%%%%%%%%%%%%%%%%%%%%%%%%%%%%%%%%%%%%%%%%%%%%%%%%%%%%%%%%%%%%%%%%%%%
%%%%%%%%%%%%%%%%%%%%%%%%%%%%%%%%%%%%%%%%%%%%%%%%%%%%%%%%%%%%%%%%%%%%%%
%%%%%%%%%%%%%%%%%%%%%%%%%%%%%%%%%%%%%%%%%%%%%%%%%%%%%%%%%%%%%%%%%%%%%%
\newpage
\section{Output files}

\subsection{Output files for spatially-resolved quantities}

Let $\big\langle Q\supt{PFT}_d\big\rangle$ be the total time-average 
power, force, or torque on body $d$. ($d$'' stands for a 
``destination'' body, as distinct from a ``source'' body $s$ 
whose fluctuating sources contributed to the $Q_d$). From 
equation (\ref{FSCEquivalence}b), we have
%====================================================================%
\begin{equation}
 \qquad \big\langle Q_d\big\rangle
 = \int_0^\infty \, \sum_s \, \Theta_s(\omega) 
   \Phi\supt{PFT}_{s\to d}(\omega)\,d\omega
\end{equation}
%====================================================================%
where $\Theta_s$ is the Bose-Einstein factor for the temperature 
of body $s$ 
where $\Phi\supt{PFT}_{s \to d}(\omega)$ is the 
(frequency-resolved, spatially-integrated) flux of power or momentum
from body $s$ to body $d$.
%====================================================================%
It is convenient to introduce a dimensionless
angular frequency $u$ according to\footnote{Note that
$u$ is the numerical quantity referred to as \texttt{Omega}
on the {\sc scuff-neq} command line and in the 
{\sc scuff-neq} source code. Some convenient
dimensionful quantities involving $\omega_0$ are:
%====================================================================%
$$
 \hbar\omega_0=0.197\text{ eV}=31.6\cdot10^{-21} \text{ Joules}, 
 \qquad 
 \hbar\omega_0^2=9.49115\,\mu\text{W} \, (\text{microWatts}).
$$}
%====================================================================%
$$ \omega=u\cdot \omega_0
   \qquad \text{where} \qquad
   \omega_0\equiv 3\cdot 10^{14}\text{ rad/sec}.
$$
Then the total time-average force PFT on body $d$ may be 
expressed as an integral over the dimensionless variable $u$:
%====================================================================%
\numeq{QAverage}
{
 \qquad \big\langle Q_d\big\rangle
 = \hbar \omega_0^2
   \int_0^\infty 
              \sum_s \, \frac{u}{e^{\beta_s u}-1}
              \Phi\supt{PFT}_{s\to d}(u)\,du
}
where 
%====================================================================%
$$\beta_s = \frac{\hbar \omega_0}{kT_s}
  \approx \frac{ 2278 }{T_s\text{ in Kelvin.}}
$$
%====================================================================%
The various constituents of equation (\ref{QAverage}) are written
by {\sc scuff-neq} to three distinct output files, as follows:
%====================================================================%
\numeq{QAverage2}
{
 \qquad \big\langle Q_d\big\rangle
 = \hbar \omega_0^2
   \underbrace{
   \Bigg[
   \int_0^\infty 
  \underbrace{\bigg\{ \sum_s \, \frac{u}{e^{\beta_s u}-1}
  \underbrace{\Phi\supt{PFT}_{s\to d}(u)}_{\texttt{.SIFlux}}
             \bigg\} }_{\texttt{.SIIntegrand}}
   \,\,du \Bigg]
              }_{\texttt{.NEQPFT}}
}
%====================================================================%
More specifically, 
%====================================================================%
\begin{itemize}
  \item The quantity $\Phi_{s\to d}\supt{PFT}(u)$ is written to the
        \texttt{.SIFlux} file. This quantity may be interpreted
        as the PFT per unit frequency per
        unit temperature exerted on body $d$ due to sources
        in body $s.$ This is a temperature-independent quantity.
        For \texttt{PFT}=power, this quantity
        is \textit{dimensionless}. For \texttt{PFT}=force,
        this quantity has units of \textit{nanoNewtons/watts}.
  \item The quantity written to the \texttt{.SIIntegrand} file
        is the PFT per unit frequency exerted on body $d$ due to 
        thermally-weighted sources in body $s$.
        (The \texttt{.SIIntegrand} file records, for each 
         destination body $d$, both the individual contributions
         of each source body $s$ and the total contributions
         of all source bodies.)
  \item The quantity written to the \texttt{.NEQPFT} file
        is the total time-average PFT exerted on body $d$ due to 
        thermally-weighted sources in body $s$.
        (The \texttt{.NEQPFT} file records, for each
         destination body $d$, both the individual contributions
         of each source body $s$ and the total contributions
         of all source bodies.)
\end{itemize}
%====================================================================%

\subsection{An explicit example}

To elucidate the significance of the various numerical
quantities reported in the {\sc scuff-neq} output files,
we now consider a specific worked example.

%====================================================================%
%====================================================================%
%===================================================================%
\newpage
\appendix
\section{Computation of DSIPFT PFT matrices}

In Section \ref{PQsFromSCBsSection} we sketched expressions for
the ``dense surface-integral PFT'' (DSIPFT)
matrices $\{\vb M\PS, \vb M\IFS, \vb M\ITS\}$ which, when sandwiched
between surface-current vectors $\vb c$, yield the total
\{power, force, torque\} on objects contained within a bounding
surface $\mc S$. In this appendix we discuss the computation of 
these matrices.

\subsection{Numerical cubature}

To estimate the elements of e.g. the matrix $\vb M\PS$, we
use a cubature rule:
%====================================================================%
\begin{align}
  M_{\alpha\beta}\PS
 &= \frac{1}{4} \int_{\mc S} 
    \bmc F^\dagger_\alpha(\vb x) 
    \,
    \bmc N\supt{P}(\vbhat{n})
    \,
    \bmc F_\beta(\vb x) 
     \, d\vb x
\nn
 &\approx \frac{1}{4} \sum_{n=1}^{N\subt{C}} w_n 
    \bmc F^\dagger_\alpha(\vb x_n) 
    \,
    \bmc N\supt{P}(\vbhat{n}_n)
    \,
    \bmc F_\beta(\vb x_n) 
\label{MPSCubature}
\end{align}
%====================================================================%
where $\{w_n, \vb x_n\}$ is an $N\subt{C}$-point cubature rule 
for the surface $\mc S$ and $\vbhat{n}_n$ is the outward-pointing
surface normal to $\mc S$ at $\vb x_n.$
 
\subsection{Precomputation of field six-vectors}

%====================================================================%
%====================================================================%
%====================================================================%
\section{Computation of VIPFT force and torque matrices}

The VIPFT matrices for the force and torque require matrix
elements of Helmholtz kernel derivatives: 
%====================================================================%
\begin{subequations}
\begin{align}
 \Big[\partial_i \vb G\Big]\supt{PQ}_{\alpha \beta}
&= \int\, d\vb x \, \int \, d\vb x^\prime \,
   b_{\alpha j}(\vb x) \partial_i \Gamma_{jk}\supt{PQ}(\vb x-\vb x^\prime)
   b_{\beta k}(\vb x^\prime)
\\[5pt]
 \Big[\partial_\theta \vb G\Big]\supt{PQ}_{\alpha \beta}
&= \int\, d\vb x \, \int \, d\vb x^\prime \,
   b_{\alpha j}(\vb x) 
   \partial_\theta \Gamma_{jk}\supt{PQ}(\vb x-\vb x^\prime)
   b_{\beta k}(\vb x^\prime)
\end{align}
\label{dGIntegrals}
\end{subequations}
%====================================================================%
where $\partial_i$ denotes the derivative of $\Gamma$ 
with respect to a infinitesimal displacement of $\vb x$
in the $i$th caresian direction, 
while $\partial_\theta$ denotes the derivative with respect to an a
infinitesimal rotation of $\vb x$ about some origin $\vb x_0$ and 
rotation axis $\vbhat{a}$ (my notation hides the dependence on 
$\vb x_0$ and $\vbhat{a}$).

Integrals of the form (\ref{dGIntegrals}) are required for 
FSC calculations of equilibrium Casimir forces and torques,
and thus their computation is implemented in my {\sc scuff-cas3d} code.
However, in the equilibrium case we only ever need to compute
matrix elements of the form (\ref{dGIntegrals}) between basis
functions that live on different objects; hence the integrals
are always nonsingular and may be evaluated by simple 
numerical quadrature, as is done in {\sc scuff-cas3d}.

A major complication in the non-equilibrium case, which is
not present in the equilibrium case, is that here we must
also compute integrals (\ref{dGIntegrals}) for basis
functions living on the \textit{same} object; these are needed 
to evaluate the self-term contributions to the non-equilibrium
force and torque. This means we must handle the case in
which the supports of the basis functions overlap, and the
integrals are thus singular. In {\sc scuff-neq} we do this
using the generalized Taylor-Duffy method reported 
in~\citeasnoun{Reid2013C}. In this section we discuss the 
forms of the $P$, $K$, and $\mc K$ functions relevant for
this computation. (The $\mc P$ functions, which are generated
by {\sc mathematica} using techniques discussed in~\citeasnoun{Reid2013C},
are much too complicated to reproduce here.)

\paragraph{Representation of integrals in generalized Taylor-Duffy form}

As usual, the kernels in (\ref{dGIntegrals}) are related to the 
dyadic Helmholtz Green's functions $\vb G, \vb C$ according to
%====================================================================%
$$ 
 \Gamma\supt{EE}_{ij}(\vb r)
= ikZ_0 Z^r G_{ij}(\vb r),
\qquad
 \Gamma\supt{EM}_{ij}(\vb r)
= ik C_{ij}(\vb r), 
$$
$$
 \Gamma\supt{ME}_{ij}(\vb r)
= -ik C_{ij}(\vb r), 
\qquad 
 \Gamma\supt{MM}_{ij}(\vb r)
= \frac{ik}{Z_0 Z^r} G_{ij}(\vb r)
$$
with 
%====================================================================%
$$
 G_{jk}(\vb r) 
= \left[ f_1(ikr)\delta_{ij} + f_2(ikr)\frac{r_i r_j}{r^2}\right] 
   \frac{e^{ikr}}{4\pi (ik)^2 r^3} 
, \qquad 
 C_{jk}(\vb r) 
= \varepsilon_{jk\ell } r_\ell f_3(ikr)
   \frac{e^{ikr}}{4\pi ik r^3},
$$
$$
 f_1(x) = 1 -x+ x^2, \qquad 
 f_2(x) = -3 + 3x - x^2, \qquad 
 f_3(x) = x - 1.
$$
%====================================================================%
In general, the matrix elements of a dyadic kernel $\bmc O$ between
RWG functions involve contributions from four panel-panel pairs:
%====================================================================%
\begin{align*}
 \Big\langle 
   \vb b_\alpha 
 \Big| 
   \bmc O
 \Big| 
   \vb b_\beta 
 \Big\rangle
&= \ell_\alpha \ell_\beta \sum \pm 
  \underbrace{
   \frac{1}{4A_\alpha^\pm A_\beta^\pm}
   \int_{\mc P_\alpha^\pm } \, d\vb x        \,
   \int_{\mc P_\beta^\pm} \, d\vb x^\prime \,
   (\vb x-\vb Q)_j (\vb x^\prime-\vb Q^\prime)_k
   \mc O_{jk}(\vb r)
             }_{\mc I(\mc O; \mc P_\alpha, \mc P_\beta)}
\end{align*}
%====================================================================%
The derivatives of the dyadic GFs are
%====================================================================%
\begin{align*}
 \partial_i G_{jk}(\vb r)
&=
  \bigg\{ f_2(ikr)\Big[ r_i\delta_{jk} + r_j\delta_{ik} + r_k\delta_{ij} \Big]
         +(ikr)^2 f_3(ikr) r_i \delta_{jk}
\\
&\hspace{0.75in}
         + f_4(ikr)\frac{r_i r_j r_k}{r^2}
  \bigg\} \frac{e^{ikr}}{4\pi (ik)^2 r^5}
\intertext{[where $ f_4(x)=15-15x+6x^2 - x^3$]}
 \partial_i C_{jk}(\vb r)
&=\varepsilon_{jk\ell}
  \bigg\{  f_3(ikr) \delta_{i\ell} 
          -f_2(ikr) \frac{r_i r_\ell}{r^2} 
  \bigg\} \frac{e^{ikr}}{4\pi (ik) r^3}
\end{align*}
%====================================================================%
With these expressions we may write the panel-panel
integrals in a form suitable for treatment by the generalized 
Taylor-Duffy method:
%====================================================================%
\begin{align*}
 \mathcal{I}(\partial_i \vb G;  \mc P_\alpha, \mc P_\beta)
&=\frac{1}{(ik)^2}
\iint \Big[  P\supt{G1}(\vb x, \vb x^\prime) K\supt{G1}(r)
             + P\supt{G2}(\vb x, \vb x^\prime) K\supt{G2}(r)
             + P\supt{G3}(\vb x, \vb x^\prime) K\supt{G3}(r)
        \Big]\,d\vb x\,d\vb x^\prime
\\
 \mathcal{I}(\partial_i \vb C; \mc P_\alpha, \mc P_\beta)
&=\frac{1}{ik}
  \iint \Big[  P\supt{C1}(\vb x, \vb x^\prime) K\supt{C1}(r)
              +P\supt{C2}(\vb x, \vb x^\prime) K\supt{C2}(r)
        \Big]\,d\vb x \, d\vb x^\prime
\end{align*}
%====================================================================%
where the $P$s and $K$s are
%====================================================================%
\begin{align*}
%--------------------------------------------------------------------%
P\supt{G1} 
 &= (\vb x - \vb x^\prime)_i 
    \Big[ (\vb x-\vb Q) \cdot (\vb x^\prime-\vb Q^\prime)\Big] 
\\
 &\quad + (\vb x-\vb Q)_i 
          \Big[ (\vb x - \vb x^\prime) \cdot (\vb x^\prime-\vb Q^\prime) \Big]
        + (\vb x^\prime-\vb Q^\prime)_i 
          \Big[ (\vb x - \vb x^\prime) \cdot (\vb x-\vb Q) \Big]
\\
%--------------------------------------------------------------------%
K\supt{G1}
&= f_2(ikr)\frac{e^{ikr}}{4\pi r^5}
\\[5pt]
%--------------------------------------------------------------------%
P\supt{G2}
&=(\vb x - \vb x^\prime)_i 
   \Big[(\vb x-\vb Q) \cdot (\vb x^\prime-\vb Q^\prime)\Big]
\\
%--------------------------------------------------------------------%
K\supt{G2}
&= (ik)^2 f_3(ikr)\frac{e^{ikr}}{4\pi r^3}
\\[5pt]
%--------------------------------------------------------------------%
P\supt{G3} 
&=(\vb x - \vb x^\prime)_i 
   \Big[ (\vb x-\vb x^\prime) \cdot (\vb x-\vb Q)  \Big]
   \Big[ (\vb x-\vb x^\prime) \cdot (\vb x^\prime-\vb Q^\prime)  \Big]
\\
%--------------------------------------------------------------------%
K\supt{G3}
&= f_4(ikr)\frac{e^{ikr}}{4\pi r^7}
\\[5pt]
%--------------------------------------------------------------------%
P\supt{C1}
&=\Big[(\vb x-\vb Q) \times (\vb x^\prime - \vb Q^\prime)\Big]_i
\\
%--------------------------------------------------------------------%
K\supt{C1}
&= f_3(ikr)\frac{e^{ikr}}{4\pi r^3}
\\[5pt]
%--------------------------------------------------------------------%
P\supt{C2}
 &= -(\vb x - \vb x^\prime)_i
     \bigg\{ (\vb x - \vb x^\prime) \cdot 
     \Big[ (\vb x-\vb Q)\times (\vb x^\prime-\vb Q^\prime) \Big]\bigg\}
\\
%--------------------------------------------------------------------%
K\supt{C2}
&= 
f_2(ikr) \frac{e^{ikr}}{4\pi r^5}
\end{align*}
%====================================================================%
The ``first integrals'' of the $K$ kernels here may be computed
as follows:
\begin{align*}
 \text{if  } K(r) &= (ikr)^p \frac{e^{ikr}}{4\pi r^q} 
\\
 \text{then  } \mc K_n(r) &\equiv\int_0^n w^n K(wX) \, dw
\\
 &=\frac{(ik)^p e^{ikX}}{4\pi(n+p+1-q)}\, X^{n+p-q} \,
   \texttt{ExpRel}\Big[n+p+1-q, -ikX\Big].
\end{align*}

%====================================================================%
%====================================================================%
%====================================================================%
\newpage
\bibliographystyle{ieeetr}
\bibliography{scuff-neq}

\end{document}
