\documentclass[letterpaper]{article}

../../../../tex/scufftex.tex

\graphicspath{{figures/}}

%------------------------------------------------------------
%------------------------------------------------------------
%- Special commands for this document -----------------------
%------------------------------------------------------------
%------------------------------------------------------------

%------------------------------------------------------------
%------------------------------------------------------------
%- Document header  -----------------------------------------
%------------------------------------------------------------
%------------------------------------------------------------
\title {{\sc scuff-rf} Implementation Notes}
\author {Homer Reid}
\date {June 27, 2012}

%------------------------------------------------------------
%------------------------------------------------------------
%- Start of actual document
%------------------------------------------------------------
%------------------------------------------------------------

\begin{document}
\pagestyle{myheadings}
\markright{Homer Reid: {\sc scuff-rf} Implementation Notes}
\maketitle

\tableofcontents

%%%%%%%%%%%%%%%%%%%%%%%%%%%%%%%%%%%%%%%%%%%%%%%%%%%%%%%%%%%%%%%%%%%%%%
%%%%%%%%%%%%%%%%%%%%%%%%%%%%%%%%%%%%%%%%%%%%%%%%%%%%%%%%%%%%%%%%%%%%%%
%%%%%%%%%%%%%%%%%%%%%%%%%%%%%%%%%%%%%%%%%%%%%%%%%%%%%%%%%%%%%%%%%%%%%%
\section{Overview}

{\sc scuff-rf} extends the functionality present in the
{\sc scuff-em} core library in three main ways:

\begin{enumerate}
  \item It introduces the notion of a \textit{port.} This is the region
        of your structure (be it an antenna, a coaxial cable, etc.)
        that interfaces with the RF circuitry. Information on ports 
        is stored internally within {\sc scuff-rf} in a data structure
        named \texttt{RWGPort}.
  \item It allows you to specify a fixed set of \textit{port currents}, 
        rather than a fixed incident field configuration, to define the
        RHS vector of the BEM scattering problem. (This basically
        involves computing the field radiated by the RWG panels on
        the driven ports.)
  \item Once the scattering problem is solved, it provides functionality
        to compute \textit{port voltages}. (This basically involves 
        computing the line integral of the scattered electric field 
        between two points on opposite sides of the port.)
\end{enumerate}

In this memo we will discuss the implementation of these features.

%%%%%%%%%%%%%%%%%%%%%%%%%%%%%%%%%%%%%%%%%%%%%%%%%%%%%%%%%%%%%%%%%%%%%%
%%%%%%%%%%%%%%%%%%%%%%%%%%%%%%%%%%%%%%%%%%%%%%%%%%%%%%%%%%%%%%%%%%%%%%
%%%%%%%%%%%%%%%%%%%%%%%%%%%%%%%%%%%%%%%%%%%%%%%%%%%%%%%%%%%%%%%%%%%%%%
\newpage
\section{The concept of an \texttt{RWGPort}}
%####################################################################%
%####################################################################%
%####################################################################%
\begin{figure}
\begin{center}
%\includegraphics{RWGPortFigure
\caption{An \texttt{RWGPort}. The port current $I$ is 
         forced into the positive port edge, while  
         an identical current is extracted from the negative
         port edge. (Note that the two port edges do not need 
         to be the same length, nor do they need to comprise  
         the same number of panels.) The figure indicates
         the positive and negative ``reference points,''
         between which the line integral of the electric
         field is computed when evaluating the port voltage.
        }
\label{RWGPortFigure}
\end{center}
\end{figure}
%####################################################################%
%####################################################################%
%####################################################################%

A key extension of the \lss core library provided by {\sc scuff-rf}
is the idea of an \texttt{RWGPort}. This is a physical region of
a structure that interfaces with RF circuitry.

Figure \ref{PortFigure} depicts an \texttt{RWGPort}. 
The port consists of two \textit{port edges}, 
one arbitrarily chosen to be the ``positive'' edge and 
the other the ``negative edge.'' Each port edge
is comprised of a number of panel edges. 
(Note that there is no need for the two port edges
to have the same length, nor must they be comprised
of the same number of panel edges.) 
If the port current for this port is $I$, then 
a total current $I$, equidistributed over the 
length $W^+$, is forced into the positive edge 
of the port, while a total current $I$ equidistributed
over the length $W^-$. 

Also shown in the figure are the positive and negative
\textit{reference points} associated with the port. 
These are the points between which the line integral
of the $\vb E$ field is computed when computing the 
port voltage (see \ref{PortVoltageSection}).

These should lie respectively on the positive and 
negative port edges, but may otherwise be chosen
arbitrarily.

%%%%%%%%%%%%%%%%%%%%%%%%%%%%%%%%%%%%%%%%%%%%%%%%%%%%%%%%%%%%%%%%%%%%%%
%%%%%%%%%%%%%%%%%%%%%%%%%%%%%%%%%%%%%%%%%%%%%%%%%%%%%%%%%%%%%%%%%%%%%%
%%%%%%%%%%%%%%%%%%%%%%%%%%%%%%%%%%%%%%%%%%%%%%%%%%%%%%%%%%%%%%%%%%%%%%
\newpage

\section{Port currents: A new type of incident field}
\label{PortCurrentSection}

%####################################################################%
%####################################################################%
%####################################################################%
\begin{figure}
\begin{center}
%\includegraphics{PortCurrentIncidentField}}
\caption{The incident field in the BEM scattering problem
         is the field radiated by the half-RWG basis functions
         associated with the edges \texttt{RWGPort}.
         Each half-RWG basis function is populated with
         a strength proportional to the port current.
        }
\label{RWGPortFigure}
\end{center}
\end{figure}
%####################################################################%
%####################################################################%
%####################################################################%

In the BEM scattering problems solved by {\sc scuff-rf}, the incident
field is the field radiated by the half-RWG basis functions associated 
with each port, with those basis functions populated by strengths
proportional to the port current.

%%%%%%%%%%%%%%%%%%%%%%%%%%%%%%%%%%%%%%%%%%%%%%%%%%%%%%%%%%%%%%%%%%%%%%
%%%%%%%%%%%%%%%%%%%%%%%%%%%%%%%%%%%%%%%%%%%%%%%%%%%%%%%%%%%%%%%%%%%%%%
%%%%%%%%%%%%%%%%%%%%%%%%%%%%%%%%%%%%%%%%%%%%%%%%%%%%%%%%%%%%%%%%%%%%%%
\newpage
\section{Port voltages: A new type of post-processing computation}
\label{PortVoltageSection}

$$ V\sups{port} = \Phi(\vb 

\end{document}
