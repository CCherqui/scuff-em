\documentclass[letterpaper]{article}

../../../../tex/scufftex.tex

\newcommand\supsstar[1]{^{\hbox{\scriptsize{#1}}*}}
\newcommand\suptstar[1]{^{\hbox{\scriptsize{#1}}*}}

\graphicspath{{figures/}}

%------------------------------------------------------------
%------------------------------------------------------------
%- Special commands for this document -----------------------
%------------------------------------------------------------
%------------------------------------------------------------

%------------------------------------------------------------
%------------------------------------------------------------
%- Document header  -----------------------------------------
%------------------------------------------------------------
%------------------------------------------------------------
\title {{\sc scuff-scatter} Implementation Notes}
\author {Homer Reid}
\date {March 18, 2012}

%------------------------------------------------------------
%------------------------------------------------------------
%- Start of actual document
%------------------------------------------------------------
%------------------------------------------------------------

\begin{document}
\pagestyle{myheadings}
\markright{Homer Reid: {\sc scuff-scatter} Implementation Notes}
\maketitle

\tableofcontents

%%%%%%%%%%%%%%%%%%%%%%%%%%%%%%%%%%%%%%%%%%%%%%%%%%%%%%%%%%%%%%%%%%%%%%
%%%%%%%%%%%%%%%%%%%%%%%%%%%%%%%%%%%%%%%%%%%%%%%%%%%%%%%%%%%%%%%%%%%%%%
%%%%%%%%%%%%%%%%%%%%%%%%%%%%%%%%%%%%%%%%%%%%%%%%%%%%%%%%%%%%%%%%%%%%%%
\newpage
\section{Concise Formulas for Scattered and Absorbed Power}

In many scattering problems we will want to compute the total
power scattered from, and the total power absorbed by, the
object(s) in a scattering geometry.  The na\"ive way to do 
this would be to integrate the normal Poynting vector over
some fictitious bounding surface surrounding the object(s);
this integral could be evaluated by numerical quadrature,
with the values of the Poynting vector at each quadrature
point computed by the usual {\sc libscuff} methods for
calculating scattered fields. {\sc scuff-scatter} uses a 
more efficient approach, obtaining the scattered and absorbed 
power \textit{directly} from matrix-vector and vector-vector 
(dot) products involving the BEM matrices and 
vectors.\footnote{To my knowledge, these formulas were first
derived by Steven Johnson.}

Throughout this section, $\mathcal{S}$ denotes the boundary
of the exterior medium (that is, the union of the outer surfaces
of all objects contained in the exterior medium), and 
$\vbhat{n}$ denotes the normal vector at a point on 
$\mathcal{S}$; $\vbhat{n}$ is taken positive pointing
into the exterior medium (away from the object).
We work at a single frequency and suppress $\omega$ arguments.

\subsubsection*{Absorbed Power}

The normal component of the \textit{total} Poynting vector 
(which includes the contributions of both the scattered and 
incident fields) 
at a point $\vb x$ on $\mathcal{S}$ is 
%====================================================================%
\begin{align*}
 \vb P\sups{tot}(\vb x) \cdot \vbhat{n}(\vb x)
  &=\frac{1}{2}\text{Re} 
    \bigg[ \vb E\supsstar{tot}(\vb x) \times \vb H\sups{tot}(\vb x)\bigg] 
    \cdot \vbhat{n}(\vb x)
%--------------------------------------------------------------------%
\intertext{which may be rewritten in the form 
           (temporarily suppressing $\vb x$ arguments)}
%--------------------------------------------------------------------%
  &=\frac{1}{4}\text{Re}
    \bigg[ \vb E\supsstar{tot} \cdot \big(\vbhat{n}\times \vb H \big)
          +\vb H\supsstar{tot} \cdot \big(-\vbhat{n}\times \vb E \big) 
    \bigg].
%--------------------------------------------------------------------%
\intertext{But the quantities in parentheses here are just the electric
           and magnetic surface currents that enter into the SIE 
           formulation of scattering problems, so we find simply}
%--------------------------------------------------------------------%
\vb P\sups{tot}(\vb x) \cdot \vbhat{n}(\vb x)
  &=-\frac{1}{4}\text{Re}
    \Big[ \vb E\supsstar{tot} \cdot \vb K 
          +\vb H\supsstar{tot} \cdot \vb N
    \Big].
\end{align*}
%====================================================================%
%
% some quick matlab code that verifies the above:
  E=rand(3,1) + sqrt(-1)*rand(3,1);
  H=rand(3,1) + sqrt(-1)*rand(3,1);
  nHat=rand(3,1); nHat=nHat/norm(nHat);
  K=cross(nHat,H);
  N=-cross(nHat,E);
  P1 = 0.5*( dot( cross(E, conj(H)), nHat ) )
  P2 = 0.5*( dot(conj(E), K)  + dot(conj(H),N) )

\begin{align*}
 P\sups{abs} 
&= - \oint \vb P\sups{tot}(\vb x) \cdot \vbhat{n} dA 
\\
&= 
\end{align*}

\end{document}
