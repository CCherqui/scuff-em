\documentclass[letterpaper]{article}
\usepackage[square,sort,comma,numbers]{natbib}
\newcommand{\citeasnoun}[1]{Ref.~\citenum{#1}}

../../../../tex/scufftex.tex

\newcommand\supsstar[1]{^{\hbox{\scriptsize{#1}}*}}
\newcommand\suptstar[1]{^{\hbox{\scriptsize{#1}}*}}
\newcommand{\IF}{^{i\text{\scriptsize F}}}
\newcommand{\IFFlux}{^{i\text{\tiny FFLUX}}}
\newcommand{\IT}{^{i\text{\scriptsize T}}}
\newcommand{\ITFlux}{^{i\text{\tiny TFLUX}}}
\newcommand{\PS}{^{\text{\scriptsize P}\mc S}}
\newcommand{\IFS}{^{i\text{\scriptsize F}\mc S}}
\newcommand{\ITS}{^{i\text{\scriptsize T}\mc S}}

\graphicspath{{figures/}}

%------------------------------------------------------------
%------------------------------------------------------------
%- Special commands for this document -----------------------
%------------------------------------------------------------
%------------------------------------------------------------

%------------------------------------------------------------
%------------------------------------------------------------
%- Document header  -----------------------------------------
%------------------------------------------------------------
%------------------------------------------------------------
\title {{\sc scuff-neq} Technical Reference}
\author {Homer Reid}
\date {July 6, 2014}

%------------------------------------------------------------
%------------------------------------------------------------
%- Start of actual document
%------------------------------------------------------------
%------------------------------------------------------------

\begin{document}
\pagestyle{myheadings}
\markright{Homer Reid: {\sc scuff-neq} Technical Reference}
\maketitle

\tableofcontents

%%%%%%%%%%%%%%%%%%%%%%%%%%%%%%%%%%%%%%%%%%%%%%%%%%%%%%%%%%%%%%%%%%%%%%
%%%%%%%%%%%%%%%%%%%%%%%%%%%%%%%%%%%%%%%%%%%%%%%%%%%%%%%%%%%%%%%%%%%%%%
%%%%%%%%%%%%%%%%%%%%%%%%%%%%%%%%%%%%%%%%%%%%%%%%%%%%%%%%%%%%%%%%%%%%%%
\newpage
\section{Theoretical background}

{\sc scuff-neq} implements the ``fluctuating-surface-current''
(FSC) approach to computational fluctuation physics. The 
FSC method was originally developed for equilibrium Casimir
computations~\cite{Reid2009, Reid2011, Reid2013B} and later
extended to non-equilibrium phenomena in 
Refs.~\citenum{Rodriguez2012C} and \citenum{Rodriguez2013B}.
Here we present a quick summary of the key equations in this
approach.

%%%%%%%%%%%%%%%%%%%%%%%%%%%%%%%%%%%%%%%%%%%%%%%%%%%%%%%%%%%%%%%%%%%%%%
%%%%%%%%%%%%%%%%%%%%%%%%%%%%%%%%%%%%%%%%%%%%%%%%%%%%%%%%%%%%%%%%%%%%%%
%%%%%%%%%%%%%%%%%%%%%%%%%%%%%%%%%%%%%%%%%%%%%%%%%%%%%%%%%%%%%%%%%%%%%%
\subsection*{Fields from surface currents}

Recall that in the surface-integral-equation (SIE) approach to
classical electromagnetism problems, we solve for tangential
\textit{surface currents} (both electric currents $\vb K$ 
and magnetic currents $\vb N$) flowing on the surfaces of 
homogeneous material bodies. In numerical solvers, we 
approximate these as finite expansions in a discrete set
of $N\subt{B}$ basis functions; using a convenient 6-vector notation 
in which $\bmc C\equiv {\vb K \choose \vb N}$, we put
%--------------------------------------------------------------------%
$$ \bmc C(\vb x)=\sum_{\alpha=1}^{N\subt{B}} 
   c_\alpha \bmc B_\alpha(\vb x) 
$$ 
%--------------------------------------------------------------------%
where $\{\bmc B_\alpha\}$ is a set of 6-vector basis functions
and $\{c_\alpha\}$ are scalar expansion coefficients.
(We work at a fixed frequency $\omega$ and suppress a factor of
$e^{-i\omega t}$ in all fields and currents.)

The electric and magnetic fields produced by these currents
are linear functions of the $\{c_\alpha\}$ coefficients.
In 6-vector notation with $\bmc F\equiv {\vb E \choose \vb H}$, we have
%--------------------------------------------------------------------%
\numeq{FFromC}
{ \bmc F(\vb x)=
   \sum_{\alpha} c_\alpha \bmc F_\alpha(\vb x)
}
%--------------------------------------------------------------------%
where $\bmc F_\alpha={\vb E_\alpha \choose \vb H_\alpha}$ 
is the six-vector of fields produced by basis function $\bmc B_\alpha$ 
populated with unit strength. These may be calculated numerically
as convolutions over basis functions,
%--------------------------------------------------------------------%
$$ \bmc F_\alpha(\vb x)
   =\int_{\sup \bmc B_\alpha} \bmc{G}(\vb x, \vb x^\prime)
    \bmc B_\alpha(\vb x^\prime) \, d\vb x^\prime
$$
%--------------------------------------------------------------------%
where $\bmc{G}$ is an appropriate $6\times 6$ dyadic Green's
function.

%%%%%%%%%%%%%%%%%%%%%%%%%%%%%%%%%%%%%%%%%%%%%%%%%%%%%%%%%%%%%%%%%%%%%%
%%%%%%%%%%%%%%%%%%%%%%%%%%%%%%%%%%%%%%%%%%%%%%%%%%%%%%%%%%%%%%%%%%%%%%
%%%%%%%%%%%%%%%%%%%%%%%%%%%%%%%%%%%%%%%%%%%%%%%%%%%%%%%%%%%%%%%%%%%%%%
\subsection*{Energy and momentum flux from field bilinears}

The Poynting flux and Maxwell stress tensor are quadratic functions
of the field components and may be conveniently written in the form
of 6-dimensional vector-matrix-vector products. 
In particular, the power flux in the direction of a unit vector 
$\vbhat{n}$ is
%====================================================================%
\begin{align}
 \vb P(\vb x) \cdot \vbhat{n}
   &=\frac{1}{2}\text{ Re }\varepsilon_{ijk}\vbhat{n}_i E^*_j(\vb x) H_k(\vb x)
\nn
   &=\frac{1}{4}\bmc{F}^\dagger(\vb x) \, \bmc{N}\supt{P}(\vbhat{n}) \, \bmc{F}(\vb x)
\label{PVMVP}
\end{align}
%====================================================================%
with
%====================================================================%
$$
   \bmc N\supt{P}=
   \left(\begin{array}{cc}
   0       & \vb N\supt{P}   \\ [3pt]
  -\vb N\supt{P} & 0
   \end{array}\right), 
\qquad 
   \vb N\supt{P}
   =
   \left(\begin{array}{ccc}
   0          &  \hat{n}_z & -\hat{n_y} \\
  -\hat{n}_z  &  0         & +\hat{n_x} \\
   \hat{n}_y  & -\hat{n}_x & 0
   \end{array}\right).
$$
%====================================================================%
Similarly, the flux of $i$-directed linear momentum is
%--------------------------------------------------------------------%
\begin{align}
 \vb T_{i}(\vb x) \cdot \vbhat{n} 
&=\frac{1}{2}\text{ Re }
  \left[ \epsilon E^*_i(\vb x) E_j(\vb x) 
             +\mu H^*_i(\vb x) H_j(\vb x) 
       -\frac{\delta_{ij}}{2}
         \Big( \epsilon |\vb E|^2
              +\mu      |\vb H|^2
         \Big)
 \right] \hat {n}_j
\nn
&= \frac{1}{4}\bmc{F}^\dagger(\vb x) 
   \, \bmc{N}\IF \, \bmc{F}(\vb x)
\label{IFVMVP}
\end{align}
with
%--------------------------------------------------------------------%
$$
   \bmc N\IF=
   \left(\begin{array}{cc}
   \epsilon \vb N\IF & 0 \\
            0        & \mu \vb N\IF 
   \end{array}\right)
$$
%--------------------------------------------------------------------%
where the $3\times 3$ matrix $N\IF$ has entries
%--------------------------------------------------------------------%
$$ N\IF_{ab} = 
   \delta_{ai} \hat{n}_b + \delta_{bi} \hat{n}_a
  - \hat{n}_i \delta_{ab}.
$$
%====================================================================%
For example, if we are computing the $x$-force ($i=x$)
we have 
%====================================================================%
$$ \vb N^{x\text{\tiny F}}=\left(\begin{array}{ccc}
   \hat{n}_x & \hat n_y   & \hat n_z \\
   \hat{n}_y & -\hat{n}_x & 0 \\
   \hat{n}_z & 0          & -\hat{n}_x
  \end{array}\right).
$$
The flux of $i$-directed \textit{angular} momentum, useful
for computations of torque about an origin $\vb x_0$,
%====================================================================%
\begin{align}
 \vb t_{i}(\vb x) \cdot \vbhat{n}
&=\frac{1}{2}
  \text{ Re }
  \varepsilon_{ijk}(\vb x-\vb x_0)_j T_{k\ell}(\vb x) \hat{n}_\ell
\\
&= \frac{1}{4}\bmc{F}^\dagger(\vb x) 
   \, \bmc{N}\IT \, \bmc{F}(\vb x)
\label{ITVMVP}
\end{align}
%====================================================================%
with 
%--------------------------------------------------------------------%
$$
   \bmc N\IT=
   \left(\begin{array}{cc}
   \epsilon \vb N\IT & 0 \\
            0        & \mu \vb N\IT 
   \end{array}\right)
$$
%--------------------------------------------------------------------%
where the $3\times 3$ matrix $N\IT$ has entries ($\vb D=\vb x-\vb x_0$)
%--------------------------------------------------------------------%
$$ N\IT_{ab}=
   \varepsilon_{ija}D_j \hat{n}_b
  +\varepsilon_{ijb}D_j \hat{n}_a
  -\delta_{ab} \varepsilon_{ijk} D_j \hat{n}_k.
$$

%%%%%%%%%%%%%%%%%%%%%%%%%%%%%%%%%%%%%%%%%%%%%%%%%%%%%%%%%%%%%%%%%%%%%%
%%%%%%%%%%%%%%%%%%%%%%%%%%%%%%%%%%%%%%%%%%%%%%%%%%%%%%%%%%%%%%%%%%%%%%
%%%%%%%%%%%%%%%%%%%%%%%%%%%%%%%%%%%%%%%%%%%%%%%%%%%%%%%%%%%%%%%%%%%%%%
\subsection*{Energy and momentum flux from surface-current bilinears}

Equations (\ref{PVMVP}), (\ref{IFVMVP}), and (\ref{ITVMVP}) express
the flux of power or momentum as bilinear products of the field
six-vectors $\bmc F$. Using (\ref{FFromC}), we can turn these into
bilinear products involving the surface-current expansion vectors
$\vb c$. For example, the power flux (\ref{PVMVP}) becomes 
%--------------------------------------------------------------------%
\begin{align}
 \vb P(\vb x) \cdot \vbhat{n}
&=\frac{1}{4}\bmc F^\dagger(\vb x) \bmc N\supt{P}(\vbhat{n}) \bmc F(\vb x)
\nn
&=\frac{1}{4}\sum_{\alpha \beta} 
  c_\alpha^* 
  \Big[ \bmc F^\dagger_\alpha(\vb x)
        \,
        \bmc N\supt{P}(\vbhat{n})
        \,
        \bmc F_\beta(\vb x)
  \Big]
  c_\beta 
\nn
&= \vb c^\dagger \vb M\supt{PFLUX}(\vb x, \vbhat{n}) \vb c 
\label{PFluxVMVP}
\end{align}
%--------------------------------------------------------------------%
where $\vb M\supt{PFLUX}(\vb x,\vbhat{n})$ is a matrix 
appropriate for $\vbhat{n}$-directed power flux in at $\vb x$.
The fluxes of $i$-directed linear and angular momentum read
similarly
%--------------------------------------------------------------------%
\begin{align}
\vb T_i(\vb x) \cdot \vbhat{n}
&= \frac{1}{4} \vb c^\dagger \vb M\IFFlux(\vb x, \vbhat{n}) \vb c 
\label{IFFluxVMVP}
\\
\vb t_i(\vb x) \cdot \vbhat{n}
&= \frac{1}{4} \vb c^\dagger \vb M\ITFlux(\vb x, \vbhat{n}) \vb c 
\label{ITFluxVMVP}
\end{align}
%--------------------------------------------------------------------%
The $\vb M$ matrices in (\ref{PFluxVMVP}), (\ref{IFFluxVMVP}), and 
(\ref{ITFluxVMVP}) are $N\subt{B}\times N\subt{B}$ matrices
whose entries are themselves 6-dimensional matrix-vector products:
\begin{align*}
M_{\alpha\beta}\supt{PFLUX}(\vb x, \vbhat{n}) 
 &= \frac{1}{4} 
    \bmc F^\dagger_\alpha(\vb x) 
    \bmc N\supt{P}(\vbhat{n})
    \bmc F_\beta(\vb x) 
\\
M_{\alpha\beta}\IFFlux(\vb x, \vbhat{n})
 &= \frac{1}{4} 
    \bmc F^\dagger_\alpha(\vb x) 
    \bmc N\supt{iF}(\vbhat{n})
    \bmc F_\beta(\vb x) 
\\
M_{\alpha\beta}\ITFlux(\vb x, \vbhat{n})
 &= \frac{1}{4} 
    \bmc F^\dagger_\alpha(\vb x) 
    \bmc N\supt{iT}(\vbhat{n})
    \bmc F_\beta(\vb x).
\end{align*}

%%%%%%%%%%%%%%%%%%%%%%%%%%%%%%%%%%%%%%%%%%%%%%%%%%%%%%%%%%%%%%%%%%%%%%
%%%%%%%%%%%%%%%%%%%%%%%%%%%%%%%%%%%%%%%%%%%%%%%%%%%%%%%%%%%%%%%%%%%%%%
%%%%%%%%%%%%%%%%%%%%%%%%%%%%%%%%%%%%%%%%%%%%%%%%%%%%%%%%%%%%%%%%%%%%%%
\subsection*{Energy and momentum transfer from surface-current bilinears}

Noting that the $\vb x$ and $\vbhat{n}$ dependence of
the flux expressions (\ref{PFluxVMVP}), (\ref{IFFluxVMVP}),
and (\ref{ITFluxVMVP}) is entirely contained in the $\vb M$
matrices, it is easy to integrate those expressions over
closed surfaces to obtain surface-current bilinears
expressing the total power or momentum transferred to 
bodies.
For example, the total power absorbed by material bodies
contained within a closed surface $\mc S$
is given by integrating the LHS of (\ref{PFluxVMVP})
over $\mc S$:
%====================================================================%
\begin{align}
  P_{\mc S}\sups{abs}&=\int_{\mc S} \vb P(\vb x) \cdot \vbhat{n} \, d\vb x
\intertext{with $\vbhat{n}$ taken to be the inward-pointing surface
           normal. Insert the RHS of (\ref{PFluxVMVP}) and pull 
           $\vb c^\dagger, \vb c$ outside the integral:}
             &=\vb c^\dagger \vb M\PS \vb c
\label{PTotVMVP}
\end{align}
where the elements of 
$\vb M\PS$
involve integrals over $\mc S$:
%====================================================================%
\numeq{MPSEntries}
{ M_{\alpha\beta}\PS
  = \frac{1}{4} \int_{\mc S} 
    \bmc F^\dagger_\alpha(\vb x) 
    \bmc N\supt{P}(\vbhat{n})
    \bmc F_\beta(\vb x) 
     \, d\vb x.
}
%====================================================================%
Similarly, the time-average $i$-directed force and torque on 
material bodies contained in $\mc S$ are
%====================================================================%
\begin{align}
 F_{i\mc S} &=\vb c^\dagger \vb M\IFS \vb c
\label{FTotVMVP}
\\
 \mc T_{i\mc S} &=\vb c^\dagger \vb M\ITS \vb c
\label{TTotVMVP}
\end{align}
%====================================================================%
where the entries of $\vb M\IFS$ and $\vb M\ITS$ are similar
to (\ref{MPSEntries}) with $\bmc N\supt{P} \to \bmc N\IF, \bmc N\IT.$

%%%%%%%%%%%%%%%%%%%%%%%%%%%%%%%%%%%%%%%%%%%%%%%%%%%%%%%%%%%%%%%%%%%%%%
%%%%%%%%%%%%%%%%%%%%%%%%%%%%%%%%%%%%%%%%%%%%%%%%%%%%%%%%%%%%%%%%%%%%%%
%%%%%%%%%%%%%%%%%%%%%%%%%%%%%%%%%%%%%%%%%%%%%%%%%%%%%%%%%%%%%%%%%%%%%%
\subsection*{Sparse alternatives to dense matrices}

For some quantities of interest it is possible to write alternative
surface-current bilinears that compute the same quantities as 
(\ref{PTotVMVP}), (\ref{FTotVMVP}), and (\ref{TTotVMVP})
but which involve \textit{sparse} matrices instead of dense
matrices. In particular, the (classical, deterministic) 
power, force, or torque on a body may be written as a bilinear 
function of the surface currents on that body alone in the form
\begin{align*} 
 P &= \frac{1}{4}vb c
\end{align*} 

where the $\vb O$ matrices are sparse ``overlap'' matrices
with nonzero entries only for pairs of basis functions
whose supports overlap. 

%%%%%%%%%%%%%%%%%%%%%%%%%%%%%%%%%%%%%%%%%%%%%%%%%%%%%%%%%%%%%%%%%%%%%%
%%%%%%%%%%%%%%%%%%%%%%%%%%%%%%%%%%%%%%%%%%%%%%%%%%%%%%%%%%%%%%%%%%%%%%
%%%%%%%%%%%%%%%%%%%%%%%%%%%%%%%%%%%%%%%%%%%%%%%%%%%%%%%%%%%%%%%%%%%%%%
\subsection{Statistical averages of surface-current bilinears}

The previous section discussed several examples of physical
quantities $Q$ which, in classical scattering problems with 
deterministic sources and induced surface currents, may be 
computed as bilinear functions of the surface currents,
$ Q(\vb c)=\vb c^\dagger \vb Q \vb c$. The heart of the FSC 
approach to nonequilibrium phenomena is a concise formula for 
the \textit{statistical average} of such surface-current bilinears,
where the averaging is performed over thermal and quantum-mechanical
fluctuations of sources

Suppose $Q(\vb c)=\vb c^\dagger \vb Q \vb c$ is a
physical quantity which---in a classical, deterministic setting---is 
computed as a bilinear function of the surface currents involving
a matrix $\vb Q$ [many examples of such quantities $Q$ and the 
corresponding matrices $\vb Q$ were discussed in the previous section.]

Then the statistical average of the contribution made by the sources 
in a body $b$ to $Q$---where the averaging is over quantum and thermal 
fluctuations of sources in the body---is given simply by
%--------------------------------------------------------------------%
\begin{align*}
 \big\langle Q\big\rangle
  &= \int_0^\infty \, \Theta(\omega) \Phi(\omega)\,d\omega 
\\
\Phi
  &= \Tr\left\{ \vb Q \vb W \overline{\vb G}_b \vb W^\dagger\right\}.
\end{align*}
%--------------------------------------------------------------------%
Here $\vb W$ is the inverse of the BEM matrix for entire system
and $\overline{\vb G}_b$ is a symmetrized version of a portion of the
BEM matrix for body $b$ alone.

%====================================================================%
\newpage
\bibliographystyle{ieeetr}
\bibliography{scuff-neq}

\end{document}
