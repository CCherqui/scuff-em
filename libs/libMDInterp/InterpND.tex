\documentclass{article}
%====================================================================%
\input{HRCommands}

\renewcommand{\vbsigma}{\boldsymbol{\sigma}}
\newcommand{\vbtau}{\boldsymbol{\tau}}
\graphicspath{{figures/}}
\begin{document}

Let $f(\vb x)$ be a scalar function of a $D$-dimensional
real variable $\vb x$. I 

\paragraph{Notation} In what follows, $\vb x=(x_1,\cdots,x\subt{D})$
is a $D$-dimensional real-valued vector and $\vb p=(p_1,\cdots,p\subt{D})$
is a $D$-tuple of integers. For monomials I use the notation
%====================================================================%
$$ \vb x^{\vb p} \equiv x_1^{p_1} x_2^{p_2} \cdots x\subt{D}^{p\subt{D}}. $$
%====================================================================%

\paragraph{$D$-dimensional interpolation grid}
Define a cartesian-product interpolation grid
with extents
$$X\sups{min}_d < x_d < X\sups{max}_d, \qquad d=1,2,\cdots,D.$$
Let $N_d$ be an integer and put
$$\Delta_d \equiv \frac{X\sups{max}_d - X\sups{min}_d}{N_d}$$
so the grid points in dimension $d$ are
$\{x_0,x_1,\cdots,x\subt{$N\subt{d}$}\}$ with
%====================================================================%
\renewcommand{\arraystretch}{1.5}
$$\begin{array}{|c|c|}\hline
  n_d & x_d                        \\\hline
  0   & X\sups{min}_d              \\\hline
  1   & X\sups{min}_d + \Delta_d   \\\hline 
  2   & X\sups{min}_d + 2\Delta_d  \\\hline
  \vdots & \vdots                  \\\hline
  N_d-1  & X\sups{min}_d + (N_d-1)\Delta_d  \\\hline
  N_d    & X\sups{max}_d \\\hline
  \end{array}
$$
\renewcommand{\arraystretch}{1.0}
$$\vb N=(N_1,\cdots,N\subt{D})$$

\paragraph{Indexing of grid points and cells}
Grid points are labeled by a $D$-tuple of integers:
$$\text{grid points:} \quad \vb n=(n_1,\cdots,n\subt{D}),
  \qquad 0 \le n_d \le N\subt{d}
$$
I identify each grid \text{cell} with its lower-left grid point.
$$\text{grid cells :} \quad \vb n=(n_1,\cdots,n\subt{D}),
  \qquad 0 \le n_d < N\subt{d}
$$

\paragraph{Polynomial interpolants}
Within the interior of each grid cell, I approximate $f$ by a polynomial
of degree $3$ in each of the variables:
%====================================================================%
$$ \text{in cell $\vb n$: \quad }
   f(\vb x) \approx P^{\vb n}(\vb x)
   \equiv \sum_{\vb p} C^{\vb n}_{\vb p} \vb x^{\vb p}
$$
%====================================================================%
where the sum is
%====================================================================%
$$ \sum_{\vb p} = \sum_{p\subt{D}=0}^3 \cdots \sum_{p_2=0}^3 \sum_{p_1=0}^3.$$
%====================================================================%

$$ f(\vb x) = \sum_{\vb p} C_{\vb p}\pf{\vb x}{\boldsymbol\Delta}^{\vb p} $$
$$ \left|
   \boldsymbol{\partial}^{\vbsigma}
   \right|_{\vb x = \vb x_{\vb n+\boldsymbol{\tau}}}
   f(\vb x)
   =\pf{2}{\boldsymbol{\Delta}}^{\vbsigma}\sum_{\vb p}
    C_{\vb p}^{\vb n}(\pm 1)
$$
The grid vertices at the corners of cell $\vb n$ may be
described as the set $\{\vb n + \vbtau\}$ where $\vbtau$
runs over $D$-tuples of integers with entries $0,1$.
Similarly, the set of all
mixed partial derivatives of degree at most 1 in any
variable may be written 
$$ 
 \Big\{
   \partial_{x_1}^{\sigma_1} \cdots \partial_{x\subt{D}}^{\sigma\subt{D}}
  f(x)
 \Big\}
  \equiv 
 \Big \{ \boldsymbol{\partial}^{\vbsigma} f(x) \Big \}
$$
where $\vbsigma$ runs over $D$-tuples of integers with entries 0,1.

%\paragraph{Determination of $\vb C_{\vb n}$ coefficients}

The number of coefficients per grid cell is $N\sups{coeff}\equiv 4\supt{D}$.
I determine these by matching the value and all partial
derivatives (of degree at most 1 in each variable) of $P^{\vb n}(\vb x)$
to those of $f(\vb x)$ at each of the grid points bordering cell $\vb{n}$.
This gives $2^D$ matching conditions for each of the $2^D$ grid points,
or $4^D$ total equations for the $4^D$ coefficients in cell $\vb n$.

\end{document}
