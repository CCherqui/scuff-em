%%%%%%%%%%%%%%%%%%%%%%%%%%%%%%%%%%%%%%%%%%%%%%%%%%%%%%%%%%%%%%%%%%%%%%
%%%%%%%%%%%%%%%%%%%%%%%%%%%%%%%%%%%%%%%%%%%%%%%%%%%%%%%%%%%%%%%%%%%%%%
%%%%%%%%%%%%%%%%%%%%%%%%%%%%%%%%%%%%%%%%%%%%%%%%%%%%%%%%%%%%%%%%%%%%%%
\newpage
\section{Equivalence-principle PFT (EPPFT)}

In the EPPFT approach, we compute the power, force and torque
on a body by considering the work done on, and the force and 
torque exerted on, the equivalent surface currents flowing on 
the body surface.

\subsection*{Work and force on a volume current distribution}

To motivate these formulas, first recall that, if we had a
\textit{volume} distribution of electric current $\vb J$ 
throughout a body $\mc B$, we could compute the power absorbed
by the body as the time-average work done on the volume currents
inside it via Joule heating:
%====================================================================%
\begin{subequations}
\begin{align}
 P\sups{abs} 
     &= \frac{1}{2}\text{ Re }
      \int_{\mc B} \vb J^*(\vb x) \cdot \vb E(\vb x)
      \,dV
%--------------------------------------------------------------------%
\intertext{Similarly, the $i$-directed force on the body follows
           from considering the Lorentz force on the charges and
           currents contained in an infinitesimal volume $dV:$}
%--------------------------------------------------------------------%
 F_i &= \frac{1}{2}\text{ Re }
        \int_{\mc B} \bigg\{ \rho(\vb x) E_i(\vb x)
                    +\frac{1}{\mu_0}
                     \Big[\vb J(\vb x) \cdot \vb H(\vb x)\Big]_i
             \bigg\} dV
\intertext{With some effort (Appendix \ref{ForceFormulaAppendix}),
           this formula may be rewritten in a form closely
           analogous to that of (\ref{VolumePFT}a):}
     &= \frac{1}{2\omega}\text{ Im }
        \int_{\mc B} \vb J^*(\vb x) \cdot \partial_i \vb E(\vb x)
                     dV.
\intertext{Similarly (Appendix \ref{ForceFormulaAppendix}), the 
           $i$-directed \textit{torque} on the body is}
     &= \frac{1}{2\omega}\text{ Im }
        \int_{\mc B} \vb J^*(\vb x) \cdot \partial_\theta \vb E(\vb x)
                     dV
\end{align}
\label{VolumePFT}
\end{subequations}
where $\partial_\theta \vb E$ denotes differentiation 
with respect to rotations about the $i$th
coordinate axis.
%====================================================================%

\subsection*{Work on a surface current distribution}

For any volume electric current distribution $\vb J$ throughout
the bulk of a compact body, there are equivalent electric and 
magnetic surface current distributions $\{\vb K, \vb N\}$ confined
to the surface of the body that produce the same fields as does
$\vb J$ both inside and outside the body. By considering
the work done and the force or torque exterted on these 
equivalent currents, we obtain surface-integral analogues 
of the volume-integral formulas (\ref{VolumePFT}). 

One complication here that is not present in the volume-integral
case is that the presence of magnetic currents complicates the
situation by introducing new terms in the force and torque that 
are not present in cases where only electric currents are present.

%====================================================================%
\begin{subequations}
\begin{align}
 P\sups{abs} 
     &= \frac{1}{2}\text{ Re }
      \int_{\partial \mc B} \Big[ \vb K^*(\vb x) \cdot \vb E(\vb x)
                                 +\vb N^*(\vb x) \cdot \vb H(\vb x)
                            \Big]
      \,dA
\nn
    &= \frac{1}{2}\text{ Re }
      \int_{\partial \mc B} \bmc C^*(\vb x) \cdot \bmc F(\vb x) \,dA
\\
 F_i
     &= \frac{1}{2\omega}\text{ Im }
      \int_{\partial \mc B} 
       \bigg[  \vb K^*(\vb x) \cdot \partial_i \vb E(\vb x)
              +\vb N^*(\vb x) \cdot \partial_i \vb H(\vb x)
\nn
&
 \hspace{1.2in} +(\vb N^*\times \vb K)_i - (\vb K^*\times \vb N)_i
       \bigg]
      \,dA
\\
    &= \frac{1}{2\omega }\text{ Im }
        \int_{\partial \mc B} \bmc C^*(\vb x) \cdot \partial_i \bmc F(\vb x) \,dA
\nn
&\hspace{1.0in}
      -\frac{1}{2\omega }\text{ Im }
        \int_{\partial \mc B} 
        \bmc C^*(\vb x) 
        \bmc{O}^{(\times)}(\vb x,\vb x^\prime) 
        \bmc C(\vb x^\prime)
        \,dA
\end{align}
\label{EPPFT0}
\end{subequations}
where 
%====================================================================%
$$ \bmc{O}^{(\times)}(\vb x,\vb x^\prime)
  = \left(\begin{array}{cc}
    0 & -\boldsymbol{\varepsilon}_{ijk} \\ 
     \boldsymbol{\varepsilon}_{ijk} & 0
   \end{array}\right)
   \delta(\vb x-\vb x^\prime).
$$
%====================================================================%
Note that these formulas only reference the tangential components
of $\vb E$ and $\vb H$, which are continuous across the body 
surface; thus it matters not whether we evaluate $\vb E$ and $\vb H$
using the computational recipes for the fields inside or outside
the surface (these recipes, of course, differ in SIE formulations).
It is easiest to use the inside recipe, in which case the fields
are a convolution over just the surface currents on the surface 
of $\mc B$:
%====================================================================%
$$ \bmc F(\vb x) 
   = \int_{\partial \mc B} \bmc G^{(\mc B)}(\vb x, \vb x^\prime) \cdot \bmc C(\vb x^\prime) 
   d\vb x^\prime
$$ 
%====================================================================%
where $\bmc G^{(\mc B)}$ is the $6\times 6$ dyadic Green's function
for the material interior to $\mc B$. The PFT formulas (\ref{EPPFT0}) 
then become double surface integrals,
%====================================================================%
\begin{subequations}
\begin{align}
 P\sups{abs} 
     &= \frac{1}{2}\text{ Re }
      \iint_{\partial \mc B} 
      \bmc C^*(\vb x) \bmc G^{(\mc B)}(\vb x, \vb x^\prime)
      \bmc C(\vb x^\prime)
      \,d\vb x\, d\vb x^\prime
\\
 F_i\sups{abs} 
     &= \frac{1}{2\omega}\text{ Im }
      \iint_{\partial \mc B}
      \bmc C^*(\vb x) \partial_i \bmc G^{(\mc B)}(\vb x, \vb x^\prime)
      \bmc C(\vb x^\prime)
      \,d\vb x \,d\vb x^\prime
\\
 \tau_i\sups{abs} 
     &= \frac{1}{2\omega}\text{ Im }
      \iint_{\partial \mc B}
      \bmc C^*(\vb x) \partial_\theta \bmc G^{(\mc B)}(\vb x, \vb x^\prime)
      \bmc C(\vb x^\prime)
      \,d\vb x\, d\vb x^\prime
\end{align}
\label{EPPFT1}
\end{subequations}
%====================================================================%
or, upon discretization,
%====================================================================%
\begin{subequations}
\begin{align}
 P\sups{abs} 
     &= \frac{1}{2}\text{ Re }
      \vb c^\dagger \vb M^{(\mc B)} \vb c
\\
 F_i\sups{abs} 
     &= \frac{1}{2\omega}\text{ Im }
      \vb c^\dagger \partial_i \vb M^{(\mc B)} \vb c
\\
 \tau_i\sups{abs} 
     &= \frac{1}{2\omega}\text{ Im }
      \vb c^\dagger \partial_\theta \vb M^{(\mc B)} \vb c.
\end{align}
\label{EPPFT}
\end{subequations}
%====================================================================%
