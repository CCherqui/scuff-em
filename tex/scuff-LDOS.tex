\documentclass[letterpaper]{article}
\usepackage[square,sort,comma,numbers]{natbib}
\newcommand{\citeasnoun}[1]{Ref.~\citenum{#1}}

../../../../tex/scufftex.tex

\newcommand\supsstar[1]{^{\hbox{\scriptsize{#1}}*}}
\newcommand\suptstar[1]{^{\hbox{\scriptsize{#1}}*}}
\newcommand{\IF}{^{i\text{\scriptsize F}}}
\newcommand{\IFFlux}{^{i\text{\tiny FFLUX}}}
\newcommand{\IT}{^{i\text{\scriptsize T}}}
\newcommand{\ITFlux}{^{i\text{\tiny TFLUX}}}
\newcommand{\PS}{^{\text{\scriptsize P}\mc S}}
\newcommand{\IFS}{^{i\text{\scriptsize F}\mc S}}
\newcommand{\ITS}{^{i\text{\scriptsize T}\mc S}}
%\newcommand{\vbchi}{\boldsymbol{\chi}}
\newcommand{\vbGamma}{\boldsymbol{\Gamma}}


\graphicspath{{figures/}}

%------------------------------------------------------------
%------------------------------------------------------------
%- Special commands for this document -----------------------
%------------------------------------------------------------
%------------------------------------------------------------

%------------------------------------------------------------
%------------------------------------------------------------
%- Document header  -----------------------------------------
%------------------------------------------------------------
%------------------------------------------------------------
\title {Computation of Green's Functions and LDOS in {\sc scuff-em}}
\author {Homer Reid}
\date {September 27, 2014}

%------------------------------------------------------------
%------------------------------------------------------------
%- Start of actual document
%------------------------------------------------------------
%------------------------------------------------------------

\begin{document}
\pagestyle{myheadings}
\markright{Homer Reid: Periodic GF and LDOS computations in {\sc scuff-em}}
\maketitle

\tableofcontents

%%%%%%%%%%%%%%%%%%%%%%%%%%%%%%%%%%%%%%%%%%%%%%%%%%%%%%%%%%%%%%%%%%%%%%
%%%%%%%%%%%%%%%%%%%%%%%%%%%%%%%%%%%%%%%%%%%%%%%%%%%%%%%%%%%%%%%%%%%%%%
%%%%%%%%%%%%%%%%%%%%%%%%%%%%%%%%%%%%%%%%%%%%%%%%%%%%%%%%%%%%%%%%%%%%%%
\newpage
\section{Green's functions and LDOS in the non-periodic case}

The scattering parts of the electric and magnetic
dyadic Green's functions of a geometry are defined by
%====================================================================%
\begin{align*}
 \mc G_{ij}\sups{E}(\omega; \vb x, \vb x^\prime)
   \equiv
   \frac{1}{ikZ_0 Z^r}
  &\left( \parbox{0.65\textwidth}
    { $i$-component of scattered $\vb E$-field at $\vb x$
      due to a unit-strength $j$-directed point \textbf{electric} 
      dipole radiator
      at $\vb x^\prime$, all quantities having time dependence
      $\sim e^{-i\omega t}$
    }
   \right)
\\[5pt]
%%--------------------------------------------------------------------%
 \mc G_{ij}\sups{M}(\omega; \vb x, \vb x^\prime)
   \equiv
   \frac{1}{ik}
  &\left( \parbox{0.65\textwidth}
    { $i$-component of scattered $\vb H$-field at $\vb x$
      due to a unit-strength $j$-directed point \textbf{magnetic}
      dipole radiator
      at $\vb x^\prime$, all quantities having time dependence
      $\sim e^{-i\omega t}$
    }
   \right)
\end{align*}
%====================================================================%
Here $k=\sqrt{\epsilon^r \mu^r }\cdot \omega$ and 
$Z^r=\sqrt{\mu^r /\epsilon^r }$ are the wavenumber and 
relative wave impedance of the material medium in which 
point $\vb x$ resides ($\epsilon^r ,\mu^r$ are its relative 
permittivity and permeability) and $Z_0\approx 377\,\Omega$  
is the impedance of vacuum. The prefactors 
$\frac{1}{ikZ_0Z^r}$ and $\frac{1}{ik}$ are inserted to 
ensure that $\bmc G\supt{E,M}$ have dimensions of inverse
length.

The enhancement of the local density of states (LDOS)
at frequency $\omega$ and at a point $\vb x$ in a 
scattering geometry is related to the scattering DGFS according 
to\footnote{K Joulain et al.,``Definition and measurement of the local 
density of electromagnetic states close to an interface,''
Physical Review B \textbf{68} 245405 (2003)}
%====================================================================%
$$ \frac{\rho(\omega; \vb x}{\rho_0(\omega)}
   \texttt{LDOS}(\omega; \vb x) 
   = 
   \frac{\pi}{k_0^2}\text{Tr }\text{Im }
   \Big[ \bmc G\supt{E}(\omega; \vb x, \vb x)
        +\bmc G\supt{M}(\omega; \vb x, \vb x)
   \Big]
$$
%====================================================================%
where $\rho_0(\omega)\equiv k^3_0/(\pi c)$ is the free-space
LDOS and $k_0=\omega/c$ is the free-space wavenumber at the
frequency in question.

In {\sc scuff-em} the dyadic GFs may be computed easily by solving a
scattering problem in which the incident fields arise from a point dipole
radiator at some source point $\vb x\supt{S}$. For example, to compute 
$\mc G\supt{E}$ we take the incident fields to be the fields of a 
unit-strength $j$-directed point electric dipole source:
%%%%%%%%%%%%%%%%%%%%%%%%%%%%%%%%%%%%%%%%%%%%%%%%%%%%%%%%%%%%%%%%%%%%%%
\begin{subequations}
\begin{align}
 \vb E\sups{inc}(\vb x) &= \vb E\supt{PD$j$}(\vb p, \vb x\supt{S}; \vb x)
\\
 \vb H\sups{inc}(\vb x) &= \vb H\supt{PD$j$}(\vb p, \vb x\supt{S}; \vb x)
\end{align}
\label{EHPD}
\end{subequations}
%%%%%%%%%%%%%%%%%%%%%%%%%%%%%%%%%%%%%%%%%%%%%%%%%%%%%%%%%%%%%%%%%%%%%%
where the fields of a unit-strength point electric dipole at $x\supt{S}$
have the components
%%%%%%%%%%%%%%%%%%%%%%%%%%%%%%%%%%%%%%%%%%%%%%%%%%%%%%%%%%%%%%%%%%%%%%
\begin{subequations}
\begin{align*}
 E_i\supt{PD$j$}(\vb x) &= ikZ_0 Z_r \vb G_{ij}(\vb x, \vb x_s) \\
 H_i\supt{PD$j$}(\vb x) &= -ik       \vb C_{ij}(\vb x, \vb x_s).
\end{align*}
\label{PDFields}
\end{subequations}
%%%%%%%%%%%%%%%%%%%%%%%%%%%%%%%%%%%%%%%%%%%%%%%%%%%%%%%%%%%%%%%%%%%%%%
(The ``PD'' superscript stands for ``point dipole.'' The $\vb G$
and $\vb C$ tensors are defined in my document \texttt{lsInnards.pdf}.)
Then we simply solve an ordinary {\sc scuff-em} scattering
problem with the incident fields given by equation 
(\ref{EHPD}) and compute the scattered---not total!---fields
at the evaluation point $\vb x\supt{D}$. The three components of the
$\vb E$-field at $\vb x\supt{D}$, divided by $ikZ_0 Z_r$, yield one 
full column of the $j$th row of the 
$3\times 3$ matrix $\bmc G\sups{E}(\omega; \vb x\supt{D}, \vb x\supt{S}.)$
Calculating $\bmc G\sups{M}$ is similar except that we use 
a point magnetic source to supply the incident field 
and compute the scattered $\vb H$ field instead of the 
scattered $\vb E$ field.

%%%%%%%%%%%%%%%%%%%%%%%%%%%%%%%%%%%%%%%%%%%%%%%%%%%%%%%%%%%%%%%%%%%%%%
%%%%%%%%%%%%%%%%%%%%%%%%%%%%%%%%%%%%%%%%%%%%%%%%%%%%%%%%%%%%%%%%%%%%%%
%%%%%%%%%%%%%%%%%%%%%%%%%%%%%%%%%%%%%%%%%%%%%%%%%%%%%%%%%%%%%%%%%%%%%%
\newpage
\section{Extension to the periodic case}

In the Bloch-periodic module of {\sc scuff-em}, \textit{all}
fields and currents are assumed to be Bloch-periodic, i.e.
if $Q(\vb x)$ denotes any field or current component at $\vb x$,
then we have the built-in assumption
%====================================================================%
\numeq{BlochCondition}
{Q(\vb x + \vb L) = e^{i\vb k\subt{B} \cdot \vb L}Q(\vb x)}
%====================================================================%
where $\vb L$ is any lattice vector and 
$\vb k\subt{B}$ is the Bloch wavevector.

The fields of a point dipole, equation (\ref{PDFields}), do \textit{not}
satisfy (\ref{BlochCondition}), and hence may not be used in
Bloch-periodic {\sc scuff-em} calculations. Instead, what we can 
simulate in the periodic case are the fields of an infinite
phased \textit{array} of point dipoles,
%%%%%%%%%%%%%%%%%%%%%%%%%%%%%%%%%%%%%%%%%%%%%%%%%%%%%%%%%%%%%%%%%%%%%%
\begin{subequations}
\begin{align*}
 \vb E\supt{PDA}(\vb p, \vb x_0, \vb k\subt{B}; \vb x)
&=\sum_{\vb L} e^{i\vb k\subt{B}\cdot \vb L}
  \vb E\supt{PD}(\vb p, \vb x_0 + \vb L; \vb x)
\\
 \vb H\supt{PDA}(\vb p, \vb x_0, \vb k\subt{B}; \vb x)
&=\sum_{\vb L} e^{i\vb k\subt{B} \cdot \vb L}
  \vb H\supt{PD}(\vb p, \vb x_0  + \vb L; \vb x)
\end{align*}
\label{EHPDA}
\end{subequations}
%%%%%%%%%%%%%%%%%%%%%%%%%%%%%%%%%%%%%%%%%%%%%%%%%%%%%%%%%%%%%%%%%%%%%%
(where ``PDA'' stands for ``point dipole array''). The quantity
we can compute

The inversion of equations (\ref{EHPDA}) recovers the fields
of a point source as an integral over the Brillouin zone of the 
lattice:\footnote{To derive these equations, multiply both sides
of (\ref{EHPDA}) by $e^{-i\vb k\subt{B} \cdot \vb L^\prime}$,
integrate both sides over the Brillouin zone, and use the
condition 
$$\int\subt{BZ} e^{i\vb k\subt{B}\cdot (\vb L-\vb L^\prime)}\,d\vb k
  =\mc{V}\subt{BZ} \, \delta(\vb L,\vb L^\prime)
$$
where $\mc V\subt{BZ}$ is the Brillouin-zone volume [for example,
a square lattice with basis vectors
$\{\vb L_1, \vb L_2\}=\{L_x\vbhat{x}, L_y\vbhat{y}\}$ has
$\mc V\subt{BZ}=4\pi^2/(L_x L_y)$].
Setting $\vb L^\prime=0$ recovers (\ref{EHPDAInverse}).}
%%%%%%%%%%%%%%%%%%%%%%%%%%%%%%%%%%%%%%%%%%%%%%%%%%%%%%%%%%%%%%%%%%%%%%
\begin{subequations}
\begin{align}
  \vb E\supt{PD}(\vb p, \vb x_0; \vb x)
&=\frac{1}{\mc V\subt{BZ}} 
   \int\subt{BZ} \vb E\supt{PDA}(\vb p, \vb x_0, \vb k\subt{B}; \vb x) 
   d\vb k\subt{B}
\\
  \vb H\supt{PD}(\vb p, \vb x_0; \vb x)
&=\frac{1}{\mc V\subt{BZ}} 
   \int\subt{BZ} \vb H\supt{PDA}(\vb p, \vb x_0, \vb k\subt{B}; \vb x) 
   d\vb k\subt{B}
\end{align}
\label{EHPDAInverse}
\end{subequations}
%%%%%%%%%%%%%%%%%%%%%%%%%%%%%%%%%%%%%%%%%%%%%%%%%%%%%%%%%%%%%%%%%%%%%%

%%%%%%%%%%%%%%%%%%%%%%%%%%%%%%%%%%%%%%%%%%%%%%%%%%%%%%%%%%%%%%%%%%%%%%
%%%%%%%%%%%%%%%%%%%%%%%%%%%%%%%%%%%%%%%%%%%%%%%%%%%%%%%%%%%%%%%%%%%%%%
%%%%%%%%%%%%%%%%%%%%%%%%%%%%%%%%%%%%%%%%%%%%%%%%%%%%%%%%%%%%%%%%%%%%%%
\newpage
\section{Example of a Brillouin-zone integral}

As a simple example of an application of Brillouin-zone integration,
let's consider the problem of computing the field of an
\textit{individual} point source by summing the fields of
many appropriately-phased point source \textit{arrays}.
This will give us some insight into how many cubature points we 
may expect

%%%%%%%%%%%%%%%%%%%%%%%%%%%%%%%%%%%%%%%%%%%%%%%%%%%%%%%%%%%%%%%%%%%%%%
%%%%%%%%%%%%%%%%%%%%%%%%%%%%%%%%%%%%%%%%%%%%%%%%%%%%%%%%%%%%%%%%%%%%%%
%%%%%%%%%%%%%%%%%%%%%%%%%%%%%%%%%%%%%%%%%%%%%%%%%%%%%%%%%%%%%%%%%%%%%%
\newpage
\section{Test: LDOS above an infinite half-space}

At a height $z$ above an infinite material half-space,
the enhancement of the electric LDOS may be expressed
as an integral over a dimensionless parameter 
$\kappa$:\footnote{K. Joulain et al.,``Definition and 
measurement of the local density of electromagnetic states
 close to an interface,'' Physical Review B \textbf{68} 245405 (2003)}
%====================================================================%
\numeq{JoulainLDOS}
{
 \frac{\rho\supt{E}(z; \omega)}{\rho_0(\omega)}
  =\frac{1}{4}\int_0^\infty \kappa F(\kappa) \, d\kappa,
}
%====================================================================%
\numeq{JoulainFKappa}
{
  F(\kappa)=
   \frac{1}{\kappa_z}
   \begin{cases}
         2 + \text{Re }\Big( r^s e^{2 i k_z z} \Big)
           + \Big(2\kappa^2-1\Big) \text{Re }\Big( r^p e^{2ik_z z}\Big)
         , \qquad &\kappa < 1 \\[10pt]
    \Big[   \text{Im }\big( r^s \big)
           + \big(2\kappa^2-1\big) \text{Im }\big( r^p \Big)
    \Big]e^{-2k_z z}, \qquad &\kappa > 1
   \end{cases}
}
%====================================================================%
where 
%====================================================================%
$$ \kappa_z \equiv 
   \begin{cases}
     \sqrt{1-\kappa^2}, \qquad &\kappa < 1, \\
     \sqrt{\kappa^2-1}, \qquad &\kappa > 1, 
   \end{cases}
   \qquad
   k_z \equiv k_0\kappa_z,
   \qquad
   k_0 \equiv \frac{\omega}{c},
$$  
%====================================================================%
$$ r_s =,
   \qquad 
   r_p =
   .
$$
%====================================================================%
The integral in (\ref{JoulainLDOS}) is computing the LDOS as a trace in 
the plane-wave basis; the integral over $\kappa$ is essentially a sum over 
all possible plane-wave wavevectors $\vb k$, and the multiple terms in 
(\ref{JoulainFKappa}) account for the two possible polarizations for each 
wavevector. More specifically, if we think in terms of the usual 
Fresnel-scattering picture of a plane wave incident on a planar interface, 
then $\kappa$ is the sine of the incident angle and $\kappa_z$ is its 
cosine, i.e. the wavevector of the plane wave 
is %====================================================================
$$ \vb k = (k_x, k_y, k_z) = k_0(\sin\theta, 0, \cos\theta)
   =k_0(\kappa, 0, \kappa_z)
$$
%====================================================================
and $r^s$ and $r^p$ are the Fresnel coefficients for reflection
of TE and TM waves from the half-space. 
This interpretation applies to the $0\le \kappa<1$ contribution to 
the integral in (\ref{JoulainLDOS}). 

However, the trace in equation (\ref{JoulainLDOS}) also contains
contributions from modes that cannot be excited by irradiating
the interface from the outside with a plane wave (corresponding
to the $\kappa>1$ portion of the integral). These are 
\textit{evanescent} waves for which the in-plane wavenumber
$k_\parallel=\kappa k_0$ is greater than the free-space wavenumber 
$k_0$. The fields associated with these modes decay exponentially
as we move away from the half-space into the vacuum region. 

\subsection*{Expression as a Brillouin-zone integral}

To write equation (\ref{JoulainLDOS}) in a form that facilitates
comparison with {\sc scuff-ldos} calcuations, I first interpret
the integral over $\kappa$ as an integral over the upper-right
quadrant of the $\vb k_\parallel=(k_x,k_y)$ plane; here
$\kappa$ is the radial component of $\vb k_\parallel$ measured in
units of $k_0$, i.e. $\kappa=\sqrt{k_x^2 + k_y^2}/k_0=k_\rho/k_0,$ 
and the integrand in (\ref{JoulainLDOS}) is independent of the 
angle $k_\theta=\atan(k_y/k_x).$ Then I have
%====================================================================%
\begin{align*}
 \frac{\rho\supt{E}(z; \omega)}{\rho_0(\omega)}
  &=\frac{1}{4}\int_0^\infty \kappa F(\kappa) \, d\kappa
\\
  &=\frac{1}{2\pi k_0^2}
    \int_0^\infty k_\rho dk_\rho \int_0^{\pi/2} \, dk_\theta F\pf{k_\rho}{k_0}
\\
  &=\frac{1}{2\pi k_0^2}
    \int_0^\infty dk_x \int_0^\infty dk_y \, F(k_x, k_y)
\end{align*}
%%%%%%%%%%%%%%%%%%%%%%%%%%%%%%%%%%%%%%%%%%%%%%%%%%%%%%%%%%%%%%%%%%%%%%
where 
%%%%%%%%%%%%%%%%%%%%%%%%%%%%%%%%%%%%%%%%%%%%%%%%%%%%%%%%%%%%%%%%%%%%%%
$$ F(k_x, k_y)=\pf{\sqrt{k_x^2 + k_y^2}}{k_0}.$$
%%%%%%%%%%%%%%%%%%%%%%%%%%%%%%%%%%%%%%%%%%%%%%%%%%%%%%%%%%%%%%%%%%%%%%
I now rewrite the integral over the infinite upper-right
quadrant of the $(k_x,k_y)$ plane as an integral over a
\textit{finite} rectangle of that plane with side lengths 
$\{\Gamma_x, \Gamma_y\}$: the integrand now consists of an 
infinite sum of terms: 
%%%%%%%%%%%%%%%%%%%%%%%%%%%%%%%%%%%%%%%%%%%%%%%%%%%%%%%%%%%%%%%%%%%%%%
\numeq{JoulainFolded}
{
    \frac{\rho\supt{E}(z; \omega)}{\rho_0(\omega)}
  = \frac{1}{2\pi k_0^2}
    \int_0^{\Gamma_x} \, dk_x \, \int_0^{\Gamma_y} \, dk_y \,
    \overline{F}(k_x, k_y),
}
$$
    \overline{F}(k_x, k_y)=\sum_{n_x,n_y=0}^\infty 
              F\Big( k_x + n_x \Gamma_x, k_y + n_y\Gamma_y \Big).
$$
%%%%%%%%%%%%%%%%%%%%%%%%%%%%%%%%%%%%%%%%%%%%%%%%%%%%%%%%%%%%%%%%%%%%%%%

%%%%%%%%%%%%%%%%%%%%%%%%%%%%%%%%%%%%%%%%%%%%%%%%%%%%%%%%%%%%%%%%%%%%%%
%%%%%%%%%%%%%%%%%%%%%%%%%%%%%%%%%%%%%%%%%%%%%%%%%%%%%%%%%%%%%%%%%%%%%%
%%%%%%%%%%%%%%%%%%%%%%%%%%%%%%%%%%%%%%%%%%%%%%%%%%%%%%%%%%%%%%%%%%%%%%
\newpage
\section{Technical implementation details}

\subsection*{Computing the fields of a point dipole array}

%The {\sc libincfield} module in {\sc scuff-em} contains a
%built-in routine (in the \texttt{PointSource} class)
%for computing the fields of a single point dipole, but

First consider a single point dipole radiator (not an array) 
at $\vb x_0$ with dipole moment $\vb p_0$ 
(as usual suppressing time-dependence factors
of $e^{-i\omega t}$). Thinking of this as an electric
current distribution of the form
%====================================================================%
$$ \vb J(\vb x) = -i\omega \vb p \delta(\vb x-\vb x_0) $$
%====================================================================%
it is easy to compute the fields by convolution with the usual
Green's functions relating currents to fields:
%====================================================================%
\begin{align*}
 E_i(\vb x) 
&= 
 \int \Gamma_{ij}\supt{EE}(\vb x, \vb x^\prime) J_j(\vb x^\prime) d\vb x^\prime
\\
&=-i\omega \Gamma_{ij}\supt{EE}(\vb x, \vb x_0) p_j
\\
&=(-i\omega)(ikZ_0Z^r) G_{ij}(\vb x, \vb x_0) p_j
\\[8pt]
H_i(\vb x) 
 &=
\int \Gamma_{ij}\supt{ME}(\vb x, \vb x^\prime) J_j(\vb x^\prime) d\vb x^\prime
\\
&=-i\omega \Gamma_{ij}\supt{ME}(\vb x, \vb x_0) p_j
\\
&=(-i\omega)(ik) C_{ij}(\vb x, \vb x_0) p_j
\end{align*}
%====================================================================%
where the $\vb G$ and $\vb C$ dyadics are related to the
scalar Helmholtz Green's function according to
%====================================================================%
$$ G_{ij}(\vb r)
   = \Big[ \delta_{ij} +\frac{1}{k^2} \partial_i \partial_j \Big] G_0(\vb r),
\qquad
   C_{ij}(\vb r)
   = \frac{1}{ik}\varepsilon_{ijk} \partial_k G_0(\vb r).
$$
%====================================================================%

\subsection*{Evaluation of BZ integrals}

For a 2D square lattice with lattice vectors 
$\vb L_1=L_x\vbhat{x}, \vb L_2=L_y\vbhat{x},$
a set of reciprocal-lattice basis vectors is 
$\vbGamma_1=\pf{2\pi}{L_x} \vbhat{x},
 \vbGamma_2=\pf{2\pi}{L_y} \vbhat{y}$,
and Brillouin-zone integrals take the form
%%%%%%%%%%%%%%%%%%%%%%%%%%%%%%%%%%%%%%%%%%%%%%%%%%%%%%%%%%%%%%%%%%%%%%
$$ \frac{1}{\mc V\subt{BZ}} 
   \int\subt{BZ} f(\vb k\subs{B}) \, d \vb k\subt{B}
  =4\int_0^{1/2} \, d u_1 \, \int_0^{1/2} \, d u_2 \, 
   f\Big( u_1\vbGamma_1 + u_y\vbGamma_2 \Big)
$$
%%%%%%%%%%%%%%%%%%%%%%%%%%%%%%%%%%%%%%%%%%%%%%%%%%%%%%%%%%%%%%%%%%%%%%

\end{document}
