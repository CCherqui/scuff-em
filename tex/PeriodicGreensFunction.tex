\documentclass[letterpaper]{article}
\usepackage[square,sort,comma,numbers]{natbib}
\newcommand{\citeasnoun}[1]{Ref.~\citenum{#1}}

../../../../tex/scufftex.tex

\newcommand\supsstar[1]{^{\hbox{\scriptsize{#1}}*}}
\newcommand\suptstar[1]{^{\hbox{\scriptsize{#1}}*}}
\newcommand{\IF}{^{i\text{\scriptsize F}}}
\newcommand{\IFFlux}{^{i\text{\tiny FFLUX}}}
\newcommand{\IT}{^{i\text{\scriptsize T}}}
\newcommand{\ITFlux}{^{i\text{\tiny TFLUX}}}
\newcommand{\PS}{^{\text{\scriptsize P}\mc S}}
\newcommand{\IFS}{^{i\text{\scriptsize F}\mc S}}
\newcommand{\ITS}{^{i\text{\scriptsize T}\mc S}}
%\newcommand{\vbchi}{\boldsymbol{\chi}}


\graphicspath{{figures/}}

%------------------------------------------------------------
%------------------------------------------------------------
%- Special commands for this document -----------------------
%------------------------------------------------------------
%------------------------------------------------------------

%------------------------------------------------------------
%------------------------------------------------------------
%- Document header  -----------------------------------------
%------------------------------------------------------------
%------------------------------------------------------------
\title {Computation of Green's Functions and LDOS in {\sc scuff-em}}
\author {Homer Reid}
\date {September 27, 2014}

%------------------------------------------------------------
%------------------------------------------------------------
%- Start of actual document
%------------------------------------------------------------
%------------------------------------------------------------

\begin{document}
\pagestyle{myheadings}
\markright{Homer Reid: Periodic GF computations in {\sc scuff-em}}
\maketitle

\tableofcontents

%%%%%%%%%%%%%%%%%%%%%%%%%%%%%%%%%%%%%%%%%%%%%%%%%%%%%%%%%%%%%%%%%%%%%%
%%%%%%%%%%%%%%%%%%%%%%%%%%%%%%%%%%%%%%%%%%%%%%%%%%%%%%%%%%%%%%%%%%%%%%
%%%%%%%%%%%%%%%%%%%%%%%%%%%%%%%%%%%%%%%%%%%%%%%%%%%%%%%%%%%%%%%%%%%%%%
\section{Green's functions and LDOS in the non-periodic case}

The scattering parts of the electric and magnetic
dyadic Green's functions of a geometry are defined by
%====================================================================%
\begin{align*}
 \mc G_{ij}\sups{E}(\omega; \vb x, \vb x^\prime)
   \equiv
   \frac{1}{ikZ_0 Z^r}
  &\left( \parbox{0.65\textwidth}
    { $i$-component of scattered $\vb E$-field at $\vb x$
      due to a unit-strength $j$-directed point \textbf{electric} 
      dipole radiator
      at $\vb x^\prime$, all quantities having time dependence
      $\sim e^{-i\omega t}$
    }
   \right)
\\[5pt]
%%--------------------------------------------------------------------%
 \mc G_{ij}\sups{M}(\omega; \vb x, \vb x^\prime)
   \equiv
   \frac{1}{ik}
  &\left( \parbox{0.65\textwidth}
    { $i$-component of scattered $\vb H$-field at $\vb x$
      due to a unit-strength $j$-directed point \textbf{magnetic}
      dipole radiator
      at $\vb x^\prime$, all quantities having time dependence
      $\sim e^{-i\omega t}$
    }
   \right)
\end{align*}
%====================================================================%
Here $k=\sqrt{\epsilon^r \mu^r }\cdot \omega$ and 
$Z^r=\sqrt{\mu^r /\epsilon^r }$ are the wavenumber and 
relative wave impedance of the material medium in which 
point $\vb x$ resides ($\epsilon^r ,\mu^r$ are its relative 
permittivity and permeability) and $Z_0\approx 377\,\Omega$  
is the impedance of vacuum. The prefactors 
$\frac{1}{ikZ_0Z^r}$ and $\frac{1}{ik}$ are inserted to 
ensure that $\bmc G\supt{E,M}$ have dimensions of inverse
length.
%====================================================================%

In {\sc scuff-em} the dyadic GFs may be computed easily by solving a
scattering problem in which the incident fields arise from a point dipole
radiator at $\vb x_0$. For example, to compute $\mc G\supt{E}$ we 
take the incident fields to be
%%%%%%%%%%%%%%%%%%%%%%%%%%%%%%%%%%%%%%%%%%%%%%%%%%%%%%%%%%%%%%%%%%%%%%
\begin{align*}
 \vb E\sups{inc}(\vb x) &= \vb E\supt{PD}(\vb p, \vb x_0; \vb x)
\\
 \vb H\sups{inc}(\vb x) &= \vb H\supt{PD}(\vb p, \vb x_0; \vb x)
\end{align*}
%%%%%%%%%%%%%%%%%%%%%%%%%%%%%%%%%%%%%%%%%%%%%%%%%%%%%%%%%%%%%%%%%%%%%%
where the fields of a point electric dipole are
%%%%%%%%%%%%%%%%%%%%%%%%%%%%%%%%%%%%%%%%%%%%%%%%%%%%%%%%%%%%%%%%%%%%%%
\begin{subequations}
\begin{align*}
\end{align*}
\label{PDFields}
\end{subequations}
%%%%%%%%%%%%%%%%%%%%%%%%%%%%%%%%%%%%%%%%%%%%%%%%%%%%%%%%%%%%%%%%%%%%%%
(The ``PD'' superscript stands for ``point dipole'')

To compute

%%%%%%%%%%%%%%%%%%%%%%%%%%%%%%%%%%%%%%%%%%%%%%%%%%%%%%%%%%%%%%%%%%%%%%
%%%%%%%%%%%%%%%%%%%%%%%%%%%%%%%%%%%%%%%%%%%%%%%%%%%%%%%%%%%%%%%%%%%%%%
%%%%%%%%%%%%%%%%%%%%%%%%%%%%%%%%%%%%%%%%%%%%%%%%%%%%%%%%%%%%%%%%%%%%%%
\section{Extension to the periodic case}

In the Bloch-periodic module of {\sc scuff-em}, \textit{all}
fields and currents are assumed to be Bloch-periodic, i.e.
if $Q(\vb x)$ denotes any field or current component at $\vb x$,
then we have the built-in assumption
%====================================================================%
\numeq{BlochCondition}
{Q(\vb x + \vb L) = e^{i\vb k\subt{B} \cdot \vb L}Q(\vb x)}
%====================================================================%
where $\vb L$ is any lattice vector and 
$\vb k\subt{B}$ is the Bloch wavevector.

The fields of a point dipole, equation (\ref{PDFields}), do \textit{not}
satisfy (\ref{BlochCondition}), and hence may not be used in
Bloch-periodic {\sc scuff-em} calculations. Instead, what we can 
simulate in the periodic case are the fields of an infinite
phased \textit{array} of point dipoles,
%%%%%%%%%%%%%%%%%%%%%%%%%%%%%%%%%%%%%%%%%%%%%%%%%%%%%%%%%%%%%%%%%%%%%%
\begin{subequations}
\begin{align*}
 \vb E\supt{PDA}(\vb p, \vb x_0, \vb k\subt{B}; \vb x)
&=\sum_{\vb L} e^{i\vb k\subt{B}\cdot \vb L}
  \vb E\supt{PD}(\vb p, \vb x_0 + \vb L; \vb x)
\\
 \vb H\supt{PDA}(\vb p, \vb x_0, \vb k\subt{B}; \vb x)
&=\sum_{\vb L} e^{i\vb k\subt{B} \cdot \vb L}
  \vb H\supt{PD}(\vb p, \vb x_0  + \vb L; \vb x)
\end{align*}
\label{EHPDA}
\end{subequations}
%%%%%%%%%%%%%%%%%%%%%%%%%%%%%%%%%%%%%%%%%%%%%%%%%%%%%%%%%%%%%%%%%%%%%%
(where ``PDA'' stands for ``point dipole array''). The quantity
we can compute

The inversion of equations (\ref{EHPDA}) recovers the fields
of a point source as an integral over the Brillouin zone of the 
lattice:\footnote{To derive these equations, multiply both sides
of (\ref{EHPDA}) by $e^{-i\vb k\subt{B} \cdot \vb L^\prime}$,
integrate both sides over the Brillouin zone, and use the
condition 
$$\int\subt{BZ} e^{i\vb k\subt{B}\cdot (\vb L-\vb L^\prime)}\,d\vb k
  =\mc{V}\subt{BZ} \, \delta(\vb L,\vb L^\prime)
$$
where $\mc V\subt{BZ}$ is the Brillouin-zone volume [for example,
a square lattice with basis vectors
$\{\vb L_1, \vb L_2\}=\{L_x\vbhat{x}, L_y\vbhat{y}\}$ has
$\mc V\subt{BZ}=4\pi^2/(L_x L_y)$].
Setting $\vb L^\prime=0$ recovers (\ref{EHPDAInverse}).}
%%%%%%%%%%%%%%%%%%%%%%%%%%%%%%%%%%%%%%%%%%%%%%%%%%%%%%%%%%%%%%%%%%%%%%
\begin{subequations}
\begin{align}
  \vb E\supt{PD}(\vb p, \vb x_0; \vb x)
&=\frac{1}{\mc V\subt{BZ}} 
   \int\subt{BZ} \vb E\supt{PDA}(\vb p, \vb x_0, \vb k\subt{B}; \vb x) 
   d\vb k\subt{B}
\\
  \vb H\supt{PD}(\vb p, \vb x_0; \vb x)
&=\frac{1}{\mc V\subt{BZ}} 
   \int\subt{BZ} \vb H\supt{PDA}(\vb p, \vb x_0, \vb k\subt{B}; \vb x) 
   d\vb k\subt{B}
\end{align}
\label{EHPDAInverse}
\end{subequations}
%%%%%%%%%%%%%%%%%%%%%%%%%%%%%%%%%%%%%%%%%%%%%%%%%%%%%%%%%%%%%%%%%%%%%%

%%%%%%%%%%%%%%%%%%%%%%%%%%%%%%%%%%%%%%%%%%%%%%%%%%%%%%%%%%%%%%%%%%%%%%
%%%%%%%%%%%%%%%%%%%%%%%%%%%%%%%%%%%%%%%%%%%%%%%%%%%%%%%%%%%%%%%%%%%%%%
%%%%%%%%%%%%%%%%%%%%%%%%%%%%%%%%%%%%%%%%%%%%%%%%%%%%%%%%%%%%%%%%%%%%%%
\newpage
\section{Technical implementation details}

\subsection*{Computing the fields of a point dipole array}

%The {\sc libincfield} module in {\sc scuff-em} contains a
%built-in routine (in the \texttt{PointSource} class)
%for computing the fields of a single point dipole, but

First consider a single point dipole radiator (not an array) 
at $\vb x_0$ with dipole moment $\vb p_0$ 
(as usual suppressing time-dependence factors
of $e^{-i\omega t}$). Thinking of this as an electric
current distribution of the form
%====================================================================%
$$ \vb J(\vb x) = -i\omega \vb p \delta(\vb x-\vb x_0) $$
%====================================================================%
it is easy to compute the fields by convolution with the usual
Green's functions relating currents to fields:
%====================================================================%
\begin{align*}
 E_i(\vb x) 
&= 
 \int \Gamma_{ij}\supt{EE}(\vb x, \vb x^\prime) J_j(\vb x^\prime) d\vb x^\prime
\\
&=-i\omega \Gamma_{ij}\supt{EE}(\vb x, \vb x_0) p_j
\\
&=(-i\omega)(ikZ_0Z^r) G_{ij}(\vb x, \vb x_0) p_j
\\[8pt]
H_i(\vb x) 
 &=
\int \Gamma_{ij}\supt{ME}(\vb x, \vb x^\prime) J_j(\vb x^\prime) d\vb x^\prime
\\
&=-i\omega \Gamma_{ij}\supt{ME}(\vb x, \vb x_0) p_j
\\
&=(-i\omega)(ik) C_{ij}(\vb x, \vb x_0) p_j
\end{align*}
%====================================================================%
where the $\vb G$ and $\vb C$ dyadics are related to the
scalar Helmholtz Green's function according to
%====================================================================%
$$ G_{ij}(\vb r)
   = \Big[ \delta_{ij} +\frac{1}{k^2} \partial_i \partial_j \Big] G_0(\vb r),
\qquad
   C_{ij}(\vb r)
   = \frac{1}{ik}\varepsilon_{ijk} \partial_k G_0(\vb r).
$$
%====================================================================%
\end{document}
