\newpage
\section{Derivation of volume integrals for the force and torque}
\label{ForceFormulaAppendix}

Consider a body throughout which exist
volume distributions of electric and magnetic currents
$\{\vb J(\vb x), \vb M(\vb x)\}$ and
electric and magnetic fields $\{\vb E(\vb x), \vb H(\vb x)\}.$
The time-average force experienced by just the \textit{electric} 
currents in an infinitesimal volume $dV$ is
%====================================================================%
\begin{align}
d\vb F^{\vb J} &= \frac{1}{2}\text{Re }
 \Big[ \rho^* \vb E + \mu_0 \vb J^* \times \vb H \Big] \, dV
\nn
\intertext{Use $\rho=\frac{1}{i\omega}(\nabla \cdot \vb J)$
           and\footnote{In the presence of magnetic currents,
                        the usual Maxwell equation for the curl of 
                        $\vb E$ is modified to read
                        $\nabla \times \vb E=i\omega \mu_0 \vb H - \vb M.$}
           $\vb H=\frac{1}{i\omega\mu_0}(\nabla \times \vb E + \vb M)$:}
 &=
 \frac{1}{2}\text{Re }\left\{
 \frac{1}{i\omega}
 \Big[ -(\nabla \cdot \vb J^*) \vb E 
       + \vb J^* \times (\nabla \times \vb E) 
       +\vb J^* \times \vb M
 \Big]\,dV \,
                      \right\} dV
\nn
 &=
 \underbrace{
              \frac{1}{2\omega}\text{Im }
              \Big[ -(\nabla \cdot \vb J^*) \vb E 
             + \vb J^* \times (\nabla \times \vb E) 
              \Big]\,dV
            }_{d \vb F^{\vb J1}}
+
 \underbrace{
   \frac{1}{2\omega}\text{Im }\Big[\vb J^* \times \vb M\Big]
            }_{d \vb F^{\vb J2}}.
\label{FJ1FJ2}
\end{align}
%====================================================================%
In component notation, the first term here reads
%====================================================================%
\begin{align}
dF^{\vb J1}_i &= -\frac{1}{2\omega}\text{Im }
 \Big[
 (\partial_j J^*_j) E_i - 
  \underbrace{ \varepsilon_{ijk}
               \varepsilon_{k\ell m}
             }_{\delta_{i\ell}\delta_{jm} - \delta_{im}\delta_{j\ell}}
  J^*_j \partial_\ell E_m
 \Big] \, dV
\nn
&= -\frac{1}{2\omega}\text{Im }
 \Big[ (\partial_j J^*_j) E_i
       - J^*_j \partial_i E_j
       + J^*_j \partial_j E_i
 \Big] \, dV.
\label{dFi}
\intertext{The total force is given by integrating over the volume:}
F_i^{\vb J1}
&= -\frac{1}{2\omega}\text{Im } \int_{\mc B_n}
 \Big[ (\partial_j J^*_j) E_i
       - J^*_j \partial_i E_j
       + J^*_j \partial_j E_i
 \Big] dV
\label{Fi}
\end{align}
%====================================================================%
The first and third terms here together read
%====================================================================%
\numeq{Argument}
{
 \int \partial_j \big(J_j^* E_i\big ) dV 
 = \int \nabla \cdot (E_i \, \vb J^*) \, dV
 = \oint E_i \vb J^* \cdot d\vb A = 0
}
%====================================================================%
because $\vb J\cdot \vbhat{n}=0$ at the surface of the object 
(no current flows from the body into space). 
Thus only the middle term in (\ref{Fi}) is nonvanishing,
and we find simply
%====================================================================%
$$
  F^{\vb J1}_i = \frac{1}{2\omega}\text{Im } \int_{\mc B_n}
         J^*_j \partial_i E_j \, dV 
      = \frac{1}{2\omega}\text{Im } 
         \int_{\mc B_n} \vb J^* \cdot \partial_i \vb E \, dV.
$$
%====================================================================%
Adding the contributions of the other term in (\ref{FJ1FJ2}), we 
find
%====================================================================%
\numeq{FJVolumeIntegral}
{ F_i^{\vb J} = \frac{1}{2\omega}\text{Im }
  \int_{\mc B_n} \Big[         \vb J^* \cdot \partial_i \vb E 
                       \, + \, (\vb J^*\times \vb M)_i
                 \Big]\, dV.
}
%====================================================================%
On the other hand, the force on the \textit{magnetic} currents in 
$dV$ is 
%====================================================================%
\begin{align}
d\vb F^{\vb M} &= \frac{1}{2}\text{Re }
 \Big[ \eta^* \vb H - \epsilon \vb M^* \times \vb E \Big] \, dV
\nn
\intertext{Use $\eta=\frac{1}{i\omega}(\nabla \cdot \vb M)$
           and
           $\vb E=-\frac{1}{i\omega\epsilon}(\nabla \times \vb H - \vb J)$:}
 &=
 \frac{1}{2}\text{Re }\left\{
 \frac{1}{i\omega}
 \Big[ -(\nabla \cdot \vb M^*) \vb H 
       + \vb M^* \times (\nabla \times \vb H) 
       +\vb M^* \times \vb J
 \Big]\,dV \,
                      \right\} dV
\nn
 &=
 \underbrace{
              \frac{1}{2\omega}\text{Im }
              \Big[ -(\nabla \cdot \vb M^*) \vb H 
             + \vb M^* \times (\nabla \times \vb H) 
              \Big]\,dV
            }_{d \vb F^{\vb M1}}
+
 \underbrace{
   \frac{1}{2\omega}\text{Im }\Big[\vb M^* \times \vb J\Big]
            }_{d \vb F^{\vb M2}}.
\label{FM1FM2}
\end{align}
%====================================================================%
Now playing the same games as above allows us to rewrite this in a 
form analogous to that of (\ref{FJVolumeIntegral}):
%====================================================================%
\numeq{FJVolumeIntegral}
{ F_i^{\vb M} = \frac{1}{2\omega}\text{Im }
  \int_{\mc B_n} \Big[         \vb M^* \cdot \partial_i \vb H 
                       \, + \, (\vb M^*\times \vb J)_i
                 \Big]\, dV.
}
%====================================================================%

\subsubsection*{Torque}

The contribution of currents in $dV$ to the \textit{torque} 
about an origin $\vb r_0$ is given by
%====================================================================%
\begin{align*} 
 d\bmc T &= (\vb r - \vb r_0) \times d \vb F
\intertext{or, in components,}
 d\mc T_i &= \varepsilon_{ijk} (\vb r - \vb r_0)_j d \vb F_k.
\intertext{Insert (\ref{dFi}):}
&= -\frac{1}{2\omega}\text{Im}
 \left\{
 \varepsilon_{ijk} (\vb r - \vb r_0)_j
 \Big[ (\partial_\ell J^*_\ell) E_k
       - J^*_\ell \partial_k E_\ell
       + J^*_\ell \partial_\ell E_k
 \Big]\right\} \, dV.
\end{align*} 
The first and third terms here integrate to zero by an 
argument similar to (\ref{Argument}), and we find
%====================================================================%
\numeq{TorqueVolumeIntegral}
{  \mc T_i 
 = \frac{1}{2\omega}\text{Im } \int_{\mc B_n}
   \epsilon_{ijk} (\vb r-\vb r_0)_j J_\ell^* \partial_k E_\ell
   \,dV
 = \frac{1}{2\omega}\text{Im } \int_{\mc B_n}
    \vb J^* \cdot \partial_\theta \vb E \, dV
}
where the symbol $\partial_\theta \vb E$ denotes the derivative
of $\vb E(\vb r)$ with respect to an infinitesimal rotation of the
point $\vb r$ about the $i$th coordinate axis with origin $\vb r_0.$

