%%%%%%%%%%%%%%%%%%%%%%%%%%%%%%%%%%%%%%%%%%%%%%%%%%%%%%%%%%%%%%%%%%%%%%
%%%%%%%%%%%%%%%%%%%%%%%%%%%%%%%%%%%%%%%%%%%%%%%%%%%%%%%%%%%%%%%%%%%%%%
%%%%%%%%%%%%%%%%%%%%%%%%%%%%%%%%%%%%%%%%%%%%%%%%%%%%%%%%%%%%%%%%%%%%%%
\newpage
\section{Evaluation of 4D Integrals over RWG Basis Functions}

As demonstrated above, the elements of the BEM matrix involve
integrals over pairs of RWG basis functions of the form
%====================================================================%
\begin{subequations}
\begin{align}
 \Big\langle \vb f_a \Big| \vb G(k) \Big | \vb f_b \Big \rangle
&\equiv 
  \int_{\sup \vb f_a} d\vb x_a \, 
  \int_{\sup \vb f_b} d\vb x_b\,
  \vb f_a(\vb x_a) \cdot 
  \vb G(k, \vb x_a-\vb x_b) \cdot 
  \vb f_b(\vb x_b)
\\
 \Big\langle \vb f_a \Big| \vb C(k) \Big | \vb f_b \Big \rangle
&\equiv 
  \int_{\sup \vb f_a} d\vb x_a \, 
  \int_{\sup \vb f_b} d\vb x_b \,
  \vb f_a(\vb x_a) \cdot 
  \vb C(k, \vb x_a-\vb x_b) \cdot 
  \vb f_b(\vb x_b).
\end{align}
\label{MatrixElementIntegrals}
\end{subequations}
%====================================================================%
\lss computes these integrals using one of two strategies depending
on how far apart the basis functions are from one another. To 
quantify this, let $d_{ab}$ be the distance between the centroids
of basis functions $\vb f_a$ and $\vb f_b$, and let 
$R\subs{max}=\texttt{max}(R_a, R_b)$ be the larger of the radii
of the two basis functions. (The ``radius'' of a compact source distribution
is the radius of the smallest sphere in which the source distribution
may be enclosed. For RWG basis functions, we take the centroid to be
the midpoint of the common edge shared by the two triangle that define
the basis function; then the radius is the greatest distance from the 
centroid to any of the four panel vertices 
(Figure \ref{RWGMidPointRadius}).)

Then the computation of the integrals (\ref{MatrixElementIntegrals})
proceeds as follows. 

\begin{enumerate}
 \item
 When $d_{ab} > \texttt{DBFTHRESHOLD}\cdot R\subs{max}$,
 we approximate (\ref{MatrixElementIntegrals}) using a 
 spherical-multipole expansion. 
 (Here \texttt{DBFThreshold}, the ``distant basis-function threshold,''
 is a dimensionless number that must be tuned to yield 
 optimal accuracy and performance; in \lss its value is set to 8.3.)
 \item
 Otherwise, we compute (\ref{MatrixElementIntegrals}) as a sum
 of four numerically-evaluated integrals over pairs of triangular
 panels.
\end{enumerate}

Each of these methods is described in the following sections.

\subsection{Matrix elements between distant basis functions: 
            spherical multipole method}
\label{SphericalMultipoleMatrixElementSection}

The spherical multipole method is based on the spherical-multipole
expansion of the $\vb G$ and $\vb C$ dyadics:

\begin{align*}
 \vb G(\vb x, \vb x^\prime)
 &= -ik\sum_{\alpha} 
      \left\{   \MInt_{\alpha}(\xInt)\MExt^*(\xExt)
              - \NInt_{\alpha}(\xInt)\NExt^*(\xExt)
      \right\} 
\\
 \vb C(\vb x, \vb x^\prime)
 &= -ik\sum_{\alpha} 
      \left\{   \MInt_{\alpha}(\xInt)\NExt^*(\xExt) 
              + \NInt_{\alpha}(\xInt)\MExt^*(\xExt)
      \right\} 
\end{align*}
where $\xInt (\xExt)$ denote whichever of $\vb x, \vb x^\prime$
is closer to (further from) the origin.
(My notation and conventions for spherical multipole functions
are summarized in Appendix \ref{SphericalHelmoltzAppendix}; 
briefly, the $\wedge$ adornment means ``interior,'' 
while $\vee$ means ``exterior,'' and the mnemonic is to 
think of $\vee$ as indicating radiation outward to infinity,
as is appropriate for exterior solutions).

Inserting into (\ref{MatrixElementIntegrals}), we have 
%====================================================================%
\begin{align*}
 \big\langle \vb f_a \big| \vb G \big| \vb f_b \big\rangle
&=-ik\sum_{\alpha} 
    \Big\{  \big\langle \vb f_a \big| \MInt_\alpha \big\rangle
            \big\langle \vb \MExt^*_\alpha \big| \vb f_b \big\rangle
           -\big\langle \vb f_a \big| \NInt_\alpha \big\rangle
            \big\langle \vb \NExt^*_\alpha \big| \vb f_b \big\rangle
    \Big\}\\
%--------------------------------------------------------------------%
 \big\langle \vb f_a \big| \vb C \big| \vb f_b \big\rangle
&=-ik\sum_{\alpha} 
    \Big\{  \big\langle \vb f_a \big| \MInt_\alpha \big\rangle
            \big\langle \vb \NExt^*_\alpha \big| \vb f_b \big\rangle
           +\big\langle \vb f_a \big| \NInt_\alpha \big\rangle
            \big\langle \vb \MExt^*_\alpha \big| \vb f_b \big\rangle
    \Big\}
%--------------------------------------------------------------------%
\end{align*}

Now use the translation matrices 
[Appendix \ref{SphericalHelmoltzAppendix}, 
 equation (\ref{TranslationMatrixDefinition})]
to rewrite inner products involving exterior functions in terms
of inner products involving interior functions:
%====================================================================%
\begin{subequations}
\begin{align}
 \big\langle \vb f_a \big| \vb G \big| \vb f_b \big\rangle
&=-ik\sum_{\alpha\beta} 
   \left(\begin{array}{c}
     \mathcal{M}_{a\alpha}  \\ \mathcal{N}_{a\alpha}
   \end{array}\right)^T
   \left(\begin{array}{cc}
   A_{\alpha\beta} & B_{\alpha\beta} \\
   B_{\alpha\beta} & -A_{\alpha\beta}
   \end{array}\right)
   \left(\begin{array}{c}
     \mathcal{M}_{b\alpha}  \\ \mathcal{N}_{b\alpha}
   \end{array}\right)
\\
%--------------------------------------------------------------------%
 \big\langle \vb f_a \big| \vb C \big| \vb f_b \big\rangle
&=-ik\sum_{\alpha\beta} 
   \left(\begin{array}{c}
     \mathcal{M}_{a\alpha}  \\ \mathcal{N}_{a\alpha}
   \end{array}\right)^T
   \left(\begin{array}{cc}
  -B_{\alpha\beta} & A_{\alpha\beta} \\
   A_{\alpha\beta} & B_{\alpha\beta}
   \end{array}\right)
   \left(\begin{array}{c}
     \mathcal{M}_{b\alpha}  \\ \mathcal{N}_{b\alpha}
   \end{array}\right)
\end{align}
\label{SphericalMultipoleMatrixElements}
\end{subequations}
%====================================================================%
where $\{ \mathcal{M}, \mathcal{N}\}$ are the spherical multipole moments
of the RWG basis functions:
\numeq{RWGSphericalMultipoleMoments}
{
   \mathcal{M}_{a\alpha}
   =
   \int_{\sup \vb f_a} \vb f_a(\vb x) \cdot \MInt_\alpha(\vb x)\, d\vb x,
   \qquad
   \mathcal{N}_{a\alpha}
   =
   \int_{\sup \vb f_a} \vb f_a(\vb x) \cdot \NInt_\alpha(\vb x)\, d\vb x.
}
The point of this decomposition is that computing the integrals
(\ref{RWGSphericalMultipoleMoments}) at a given frequency
requires $O(\texttt{NBF})$ numerical cubatures, in contrast to 
the $O(\texttt{NBF}^2)$ numerical cubatures that would na\"ively 
be required to evaluate (\ref{MatrixElementIntegrals}) for 
all pairs of basis functions.

%====================================================================%

\subsection{Matrix elements between nearby basis functions: 
            panel-panel integration method}

When the supports of the basis functions $\vb f_a$ and $\vb f_b$ 
are relatively close to each other, we evaluate each of the 
integrals in (\ref{MatrixElementIntegrals}) as a sum of four
integrals over pairs of triangular panels: 
%====================================================================%
\begin{subequations}
\begin{align}
&\hspace{-0.1in}
 \Big\langle \vb f_a \Big| \vb G(k) \Big | \vb f_b \Big \rangle
\nn
&=
  l_a l_b 
  \sum_{\sigma, \tau=-}^+ 
  \frac{\sigma\tau}{4A^\sigma_a A^\tau_b}
  \int_{\pan_a^\sigma}  d\vb x_a \,
  \int_{\pan_b^\tau}  d\vb x_b \,
   \Big[                 h_{\bullet}(\vb x_a, \vb x_b)
         - \frac{1}{k^2} h_{\nabla}(\vb x_a, \vb x_b)
   \Big]
   \phi\big(k, |\vb x_a - \vb x_b|\big)
\\[5pt]
%--------------------------------------------------------------------%
&\hspace{-0.1in}
 \Big\langle \vb f_a \Big| \vb C(k) \Big | \vb f_b \Big \rangle
\nn
&=
  \frac{l_a l_b}{ik}
  \sum_{\sigma, \tau=-}^+ 
  \frac{\sigma \tau}{4A^\sigma_a A^\tau_b}
  \int_{\pan_a^\sigma}  d\vb x_a \,
  \int_{\pan_b^\tau}  d\vb x_b \,
   h_{\times}(\vb x_a, \vb x_b)
   \psi\big(k, |\vb x_a - \vb x_b|\big)
\end{align}
\label{PanelPanelIntegrals}
\end{subequations}
%====================================================================%
where\footnote{\textit{Warning:} My notation for the $h$ functions
hides the fact that $h_\bullet$ and $h_\times$ depend on the current 
source/sink nodes $\vb Q_{ab}^\pm$ within the two triangles.}
%====================================================================%
\begin{align*}
h_{\bullet}(\vb x_a, \vb x_b)
&= (\vb x_a - \vb Q_a) \cdot (\vb x_b - \vb Q_b) 
\\
%--------------------------------------------------------------------%
h_{\nabla}(\vb x_a, \vb x_b)
&= 4
\\
%--------------------------------------------------------------------%
h_{\times}(\vb x_a, \vb x_b)
&=
\Big[ (\vb x_a - \vb Q_a) \times (\vb x_b - \vb Q_b) \Big]
      \cdot (\vb x_a - \vb x_b)
\\
&=
      (\vb x_a \times \vb x_b) \cdot (\vb Q_a - \vb Q_b)
    + (\vb Q_a \times \vb Q_b) \cdot (\vb x_a - \vb x_b)
\\
%--------------------------------------------------------------------%
\phi(k, r)&=\frac{e^{ikr}}{4\pi r}, 
\\
%--------------------------------------------------------------------%
\psi(k, r)&=(ikr-1)\frac{e^{ikr}}{4\pi r^3}.
\end{align*}
%====================================================================%
[Note that $\psi$ is defined such that 
$\nabla \phi(k, |\vb r|) = \vb r \psi(k, |\vb r|)$].
%====================================================================%

The point of this notation is that it expresses the 
integrands of the panel-panel integrals in (\ref{PanelPanelIntegrals}) 
as products of benign polynomials in $\vb x_a, \vb x_b$
(the $h$ functions) times kernel functions that depend only 
on the distance $\vb x_a-\vb x_b$ and have singularities 
when this distance vanishes (the $\phi, \psi$ kernels).
This decomposition facilitates the desingularization procedure
described below.

In some cases I will write (\ref{PanelPanelIntegrals}) 
using the notation
\begin{subequations}
\begin{align}
 \Big\langle \vb f_a \Big| \vb G(k) \Big | \vb f_b \Big \rangle
&=
  l_a l_b 
  \sum_{\sigma, \tau=-}^+ 
  \sigma\tau
  \left\{               \texttt{PPI}\Big(\pan_a^\sigma, \pan_b^\tau, h_\bullet, \phi\Big) 
         -\frac{1}{k^2} \texttt{PPI}\Big(\pan_a^\sigma, \pan_b^\tau, h_\nabla,  \phi\Big) 
  \right\}
\\
%--------------------------------------------------------------------%
 \Big\langle \vb f_a \Big| \vb C(k) \Big | \vb f_b \Big \rangle
&=
  \frac{l_a l_b }{ik}
  \sum_{\sigma, \tau=-}^+ \sigma \tau \cdot 
   \texttt{PPI}\Big(\pan_a^\sigma, \pan_b^\tau, h_\times, \psi\Big) 
\end{align}
\label{PPIs}
\end{subequations}
with the ``panel-panel integral'' functions defined by 
\numeq{PPIDefinition}
{
   \texttt{PPI}\Big(\pan, \pan^\prime, h, g\Big)
   \equiv 
   \frac{1}{4A A^\prime}
   \int_{\pan} d\vb x \, 
   \int_{\pan^\prime} d\vb x^\prime \, 
   h(\vb x, \vb x^\prime) \, g(|\vb x-\vb x^\prime|).
}
%Sometimes I will use the alternative notation
%$$
%   H_{\bullet, \nabla, \times} \Big(\pan_a, \pan_b, g\Big)
%   \equiv 
%   \int_{\pan_a} d\vb x_a \, 
%   \int_{\pan_b} d\vb x_b \, 
%   h_{\bullet, \nabla, \times}(\vb x_a, \vb x_b) \, 
%   g(|\vb x_a-\vb x_b|)
%$$
%as well as 
%$$ H_{+}\Big(\pan_a, \pan_b, g\Big) \equiv 
%   H_{\bullet}\Big(\pan_a, \pan_b, g\Big)
%   -\frac{1}{k^2} H_{\nabla }\Big(\pan_a, \pan_b, g\Big).
%$$
The quantities (\ref{PPIDefinition}) are what are computed 
by the \texttt{GetPanelPanelInteractions()} routine in \ls. 
The responsibility of assembling these quantities together with
the requisite prefactors to obtain the full inner products
(\ref{PPIs}) is handled by the routine 
\texttt{GetEdgeEdgeInteractions()}.

\subsubsection*{Evaluation of distant panel-panel integrals: Cubature}

When the panels $\pan, \pan^\prime$ in 
(\ref{PPIDefinition}) are relative far away from each other, we
evaluate the four-dimensional integral using numerical cubature.
For this purpose it is convenient to parameterize points in the
triangles using the prescription (Figure \ref{uvDefinitionFigure})
\numeq{uvDefinition}
{\begin{array}{cclll}
 \vb x &=&\vb V_1 + u\vb A + v\vb B, 
 \qquad &0\le u \le 1, 
 \qquad &0 \le v \le u 
\\[5pt]
 \vb x^\prime &=& \vb V_1^\prime 
                 + u^\prime\vb A^\prime 
                 + v^\prime\vb B^\prime
 \qquad &0\le u^\prime \le 1, 
 \qquad &0 \le v^\prime \le u^\prime.
 \end{array}
}
The integral (\ref{PPIDefinition}) becomes 
\numeq{uvIntegral}
{
   \texttt{PPI}\Big(\pan, \pan^\prime, h, g\Big)
   =
   \int_0^1 du \, \int_0^u dv \, 
   \int_0^1 du^\prime \, \int_0^{u^\prime} dv^\prime \, 
   h(u, v, u^\prime, v^\prime)
   g(u, v, u^\prime, v^\prime)
}
where the prefactor $\frac{1}{4AA^\prime}$ is cancelled
by the Jacobian of the variable transformation (\ref{uvDefinition}),
and where I put
$$ h(u,v,u^\prime,v^\prime) = 
   h\Big( \vb x(u,v), \vb x^\prime(u^\prime,v^\prime) \Big), 
   \qquad
   g(u,v,u^\prime,v^\prime) = 
   g\Big( \big|\vb x(u,v) - \vb x^\prime(u^\prime,v^\prime) \big|\Big).
$$
The integral (\ref{uvIntegral}) may be evaluated numerically in 
one of two ways:
\begin{enumerate}
 \item by nesting two cubature rules for the standard triangle 
       with vertices $\{(0,0), (0,1), (1,0)\}$, or
 \item by mapping the domain of integration to the four-dimensional 
       hypercube $[0,1] \times [0,1] \times [0,1] \times [0,1]$
       and using a four-dimensional cubature rule for this hypercube.
       (The variable transformation that enables this mapping is
       $$v=ut, \qquad dv=udt, \qquad \int_0^u f(v) dv = \int_0^1 u f(ut) dt$$
       and similarly $v^\prime=u^\prime t^\prime.$)
\end{enumerate}
The former of these two strategies is slightly more efficient, but
the latter has the advantage of allowing the use of standard codes
for adaptive cubature over 
hypercubes.\footnote{\texttt{http://ab-initio.mit.edu/cubature}}
Both strategies are used in \ls.

\subsubsection*{Evaluation of nearby panel-panel integrals: Desingularization}

When the panels in (\ref{PPIDefinition}) have one or more common
vertices, the integrand becomes singular at one or more points in 
the domain of integration. Although these are \textit{integrable}
singularities, their existence precludes application of the na\"ive
numerical cubature scheme discussed above, and instead we must
resort to more complicated, and more costly, numerical methods.

On the other hand, when the panels have no common vertices but are
nearby one another, the integrand in (\ref{PPIDefinition}) is
nonsingular but rapidly varying over the domain of integration, 
and evaluation of the integral by numerical cubature is technically 
possible but expensive due to the large number of cubature points 
required to obtain decent accuracy.

%====================================================================%
\begin{subequations}
\begin{align}
 \iint h \phi
&=
 \iint h \phi\supt{DS}
+\sum_{n=0}^3 \frac{(ik)^n}{4\pi} A_n  
 \iint h r^{n-1}
\\
%--------------------------------------------------------------------%
 \iint h\psi
&=
 \iint h\psi\supt{DS}
+\sum_{n=0}^4 \frac{(ik)^n}{4\pi} B_n 
 \iint h r^{n-3}
\end{align}
\label{DesingularizedPPIs}
\end{subequations}
%====================================================================%
Here we are using a shorthand in which 
\begin{itemize}
 \item $h$ is any of the functions $\{h_\bullet, h_\nabla, h_\times\}$,
 \item $\iint$ is shorthand for 
       $\int_{\pan_a} \, d\vb x_a \, \int_{\pan_b} \, d\vb x_b, $
 \item $\iint h r^p$ is shorthand for 
       $\int_{\pan_a} \, d\vb x_a \, \int_{\pan_b} \, d\vb x_b\,
        \big\{h(\vb x_a, \vb x_b) |\vb x_a - \vb x_b|^p\big\},$
 \item the $A$ coefficients in (\ref{DesingularizedPPIs}a) are 
       $$A_n=\frac{1}{n!}$$
 \item the $B$ coefficients in (\ref{DesingularizedPPIs}b) are 
       $$ B_0 = -1, \qquad B_1 = 0, \qquad B_2 = \frac{1}{2}, \qquad
          B_3 = \frac{1}{3}, \qquad B_4 = \frac{1}{6}
       $$
 \item the desingularized kernels are 
\begin{subequations}
\begin{align}
\phi\supt{DS}(k, r) 
 &=
\frac{e^{ikr} - 1 - ikr - \frac{1}{2}(ikr)^2 - \frac{1}{6}(ikr)^3 }
     {4\pi r}, 
\nonumber \\
&\equiv\frac{\texttt{ExpRel}(ikr, 4)}{4\pi r}
\\
\intertext{and similarly}
%--------------------------------------------------------------------%
\psi\supt{DS}(k, r)&\equiv(ikr-1)\frac{\texttt{ExpRel}(ikr, 4)}{4\pi r^3}.
\end{align}
\label{PhiDSPsiDS}
\end{subequations}
%====================================================================%
\end{itemize}
When computing the ``relative exponential'' functions 
$\texttt{ExpRel}(x,n)$ in (\ref{PhiDSPsiDS}) it is 
important that we \textit{not} simply compute 
$\texttt{exp(x)}$ and then subtract off the first $n$ terms
in its Taylor series, as doing so could lead to catastrophic loss
of numerical precision. Instead, a better-behaved procedure is 
simply to sum the Taylor series for the exponential starting 
with its $(n+1)$th term.
