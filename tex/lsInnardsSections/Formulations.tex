
%%%%%%%%%%%%%%%%%%%%%%%%%%%%%%%%%%%%%%%%%%%%%%%%%%%%%%%%%%%%%%%%%%%%%%
%%%%%%%%%%%%%%%%%%%%%%%%%%%%%%%%%%%%%%%%%%%%%%%%%%%%%%%%%%%%%%%%%%%%%%
%%%%%%%%%%%%%%%%%%%%%%%%%%%%%%%%%%%%%%%%%%%%%%%%%%%%%%%%%%%%%%%%%%%%%%
\newpage
\section{BEM Formulations in \ls}

In previous sections we discussed how \lss represents
surface currents and how \lss uses surface currents to 
compute scattered fields. In this section we discuss
how \lss actually computes the surface currents for a 
given incident field. To this end, \lss employs a 
variety of BEM formulations, some of which may coexist
in a single problem.

%%%%%%%%%%%%%%%%%%%%%%%%%%%%%%%%%%%%%%%%%%%%%%%%%%%%%%%%%%%%%%%%%%%%%%
%%%%%%%%%%%%%%%%%%%%%%%%%%%%%%%%%%%%%%%%%%%%%%%%%%%%%%%%%%%%%%%%%%%%%%
%%%%%%%%%%%%%%%%%%%%%%%%%%%%%%%%%%%%%%%%%%%%%%%%%%%%%%%%%%%%%%%%%%%%%%
\subsection{Continuous forms of the integral equations
            solved by \lss}

\subsubsection{The equation imposed at points on PEC surfaces}

For PEC surfaces, \lss imposes the \textit{electric field
integral equation} (EFIE). This means that at each point 
$\vb x$ on a PEC surface $\mc S$ we require that the tangential 
components of the total (incident + scattered) field vanish:
%====================================================================%
\numeq{PECEquation1}
{ \vbhat{n}\times \vb E\sups{total} (\vb x)=0, \qquad
  \vb x \in \mc S 
}
%====================================================================%
where $\vbhat{n}$ is the normal to the object surface
at $\vb x$. 

To obtain an integral equation from (\ref{PECEquation1}), 
suppose that $\mc S$ is embedded in a homogeneous 
material region $\mc R$. 
%[$\mc S$ may be an open
%or closed; in the latter case the region interior to
%$\mc S$ is irrelevant and does not enter the problem
%formulation.] 
Then the total field in (\ref{PECEquation1})
is a sum of scattered and incident contributions; 
the former involve convolutions of the surface-current
distributions on all surfaces bounding $\mc R$
(including $\mc S$ plus any other surfaces that 
may be part of the boundary of $\mc R$), while the 
latter involve only those incident-field sources
lying interior to $\mc R$. Equation (\ref{PECEquation1})
then reads
\begin{align}
   \sum_{\mathcal{S}_s \subset \partial \mc R}
   \text{sgn}\big(\mc S_s,R\big)
   \oint_{\mc S_s}
   \left(\begin{array}{cc}
   \BG\supt{EE}(\mc R) & \BG\supt{EM}(\mc R)  \\
   \BG\supt{ME}(\mc R) & \BG\supt{MM}(\mc R)  \\
   \end{array}\right)
   \left(\begin{array}{cc} \vb K \\ \vb N \end{array}\right)
   dA
  &=
  -
   \left(\begin{array}{c} \vb E\incr \\ \vb H\incr \end{array}\right)
\\
\intertext{or, in convenient shorthand,}
   \sum_{\mathcal{S}_s \subset \partial \mc R}
   \text{sgn}\big(\mc S_s,R\big)
   \left[ 
         \left(\begin{array}{cc}
         \BG\supt{EE}(\mc R) & \BG\supt{EM}(\mc R)  \\
         \BG\supt{ME}(\mc R) & \BG\supt{MM}(\mc R)  \\
         \end{array}\right) 
   \star 
   \left(\begin{array}{cc} \vb K_s \\ \vb N_s \end{array}\right)
   \right]
  &=
  -
   \left(\begin{array}{c} \vb E\incr \\ \vb H\incr \end{array}\right)
\label{EFIE}
\end{align}
where the sum is over all surfaces $\mc S_s$ that 
constitute the boundary $\partial \mc R$ of the region $\mc R$
(including $\mathcal S$), $\star$ denotes convolution,
the $s$ subscript on $\vb K,\vb N$ indicates the restriction
of the current distributions to $\mathcal {S}_s$, and 
$$\text{sgn}\big(\mc S,\mc R\big)\equiv
   \begin{cases} 
   +1  \qquad &\text{if $\mc R$ is the ``exterior'' 
                     region for $\mc S$}
   \\
   -1  \qquad &\text{if $\mc R$ is the ``interior'' 
                     region for $\mc S$.}
   \end{cases}
$$
As discussed in Section \ref{GeometriesSection},
the notions of ``exterior'' and ``interior'' are
defined even when $\mc S$ is an open surface; in
that case its ``exterior'' region is the region
into which its surface normal points.

Note that any PEC surfaces in the sum in (\ref{EFIE}), 
including $\mc S$ itself, have $\vb N=0$ identically.

Note that equation (\ref{EFIE}) may be written using
a shorthand notation:
\numeq{EFIEShorthand}
{  \sum_{\mc S_s \subset \partial \mc R}
   \text{sgn}\big(\mc S_s,\mc R\big)
   \Big[ \bmc G(\mc R) \star \bmc C_s\Big]
    = 
    -\bmc F\incr
}
where $\bmc C$, $\bmc F$, and $\bmc G$ denote
six-vector surface currents, six-vector fields, and
$6\times 6$ dyadic Green's functions:
$$ \bmc C 
   \equiv
   \left(\begin{array}{c} \vb K \\ \vb N \end{array}\right),
   \qquad
   %--------------------------------------------------------------------%
   \bmc F
   \equiv
   \left(\begin{array}{c} \vb E \\ \vb H \end{array}\right), 
   \qquad
   %--------------------------------------------------------------------%
   \bmc G(\mathcal{R})\equiv
   \left(\begin{array}{cc}
   \BG\supt{EE}(\mathcal{R}) & \BG\supt{EM}(\mathcal{R})  \\
   \BG\supt{ME}(\mathcal{R}) & \BG\supt{MM}(\mathcal{R})  \\
   \end{array}\right).
$$
We will use this 6-vector shorthand notation frequently
in what follows.

%%%%%%%%%%%%%%%%%%%%%%%%%%%%%%%%%%%%%%%%%%%%%%%%%%%%%%%%%%%%%%%%%%%%%%
%%%%%%%%%%%%%%%%%%%%%%%%%%%%%%%%%%%%%%%%%%%%%%%%%%%%%%%%%%%%%%%%%%%%%%
%%%%%%%%%%%%%%%%%%%%%%%%%%%%%%%%%%%%%%%%%%%%%%%%%%%%%%%%%%%%%%%%%%%%%%
\subsubsection{The equation imposed at points on 
               dielectric object surfaces: PMCHWT formulation}

By default, \lss adopts the PMCHWT formulation of the 
BEM for dielectric surfaces $\mc S$. 
(Alternative formulations may be selected by setting
internal \lss variables; see below.) In the PMCHWT formulation, 
we require that, at each point $\vb x\in\mc S,$ the 
tangential components of the total $\vb E$ and $\vb H$ 
fields be continuous as we pass through $\mc S$:
%====================================================================%
\begin{subequations}
\begin{align}
 \lim_{\eta \to 0}
 \Big[ \vb E\sups{total}(\vb x + \eta\vbhat{n})
      -\vb E\sups{total}(\vb x - \eta\vbhat{n})
 \Big]_{\parallel} 
&= 0 
\\
 \lim_{\eta \to 0}
 \Big[ \vb H\sups{total}(\vb x + \eta\vbhat{n})
      -\vb H\sups{total}(\vb x - \eta\vbhat{n})
 \Big]_{\parallel} 
&= 0.
\label{PMCHWTConditions}
\end{align}
\end{subequations}
%====================================================================%
where the subscript ${\parallel}$ extracts the 
vector components tangential to $\mc S$ at $\vb x$.

To write an integral-equation version of this, analogous to
(\ref{EFIEShorthand}), let $\mc R_1$ and $\mc R_2$ be
the two regions at whose interface $\mc S$ lies, with 
the surface normal to $\mc S$ taken to point away from
$\mc R_2$ and into $\mc R_1$. (Thus, if $\mc S$ is closed,
$\mc R_1$ is its exterior region and $\mc R_2$ is its
interior region; if $\mc S$ is not closed then everything
else goes through in the same way, just without the 
classification of $\mathcal{R}_{1,2}$ as interior or
exterior). Then the integral-equation version of 
(\ref{PMCHWTConditions}) reads
%====================================================================%
\begin{align}
& \sum_{\mc S_s \subset \partial \mc R_1}
   \text{sgn}\big(\mc S_s,\mc R_1\big)
   \Big[ \bmc G(\mc R_1) \star \bmc C_s\Big]_{\parallel}
 -\sum_{\mc S_s \subset \partial \mc R_2} 
   \text{sgn}\big(\mc S_s,\mc R_2\big)
   \Big[ \bmc G(\mc R_2) \star \bmc C_s\Big]_{\parallel}
\nn
&\qquad
=
  \Big[-\bmc F\incro + \bmc F\incrt \Big]_{\parallel}
\label{PMCHWT}
\end{align}
%====================================================================%
The RHS of (\ref{PMCHWT}) describes the fields contributed
by sources inside $\mc R_2$ minus the fields contributed by 
sources inside $\mc R_1$.

Note that, when we consider the contribution made by 
surface $\mc S$ itself to the sums on the LHS,
we have 
$\text{sgn}\big(\mc S, \mc R_1\big)=
 -\text{sgn}\big(\mc S, \mc R_2\big)$,
so the two terms on the LHS wind up \textit{adding}, 
not subtracting, for that surface.

%%%%%%%%%%%%%%%%%%%%%%%%%%%%%%%%%%%%%%%%%%%%%%%%%%%%%%%%%%%%%%%%%%%%%%
%%%%%%%%%%%%%%%%%%%%%%%%%%%%%%%%%%%%%%%%%%%%%%%%%%%%%%%%%%%%%%%%%%%%%%
%%%%%%%%%%%%%%%%%%%%%%%%%%%%%%%%%%%%%%%%%%%%%%%%%%%%%%%%%%%%%%%%%%%%%%
\subsubsection{The equation imposed at points on 
               dielectric object surfaces: N-M\"uller formulation}

The N-M\"uller formulation is an alternative to the PMCHWT
approach that yields a different set of integral equations
whose discretization exhibits different numerical 
behavior.\footnote{
P. Yla-Oijala and M. Taskinen, ``Well-conditioned 
M\"uller formulation for electromagnetic scattering by 
dielectric objects,'' \textit{IEEE Transactions on Antennas
and Propagation}, \textbf{53} 3316 (2005)} To derive 
the equations of this formulation, consider a surface $\mc S$
at the interface of two regions $\mc R_{1,2}$, and let
$\vbhat n$ be the normal to $\mc S$ pointing into 
$\mc R_1$ and away from $\mc R_2$.\footnote{In contrast to
other derivations, such as that of the previous footnote, 
I do \textit{not} use two different symbols to denote 
the two opposite orientations of the surface normal
vector. Throughout this derivation my $\vbhat{n}$ vector 
points always in the \textit{same} direction, namely, out
of $\mc R_2$ into $\mc R_1$. Similarly, there are not two
different $\bmc N$ matrices; there is only one $\bmc N$ 
matrix.} One way to express the 
surface currents is to compute the total fields in $\mc R_1$ 
at points approaching the surface from within $\mc R_1$:
\begin{align}
  \left(\begin{array}{c}
   \vb K(\vb x) \\[3pt] 
   \vb N(\vb x) 
  \end{array}\right)
 &=
  \lim_{\eta \to 0}
  \left(\begin{array}{c}
  +\vbhat{n} \times \vb H\sups{tot}\big(\vb x + \eta\vbhat{n}\big) 
   \nn[3pt]
  -\vbhat{n} \times \vb E\sups{tot}\big(\vb x + \eta\vbhat{n}\big)
  \end{array}\right)
\nonumber
\intertext{or, using the 6-vector notation introduced above,}
  \bmc C 
 &=
  \lim_{\eta \to 0}
  \bmc N \bmc F\big( \vb x + \eta\vbhat{n} \big)
\nn
 &=
  \lim_{\eta \to 0}
  \bmc N \Big[ \bmc F\incro\big( \vb x + \eta\vbhat{n} \big)
               \,\, + \,\, \bmc G(\mc R_1) \star \bmc C
         \Big]
\label{NMuller1}
\end{align}
where 
$$\bmc N = \left(\begin{array}{cc}
  0 & \vbhat{n}\times \\ 
 -\vbhat{n}\times & 0 \\ 
 \end{array}\right)
$$
and again $\bmc F\incro$ are the incident fields arising
only from those field sources inside $\mc R_1$.

On the other hand, another way to derive the same 
surface currents is to compute the total fields in $\mc R_2$ 
at points approaching the surface from within $\mc R_2$:
\numeq{NMuller2}
{\bmc C =
    \bmc N \Big[ \bmc F\incrt\big( \vb x - \eta\vbhat{n} \big)
                  \,\, - \,\, \bmc G(\mc R_2) \star \bmc C
           \Big]
}
Equation (\ref{NMuller1}) and (\ref{NMuller2}) 
are two distinct equations that must hold simultaneously.
If we subtract equation (\ref{NMuller2}) from (\ref{NMuller1})
and operate on both sides with $-\bmc N$, we recover the
PMCHWT equation (\ref{PMCHWT}). On the other hand, 
if we multiply 
(\ref{NMuller1}) by $\vbchi_1$ and 
(\ref{NMuller1}) by $\vbchi_2$---where 
$\vbchi_r$ is the $6\times 6$ constant diagonal matrix 
$$ \vbchi_r 
   = 
   \left(\begin{array}{cc} 
      \mu_r & 0 \\ 
      0 & -\epsilon_r \end{array}\right)
$$
with $\epsilon_r,\mu_r$ the relative material properties of
region $\mc R$---and \textit{add} the two equations instead of
subtracting, we obtain
%====================================================================%
\renewcommand{\arraystretch}{1.5}
$$
 \Big(\vbchi_1 + \vbchi_2\Big)\bmc C
=    \vbchi_1 \bmc N \bmc F\incro 
   + \vbchi_2 \bmc N \bmc F\incrt 
   + \Big[    \vbchi_1 \bmc N \bmc G_1 
            - \vbchi_2 \bmc N \bmc G_2
     \Big]\star \bmc C
$$
%--------------------------------------------------------------------%
or, breaking out the 3x3 block components,
%--------------------------------------------------------------------%
\begin{align}
&
\left(\begin{array}{c}
  \!\!\!\!
  (\mu_1      + \mu_2) \vb K
  \!\!\!\!
  \\
  \!\!\!\!
 -(\epsilon_1 + \epsilon_2) \vb N
  \!\!\!\! 
\end{array}\right)
-
\left(\begin{array}{cc}
  \vbhat{n}\times 
   \big[\mu_1 \BG\supt{ME}_1 - \mu_2 \BG\supt{ME}_2 \big]
&
  \vbhat{n}\times 
   \big[\mu_1 \BG\supt{MM}_1 - \mu_2 \BG\supt{MM}_2 \big]
\\
  \vbhat{n}\times 
   \big[\epsilon_1 \BG\supt{EE}_1 - \epsilon_2 \BG\supt{EE}_2 \big]
  \!\!\!\!
&
  \vbhat{n}\times 
   \big[\epsilon_1 \BG\supt{EM}_1 - \epsilon_2 \BG\supt{EM}_2 \big]
  \!\!\!\!
\end{array}\right) 
 \!\! \star \!\!
\left(\begin{array}{c}
  \!\!
  \vb K 
  \!\! 
  \\
  \!\!
  \vb N 
  \!\!
\end{array}\right)
\nn
%--------------------------------------------------------------------%
&\qquad=
\left(\begin{array}{c}
  \vbhat{n}\times \big[\mu_1 \vb H\incro + \mu_2 \vb H\incrt\big] \\
  \vbhat{n}\times \big[\epsilon_1 \vb E\incro + \epsilon_2 \vb E\incrt\big]
\end{array}\right).
\label{NMuller0}
\end{align}
\renewcommand{\arraystretch}{1.0}

There is a subtlety in equation (\ref{NMuller0})
that is not present in (\ref{PMCHWT}):
In evaluating the convolutions on the LHS of (\ref{NMuller0}),
we must account for the $\delta-$function singularities in 
$\BG\supt{ME}$ and $\BG\supt{EM}$. [These terms 
\textit{cancel} out of (\ref{PMCHWT}), which is why we
didn't need to consider them above]. I think the easiest
way to work out what these are is to consider the 
magnetic field due to an infinite constant sheet 
of $\vb x$-directed electric surface current 
$\vb K$ confined to the $xy$ plane. Then, by a simple 
application of the right-hand rule, the magnetic 
field in the \textit{upper} half space points in the 
\textit{negative} $y$-direction, 
while the magnetic field in the \textit{lower} half 
space points in the \textit{positive} $y$-direction.
Let the upper (lower) half-space
be $\mc R_1$ $(\mc R_2)$. Then we have 
%====================================================================%
$$ \vbhat{n} = K \vbhat{z}, \qquad \vb K= K \vbhat{x}. $$
%====================================================================%
The magnetic field in the upper half-space is
$$ \BG\supt{ME}_1 \star \vb K 
   = -\frac{1}{2}K \vbhat{y}, \qquad  z\ge 0.
$$
In the lower half-space, we find instead
$$ \BG\supt{ME}_2 \star \vb K 
   = +\frac{1}{2}K \vbhat{y}, \qquad  z\le 0.
$$
%====================================================================%
Thus, at $z=0$, we find
\begin{align*}
 \vbhat{n}\times
   \big[\mu_1 \BG\supt{ME}_1 - \mu_2 \BG\supt{ME}_2 \big]\star \vb K
   &=-\frac{(\mu_1 + \mu_2)K}{2}\vbhat{z} \times \vbhat{y}
\\
   &=+\frac{(\mu_1 + \mu_2)K}{2}\vbhat{x}
\\
   &=\frac{(\mu_1 + \mu_2)}{2}\vb K.
\end{align*}
By analogous arguments we find 
$$
 \vbhat{n}\times
   \big[\epsilon_1 \BG\supt{EM}_1 - \epsilon_2 \BG\supt{EM}_2 \big]\star \vb N
   =-\frac{(\epsilon_1 + \epsilon_2)}{2}\vb N.
$$

%%%%%%%%%%%%%%%%%%%%%%%%%%%%%%%%%%%%%%%%%%%%%%%%%%%%%%%%%%%%%%%%%%%%%%
%%%%%%%%%%%%%%%%%%%%%%%%%%%%%%%%%%%%%%%%%%%%%%%%%%%%%%%%%%%%%%%%%%%%%%
%%%%%%%%%%%%%%%%%%%%%%%%%%%%%%%%%%%%%%%%%%%%%%%%%%%%%%%%%%%%%%%%%%%%%%
\subsubsection{Modifications for nonzero surface impedance}

\medskip

\subsubsection*{PEC bodies} 

In the presence of a nonzero surface impedance $Z_s(\vb x)>0$
(corresponding to a finite surface conductivity
 $G_s(\vb x) = \frac{1}{Z_s(\vb x)} < \infty$),
equation (\ref{PECEquation1}) is modified to read 

\numeq{PECEquationZS}
{ \vbhat{n}\times \vb E\sups{total} (\vb x) 
  = 
  \frac{1}{G_s(\vb x)} \vbhat{n}\times \vb K(\vb x)
}
where $\vb K(\vb x)$ is the (unknown) electric surface current
at $\vb x$. 

Note that we use the symbol $G$, not $\sigma$, for surface
conductivity, because this quantity has the dimensions of a 
conductance (current/voltage), not the dimensions of a conductivity
[current/(voltage$\cdot$length)].

\subsubsection*{Dielectric bodies, PMCHWT Formulation} 

In the presence of a nonvanishing surface impedance $Z_s(\vb x)$
(corresponding to a finite surface conductivity, 
 $G_s(\vb x) = \frac{1}{Z_s(\vb x)} < \infty$),
the $\vb E$-field continuity equation [the first of equations
(\ref{PMCHWT})] is unchanged, while the $\vb H$-field
continuity equation is modified to read
%====================================================================%
\begin{subequations}
\begin{align}
 \lim_{\eta \to 0} \vbhat{n}\times 
 \Big[ \vb H\sups{total}(\vb x + \eta\vbhat{n})
      -\vb H\sups{total}(\vb x - \eta\vbhat{n})
 \Big] 
&= G(\vb x) \vb E\sups{total}(\vb x)
\\
&= -G(\vb x) \vbhat{n} \times \vb{N}(\vb x)
\label{DielectricEquationZS}
\end{align}
\end{subequations}
is the (unknown) magnetic surface current at $\vb x$.

%%%%%%%%%%%%%%%%%%%%%%%%%%%%%%%%%%%%%%%%%%%%%%%%%%%%%%%%%%%%%%%%%%%%%%
%%%%%%%%%%%%%%%%%%%%%%%%%%%%%%%%%%%%%%%%%%%%%%%%%%%%%%%%%%%%%%%%%%%%%%
%%%%%%%%%%%%%%%%%%%%%%%%%%%%%%%%%%%%%%%%%%%%%%%%%%%%%%%%%%%%%%%%%%%%%%
\newpage
\section{Introduction of the $-Z_0$ Prefactor in the 
         Magnetic Current Expansion} 

In rough schematic form, the equations derived in the 
previous section take the form 
%====================================================================%
$$ \left( \begin{array}{cc}
   \BG\supt{EE} & \BG\supt{EM} \\
   \BG\supt{ME} & \BG\supt{MM} \\
   \end{array} \right)
   *
   \left( \begin{array}{c} \vb K \\  \vb N\end{array}\right)
   = 
   -
   \left( \begin{array}{c}
   \vb E\sups{inc} \\
   \vb H\sups{inc} \\
   \end{array}\right)
$$
%====================================================================%
where $*$ denotes a convolution operation.

The matrix kernel on the LHS is not symmetric because 
$\BG\supt{EM} = -\BG\supt{ME}$. To remedy this, I 
\textbf{(a)} scale the magnetic current by $-1/Z_0$, and then
\textbf{(b)} divide the upper row of the system by $Z_0$ to obtain
the following symmetric 
system\footnote{L.~N. Medgyesi-Mitschang, J.~M. Putnam, and M.~B. Gedera, ``Generalized method
  of moments for three-dimensional penetrable scatterers,'' \emph{J. Opt. Soc.
  Am. A}, vol.~11, no.~4, pp. 1383--1398, Apr 1994.}:
%====================================================================%
\renewcommand{\arraystretch}{2.0}
\numeq{PMCHWTSystem}
{ \left( \begin{array}{cc}
   \displaystyle{ \frac{1}{Z_0} \BG\supt{EE}} 
   & 
   \displaystyle{-\BG\supt{EM}}
   \\
   \displaystyle{\BG\supt{ME}}
   & 
   \displaystyle{-Z_0\BG\supt{MM}} \\
   \end{array} \right)
   *
   \left( \begin{array}{c} 
      \displaystyle{\vb K} 
      \\
      \displaystyle{-\frac{1}{Z_0}\vb N}
   \end{array}\right)
   = 
   -
   \left( \begin{array}{c}
   \displaystyle{\frac{1}{Z_0} \vb E\sups{inc}} 
   \\
   \displaystyle{\vb H\sups{inc}}
   \end{array}\right)
}
\renewcommand{\arraystretch}{1.0}
Equation (\ref{PMCHWTSystem}) is the actual linear system
solved by \ls. 

%%%%%%%%%%%%%%%%%%%%%%%%%%%%%%%%%%%%%%%%%%%%%%%%%%%%%%%%%%%%%%%%%%%%%%
%%%%%%%%%%%%%%%%%%%%%%%%%%%%%%%%%%%%%%%%%%%%%%%%%%%%%%%%%%%%%%%%%%%%%%
%%%%%%%%%%%%%%%%%%%%%%%%%%%%%%%%%%%%%%%%%%%%%%%%%%%%%%%%%%%%%%%%%%%%%%
\newpage
\section{The Equations Solved by \ls: Discrete Forms}

\subsubsection*{Discretization Procedure} 

\lss uses a two-step procedure to discretize the integral equations 
derived in the previous sections. 
Briefly, we approximate the $\vb K$ and $\vb N$ surface-current 
distributions as expansions in RWG basis functions, then 
Galerkin-test the resulting equations again with the RWG basis
functions. 

In more detail,

\begin{enumerate}
 \item First, \lss approximates surface currents as expansions 
       in RWG basis functions.

       For electric currents on the surfaces of PEC objects, 
       we have the expansion
       \begin{subequations}
       %====================================================================%
       \begin{equation}
         \vb K(\vb x) \approx \sum K_{\alpha n} \vb f_{\alpha n}(\vb x)
       \end{equation}
       %====================================================================%
       where $\alpha$ runs over all PEC objects in the geometry
       and $n$ runs over all RWG basis functions on object 
       $\mathcal{O}_\alpha$.

       For electric and magnetic currents on the surfaces of 
       dielectric objects, we have the expansions
       %====================================================================%
       \begin{equation}
           \vb K(\vb x) \approx \sum K_{\beta n} \vb f_{\beta n}(\vb x), 
           \qquad 
           \vb N(\vb x) \approx -Z_0 \sum N_{\beta n} \vb f_{\beta n}(\vb x)
       \end{equation}
       %====================================================================%
       \label{KNExpansions}
       \end{subequations}
       \noindent where $\beta $ runs over all PEC objects in the geometry
       and $n$ runs over all RWG basis functions on object
       $\mathcal{O}_\beta$. (The rationale for the prefactor $-Z_0$ 
       was explained in the previous section).

 \item Second, \lss obtains one equation for each of the unknown
       $K$ and $N$ coefficients in (\ref{KNExpansions}) by proceeding
       as follows.

       First, we take the inner product of equation
       (\ref{PECEquationBreakdown})
       with each RWG basis function $\vb f_{\alpha n}$ defined on 
       the surface of each PEC object. This gives us one equation
       for each of the $K_{\alpha n}$ coefficients in 
       (\ref{KNExpansions}a).

       Second, we take the inner product of equation
       (\ref{DielectricEquationBreakdown})
       with each RWG basis function $\vb f_{\beta n}$ defined on 
       the surface of each dielectric object. We associate the 
       resulting equation with the coefficient $K_{\beta n}$ in
       (\ref{KNExpansions}b). Then, we take the inner product of 
       the magnetic analogue of 
       (\ref{DielectricEquationBreakdown}) (which, as stated above,
       is identical to (\ref{DielectricEquationBreakdown}) with
       the replacements $\{\BG\supt{EE}, \BG\supt{EM}\} \to 
       \{\BG\supt{ME}, \BG\supt{MM}\}$) with $\vb f_{\beta n})$
       and associate the resulting equation with the coefficient
       $N_{\beta n}$ in (\ref{KNExpansions}b).
\end{enumerate}

\subsubsection*{Discretized Version of Equation 
                (\ref{PECEquationBreakdown})}

We consider again the setting of Figure \ref{EmbeddedObjects}:
We have a PEC object $\mathcal{O}_\beta$, embedded in an object
$\mathcal{O}_\alpha$ (which may be the environment 
$\mathcal{O}_e$); also embedded in $\mathcal{O}_\alpha$ are
additional PEC object(s) $\mathcal{O}_{\beta^\prime}$ and dielectric 
object(s) $\mathcal{O}_\gamma.$ 
Inserting expansions (\ref{KNExpansions}) 
into (\ref{PECEquationBreakdown})
and Galerkin-testing with a basis function $\vb f_{\beta m}$ on 
the surface of $\mathcal{O}_\beta$ yields 
(after dividing both 
sides of the equation by $Z_0$ as per equation (\ref{PMCHWTSystem}):
%====================================================================%
\begin{align}
&\sum_{n=1}^{\texttt{NEdges}(\beta)} 
   \Big\langle 
         \vb f_{\beta m} 
   \Big| \frac{1}{Z_0} \BG\supt{EE}(\alpha) \Big|
         \vb f_{\beta n}
   \Big\rangle 
         K_{\beta n}
\nn
%--------------------------------------------------------------------%
+&\sum_{n=1}^{\texttt{NEdges}(\beta^\prime)} 
   \Big\langle 
         \vb f_{\beta m} 
   \Big| \frac{1}{Z_0} \BG\supt{EE}(\alpha) \Big| 
         \vb f_{\beta^\prime n}
   \Big\rangle 
         K_{\beta^\prime n}
\nn
%--------------------------------------------------------------------%
+&\sum_{n=1}^{\texttt{NEdges}(\gamma)} \bigg\{
   \Big\langle 
         \vb f_{\beta m} 
   \Big| \frac{1}{Z_0} \BG\supt{EE}(\alpha) \Big| 
         \vb f_{\gamma n}
   \Big\rangle 
         K_{\gamma n}
   \,-
   \Big\langle 
         \vb f_{\beta m} 
   \Big| \BG\supt{EM}(\alpha) \Big| 
         \vb f_{\gamma n}
   \Big\rangle 
         N_{\gamma n}
  \bigg\}
\nn
%--------------------------------------------------------------------%
-&\sum_{n=1}^{\texttt{NEdges}(\alpha)} \bigg\{
   \Big\langle 
         \vb f_{\beta m} 
   \Big| \frac{1}{Z_0} \BG\supt{EE}(\alpha) \Big| 
         \vb f_{\alpha n}
   \Big\rangle 
         K_{\alpha n}
   \,-
   \Big\langle 
         \vb f_{\beta m} 
   \Big| \BG\supt{EM}(\alpha) \Big| 
         \vb f_{\alpha n}
   \Big\rangle 
         N_{\alpha n}
  \bigg\}
\nn
%--------------------------------------------------------------------%
&=-\frac{1}{Z_0} \chi_\alpha\sups{inc} 
   \Big\langle \vb f_{\beta m} \Big| \vb E\sups{inc} \Big\rangle.
 \label{DiscretizedPECEquation}
\end{align}
%====================================================================%

\subsubsection*{Discretized Version of Equation
                (\ref{DielectricEquationBreakdown}) and its 
                Magnetic Analogue}

Again in the setting of Figure \ref{EmbeddedObjects}, we 
Galerkin-test equation (\ref{DielectricEquationBreakdown}) 
with a basis function $\vb f_{\gamma m}$ on the surface
of object $\mathcal{O}_\gamma$ to find (again, dividing through
by $Z_0$ as per (\ref{PMCHWTSystem})):
%====================================================================%
\begin{subequations}
\begin{align}
&\sum_{n=1}^{\texttt{NEdges}(\gamma)}  \bigg\{
   \Big\langle 
         \vb f_{\gamma m} 
   \Big| \frac{1}{Z_0} \BG\supt{EE}(\alpha) + \frac{1}{Z_0} \BG\supt{EE}(\gamma) \Big|
         \vb f_{\gamma n}
   \Big\rangle 
         K_{\gamma n}
   -
   \Big\langle 
         \vb f_{\gamma m} 
   \Big| \BG\supt{EM}(\alpha) + \BG\supt{EM}(\gamma) \Big|
         \vb f_{\gamma n}
   \Big\rangle 
         N_{\gamma n}
  \bigg\}
\nn
%--------------------------------------------------------------------%
&\sum_{n=1}^{\texttt{NEdges}(\gamma^\prime)}  \bigg\{
   \Big\langle 
         \vb f_{\gamma m} 
   \Big| \frac{1}{Z_0}\BG\supt{EE}(\alpha) \Big|
         \vb f_{\gamma^\prime n}
   \Big\rangle 
         K_{\gamma^\prime n}
   -
   \Big\langle 
         \vb f_{\gamma m} 
   \Big| \BG\supt{EM}(\alpha) \Big|
         \vb f_{\gamma^\prime n}
   \Big\rangle 
         N_{\gamma^\prime n}
  \bigg\}
\nn
%--------------------------------------------------------------------%
+&\sum_{n=1}^{\texttt{NEdges}(\beta)} 
   \Big\langle 
         \vb f_{\gamma m} 
   \Big| \frac{1}{Z_0} \BG\supt{EE}(\alpha) \Big| 
         \vb f_{\beta n}
   \Big\rangle 
         K_{\beta n}
\nn
%--------------------------------------------------------------------%
-&\sum_{n=1}^{\texttt{NEdges}(\alpha)} \bigg\{
   \Big\langle 
         \vb f_{\gamma m} 
   \Big| \frac{1}{Z_0} \BG\supt{EE}(\alpha) \Big| 
         \vb f_{\alpha n}
   \Big\rangle 
         K_{\alpha n}
   \,-
   \Big\langle 
         \vb f_{\gamma m} 
   \Big| \BG\supt{EM}(\alpha) \Big| 
         \vb f_{\alpha n}
   \Big\rangle 
         N_{\alpha n}
  \bigg\}
\nn
%--------------------------------------------------------------------%
-&\sum_{n=1}^{\texttt{NEdges}(\delta)} \bigg\{
   \Big\langle 
         \vb f_{\gamma m} 
   \Big| \frac{1}{Z_0} \BG\supt{EE}(\gamma) \Big| 
         \vb f_{\delta n}
   \Big\rangle 
         K_{\delta n}
   \,-
   \Big\langle 
         \vb f_{\gamma m} 
   \Big| \BG\supt{EM}(\gamma) \Big| 
         \vb f_{\delta n}
   \Big\rangle 
         N_{\delta n}
  \bigg\}
\nn
%--------------------------------------------------------------------%
&=\frac{1}{Z_0} \Big[\chi_\gamma\sups{inc} - \chi_\alpha\sups{inc}\Big] 
   \Big\langle \vb f_{\gamma m} \Big| \vb E\sups{inc} \Big\rangle.
\end{align}

\subsubsection*{Discretized Version of the Magnetic Analogue of 
                Equation (\ref{DielectricEquationBreakdown})}

Finally,, we Galerkin-test the magnetic-field analogue of 
(\ref{DielectricEquationBreakdown}) to find
\begin{align}
&\sum_{n=1}^{\texttt{NEdges}(\gamma)}  \bigg\{
   \Big\langle 
         \vb f_{\gamma m} 
   \Big| \BG\supt{ME}(\alpha) + \BG\supt{ME}(\gamma) \Big|
         \vb f_{\gamma n}
   \Big\rangle 
         K_{\gamma n}
   -
   \Big\langle 
         \vb f_{\gamma m} 
   \Big| Z_0\BG\supt{MM}(\alpha) + Z_0\BG\supt{MM}(\gamma) \Big|
         \vb f_{\gamma n}
   \Big\rangle 
         N_{\gamma n}
  \bigg\}
\nn
%--------------------------------------------------------------------%
&\sum_{n=1}^{\texttt{NEdges}(\gamma^\prime)}  \bigg\{
   \Big\langle 
         \vb f_{\gamma m} 
   \Big| \BG\supt{ME}(\alpha) \Big|
         \vb f_{\gamma^\prime n}
   \Big\rangle 
         K_{\gamma^\prime n}
   -
   \Big\langle 
         \vb f_{\gamma m} 
   \Big| Z_0\BG\supt{MM}(\alpha) \Big|
         \vb f_{\gamma^\prime n}
   \Big\rangle 
         N_{\gamma^\prime n}
  \bigg\}
\nn
%--------------------------------------------------------------------%
+&\sum_{n=1}^{\texttt{NEdges}(\beta)} 
   \Big\langle 
         \vb f_{\gamma m} 
   \Big| \BG\supt{ME}(\alpha) \Big| 
         \vb f_{\beta n}
   \Big\rangle 
         K_{\beta n}
\nn
%--------------------------------------------------------------------%
-&\sum_{n=1}^{\texttt{NEdges}(\alpha)} \bigg\{
   \Big\langle 
         \vb f_{\gamma m} 
   \Big| \BG\supt{ME}(\alpha) \Big| 
         \vb f_{\alpha n}
   \Big\rangle 
         K_{\alpha n}
   \,-
   \Big\langle 
         \vb f_{\gamma m} 
   \Big| Z_0 \BG\supt{MM}(\alpha) \Big| 
         \vb f_{\alpha n}
   \Big\rangle 
         N_{\alpha n}
  \bigg\}
\nn
%--------------------------------------------------------------------%
-&\sum_{n=1}^{\texttt{NEdges}(\delta)} \bigg\{
   \Big\langle 
         \vb f_{\gamma m} 
   \Big| \BG\supt{ME}(\gamma) \Big| 
         \vb f_{\delta n}
   \Big\rangle 
         K_{\delta n}
   \,-
   \Big\langle 
         \vb f_{\gamma m} 
   \Big| Z_0 \BG\supt{MM}(\gamma) \Big| 
         \vb f_{\delta n}
   \Big\rangle 
         N_{\delta n}
  \bigg\}
\nn
%--------------------------------------------------------------------%
&=\Big[\chi_\gamma\sups{inc} - \chi_\alpha\sups{inc}\Big] 
   \Big\langle \vb f_{\gamma m} \Big| \vb H\sups{inc} \Big\rangle.
\end{align}
\label{DiscretizedDielectricEquation}
\end{subequations}
%====================================================================%
%%%%%%%%%%%%%%%%%%%%%%%%%%%%%%%%%%%%%%%%%%%%%%%%%%%%%%%%%%%%%%%%%%%%%%
%%%%%%%%%%%%%%%%%%%%%%%%%%%%%%%%%%%%%%%%%%%%%%%%%%%%%%%%%%%%%%%%%%%%%%
%%%%%%%%%%%%%%%%%%%%%%%%%%%%%%%%%%%%%%%%%%%%%%%%%%%%%%%%%%%%%%%%%%%%%%
\newpage
\section{Structure of the BEM System}

The discretization procedure of the previous section results in a 
linear system of the form
\numeq{BEMSystem}
{ \Big[ \vb M \Big]  \cdot \Big[ \vb{KN} \Big] = \Big[ \vb{RHS} \Big ] }
where the vector $\big[\vb{KN}\big]$ contains the unknown 
$K$ and $N$ coefficients from equation (\ref{KNExpansions}),
the matrix $\big[\vb M\big]$ is the ``BEM matrix,''
and the right-hand-side vector $\big[\vb{RHS}\big]$ 
depends on the incident fields.

In the remainder of this section we will describe the structure
of each of the entities in equation (\ref{BEMSystem}). 

%%%%%%%%%%%%%%%%%%%%%%%%%%%%%%%%%%%%%%%%%%%%%%%%%%%%%%%%%%%%%%%%%%%%%%
%%%%%%%%%%%%%%%%%%%%%%%%%%%%%%%%%%%%%%%%%%%%%%%%%%%%%%%%%%%%%%%%%%%%%%
%%%%%%%%%%%%%%%%%%%%%%%%%%%%%%%%%%%%%%%%%%%%%%%%%%%%%%%%%%%%%%%%%%%%%%
\subsection{Structure of the coefficient vector}

The $\vb{KN}$ vector contains the $K$ and $N$ coefficients
in (\ref{KNExpansions}), ordered as follows:

\begin{enumerate}
 \item All coefficients for object $\mathcal{O}_1$ come first, 
       followed by all coefficients for object $\mathcal{O}_2$, etc.

       (Object indices correspond with the order in which the 
        objects were specified in the \texttt{.rwggeo} file 
        used as input to \ls.)
       
 \item Within the portion of the vector corresponding to a 
       dielectric object, the electric and magnetic coefficients
       for the first RWG basis function come first, followed
       by the electric and magnetic coefficients
       for the second RWG basis function, etc.

\end{enumerate}

Thus, for a geometry consisting of object $\mathcal{O}_1$ (PEC)
with $M$ interior edges in its surface discretization
and object $\mathcal{O}_2$ (dielectric) with $N$ interior edges
in its surface discretization, the $\vb{KN}$ vector has dimension
$M+2N$ and looks like
%====================================================================%
$$ \vb{KN}=\left(\begin{array}{c}
   K_{11} \\ 
   K_{12} \\ 
   \vdots \\
   K_{1M} \\ 
   K_{21} \\ 
   N_{21} \\ 
   \vdots \\
   K_{2N} \\ 
   N_{2N}
   \end{array}\right).
$$
%====================================================================%

To compute the index of any given coefficient $K_{\alpha n}$ 
or $N_{\alpha n}$ within the $\vb{KN}$ vector,
it is useful first to define functions \texttt{NBF}($\alpha$) 
and \texttt{BFIndexOffset}($\alpha$). The former of these 
is just the number of basis functions on object $\alpha$,
i.e.
$$ \texttt{NBF}(\alpha) = 
   \begin{cases}
   \texttt{NEdges}(\alpha), \qquad &\text{if object $\mathcal{O}_\alpha$
                                          is PEC} \\
   2\cdot \tt{NEdges}(\alpha), 
                        \qquad &\text{if object $\mathcal{O}_\alpha$
                                      is dielectric.} \\
   \end{cases}
$$
(Here \texttt{NEdges}($\alpha$) is the number of interior edges 
in the surface discretization of object $\mathcal{O}_\alpha$.)

The function \texttt{BFIndexOffset}($\alpha$) is the index
within the $\vb{KN}$ vector of the first coefficient corresponding
to object $\mathcal{O}_\alpha$; thus 
\begin{align*}
 \texttt{BFIndexOffset}(1) &= 1 \\
 \texttt{BFIndexOffset}(2) &= 1 + \tt{NBF}(1) \\
 \texttt{BFIndexOffset}(3) &= 1 + \tt{NBF}(1) + \tt{NBF}(2)
\end{align*}
et cetera.

Then I can write the following relations for the indices 
with the $\vb{KN}$ vector of individual $K, N$ coefficients.
(Note that these are one-based indices, which must be translated
into zero-based indices for use in \texttt{C++} code.)
%====================================================================%
\begin{align*}
\mathcal{I}(K, \alpha, n)
&\equiv \text{Index of coefficient $K_{\alpha n}$
              within the $\vb{KN}$ vector}
\\
&=\begin{cases}
   \texttt{BFIndexOffset}(\alpha) + n-1, 
   \qquad &\text{if $\mathcal{O}_\alpha$ is PEC} \\
   \texttt{BFIndexOffset}(\alpha) + 2(n-1),
   \qquad &\text{if $\mathcal{O}_\alpha$ is dielectric}
  \end{cases}
\\[5pt]
\mathcal{I}(N, \alpha, n)
&\equiv \text{Index of coefficient $N_{\alpha n}$ 
              within the $\vb{KN}$ vector}
\\
&=\texttt{BFIndexOffset}(\alpha) + 2(n-1)+1.
\end{align*}
%====================================================================%

%%%%%%%%%%%%%%%%%%%%%%%%%%%%%%%%%%%%%%%%%%%%%%%%%%%%%%%%%%%%%%%%%%%%%%
%%%%%%%%%%%%%%%%%%%%%%%%%%%%%%%%%%%%%%%%%%%%%%%%%%%%%%%%%%%%%%%%%%%%%%
%%%%%%%%%%%%%%%%%%%%%%%%%%%%%%%%%%%%%%%%%%%%%%%%%%%%%%%%%%%%%%%%%%%%%%
\subsection{Structure of the RHS vector}

The structure of the RHS vector mirrors the structure of 
the coefficient vector: 

\begin{itemize}
 \item If the $n$th element of the coefficient vector is a $K$ 
       coefficient (i.e. the electric surface-current expansion 
       coefficient associated with some RWG basis function $\vb b$), 
       then the $n$th element of the RHS vector is minus the 
       inner product of the incident electric field with $\vb b$,
       divided by $Z_0$.
 \item If the $n$th element of the coefficient vector is an $N$ 
       coefficient (i.e. the magnetic surface-current expansion 
       coefficient associated with some RWG basis function $\vb b$),
       then the $n$th element of the RHS vector is minus the 
       inner product of the incident magnetic field with $\vb b$.
\end{itemize}

For the example considered above, consisting of a PEC surface
with $M$ interior edges and a dielectric surface with $N$ 
interior edges, the elements of the RHS vector would be 

%====================================================================%
\setlength{\extrarowheight}{3pt}
$$ \vb{RHS}=-\left(\begin{array}{l}
   \langle\vb E\sups{inc}, \vb b_{11} \rangle  / Z_0  
\\
   \langle\vb E\sups{inc}, \vb b_{12} \rangle  / Z_0 
\\
   \vdots 
\\
   \langle\vb E\sups{inc}, \vb b_{1M} \rangle  / Z_0 
\\
   \langle\vb E\sups{inc}, \vb b_{21} \rangle  / Z_0 
\\
   \langle\vb H\sups{inc}, \vb b_{21} \rangle 
\\
   \langle\vb E\sups{inc}, \vb b_{22} \rangle / Z_0 
\\
   \vdots
\\
   \langle\vb H\sups{inc}, \vb b_{2N} \rangle
   \end{array}\right).
$$

%%%%%%%%%%%%%%%%%%%%%%%%%%%%%%%%%%%%%%%%%%%%%%%%%%%%%%%%%%%%%%%%%%%%%%
%%%%%%%%%%%%%%%%%%%%%%%%%%%%%%%%%%%%%%%%%%%%%%%%%%%%%%%%%%%%%%%%%%%%%%
%%%%%%%%%%%%%%%%%%%%%%%%%%%%%%%%%%%%%%%%%%%%%%%%%%%%%%%%%%%%%%%%%%%%%%
\subsection{Structure of the BEM matrix}

Consider two basis functions: $\vb f_{\alpha m}$, corresponding
to the $m$th interior edge of object $\mathcal{O}_\alpha$,
and $\vb f_{\beta n}$, corresponding to the $n$th interior edge
of object $\mathcal{O}_\beta.$ 

Let $\vb A$ be the index of the medium through which the 
objects interact. (If we have $\texttt{NObj}$ objects in 
our geometry, then either $1\le \vb A \le \texttt{NObj}$ 
or else $\vb A=e$ for the external medium.) Thus,

\begin{itemize}
 \item If $\mathcal{O}_\alpha$ is contained in $\mathcal{O}_\beta$, 
       then $\vb A=\beta$.
 \item If $\mathcal{O}_\beta$ is contained in $\mathcal{O}_\alpha$, 
       then $\vb A=\alpha$.
 \item If $\mathcal{O}_\alpha$ and $\mathcal{O}_\alpha$ are both
       contained in the same object $\mathcal{O}_\gamma$ (which may
       be the external medium $\mathcal{O}_e$), then  
       then $\vb A=\gamma$.
 \item If none of the above are true, then the two objects do not
       interact and the corresponding block of the BEM matrix is
       zero.
\end{itemize}

Define a symbol $\texttt{Sign}$ to have value $-1$ in the first
two cases (i.e. when one of $\mathcal{O}_\alpha$, $\mathcal{O}_\beta$
is contained inside the other), while $\texttt{Sign}=+1$ otherwise.

Then the two basis elements contribute a 
$1\times 1$, $1\times 2$, $2\times 1$, or $2\times 2$
block of matrix elements to the BEM matrix, which may be determined
by looking at equations 
(\ref{DiscretizedPECEquation}) 
and 
(\ref{DiscretizedDielectricEquation}) 
as follows:
%%%%%%%%%%%%%%%%%%%%%%%%%%%%%%%%%%%%%%%%%%%%%%%%%%%%%%%%%%%%%%%%%%%%%%
%%%%%%%%%%%%%%%%%%%%%%%%%%%%%%%%%%%%%%%%%%%%%%%%%%%%%%%%%%%%%%%%%%%%%%
%%%%%%%%%%%%%%%%%%%%%%%%%%%%%%%%%%%%%%%%%%%%%%%%%%%%%%%%%%%%%%%%%%%%%%
\begin{enumerate}
%====================================================================%
\item \textbf{Both objects are PEC:}
%====================================================================%
$$\begin{array}{lclcl}
   M\Big( \mathcal{I}^K_{\alpha m}, \mathcal{I}^K_{\beta n} \Big) 
  &=&
  \displaystyle{ \frac{1}{Z_0} 
                 \Big\langle 
                 \vb f_{\alpha m} 
                 \Big| \BG\supt{EE}(\vb A) \Big|
                 \vb f_{\beta n} 
                 \Big\rangle
               }
%--------------------------------------------------------------------%
  &=&
  \displaystyle{
  ik\supt{A} Z\supt{A}
                \Big\langle 
                \vb f_{\alpha m} 
                \Big| \vb G(k\supt{A}) \Big|
                \vb f_{\beta n} 
                \Big\rangle
               }
\end{array}$$
%====================================================================%
\item \textbf{ $\mathcal{O}_\alpha$ is PEC, 
               $\mathcal{O}_\beta$ is dielectric:}
%====================================================================%
$$\begin{array}{lclcl}
 M\Big( \mathcal{I}^K_{\alpha m}, \mathcal{I}^K_{\beta n} \Big) 
  &=& 
  \displaystyle{ \frac{\texttt{Sign}}{Z_0} 
                 \Big\langle \vb f_{\alpha m} 
                 \Big| \BG\supt{EE}(\vb A) \Big|
                 \vb f_{\beta n} 
                 \Big\rangle
               }
%--------------------------------------------------------------------%
  &=&
  \displaystyle{ \texttt{Sign}\cdot ik\supt{A} Z\supt{A}
                 \Big\langle \vb f_{\alpha m} 
                 \Big| \vb G(k\supt{A}) \Big|
                 \vb f_{\beta n} 
                 \Big\rangle
               }
\\[10pt]
%--------------------------------------------------------------------%
%--------------------------------------------------------------------%
 M\Big( \mathcal{I}^K_{\alpha m}, \mathcal{I}^N_{\beta n} \Big) 
  &=&
  \displaystyle{ -\texttt{Sign}\cdot
                 \Big\langle 
                 \vb f_{\alpha m} 
                 \Big| \BG\supt{EM}(\vb A) \Big|
                 \vb f_{\beta n} 
                 \Big\rangle
               }
%--------------------------------------------------------------------%
  &=&
  \displaystyle{ -\texttt{Sign}\cdot ik\supt{A}
                  \Big\langle 
                  \vb f_{\alpha m} 
                  \Big| \vb C(k\supt{A})\Big|
                  \vb f_{\beta n} 
                  \Big\rangle
               }
\end{array}$$
%====================================================================%
\item \textbf{ $\mathcal{O}_\alpha$ is dielectric, 
               $\mathcal{O}_\beta$ is PEC:}
%====================================================================%
$$\begin{array}{lclcl}
 M\Big( \mathcal{I}^K_{\alpha m}, \mathcal{I}^K_{\beta n} \Big) 
  &=& 
  \displaystyle{ \frac{\texttt{Sign}}{Z_0} 
                 \Big\langle \vb f_{\alpha m} 
                 \Big| \BG\supt{EE}(\vb A) \Big|
                 \vb f_{\beta n} 
                 \Big\rangle
               }
%--------------------------------------------------------------------%
  &=&
  \displaystyle{ \texttt{Sign}\cdot ik\supt{A} Z\supt{A}
                 \Big\langle \vb f_{\alpha m} 
                 \Big| \vb G(k\supt{A}) \Big|
                 \vb f_{\beta n} 
                 \Big\rangle
               }
\\[10pt]
%--------------------------------------------------------------------%
%--------------------------------------------------------------------%
 M\Big( \mathcal{I}^N_{\alpha m}, \mathcal{I}^K_{\beta n} \Big) 
  &=&
  \displaystyle{ \texttt{Sign}\cdot
                 \Big\langle 
                 \vb f_{\alpha m} 
                 \Big| \BG\supt{ME}(\vb A) \Big|
                 \vb f_{\beta n} 
                 \Big\rangle
               }
%--------------------------------------------------------------------%
  &=&
  \displaystyle{ -\texttt{Sign}\cdot ik\supt{A}
                  \Big\langle 
                  \vb f_{\alpha m} 
                  \Big| \vb C(k\supt{A})\Big|
                  \vb f_{\beta n} 
                  \Big\rangle
               }
\end{array}$$
%====================================================================%
\item \textbf{ $\mathcal{O}_\alpha, \mathcal{O}_\beta$ are both
               dielectric and $\mathcal{O}_\alpha \ne \mathcal{O}_\beta$:}
%====================================================================
$$\begin{array}{lclcl}
 M\Big( \mathcal{I}^K_{\alpha m}, \mathcal{I}^K_{\beta n} \Big) 
  &=& 
  \displaystyle{ \frac{\texttt{Sign}}{Z_0} 
                 \Big\langle \vb f_{\alpha m} 
                 \Big| \BG\supt{EE}(\vb A) \Big|
                 \vb f_{\beta n} 
                 \Big\rangle
               }
%--------------------------------------------------------------------%
  &=&
  \displaystyle{ \texttt{Sign}\cdot ik\supt{A} Z\supt{A}
                 \Big\langle \vb f_{\alpha m} 
                 \Big| \vb G(k\supt{A}) \Big|
                 \vb f_{\beta n} 
                 \Big\rangle
               }
\\[10pt]
%--------------------------------------------------------------------%
%--------------------------------------------------------------------%
 M\Big( \mathcal{I}^K_{\alpha m}, \mathcal{I}^N_{\beta n} \Big) 
  &=&
  \displaystyle{ -\texttt{Sign}\cdot
                 \Big\langle 
                 \vb f_{\alpha m} 
                 \Big|\BG\supt{EM}(\vb A) \Big|
                 \vb f_{\beta n} 
                 \Big\rangle
               }
%--------------------------------------------------------------------%
  &=&
  \displaystyle{ -\texttt{Sign}\cdot ik\supt{A}
                  \Big\langle 
                  \vb f_{\alpha m} 
                  \Big| \vb C(k\supt{A})\Big|
                  \vb f_{\beta n} 
                  \Big\rangle
               }
\\[10pt]
%--------------------------------------------------------------------%
%--------------------------------------------------------------------%
 M\Big( \mathcal{I}^N_{\alpha m}, \mathcal{I}^K_{\beta n} \Big) 
  &=&
  \displaystyle{ \texttt{Sign}\cdot
                 \Big\langle 
                 \vb f_{\alpha m} 
                 \Big|\BG\supt{ME}(\vb A) \Big|
                 \vb f_{\beta n} 
                 \Big\rangle
               }
%--------------------------------------------------------------------%
  &=&
  \displaystyle{ -\texttt{Sign}\cdot ik\supt{A}
                  \Big\langle 
                  \vb f_{\alpha m} 
                  \Big| \vb C(k\supt{A})\Big|
                  \vb f_{\beta n} 
                  \Big\rangle
               }
\\[10pt]
%--------------------------------------------------------------------%
%--------------------------------------------------------------------%
 M\Big( \mathcal{I}^N_{\alpha m}, \mathcal{I}^N_{\beta n} \Big) 
  &=&
  \displaystyle{ -\texttt{Sign}\cdot Z_0 
                 \Big\langle 
                 \vb f_{\alpha m} 
                 \Big|\BG\supt{MM}(\vb A) \Big|
                 \vb f_{\beta n} 
                 \Big\rangle
               }
%--------------------------------------------------------------------%
  &=&
  \displaystyle{ -\frac{\texttt{Sign}\cdot ik\supt{A}}{Z\supt{A}}
                  \Big\langle 
                  \vb f_{\alpha m} 
                  \Big| \vb G(k\supt{A})\Big|
                  \vb f_{\beta n} 
                  \Big\rangle
               }
\end{array}$$
%====================================================================%
\item \textbf{ $\mathcal{O}_\alpha, \mathcal{O}_\beta$ are both
               dielectric and $\mathcal{O}_\alpha = \mathcal{O}_\beta$:}

%====================================================================%
This case is identical to the previous case, but now the matrix elements
are augmented by additional contributions describing the basis functions
interacting through the medium inside the object. Let the index
of this medium be $\vb B$. (We have $\vb B=\alpha=\beta$.) Then
the matrix elements are 
%====================================================================
\begin{align*}
 M\Big( \mathcal{I}^K_{\alpha m}, \mathcal{I}^K_{\beta n} \Big) 
 &= \frac{1}{Z_0}
    \Big\langle \vb f_{\alpha m} 
    \Big| \BG\supt{EE}(\vb A) + \BG\supt{EE}(\vb B) \Big|
                \vb f_{\beta n} 
    \Big\rangle
\\[5pt]
 &= ik\supt{A} Z\supt{A}
    \Big\langle \vb f_{\alpha m} 
                 \Big| \vb G(k\supt{A}) \Big|
                 \vb f_{\beta n} 
    \Big\rangle
    +
    ik\supt{B} Z\supt{B}
    \Big\langle \vb f_{\alpha m} 
                 \Big| \vb G(k\supt{B}) \Big|
                 \vb f_{\beta n} 
    \Big\rangle
\\[10pt]
%--------------------------------------------------------------------%
 M\Big( \mathcal{I}^K_{\alpha m}, \mathcal{I}^N_{\beta n} \Big) 
 &= -\Big\langle \vb f_{\alpha m} 
    \Big| \BG\supt{EM}(\vb A) + \BG\supt{EM}(\vb B) \Big|
                \vb f_{\beta n} 
    \Big\rangle
\\[5pt]
 &= -ik\supt{A}
    \Big\langle \vb f_{\alpha m} 
                 \Big| \vb C(k\supt{A}) \Big|
                 \vb f_{\beta n} 
    \Big\rangle
    -ik\supt{B}
    \Big\langle \vb f_{\alpha m} 
                 \Big| \vb C(k\supt{B}) \Big|
                 \vb f_{\beta n} 
    \Big\rangle
\\[10pt]
%--------------------------------------------------------------------%
 M\Big( \mathcal{I}^N_{\alpha m}, \mathcal{I}^K_{\beta n} \Big) 
 &= \Big\langle \vb f_{\alpha m} 
    \Big| \BG\supt{ME}(\vb A) + \BG\supt{ME}(\vb B) \Big|
                \vb f_{\beta n} 
    \Big\rangle
\\[5pt]
 &= -ik\supt{A}
    \Big\langle \vb f_{\alpha m} 
                 \Big| \vb C(k\supt{A}) \Big|
                 \vb f_{\beta n} 
    \Big\rangle
    -ik\supt{B}
    \Big\langle \vb f_{\alpha m} 
                 \Big| \vb C(k\supt{B}) \Big|
                 \vb f_{\beta n} 
    \Big\rangle
\\[10pt]
%--------------------------------------------------------------------%
 M\Big( \mathcal{I}^N_{\alpha m}, \mathcal{I}^N_{\beta n} \Big) 
 &= -\Big\langle \vb f_{\alpha m} 
    \Big| \BG\supt{MM}(\vb A) + \BG\supt{MM}(\vb B) \Big|
                \vb f_{\beta n} 
    \Big\rangle
\\[5pt]
 &= -\frac{ik\supt{A}}{Z\supt{A}}
    \Big\langle \vb f_{\alpha m} 
                 \Big| \vb G(k\supt{A}) \Big|
                 \vb f_{\beta n} 
    \Big\rangle
    -
    \frac{ik\supt{B}}{Z\supt{B}}
    \Big\langle \vb f_{\alpha m} 
                 \Big| \vb G(k\supt{B}) \Big|
                 \vb f_{\beta n}
    \Big\rangle.
\\[10pt]
\end{align*}
%====================================================================%
\end{enumerate}

