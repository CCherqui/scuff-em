%%%%%%%%%%%%%%%%%%%%%%%%%%%%%%%%%%%%%%%%%%%%%%%%%%%%%%%%%%%%%%%%%%%%%%
%%%%%%%%%%%%%%%%%%%%%%%%%%%%%%%%%%%%%%%%%%%%%%%%%%%%%%%%%%%%%%%%%%%%%%
%%%%%%%%%%%%%%%%%%%%%%%%%%%%%%%%%%%%%%%%%%%%%%%%%%%%%%%%%%%%%%%%%%%%%%
\newpage
\section{Equivalence-principle PFT (EPPFT)}

In the EPPFT approach, we compute the power or force 
transfer

in terms of the total work done,

\subsection*{Work on a volume current distribution}

To motivate these formulas, first recall that, if we had a
\textit{volume} distribution of electric current $\vb J$ 
throughout a body $\mc B$, we could compute the power absorbed
by the body as the time-average work done on the volume currents
inside it via Joule heating:
%====================================================================%
\begin{subequations}
\begin{align}
 P\sups{abs} 
     &= \frac{1}{2}\text{ Re }
      \int_{\mc B} \vb J^*(\vb x) \cdot \vb E(\vb x)
      \,dV
%--------------------------------------------------------------------%
\intertext{Similarly, the $i$-directed force on the body follows
           from considering the Lorentz force on the charges and
           currents contained in an infinitesimal volume $dV:$}
%--------------------------------------------------------------------%
 F_i &= \frac{1}{2}\text{ Re }
        \int_{\mc B} \bigg\{ \rho(\vb x) E_i(\vb x)
                    +\frac{1}{\mu_0}
                     \Big[\vb J(\vb x) \cdot \vb H(\vb x)\Big]_i
             \bigg\} dV
\intertext{With some effort (Appendix \ref{ForceFormulaAppendix}),
           this formula may be rewritten in a form closely
           analogous to that of (\ref{VolumePFT}a):}
     &= \frac{1}{2\omega}\text{ Im }
        \int_{\mc B} \vb J^*(\vb x) \cdot \partial_i \vb E(\vb x)
                     dV.
\intertext{Similarly (Appendix \ref{ForceFormulaAppendix}), the 
           $i$-directed \textit{torque} on the body is}
     &= \frac{1}{2\omega}\text{ Im }
        \int_{\mc B} \vb J^*(\vb x) \cdot \partial_\theta \vb E(\vb x)
                     dV
\end{align}
\label{VolumePFT}
\end{subequations}
where $\partial_\theta \vb E$ denotes differentiation 
with respect to rotations about the $i$th
coordinate axis.
%====================================================================%

\subsection*{Work on a surface current distribution}

We now simply write down analogues of formulas (\ref{VolumePFT})
for the case in which the volume electric current $\vb J$
existing throughout the bulk of the body is replaced by 
surface electric and magnetic currents $\{\vb K, \vb N\}$
present only on the surface:
%====================================================================%
\begin{subequations}
\begin{align}
 P\sups{abs} 
     &= \frac{1}{2}\text{ Re }
      \int_{\partial \mc B} \Big[ \vb K^*(\vb x) \cdot \vb E(\vb x)
                                 +\vb N^*(\vb x) \cdot \vb H(\vb x)
                            \Big]
      \,dA
\nn
    &= \frac{1}{2}\text{ Re }
      \int_{\partial \mc B} \bmc C^*(\vb x) \cdot \bmc F(\vb x) \,dA
\\
 F_i
     &= \frac{1}{2\omega}\text{ Im }
      \int_{\partial \mc B} \Big[ \vb K^*(\vb x) \cdot \partial_i \vb E(\vb x)
                                 +\vb N^*(\vb x) \cdot \partial_i \vb H(\vb x)
                            \Big]
      \,dA
\nn
    &= \frac{1}{2\omega }\text{ Im }
      \int_{\partial \mc B} \bmc C^*(\vb x) \cdot \partial_i \bmc F(\vb x) \,dA
\intertext{and similarly}
 \tau_i
    &= \frac{1}{2\omega}\text{ Im }
      \int_{\partial \mc B} \bmc C^*(\vb x) \cdot \partial_\theta \bmc F(\vb x)
      \,dA.
\end{align}
\label{EPPFT0}
\end{subequations}
%====================================================================%
Note that these formulas only reference the tangential components
of $\vb E$ and $\vb H$, which are continuous across the body 
surface; thus it matters not whether we evaluate $\vb E$ and $\vb H$
using the computational recipes for the fields inside or outside
the surface (these recipes, of course, differ in SIE formulations).
It is easiest to use the inside recipe, in which case the fields
are a convolution over just the surface currents on the surface 
of $\mc B$:
%====================================================================%
$$ \bmc F(\vb x) 
   = \int_{\partial \mc B} \bmc G^{(\mc B)}(\vb x, \vb x^\prime) \cdot \bmc C(\vb x^\prime) 
   d\vb x^\prime
$$ 
%====================================================================%
where $\bmc G^{(\mc B)}$ is the $6\times 6$ dyadic Green's function
for the material interior to $\mc B$. The PFT formulas (\ref{EPPFT0}) 
then become double surface integrals,
%====================================================================%
\begin{subequations}
\begin{align}
 P\sups{abs} 
     &= \frac{1}{2}\text{ Re }
      \iint_{\partial \mc B} 
      \bmc C^*(\vb x) \bmc G^{(\mc B)}(\vb x, \vb x^\prime)
      \bmc C(\vb x^\prime)
      \,d\vb x\, d\vb x^\prime
\\
 F_i\sups{abs} 
     &= \frac{1}{2\omega}\text{ Im }
      \iint_{\partial \mc B}
      \bmc C^*(\vb x) \partial_i \bmc G^{(\mc B)}(\vb x, \vb x^\prime)
      \bmc C(\vb x^\prime)
      \,d\vb x \,d\vb x^\prime
\\
 \tau_i\sups{abs} 
     &= \frac{1}{2\omega}\text{ Im }
      \iint_{\partial \mc B}
      \bmc C^*(\vb x) \partial_\theta \bmc G^{(\mc B)}(\vb x, \vb x^\prime)
      \bmc C(\vb x^\prime)
      \,d\vb x\, d\vb x^\prime
\end{align}
\label{EPPFT1}
\end{subequations}
%====================================================================%
or, upon discretization,
%====================================================================%
\begin{subequations}
\begin{align}
 P\sups{abs} 
     &= \frac{1}{2}\text{ Re }
      \vb c^\dagger \vb M^{(\mc B)} \vb c
\\
 F_i\sups{abs} 
     &= \frac{1}{2\omega}\text{ Im }
      \vb c^\dagger \partial_i \vb M^{(\mc B)} \vb c
\\
 \tau_i\sups{abs} 
     &= \frac{1}{2\omega}\text{ Im }
      \vb c^\dagger \partial_\theta \vb M^{(\mc B)} \vb c.
\end{align}
\label{EPPFT}
\end{subequations}
%====================================================================%
