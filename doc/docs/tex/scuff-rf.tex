\documentclass[letterpaper]{article}

../../../../tex/scufftex.tex

\graphicspath{{figures/}}

%------------------------------------------------------------
%------------------------------------------------------------
%- Special commands for this document -----------------------
%------------------------------------------------------------
%------------------------------------------------------------

%------------------------------------------------------------
%------------------------------------------------------------
%- Document header  -----------------------------------------
%------------------------------------------------------------
%------------------------------------------------------------
\title {{\sc librfsolver} and {\sc scuff-rf}: Support for
        RF device modeling in {\sc scuff-em}}
\author {Homer Reid}
\date {April 16, 2018}

%------------------------------------------------------------
%------------------------------------------------------------
%- Start of actual document
%------------------------------------------------------------
%------------------------------------------------------------

\begin{document}
\pagestyle{myheadings}
\markright{Homer Reid: The {\sc scuff-em} RF module}
\maketitle

\tableofcontents

%%%%%%%%%%%%%%%%%%%%%%%%%%%%%%%%%%%%%%%%%%%%%%%%%%%%%%%%%%%%%%%%%%%%%%
%%%%%%%%%%%%%%%%%%%%%%%%%%%%%%%%%%%%%%%%%%%%%%%%%%%%%%%%%%%%%%%%%%%%%%
%%%%%%%%%%%%%%%%%%%%%%%%%%%%%%%%%%%%%%%%%%%%%%%%%%%%%%%%%%%%%%%%%%%%%%
\section{Overview}

The {\sc scuff-em} package includes a module for RF device modeling
within the framework of integral-equation electromagnetism.
The core functionality of the module is implemented 
by the {\sc librfsolver} library packaged with the {\sc scuff-em}
distribution, and may be accessed either \textbf{(a)} in API form 
from C++ or python programs, or \textbf{(b)} from the command line
via the {\sc scuff-rf} binary application code.

The {\sc scuff-em} RF module extends the existing functionality of the
{\sc scuff-em} core library in two main ways:

\begin{enumerate}
  \item It introduces the notion of a \textit{port.} This is a region
        of your structure (be it an antenna, a coaxial cable, etc.)
        that interfaces with RF circuitry; more specifically, a port
        consists of a positive terminal, into which a current of arbitrary
        complex amplitude is injected,
        and a negative terminal from which the same current is extracted.
        The fields radiated by these currents define the incident field
        in the SIE scattering problem solved by {\sc scuff-em}.

        Ports are defined by geometric entities (points, lines, or
        polygons) identifying regions of meshed structures.
        Information on ports is stored internally in the {\sc scuff-em}
        RF module via data structures named \texttt{RWGPort} and 
        \texttt{RWGPortEdge}.
  \item It introduces a new type of post-processing operation
        to compute the \textit{impedance parameters} of a multiport geometry.
        (All of the varous other types of post-processing calculations offered by
        core {\sc scuff-em}---including computation and visualization of fields,
        induced moments, power, etc.---are available as well.)
\end{enumerate}

In this memo we will discuss the implementation of these features.

%%%%%%%%%%%%%%%%%%%%%%%%%%%%%%%%%%%%%%%%%%%%%%%%%%%%%%%%%%%%%%%%%%%%%%
%%%%%%%%%%%%%%%%%%%%%%%%%%%%%%%%%%%%%%%%%%%%%%%%%%%%%%%%%%%%%%%%%%%%%%
%%%%%%%%%%%%%%%%%%%%%%%%%%%%%%%%%%%%%%%%%%%%%%%%%%%%%%%%%%%%%%%%%%%%%%
\newpage
\section{The concept of an \texttt{RWGPort}}
%####################################################################%
%####################################################################%
%####################################################################%
\begin{figure}
\begin{center}
%\includegraphics{RWGPortFigure
\caption{An \texttt{RWGPort}. The port current $I$ is
         forced into the positive port terminal, while
         an identical current is extracted from the negative
         port terminal. Each port terminal is comprised of one or
         more exterior edges in a meshed geometry.
        }
\label{RWGPortFigure}
\end{center}
\end{figure}
%####################################################################%
%####################################################################%
%####################################################################%

The key extension of the \lss core library provided by {\sc scuff-rf}
is the idea of an \texttt{RWGPort}. This is a physical region of
a material body that interfaces with RF circuitry. More specifically,
a port consists of positive and negative terminals, where each
terminal is a small region of a surface over which flows an externally
injected or extracted current; the fields radiated by the resulting
spatially localized current distributions then constitute the incident
field in a scattering problem.

Internally, each terminal of an RWGPort is simply a set of
{\sc half-RWG} basis functions

Figure \ref{PortFigure} shows some examples of \texttt{RWGPort} 
terminals.
As this figure makes clear, internally each terminal of a \texttt{RWGPort}
is nothing but a collection of \textit{half-RWG} basis functions.
A half-RWG function is just what it sounds like---it is defined by
a single triangle, with a distinguished choice of vertex, and it describes
a surface-current density supported only on that triangle,
emanating from the vertex and flowing normally outward through the opposite edge.
In a \textit{full}-RWG function this outflow of current is
sunk into the adjacent panel (which, in {\sc scuff-em} parlance, would be
the `negative panel' of the full RWG function), and thus full RWG
functions carry no net current; half-RWG functions, on the other hand,
have no negative panel and describe currents that appear from out of nowhere,
as is appropriate for representing ports driven by a given current injected
by fixed external sources.

Any meshed representation of an open surface in a {\sc scuff-em} geometry 
is automatically assigned a set of half-RWG basis functions, one 
for each exterior panel edge; these are stored 
within the \texttt{RWGSurface} structure for the surface
in an array called \texttt{HalfRWGEdges}. 
(Full RWG edges are stored separately, in the \texttt{Edges} array.)

Ordinarily, the half-RWG functions assigned to exterior edges are inert
in {\sc scuff-em} scattering problems; they are identified when the 
geometry is read and their properties are stored in the \texttt{HalfRWGEdges} 
array, but they are never populated with surface-current weights 
and do not contribute to the SIE system or to post-processing quantities. 
However, when a half-RWG function for an exterior edge is identified as 
belonging to a port terminal, it is promoted to an active participant 
in scattering calculations, contributing to the RHS vector and to 
post-processing quantities like scattered fields.

In addition to ports defined on boundaries of open surfaces, it is
also possible to define ports based at individual vertices
in the \textit{interior} of surfaces. In this case, although the panels
that share that vertex already belong to full-RWG functions,
we create separate new half-RWG functions for them to describe 
their their role in supporting the port-current distribution.
Thus, each instantiation of a point-based port results in the 
creation of new \texttt{HalfRWGEdge} structures that are tacked on 
to the end of the \texttt{HalfRWGEdge} array for the \texttt{RWGSurface}
in question.

The \textit{perimeter} of a port terminal is the sum of the lengths of
all half-RWG edges it contains. This number is relevant for determining
the weight with which each half-RWG function in a port terminal is
populated to describe a port current $I$.
If $L_{p}^\pm$ are the perimeters of the positive and negative ports
of port $p$, then a port current $I_p$ is described by weighting
each basis function in the positive terminal with weight
$+\frac{I_p}{L_p^+}$ and 
each basis function in the negative terminal with weight
$-\frac{I_p}{L_p^-}$. 
Thus, 
\begin{align}
\intertext{
spatial distribution of electric surface-current density due to port $p$
driven by current $I_p$:
}
\vb K(\vb x)&=
   \frac{I_p}{L_p^+} \sum_{\alpha^+} \vb h_\alpha(\vb x)
  -\frac{I_p}{L_p^-} \sum_{\alpha^-} \vb h_\alpha(\vb x).
\end{align}

%%%%%%%%%%%%%%%%%%%%%%%%%%%%%%%%%%%%%%%%%%%%%%%%%%%%%%%%%%%%%%%%%%%%%%
%%%%%%%%%%%%%%%%%%%%%%%%%%%%%%%%%%%%%%%%%%%%%%%%%%%%%%%%%%%%%%%%%%%%%%
%%%%%%%%%%%%%%%%%%%%%%%%%%%%%%%%%%%%%%%%%%%%%%%%%%%%%%%%%%%%%%%%%%%%%%
\newpage

\section{RF device modeling in the integral-equation framework}
\label{RFDeviceModeling}

In RF device modeling, structures are excited by user-specified input currents
forced into one or more ports of a structure, and the objective is to compute
impedance parameters (or admittance parameters or $S$-parameters) for the 
multiport network (possibly accompanied by other quantities such as 
radiated-field profiles).

On the other hand, in the usual surface-integral-equation (SIE)
approach to electromagnetic scattering implemented by the {\sc scuff-em}
core library, the excitation is provided by a user-specified
incident electromagnetic field configuration (such as a plane wave
or the field of a point dipole),
and the objective is to compute quantities such as absorbed or scattered
power.

How do we squeeze the former problem into the latter framework?
In essence, we take the incident field in the SIE scattering problem
to be the field radiated by the currents forced into the ports
of a structure, and we obtain impedance-matrix elements by calculating
complex power dissipation in the structure.
In this section I discuss how these steps are implemented
in the {\sc scuff-em} RF module.
(The problem of computing fields radiated by port-driven structures
requires no new implementation beyond the existing {\sc scuff-em} 
algorithms for computing fields radiated by excited structures.)

\subsection{Port currents: A new type of incident field}

%####################################################################%
%####################################################################%
%####################################################################%
\begin{figure}
\begin{center}
%\includegraphics{PortCurrentIncidentField}}
\caption{The incident field in the BEM scattering problem
         is the field radiated by half-RWG basis functions
         associated with the exterior edges comprising the
         \texttt{RWGPort}.
         The weight of each
half-RWG basis function is populated with
         a strength proportional to the port current.
        }
\label{PortCurrentFigure}
\end{center}
\end{figure}
%####################################################################%
%####################################################################%
%####################################################################%

To model RF devices excited by currents forced into ports,
the {\sc scuff-em} RF module solves an integral-equation scattering problem
in which the incident field is the field radiated by the currents
forced into the ports.
half-RWG basis
functions associated 
with each port, with those basis functions populated by strengths
proportional to the port current.

The positive port current is assumed to be equally distributed over 
all edges on the positive side of the port, while the 
negative port current is assumed to be equally distributed over 
all edges on the negative side of the port.
Thus, if the port current in Figure \ref{PortCurrentFigure}) is $I$, 
then the coefficients of the three half-RWG functions on the 
positive side of the port are
$$K_{15}=K_{29}=K_{37}=-\frac{I}{W^+},$$
while the coefficients of the three half-RWG functions on the 
positive side of the port are
$$K_{61}=K_{82}=+\frac{I}{W^-}.$$
Note that our sign convention is that the current carried
by a half-RWG function always flows \textit{from} the vertex 
\textit{to} the edge, which is why the half-RWG functions
on the positive port edge have negative coefficients.

%####################################################################%
%####################################################################%
%####################################################################%
\begin{figure}
\begin{center}
%\includegraphics{PortCurrentIncidentField}}
\caption{The fields radiated by a half-RWG basis function
         contain one contribution from surface sources 
         (surface current and associated surface charge)
         on the triangular panel $\pan$, and a second contribution
         from the line charge that accumulates on the edge 
         $\vb L.$
        }
\label{HalfRWGBasisFunction}
\end{center}
\end{figure}
%####################################################################%
%####################################################################%
%####################################################################%

\subsection*{Fields radiated by a half-RWG basis function}

Consider a half-RWG basis function $h_\alpha(\vb x)$, as depicted in 
Figure \ref{HalfRWGBasisFunction}.  Suppose the basis function
is weighted with coefficient $k_\alpha$. 
($k_\alpha$ has dimensions of current/length.) 
The $\vb E$ field may be expressed in terms of the Lorentz-gauge potentials,
\begin{align*}
  \vb E(\vb x) 
 &= iw\vb A(\vb x) - \nabla \Phi(\vb x),,,,
\end{align*}
where the vector potential is obtained by integrating over surface 
currents on $\pan_\alpha$,
\begin{align*}
 \vb A(\vb x) 
&=\mu\int_{\pan_\alpha} G(\vb x, \vb x^\prime) \vb K(\vb x^\prime) 
  d\vb x^\prime
\intertext{but the scalar potential contains contributions from both
           the surface charge on $\pan_\alpha$ and the line charge 
           on $\vb L_\alpha$:}
 \vb \phi(\vb x) 
&=+\frac{1}{\epsilon}
  \int_{\pan_\alpha} G(\vb x, \vb x^\prime) \sigma(\vb x^\prime) dA 
+\,\frac{1}{\epsilon}
  \int_{\vb L_\alpha} G(\vb x, \vb x^\prime) \lambda(\vb x^\prime) dl
\end{align*}
For a half-RWG function populated by strength $k_\alpha$ we have 
$$ \left.
   \begin{array}{lcl}
   \vb K(\vb x)  &=& \frac{lk_\alpha}{2A}(\vb x-\vb Q)
   \\[8pt]
   \sigma(\vb x) &=& \frac{lk_\alpha}{i\omega A}
   \end{array}
   \right\} \qquad \vb x\in\pan_\alpha 
$$
and 
$$ \lambda(\vb x) = \frac{k_\alpha}{i\omega}, \qquad 
   \vb x \in \vb L 
$$
and hence the 

\subsection{Evaluation of RHS vector}

The elements of the RHS vector in a {\sc scuff-em} scattering problem
are (the negatives of) the inner products of the incident fields
with the RWG basis functions of the geometry.

%####################################################################%
%####################################################################%
%####################################################################%
\subsection{Calculation of impedance matrix}

For an $N\subt{P}$-port system, the
$N\subt{P}\times N\subt{P}$ impedance matrix $\vb Z$ is defined by
JStar E = (Jr-iJi)(Er+iEi) JrEr + JiEi + i(JrEi-JiEr)
J E = (Jr+iJi)(Er+iEi) (JrEr - JiEi) + i(JrEi+JiEr)

$$ \vb J_p(\vb x) =
   \sum_{a\in\mc{P}_p}
   \bigg\{ \Big[ 
     -\sum_{\alpha\beta} \vb b_\alpha(\vb x) M_{\alpha\beta} R_{\beta a}
           \Big]
           +\vb b_a(\vb x)
   \bigg\}w_a,
 \quad 
   \vb E_q(\vb x) =
   \sum_{b\in\mc{P}_q}
   \bigg\{ \Big[ 
     -\sum_{\gamma\delta} \mb E_\gamma(\vb x) M_{\gamma\delta} R_{\delta b}
           \Big]
           +\mb E_b(\vb x)
   \bigg\}w_b
 \qquad 
$$
%====================================================================%
\begin{align*}
 Z_{pq} 
 &\equiv \frac{1}{2I_p I_q}\int \vb J_p(\vb x) \vb E_q(\vb x)\,d\vb x
\\
 &= T_1 + T_2 + T_3 + T_4 
\\
 T_1 &\equiv \sum_{\alpha\beta\gamma\delta}
  R_{\delta q}
  M_{\alpha\beta}
 \underbrace{\Inp{\vb b_\alpha}{\vb E_\gamma}}_{M_{\alpha\gamma}^{-1}}
  M_{\gamma\delta}
  R_{\beta p}
\\
&= \frac{1}{2} \vb R_q \vb M \vb R_p
\\
 T_2 &\equiv -\sum_{\alpha\beta}
 \underbrace{\Inp{\vb b_\alpha}{\vb E_b}}_{R_{\alpha q}}
  M_{\alpha\beta}
  R_{\beta p}
 &= -\frac{1}{2}\vb R_q \vb M \vb R_p
\end{align*}
%====================================================================%

%%%%%%%%%%%%%%%%%%%%%%%%%%%%%%%%%%%%%%%%%%%%%%%%%%%%%%%%%%%%%%%%%%%%%%
%%%%%%%%%%%%%%%%%%%%%%%%%%%%%%%%%%%%%%%%%%%%%%%%%%%%%%%%%%%%%%%%%%%%%%
%%%%%%%%%%%%%%%%%%%%%%%%%%%%%%%%%%%%%%%%%%%%%%%%%%%%%%%%%%%%%%%%%%%%%%
\newpage
\appendix
\subsection{Port voltages: Alternative computation of impedance matrix}
%####################################################################%
%####################################################################%
%####################################################################%
\begin{figure}
\begin{center}
%\includegraphics{PortCurrentIncidentField}}
\caption{The port voltage is the line integral of the total $\vb E$ 
         field along the straight line connecting the port's 
         positive reference point to its negative reference point.
        }
\label{LineIntegralFigure}
\end{center}
\end{figure}
%####################################################################%
%####################################################################%
%####################################################################%

\label{PortVoltageSection}

The port voltage is defined to be the directed line integral 
of the total $\vb E$ field over the straight line connecting
the port's positive reference point to its negative reference
point (Figure \ref{LineIntegralFigure}).

\begin{align*}
  V\sups{port}
&=\int_{P^+}^{P^-} \vb E(\vb x) \cdot d \vb l 
\\
&=\int_{P^+}^{P^-} \Big[ i\omega \vb A(\vb x) - \nabla \Phi(\vb x)\Big]
                   \cdot d \vb l
\\
&= \Big[ \Phi\big(P^+\big) - \Phi\big(P^-\big)\Big]
   +i\omega \int_{P^+}^{P^-} \vb A(\vb x) \cdot d\vb l
\end{align*}

\section*{Averaging the potential due to $\mathcal{P}$ over an edge $\vb L$}

The electrostatic potential due to a unit charge density on panel $\mc P$,
averaged over the length of some panel edge $\vb L$, is 

\begin{align*}
 \Big\langle \phi^{\mc P} \Big \rangle_{\vb L}
&= \frac{1}{4\pi \epsilon_0 |\vb L|}\int_{\vb L} d\vb x \int_{\mc P} d\vb x^\prime 
   \frac{1}{|\vb x - \vb x^\prime|} 
\end{align*}

\end{document}
