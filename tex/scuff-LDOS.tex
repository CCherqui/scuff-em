\documentclass[letterpaper]{article}
\usepackage[square,sort,comma,numbers]{natbib}
\newcommand{\citeasnoun}[1]{Ref.~\citenum{#1}}

../../../../tex/scufftex.tex

\newcommand\supsstar[1]{^{\hbox{\scriptsize{#1}}*}}
\newcommand\suptstar[1]{^{\hbox{\scriptsize{#1}}*}}
\newcommand{\IF}{^{i\text{\scriptsize F}}}
\newcommand{\IFFlux}{^{i\text{\tiny FFLUX}}}
\newcommand{\IT}{^{i\text{\scriptsize T}}}
\newcommand{\ITFlux}{^{i\text{\tiny TFLUX}}}
\newcommand{\PS}{^{\text{\scriptsize P}\mc S}}
\newcommand{\IFS}{^{i\text{\scriptsize F}\mc S}}
\newcommand{\ITS}{^{i\text{\scriptsize T}\mc S}}
%\newcommand{\vbchi}{\boldsymbol{\chi}}
\newcommand{\vbGamma}{\boldsymbol{\Gamma}}
\newcommand{\wh}{\widehat}

\graphicspath{{figures/}}

%------------------------------------------------------------
%------------------------------------------------------------
%- Special commands for this document -----------------------
%------------------------------------------------------------
%------------------------------------------------------------

%------------------------------------------------------------
%------------------------------------------------------------
%- Document header  -----------------------------------------
%------------------------------------------------------------
%------------------------------------------------------------
\title {Computation of Green's Functions and LDOS in {\sc scuff-em}}
\author {Homer Reid}
\date {September 27, 2014}

%------------------------------------------------------------
%------------------------------------------------------------
%- Start of actual document
%------------------------------------------------------------
%------------------------------------------------------------

\begin{document}
\pagestyle{myheadings}
\markright{Homer Reid: Periodic GF and LDOS computations in {\sc scuff-em}}
\maketitle

\tableofcontents

%%%%%%%%%%%%%%%%%%%%%%%%%%%%%%%%%%%%%%%%%%%%%%%%%%%%%%%%%%%%%%%%%%%%%%
%%%%%%%%%%%%%%%%%%%%%%%%%%%%%%%%%%%%%%%%%%%%%%%%%%%%%%%%%%%%%%%%%%%%%%
%%%%%%%%%%%%%%%%%%%%%%%%%%%%%%%%%%%%%%%%%%%%%%%%%%%%%%%%%%%%%%%%%%%%%%
\newpage
\section{Green's functions and LDOS in the non-periodic case}

The scattering parts of the electric and magnetic
dyadic Green's functions (DGFs) of a geometry are defined by
%====================================================================%
\begin{align*}
 \mc G_{ij}\sups{E}(\omega; \vb x, \vb x^\prime)
   \equiv
   \frac{1}{ikZ_0 Z^r}
  &\left( \parbox{0.65\textwidth}
    { $i$-component of scattered $\vb E$-field at $\vb x$
      due to a unit-strength $j$-directed point \textbf{electric} 
      dipole radiator
      at $\vb x^\prime$, all quantities having time dependence
      $\sim e^{-i\omega t}$
    }
   \right)
\\[5pt]
%%--------------------------------------------------------------------%
 \mc G_{ij}\sups{M}(\omega; \vb x, \vb x^\prime)
   \equiv
   \frac{1}{ik}
  &\left( \parbox{0.65\textwidth}
    { $i$-component of scattered $\vb H$-field at $\vb x$
      due to a unit-strength $j$-directed point \textbf{magnetic}
      dipole radiator
      at $\vb x^\prime$, all quantities having time dependence
      $\sim e^{-i\omega t}$
    }
   \right)
\end{align*}
%====================================================================%
Here $k=\sqrt{\epsilon^r \mu^r }\cdot \omega$ and 
$Z^r=\sqrt{\mu^r /\epsilon^r }$ are the wavenumber and 
relative wave impedance of the material medium in which 
point $\vb x$ resides ($\epsilon^r ,\mu^r$ are its relative 
permittivity and permeability) and $Z_0\approx 377\,\Omega$  
is the impedance of vacuum. The prefactors 
$\frac{1}{ikZ_0Z^r}$ and $\frac{1}{ik}$ are inserted to 
ensure that $\bmc G\supt{E,M}$ have dimensions of inverse
length.

The enhancement of the local density of states (LDOS)
at frequency $\omega$ and at a point $\vb x$ in a 
scattering geometry is related to the scattering DGFs according 
to\footnote{K Joulain et al.,``Definition and measurement of the local 
density of electromagnetic states close to an interface,''
Physical Review B \textbf{68} 245405 (2003)}
%====================================================================%
$$
   \texttt{LDOS}(\omega; \vb x)
   \equiv 
   \frac{\rho(\omega; \vb x)}{\rho_0(\omega)}
   \equiv 
   \frac{\pi}{k_0^2}\text{Tr }\text{Im }
   \Big[ \bmc G\supt{E}(\omega; \vb x, \vb x)
        +\bmc G\supt{M}(\omega; \vb x, \vb x)
   \Big]
$$
%====================================================================%
where $\rho_0(\omega)\equiv k^3_0/(\pi c)$ is the free-space
LDOS and $k_0=\omega/c$ is the free-space wavenumber at the
frequency in question.

In {\sc scuff-em} the dyadic GFs may be computed easily by solving a
scattering problem in which the incident fields arise from a point dipole
radiator at a source point $\vb x\subt{S}$.
For example, to compute $\mc G\supt{E}$ we take the incident fields 
to be the fields of a unit-strength $j$-directed point electric dipole 
source at $\vb s\supt{S}$:
%%%%%%%%%%%%%%%%%%%%%%%%%%%%%%%%%%%%%%%%%%%%%%%%%%%%%%%%%%%%%%%%%%%%%%
\numeq{EHED}
{
 \vb E\sups{inc}(\vb x) =
 \vb E\supt{ED}\Big(\vb x; \{\vb x\subt{S}, \vbhat{x}_j\} \Big),
 \qquad
 \vb H\sups{inc}(\vb x) =
 \vb H\supt{ED}\Big(\vb x; \{\vb x\subt{S}, \vbhat{x}_j\} \Big)
}
%%%%%%%%%%%%%%%%%%%%%%%%%%%%%%%%%%%%%%%%%%%%%%%%%%%%%%%%%%%%%%%%%%%%%%
where $\{\vb E, \vb H\}\supt{ED}\Big(\vb x; \{\vb x_0, \vb p_0\}\Big)$
are the fields of a point electric dipole radiator 
at $\vb x_0$ with dipole moment $\vb p_0$. 
(Expressions for these fields are given in 
Appendix \ref{DipoleFieldsAppendix}).
Then we simply solve an ordinary {\sc scuff-em} scattering
problem with the incident fields given by equation 
(\ref{EHED}) and compute the scattered---not total!---fields
at the evaluation point $\vb x\subt{D}$. The three components of the
$\vb E$-field at $\vb x\subt{D}$, divided by $ikZ_0 Z_r$, yield the
three vertical entries of the $j$th column of the
$3\times 3$ matrix 
$\bmc G\sups{E}(\omega; \vb x\subt{D}, \vb x\subt{S}.)$
Calculating $\bmc G\sups{M}$ is similar except that we use 
a point magnetic source to supply the incident field 
and compute the scattered $\vb H$ field instead of the 
scattered $\vb E$ field.

%%%%%%%%%%%%%%%%%%%%%%%%%%%%%%%%%%%%%%%%%%%%%%%%%%%%%%%%%%%%%%%%%%%%%%
%%%%%%%%%%%%%%%%%%%%%%%%%%%%%%%%%%%%%%%%%%%%%%%%%%%%%%%%%%%%%%%%%%%%%%
%%%%%%%%%%%%%%%%%%%%%%%%%%%%%%%%%%%%%%%%%%%%%%%%%%%%%%%%%%%%%%%%%%%%%%
\newpage
\section{Extension to the periodic case}

In the Bloch-periodic module of {\sc scuff-em}, \textit{all}
fields and currents are assumed to be Bloch-periodic, i.e.
if $Q(\vb x)$ denotes any field or current component at $\vb x$,
then we have the built-in assumption
%====================================================================%
\numeq{BlochCondition}
{Q(\vb x + \vb L) = e^{i\vb k\subt{B} \cdot \vb L}Q(\vb x)}
%====================================================================%
where $\vb L$ is any lattice vector and 
$\vb k\subt{B}$ is the Bloch wavevector.

The fields of a point dipole, equation (\ref{EHED}), do \textit{not}
satisfy (\ref{BlochCondition}), and hence may not be used in
Bloch-periodic {\sc scuff-em} calculations. Instead, what we can 
simulate in the periodic case are the fields of an infinite
phased \textit{array} of point electric dipoles,
%%%%%%%%%%%%%%%%%%%%%%%%%%%%%%%%%%%%%%%%%%%%%%%%%%%%%%%%%%%%%%%%%%%%%%
\begin{subequations}
\begin{align}
 \vb E\supt{EDA}\Big(\vb x; \{ \vb x_0, \vb p_0, \vb k\subt{B}\}\Big)
&=\sum_{\vb L} e^{i\vb k\subt{B}\cdot \vb L}
  \, \vb E\supt{ED}\Big(\vb x; \{\vb x\subt{0}+\vb L, \vb p_0\} \Big),
\\
 \vb H\supt{EDA}\Big(\vb x; \{ \vb x_0, \vb p_0, \vb k\subt{B}\}\Big)
&=\sum_{\vb L} e^{i\vb k\subt{B}\cdot \vb L}
  \, \vb H\supt{ED}\Big(\vb x; \{\vb x\subt{0}+\vb L, \vb p_0\} \Big),
\end{align}
\label{EHEDA}
\end{subequations}
%%%%%%%%%%%%%%%%%%%%%%%%%%%%%%%%%%%%%%%%%%%%%%%%%%%%%%%%%%%%%%%%%%%%%%
(where ``EDA'' stands for ``electric dipole array''). The quantities
we can compute in a single {\sc scuff-em} scattering calculation
are now the periodically phased versions of the DGFs, i.e.
(suppressing $\omega$ arguments),
%%%%%%%%%%%%%%%%%%%%%%%%%%%%%%%%%%%%%%%%%%%%%%%%%%%%%%%%%%%%%%%%%%%%%%
\numeq{ScriptGBarDef}
{
 \overline{\mc G_{ij}\supt{E}}(\vb x, \vb x^\prime, \vb k\subt{B})
 \equiv \sum_{\vb L} e^{i\vb k\subt{B}\cdot \vb L}
  \mc G_{ij}\sups{E}(\vb x, \vb x^\prime-\vb L), 
}
%%%%%%%%%%%%%%%%%%%%%%%%%%%%%%%%%%%%%%%%%%%%%%%%%%%%%%%%%%%%%%%%%%%%%%
with $\overline{\mc G_{ij}\supt{M}}$ defined similarly.
(Here and elsewhere, barred symbols denote Bloch-periodic 
quantities.) To recover the non-periodic Green's function---that
is, the response of our periodic geometry to a \textit{non-periodic}
point source---we must perform a Brillouin-zone 
integration:\footnote{To derive these equations, multiply both sides
of (\ref{ScriptGBarDef}) by $e^{-i\vb k\subt{B} \cdot \vb L^\prime}$,
integrate both sides over the Brillouin zone, and use the
condition 
%%%%%%%%%%%%%%%%%%%%%%%%%%%%%%%%%%%%%%%%%%%%%%%%%%%%%%%%%%%%%%%%%%%%%%
$$\int\subt{BZ} e^{i\vb k\subt{B}\cdot (\vb L-\vb L^\prime)}\,d\vb k
  =\mc{V}\subt{BZ} \, \delta(\vb L,\vb L^\prime)
$$
%%%%%%%%%%%%%%%%%%%%%%%%%%%%%%%%%%%%%%%%%%%%%%%%%%%%%%%%%%%%%%%%%%%%%%
where $\mc V\subt{BZ}$ is the Brillouin-zone volume [for example,
a square lattice with basis vectors
$\{\vb L_1, \vb L_2\}=\{L_x\vbhat{x}, L_y\vbhat{y}\}$ has
$\mc V\subt{BZ}=4\pi^2/(L_x L_y)$].
Setting $\vb L^\prime=0$ recovers (\ref{BZIntegration}).}
%%%%%%%%%%%%%%%%%%%%%%%%%%%%%%%%%%%%%%%%%%%%%%%%%%%%%%%%%%%%%%%%%%%%%%
\numeq{BZIntegration}
{
  \mc G_{ij}\supt{E}(\vb x, \vb x^\prime)
 =\frac{1}{\mc V\subt{BZ}} 
   \int\subt{BZ} 
   \overline{\mc G_{ij}\supt{E}}(\vb x, \vb x^\prime, \vb k\subt{B})
   \, d\vb k\subt{B}.
}
%%%%%%%%%%%%%%%%%%%%%%%%%%%%%%%%%%%%%%%%%%%%%%%%%%%%%%%%%%%%%%%%%%%%%%
and similarly for $\mc G\supt{M}.$

\subsection*{Evaluation of BZ integrals}

For a 2D square lattice with lattice vectors 
$\vb L_1=L_x\vbhat{x}, \vb L_2=L_y\vbhat{x},$
a set of reciprocal-lattice basis vectors is 
$\vbGamma_1=\pf{2\pi}{L_x} \vbhat{x},
 \vbGamma_2=\pf{2\pi}{L_y} \vbhat{y}$,
and Brillouin-zone integrals take the form
%%%%%%%%%%%%%%%%%%%%%%%%%%%%%%%%%%%%%%%%%%%%%%%%%%%%%%%%%%%%%%%%%%%%%%
$$ \frac{1}{\mc V\subt{BZ}} 
   \int\subt{BZ} f(\vb k\subs{B}) \, d \vb k\subt{B}
  =4\int_0^{1/2} \, d u_1 \, \int_0^{1/2} \, d u_2 \, 
   f\Big( u_1\vbGamma_1 + u_y\vbGamma_2 \Big)
$$
%%%%%%%%%%%%%%%%%%%%%%%%%%%%%%%%%%%%%%%%%%%%%%%%%%%%%%%%%%%%%%%%%%%%%%

%%%%%%%%%%%%%%%%%%%%%%%%%%%%%%%%%%%%%%%%%%%%%%%%%%%%%%%%%%%%%%%%%%%%%%
%%%%%%%%%%%%%%%%%%%%%%%%%%%%%%%%%%%%%%%%%%%%%%%%%%%%%%%%%%%%%%%%%%%%%%
%%%%%%%%%%%%%%%%%%%%%%%%%%%%%%%%%%%%%%%%%%%%%%%%%%%%%%%%%%%%%%%%%%%%%%
\newpage
\section{Test: LDOS above an infinite half-space}

At a height $z$ above an infinite dielectric half-space
with relative permittivity $\epsilon$ (and $\mu=1$)
the enhancement of the electric LDOS may be expressed
as an integral over a dimensionless parameter 
$\kappa$:\footnote{K. Joulain et al.,``Definition and 
measurement of the local density of electromagnetic states
 close to an interface,'' Physical Review B \textbf{68} 245405 (2003)}
%====================================================================%
\numeq{JoulainLDOS}
{
 \frac{\rho\supt{E}(z; \omega)}{\rho_0(\omega)}
  =\frac{1}{4}\int_0^\infty \kappa F(\kappa) \, d\kappa,
}
%====================================================================%
\numeq{JoulainFKappa}
{
  F(\kappa)=
   \frac{1}{\kappa_z}
   \begin{cases}
         2 + \text{Re }\Big( r\subt{TE} e^{2 i k_z z} \Big)
           + \Big(2\kappa^2-1\Big) \text{Re }\Big( r\subt{TM} e^{2ik_z z}\Big)
         , \qquad &\kappa < 1 \\[10pt]
    \Big[   \text{Im }\big( r\subt{TE} \big)
           + \big(2\kappa^2-1\big) \text{Im }\big( r\subt{TM} \Big)
    \Big]e^{-2k_z z}, \qquad &\kappa > 1
   \end{cases}
}
%====================================================================%
where 
%====================================================================%
$$ \kappa_z \equiv 
   \begin{cases}
     \sqrt{1-\kappa^2}, \qquad &\kappa < 1, \\
     \sqrt{\kappa^2-1}, \qquad &\kappa > 1, 
   \end{cases}
   \qquad
   \kappa^\prime_z \equiv 
   \begin{cases}
     \sqrt{\epsilon-\kappa^2}, \qquad &\kappa < 1, \\
     \sqrt{\kappa^2-\epsilon}, \qquad &\kappa > 1, 
   \end{cases}
   \qquad
   k_z \equiv k_0\kappa_z,
   \qquad
   k_0 \equiv \frac{\omega}{c},
$$  
%====================================================================%
$$ r\subt{TE} = \frac{\kappa_z - \kappa_z^\prime}{\kappa_z + \kappa_z^\prime}
   \qquad
   r\subt{TM} = \frac{\epsilon \kappa_z - \kappa_z^\prime}{\epsilon \kappa_z + \kappa_z^\prime}
$$
%====================================================================%
The integral in (\ref{JoulainLDOS}) is computing the LDOS as a trace in 
the plane-wave basis; the integral over $\kappa$ is essentially a sum over 
all possible plane-wave wavevectors $\vb k$, and the multiple terms in 
(\ref{JoulainFKappa}) account for the two possible polarizations for each 
wavevector. More specifically, if we think in terms of the usual 
Fresnel-scattering picture of a plane wave incident on a planar interface, 
then $\kappa$ is the sine of the incident angle and $\kappa_z$ is its 
cosine, i.e. the wavevector of the plane wave 
is 
%====================================================================
$$ \vb k = (k_x, k_y, k_z) = k_0(\sin\theta, 0, \cos\theta)
   =k_0(\kappa, 0, \kappa_z)
$$
%====================================================================
and $r\subt{TE}$ and $r\subt{TM}$ are the Fresnel coefficients for reflection
of TE and TM waves from the half-space. 
This interpretation applies to the $0\le \kappa<1$ contribution to 
the integral in (\ref{JoulainLDOS}). 

However, the trace in equation (\ref{JoulainLDOS}) also contains
contributions from modes that cannot be excited by irradiating
the interface from the outside with a plane wave (corresponding
to the $\kappa>1$ portion of the integral). These are 
\textit{evanescent} waves for which the in-plane wavenumber
$k_\parallel=\kappa k_0$ is greater than the free-space wavenumber 
$k_0$. The fields associated with these modes decay exponentially
as we move away from the half-space into the vacuum region. 

\subsection*{Expression as a Brillouin-zone integral}

To write equation (\ref{JoulainLDOS}) in a form that facilitates
comparison with {\sc scuff-ldos} calculations, I first interpret
the integral over $\kappa$ as an integral over the upper-right
quadrant of the $\vb k_\parallel=(k_x,k_y)$ plane; here
$\kappa$ is the radial component of $\vb k_\parallel$ measured in
units of $k_0$, i.e. $\kappa=\sqrt{k_x^2 + k_y^2}/k_0=k_\rho/k_0,$ 
and the integrand in (\ref{JoulainLDOS}) is independent of the 
angle $k_\theta=\atan(k_y/k_x).$ Then I have
%====================================================================%
\begin{align*}
 \frac{\rho\supt{E}(z; \omega)}{\rho_0(\omega)}
  &=\frac{1}{4}\int_0^\infty \kappa F(\kappa) \, d\kappa
\\
  &=\frac{1}{2\pi k_0^2}
    \int_0^\infty k_\rho dk_\rho \int_0^{\pi/2} \, dk_\theta F\pf{k_\rho}{k_0}
\\
  &=\frac{1}{2\pi k_0^2}
    \int_0^\infty dk_x \int_0^\infty dk_y \, F(k_x, k_y)
\end{align*}
%%%%%%%%%%%%%%%%%%%%%%%%%%%%%%%%%%%%%%%%%%%%%%%%%%%%%%%%%%%%%%%%%%%%%%
where 
%%%%%%%%%%%%%%%%%%%%%%%%%%%%%%%%%%%%%%%%%%%%%%%%%%%%%%%%%%%%%%%%%%%%%%
$$ F(k_x, k_y)=\pf{\sqrt{k_x^2 + k_y^2}}{k_0}.$$
%%%%%%%%%%%%%%%%%%%%%%%%%%%%%%%%%%%%%%%%%%%%%%%%%%%%%%%%%%%%%%%%%%%%%%
I now rewrite the integral over the infinite upper-right
quadrant of the $(k_x,k_y)$ plane as an integral over a
\textit{finite} rectangle of that plane with side lengths 
$\{\Gamma_x, \Gamma_y\}$: the integrand now consists of an 
infinite sum of terms: 
%%%%%%%%%%%%%%%%%%%%%%%%%%%%%%%%%%%%%%%%%%%%%%%%%%%%%%%%%%%%%%%%%%%%%%
\numeq{JoulainFolded}
{
    \frac{\rho\supt{E}(z; \omega)}{\rho_0(\omega)}
  = \frac{1}{2\pi k_0^2}
    \int_0^{\Gamma_x} \, dk_x \, \int_0^{\Gamma_y} \, dk_y \,
    \overline{F}(k_x, k_y),
}
$$
    \overline{F}(k_x, k_y)=\sum_{n_x,n_y=0}^\infty 
              F\Big( k_x + n_x \Gamma_x, k_y + n_y\Gamma_y \Big).
$$
%%%%%%%%%%%%%%%%%%%%%%%%%%%%%%%%%%%%%%%%%%%%%%%%%%%%%%%%%%%%%%%%%%%%%%%

%%%%%%%%%%%%%%%%%%%%%%%%%%%%%%%%%%%%%%%%%%%%%%%%%%%%%%%%%%%%%%%%%%%%%%
%%%%%%%%%%%%%%%%%%%%%%%%%%%%%%%%%%%%%%%%%%%%%%%%%%%%%%%%%%%%%%%%%%%%%%
%%%%%%%%%%%%%%%%%%%%%%%%%%%%%%%%%%%%%%%%%%%%%%%%%%%%%%%%%%%%%%%%%%%%%%
\appendix
\newpage 
\section{Fields of a phased array of point dipole radiators}
\label{DipoleFieldsAppendix}

To compute dyadic Green's functions in periodic geometries,
{\sc scuff-ldos} solves a scattering problem in which the
incident fields originate from a an infinite phased array
of point sources. Here I describe the calculation of these
infinite fields. This calculation implemented by the 
\texttt{PointSource} 
class in the {\sc libincfield} module in {\sc scuff-em}.

\subsection*{Fields of a single point dipole}

First consider a single point electric dipole radiator (not an array)
with dipole moment $\vb p_0$ at a point $\vb x_0$ in a
medium with relative permittivity and permeability
$\epsilon^r, \mu^r$
(as usual suppressing time-dependence factors
of $e^{-i\omega t}$). The fields at $\vb x$ due to this source are 
%====================================================================%
\begin{align*}
  \vb E\supt{ED}(\vb x; \vb x_0, \vb p_0)
   &= \frac{|\vb p_0|}{\epsilon_0 \epsilon^r}
      \cdot \frac{e^{ikr}}{4\pi r^3} \cdot 
      \Big[ f_1(ikr) \vbhat{p_0} + f_2(ikr)(\vbhat{r}\cdot \vbhat{p_0})\vbhat{r} \Big]
\\
  \vb H\supt{ED}(\vb x; \vb x_0, \vb p_0)
   &= \frac{1}{Z_0 Z^r }\cdot \frac{|\vb p_0|}{\epsilon_0 \epsilon^r}
      \cdot \frac{e^{ikr}}{4\pi r^3} \cdot 
      \Big[ f_3(ikr) (\vbhat{r}\times\vbhat{p_0})\Big]
\end{align*}
%====================================================================%
$$ \vb r=|\vb x-\vb x_0|, \qquad r=|\vb r|, \qquad \vbhat{r}=\frac{\vb r}{r},$$
$$ f_1(x)=-1+x-x^2, \qquad f_2(x)=3-3x+x^2, \qquad f_3(x)=x-x^2.$$
%====================================================================%
An alternative way to understand these fields is to think of the point dipole
$\vb p_0$ at $\vb x_0$ as a localized volume current distribution, 
%====================================================================%
$$ \vb J(\vb x) = -i\omega \vb p_0 \delta(\vb x-\vb x_0) $$
%====================================================================%
in which case it is easy to compute the fields at $\vb x$ by
convolving with the usual (free-space) dyadic
Green's functions relating currents to fields:
%====================================================================%
\begin{subequations}
\begin{align}
 E_i(\vb x) 
&= 
 \int \Gamma_{ij}\supt{EE}(\vb x, \vb x^\prime) J_j(\vb x^\prime) d\vb x^\prime
\nn
&=-i\omega \Gamma_{ij}\supt{EE}(\vb x, \vb x_0) p_{0j}
\nn
&=(-i\omega)(ikZ_0Z^r) G_{ij}(\vb x, \vb x_0) p_{0j}
\nn
&=+k^2 \cdot \frac{|\vb p_0|}{\epsilon_0 \epsilon^r } \cdot G_{ij}(\vb x, \vb x_0) \hat{p}_{0j}
\\[8pt]
H_i(\vb x) 
 &=
\int \Gamma_{ij}\supt{ME}(\vb x, \vb x^\prime) J_j(\vb x^\prime) d\vb x^\prime
\nn
&=-i\omega \Gamma_{ij}\supt{ME}(\vb x, \vb x_0) p_{0j}
\nn
&=(-i\omega)(-ik)  C_{ij}(\vb x, \vb x_0) p_{0j}
\nn
&=-\frac{k^2}{Z_0 Z^r} 
   \cdot 
   \frac{|\vb p|}{\epsilon_0 \epsilon^r}
   \cdot C_{ij}(\vb x, \vb x_0) \hat{p}_{0j}
\end{align}
\label{DipoleFieldsFromGC}
\end{subequations}
%====================================================================%
where the $\vb G$ and $\vb C$ dyadics are related to the
scalar Helmholtz Green's function according to
%====================================================================%
\numeq{GCFromG0}
{
  G_{ij}(\vb r)
   = \Big[ \delta_{ij} +\frac{1}{k^2} \partial_i \partial_j \Big] G_0(\vb r),
\qquad
   C_{ij}(\vb r)
   = \frac{1}{ik}\varepsilon_{ijk} \partial_k G_0(\vb r).
}
%====================================================================%

\subsection*{Fields of a phased array of point dipoles, take 1}

Now consider the fields of a phased array of electric dipoles
of dipole moment $\vb p_0$ located at $\vb x_0$ in the lattice 
unit cell.
A first way to get the fields of this array is 
to start with equations (\ref{DipoleFieldsFromGC}) and
(\ref{GCFromG0}), but replace the non-periodic scalar
Green's function $G_0$ with its Bloch-periodic version,
%%%%%%%%%%%%%%%%%%%%%%%%%%%%%%%%%%%%%%%%%%%%%%%%%%%%%%%%%%%%%%%%%%%%%%
$$ G_0(\vb x-\vb x^\prime) \qquad \longrightarrow \qquad 
   \overline{G_0}(\vb x, \vb x^\prime; \vb k\subt{B}) \equiv
   \sum_{\vb L} e^{i \vb k\subt{B} \cdot \vb L} \, G_0(\vb x- \vb x^\prime-\vb L).
$$
%%%%%%%%%%%%%%%%%%%%%%%%%%%%%%%%%%%%%%%%%%%%%%%%%%%%%%%%%%%%%%%%%%%%%%
Then the components of the fields of an electric dipole array, 
equation (\ref{EHEDA}), read
%%%%%%%%%%%%%%%%%%%%%%%%%%%%%%%%%%%%%%%%%%%%%%%%%%%%%%%%%%%%%%%%%%%%%%
\begin{align*}
 E_i\supt{EDA}(\vb x)
 &=k^2 \cdot \frac{|\vb p_0|}{\epsilon_0 \epsilon^r } 
    \left[ \delta_{ij} + \frac{1}{k^2}\partial_i \partial_j
    \right] \overline{G}_{0}(\vb x-\vb x^\prime)\hat{p}_{0j}
\\
 H_i\supt{EDA}(\vb x)
 &=\frac{ik}{Z_0 Z^r}
    \cdot \frac{|\vb p_0|}{\epsilon_0 \epsilon^r } \cdot
    \epsilon_{ijk} \partial_k \overline{G}_{0}(\vb x-\vb x^\prime)
    \hat{p}_{0j}.
\end{align*}
%%%%%%%%%%%%%%%%%%%%%%%%%%%%%%%%%%%%%%%%%%%%%%%%%%%%%%%%%%%%%%%%%%%%%%

\subsection*{Fields of a phased array of point dipoles, take 2}

An alternative way to get the fields of a point array of dipoles,
which is useful for the half-space calculation of the following
section, is to start with the two-dimensional Fourier representation 
of the (non-periodic) homogeneous dyadic Green's functions:
%%%%%%%%%%%%%%%%%%%%%%%%%%%%%%%%%%%%%%%%%%%%%%%%%%%%%%%%%%%%%%%%%%%%%%
\numeq{GCInfiniteKIntegrals}
{
\vb G(\vbrho, z)
 = \int_{\mathbb{R}^2}
     \frac{d\vb k}{(2\pi)^2} \vb g(\vbrho, z; \vb k),
\qquad
\vb C(\vbrho, z)
 = \int_{\mathbb{R}^2}
     \frac{d\vb k}{(2\pi)^2} \vb c(\vbrho, z; \vb k),
}
%%%%%%%%%%%%%%%%%%%%%%%%%%%%%%%%%%%%%%%%%%%%%%%%%%%%%%%%%%%%%%%%%%%%%%
\begin{align*}
\vb g(\vbrho, z; \vb k)
&= \pf{i}{2k_0^2 k_z}
   \left(\begin{array}{ccc}
    k_0^2-k_x^2      & k_x k_y      & \pm k_z k_x \\
    k_y k_x          & k_0^2-k_y^2  & \pm k_z k_y \\
    \pm  k_x k_z     & \pm k_y k_z  & k_0^2-k_z^2
    \end{array}\right) e^{i\vb k \cdot \vbrho} e^{\pm i k_z z}
\\[5pt]
\vb c(\vbrho, z; \vb k)
 &= -\pf{1}{2k_0 k_z}
     \left(\begin{array}{ccc}
         0 & \pm k_z & -k_y \\
  \mp k_z  & 0       &  k_x \\
      k_y  & -k_x    &  0 
     \end{array}\right) e^{i\vb k \cdot \vbrho} e^{\pm i k_z z}
\end{align*}
%====================================================================%
$$
   k_z \equiv 
   \begin{cases}
     \sqrt{k_0^2 - |\vb k|^2}, \qquad &|\vb k| \le \vb k_0 \\
    i\sqrt{|\vb k|^2-k_0^2},   \qquad &|\vb k| > \vb k_0
   \end{cases},
   \qquad
   \pm \equiv
   \begin{cases}
     +, \quad & z\ge 0 \\ 
     -, \quad & z<   0
   \end{cases}.
$$
%%%%%%%%%%%%%%%%%%%%%%%%%%%%%%%%%%%%%%%%%%%%%%%%%%%%%%%%%%%%%%%%%%%%%%
Now reinterpret the infinite integrals over the entire $k_x, k_y$ plane in
(\ref{GCInfiniteKIntegrals}) as finite integrals over just the Brillouin
zone;
%%%%%%%%%%%%%%%%%%%%%%%%%%%%%%%%%%%%%%%%%%%%%%%%%%%%%%%%%%%%%%%%%%%%%%
\numeq{GCFiniteKIntegrals}
{
\vb G(\vbrho, z)
 = \int_0^{\Gamma_x} \, dk_x \, \int_0^{\Gamma_y} \, dk_y \,
     \frac{d\vb k}{(2\pi)^2} \overline{\vb g}(\vbrho, z; \vb k),
\qquad
\vb C(\vbrho, z)
 = \int_0^{\Gamma_x} \, dk_x \, \int_0^{\Gamma_y} \, dk_y \,
     \frac{d\vb k}{(2\pi)^2} \overline{\vb c}(\vbrho, z; \vb k),
}
%%%%%%%%%%%%%%%%%%%%%%%%%%%%%%%%%%%%%%%%%%%%%%%%%%%%%%%%%%%%%%%%%%%%%%

%%%%%%%%%%%%%%%%%%%%%%%%%%%%%%%%%%%%%%%%%%%%%%%%%%%%%%%%%%%%%%%%%%%%%%
\begin{align*}
 \overline{\vb g}(\vbrho, z; k_x, k_y)
 &= \sum_{n_x, n_y=-\infty}^{\infty}
     \vb g\Big(\vbrho, z; k_x + n_x\vbGamma_x, k_y + n_y \vbGamma_y\Big), 
\\
 \overline{\vb c}(\vbrho, z; k_x, k_y) 
 &= \sum_{n_x, n_y=-\infty}^{\infty}
     \vb c\Big(\vbrho, z; k_x + n_x\vbGamma_x, k_y + n_y \vbGamma_y\Big).
\end{align*}
%%%%%%%%%%%%%%%%%%%%%%%%%%%%%%%%%%%%%%%%%%%%%%%%%%%%%%%%%%%%%%%%%%%%%%
If I think of (\ref{GCFiniteKIntegrals}) as equations of the form
(\ref{BZIntegration}), i.e. equations relating non-barred quantities
to Brillouin-zone integrals over barred quantities, I can
identify the Bloch-periodic versions of the dyadic Green's functions
as 
%%%%%%%%%%%%%%%%%%%%%%%%%%%%%%%%%%%%%%%%%%%%%%%%%%%%%%%%%%%%%%%%%%%%%%
$$
\overline{\vb G}\Big(\vbrho, z; \vb k\supt{B}\Big)
=\frac{\mc V\supt{BZ}}{(2\pi)^2}
  \overline{\vb g}\big(\vbrho, z; \vb k\supt{B}\big), 
\qquad
\overline{\vb C}\Big(\vbrho, z; \vb k\supt{B}\Big)
=\frac{\mc V\supt{BZ}}{(2\pi)^2}
  \overline{\vb c}\big(\vbrho, z; \vb k\supt{B}\big).
$$
%%%%%%%%%%%%%%%%%%%%%%%%%%%%%%%%%%%%%%%%%%%%%%%%%%%%%%%%%%%%%%%%%%%%%%

%%%%%%%%%%%%%%%%%%%%%%%%%%%%%%%%%%%%%%%%%%%%%%%%%%%%%%%%%%%%%%%%%%%%%%
%%%%%%%%%%%%%%%%%%%%%%%%%%%%%%%%%%%%%%%%%%%%%%%%%%%%%%%%%%%%%%%%%%%%%%
%%%%%%%%%%%%%%%%%%%%%%%%%%%%%%%%%%%%%%%%%%%%%%%%%%%%%%%%%%%%%%%%%%%%%%
\newpage
\newcommand{\EE}{\mathbb{E}}
\section{Analytical expressions for dyadic Green's functions}

\subsection{General formalism}

$$ \EE\sups{reg}_{p\alpha}(\vb r), \EE\sups{out}_{p\alpha}(\vb r)$$

%%%%%%%%%%%%%%%%%%%%%%%%%%%%%%%%%%%%%%%%%%%%%%%%%%%%%%%%%%%%%%%%%%%%%%
%%%%%%%%%%%%%%%%%%%%%%%%%%%%%%%%%%%%%%%%%%%%%%%%%%%%%%%%%%%%%%%%%%%%%%
%%%%%%%%%%%%%%%%%%%%%%%%%%%%%%%%%%%%%%%%%%%%%%%%%%%%%%%%%%%%%%%%%%%%%%
\numeq{GExpansion1}
{ \vb G(\vb r, \vb r^\prime) 
  =
  \sum_{p\alpha}
  \EE\sups{reg}_{p\alpha}(\vb r_<) \EE\sups{out}_{p\alpha}(\vb r_>)
}
%%%%%%%%%%%%%%%%%%%%%%%%%%%%%%%%%%%%%%%%%%%%%%%%%%%%%%%%%%%%%%%%%%%%%%
%%%%%%%%%%%%%%%%%%%%%%%%%%%%%%%%%%%%%%%%%%%%%%%%%%%%%%%%%%%%%%%%%%%%%%
%%%%%%%%%%%%%%%%%%%%%%%%%%%%%%%%%%%%%%%%%%%%%%%%%%%%%%%%%%%%%%%%%%%%%%
\subsection*{DGFs at exterior points}

If the object is irradiated by $\EE\sups{reg}_{p\alpha}$
then the scattered field is
$\sum_{p^\prime} T_{p^\prime p} \EE\sups\out_{p^\prime\alpha}$
where $T_{p^\prime p}$ is the $T$ matrix.

\numeq{GExpansion2}
{ \bmc G(\vb r, \vb r^\prime)
 = \sum_{pp^\prime \alpha}
   T_{pp^\prime}
   \EE\sups{out}_{p\alpha}(\vb r)
   \EE\sups{out}_{p^\prime\alpha}(\vb r^\prime)
}

%%%%%%%%%%%%%%%%%%%%%%%%%%%%%%%%%%%%%%%%%%%%%%%%%%%%%%%%%%%%%%%%%%%%%%
%%%%%%%%%%%%%%%%%%%%%%%%%%%%%%%%%%%%%%%%%%%%%%%%%%%%%%%%%%%%%%%%%%%%%%
%%%%%%%%%%%%%%%%%%%%%%%%%%%%%%%%%%%%%%%%%%%%%%%%%%%%%%%%%%%%%%%%%%%%%%
\subsection*{DGFs at interior points}

If the object is irradiated from within by $\EE\sups{out}_{p\alpha}$
then the scattered field (at interior points) is
$\sum_{p^\prime} \overline{T}_{p^\prime p} \EE\sups\out_{p^\prime\alpha}$
where $\overline{T}_{p^\prime p}$ is a sort of modified
version of the $T$ matrix. [This is the quantity labeled
$\mc F^{ii}$ in Rahi et al., PRD \textbf{80} 085021 (2009).]

\subsection*{Curvilinear to cartesian conversion}

$$ \bmc G\sups{cartesian}(\vb r, \vb r^\prime) 
   = 
   \vbLambda(\vb r) \bmc G\sups{curvilinear}(\vb r, \vb r^\prime)
   \vbLambda^\dagger(\vb r^\prime)
$$

where $\vbmLambda(\vb r)$ is the $3\times 3$ matrix that 
converts a $3$-vector of cartesian vector components into
a $3$-vector of curvilinear components of the same vector
at $\vb r$.

%%%%%%%%%%%%%%%%%%%%%%%%%%%%%%%%%%%%%%%%%%%%%%%%%%%%%%%%%%%%%%%%%%%%%%
%%%%%%%%%%%%%%%%%%%%%%%%%%%%%%%%%%%%%%%%%%%%%%%%%%%%%%%%%%%%%%%%%%%%%%
%%%%%%%%%%%%%%%%%%%%%%%%%%%%%%%%%%%%%%%%%%%%%%%%%%%%%%%%%%%%%%%%%%%%%%
\subsection{Dielectric sphere}

In this case $\alpha=\{\ell m\}$ where both $\ell,m$ are discrete.

%%%%%%%%%%%%%%%%%%%%%%%%%%%%%%%%%%%%%%%%%%%%%%%%%%%%%%%%%%%%%%%%%%%%%%
%%%%%%%%%%%%%%%%%%%%%%%%%%%%%%%%%%%%%%%%%%%%%%%%%%%%%%%%%%%%%%%%%%%%%%
%%%%%%%%%%%%%%%%%%%%%%%%%%%%%%%%%%%%%%%%%%%%%%%%%%%%%%%%%%%%%%%%%%%%%%
\subsection{Dielectric cylinder}

In this case $\alpha=\{\nu k_z}$, where $\nu$ is discrete and 
$k_z$ is continuous. The regular and outgoing waves are 

%%%%%%%%%%%%%%%%%%%%%%%%%%%%%%%%%%%%%%%%%%%%%%%%%%%%%%%%%%%%%%%%%%%%%%
$$
 \mathbb{E}_{M\alpha}(\rho, \varphi, z)
 = \Big[   i\nu\frac{Z_\nu(\zeta)}{\zeta}\vbhatt{\rho}
                  - Z_\nu^\prime(\zeta) \vbhatt{\varphi}
      \Big]e^{i\nu\varphi} e^{ik_z z} 
\\
 \mathbb{E}_{N\alpha}(\rho, \varphi, z)
 =\frac{1}{ik_0}
      \Big[ -ik_z Z_\nu^\prime(\zeta)          \vbhatt{\rho}
            +\nu k_z \frac{Z_\nu(\zeta)}{\zeta} \vbhatt{\varphi}
            - k_\rho Z_\nu(\zeta)\vbhat{z}
      \Big] e^{i\nu\varphi} e^{ik_z z} 
$$
%%%%%%%%%%%%%%%%%%%%%%%%%%%%%%%%%%%%%%%%%%%%%%%%%%%%%%%%%%%%%%%%%%%%%%
with
$$ 
 \zeta \equiv k_\rho \rho,
 \qquad
 k_\rho\equiv \sqrt{k_0^2 - k_z^2},
 \qquad
 Z(\zeta) =
 \begin{cases}
   H^{(1)}_\nu(\zeta), \qquad &\text{outgoing} \\
   H^{(2)}_\nu(\zeta), \qquad &\text{incoming} \\
   J_\nu      (\zeta), \qquad &\text{regular}
 \end{cases}
$$
%%%%%%%%%%%%%%%%%%%%%%%%%%%%%%%%%%%%%%%%%%%%%%%%%%%%%%%%%%%%%%%%%%%%%%
The $T$-matrix elements are 

\subsection{Dielectric half-space}

In this case $\alpha=\{k_x, k_y\}=\vb k$ where
both $k_x, k_y$ are continuous.

For testing purposes it is useful to have analytical expressions
for the dyadic Green's functions above an infinite half space.

\subsection*{Plane-wave decomposition of point-source fields}

Note: In what follows,
$\vb k=(k_x,k_y)$ is a \textit{two-dimensional} vector and we have
%====================================================================%
$$ k_z \equiv \sqrt{k_0^2 - |\vb k|^2}, 
%--------------------------------------------------------------------%
   \qquad
%--------------------------------------------------------------------%
   \vb k\subt{3D}\equiv
   \left(\begin{array}{c}k_x \\ k_y \\ \pm k_z \end{array}\right), 
%--------------------------------------------------------------------%
   \qquad
%--------------------------------------------------------------------%
   \pm \equiv
   \begin{cases}
     +, \quad & z\ge 0 \\ 
     -, \quad & z<   0
   \end{cases}.
$$
%====================================================================%
For arbitrary $\vb k$ I now define generalized\footnote{These 
are ``generalized'' plane waves in the sense that $k_z$ is 
imaginary for sufficiently large $\vb k$, in which case the 
waves are evanescent.} transverse-electric and
transverse-magnetic plane waves propagating in the direction
of $\vb k\subt{3D}$:
%====================================================================%
$$\begin{array}{ccccccc}
 \vb E^{\pm} \subt{TE}(\vb x; \vb k)
   &\equiv& E_0 \vb P(\vb k) e^{i\vb k \cdot \vbrho \, \pm \, ik_z z},
   &\qquad&
 \vb H^{\pm} \subt{TE}(\vb x; \vb k)
   &\equiv& H_0 \overline{\vb P}(\vb k) e^{i\vb k \cdot \vbrho \, \pm \, ik_z z},
\\[8pt]
%--------------------------------------------------------------------%
 \vb E^{\pm} \subt{TM}(\vb x; \vb k)
   &\equiv& -E_0 \overline{\vb P}(\vb k) e^{i\vb k \cdot \vbrho \, \pm \, ik_z z},
   &\qquad&
 \vb H^{\pm} \subt{TM}(\vb x; \vb k)
   &\equiv& H_0 \vb P(\vb k) e^{i\vb k \cdot \vbrho \, \pm \, ik_z z},
\end{array}$$
%====================================================================%
$$ E_0 \equiv 1 \text{ volt/$\mu$m}, \qquad H_0\equiv \frac{E_0}{Z_0}.$$
%====================================================================%
where $\vb P$ and $\overline{\vb P}$ are unit-magnitude
polarization vectors with the properties
that \textbf{(1)} both $\vb P$ and $\overline{\vb P}$ are
orthogonal to $\vb k\subt{3D}$, and
\textbf{(2)} $\vb P$ is orthogonal to $\vbhat{z}$ (i.e. ``transverse'').
%====================================================================%
$$ \vb P(\vb k) \equiv
   \frac{1}{|\vb k|}
   \left(\begin{array}{c}-k_y \\ k_x \\ 0 \end{array}\right),
   \qquad
   \overline{\vb P}(\vb k) \equiv
   \frac{1}{k_0}\Big[ \vb k\subt{3D} \times \vb P(\vb k) \Big]
   =
   \frac{1}{k_0 |\vb k|}
   \left(\begin{array}{c} \pm k_x k_z \\ \pm k_y k_z \\ k_x^2 + k_y^2 
         \end{array}\right)
$$
%====================================================================%
The fields of a point source may be written in the form
%====================================================================%
\begin{align*}
 \vb E\supt{ED}\Big(\vb x; \big\{ z_0 \vbhat{z}, \vbhat{x} \big\}\Big)
&= \int \frac{d \vb k}{2\pi} 
   \bigg\{ C^{x}\subt{TE}(\vb k)
           \vb E^\pm\subt{TE}\Big(\vb x-z_0\vbhat{z}; \vb k\Big)
          + 
           C^{x}\subt{TM}(\vb k)
           \vb E^\pm\subt{TM}\Big(\vb x-z_0\vbhat{z}; \vb k\Big)
   \bigg\}
\\
 \vb E\supt{ED}\Big(\vb x; \big\{ z_0 \vbhat{z}, \vbhat{y} \big\}\Big)
&= \int \frac{d \vb k}{2\pi} 
   \bigg\{ C^{y}\subt{TE}(\vb k)
           \vb E^\pm\subt{TE}\Big(\vb x-z_0\vbhat{z}; \vb k\Big)
          + 
           C^{y}\subt{TM}(\vb k)
           \vb E^\pm\subt{TM}\Big(\vb x-z_0\vbhat{z}; \vb k\Big)
   \bigg\}
\\
 \vb E\supt{ED}\Big(\vb x; \big\{ z_0 \vbhat{z}, \vbhat{z} \big\}\Big)
&= \int \frac{d \vb k}{2\pi} 
           C^{z}\subt{TM}(\vb k)
           \vb E^\pm\subt{TM}\Big(\vb x-z_0\vbhat{z}; \vb k\Big)
\end{align*}
%====================================================================%
where the scalar coefficients are 
%====================================================================%
$$\begin{array}{ccccccc}
 \displaystyle{ C^{x}\subt{TE} }
 &=& 
  \displaystyle{ \frac{ ik_y |\vb k|}{2k_0^2 k_z} }
 &\qquad&
 \displaystyle{ C^{x}\subt{TM} }
 &=& 
 \displaystyle{ \frac{ \pm ik_x }{2k_0^2 |\vb k|^2} }
\\[10pt]
%--------------------------------------------------------------------%
 \displaystyle{ C^{y}\subt{TE} }
 &=& 
  \displaystyle{ \frac{ -ik_x |\vb k|}{2k_0^2 k_z} }
 &\qquad&
 \displaystyle{ C^{y}\subt{TM} }
 &=& 
 \displaystyle{ \frac{ \pm ik_y }{2k_0^2 |\vb k|^2} }
\\[10pt]
%--------------------------------------------------------------------%
 \displaystyle{ C^{z}\subt{TE} }
 &=& 0
 &\qquad&
 \displaystyle{ C^{z}\subt{TM} }
 &=& \displaystyle{ \frac{ \pm ikz  }{2k_0^2 |\vb k|^2} }.
\end{array}$$
%====================================================================%

\subsection*{Plane-wave decomposition of DGFs}

The point of this decomposition is that each
plane wave is reflected from the dielectric interface

\subsection*{Converting $k_\rho$ integrals to Brillouin-zone integrals}

\begin{align*}
 Q &= \int_0^\infty \, dk_\rho \, k_\rho F(k_\rho) 
\\
   &= \frac{1}{2\pi} \int_0^{2\pi} \, dk_\theta \,
                     \int_0^\infty \, dk_\rho\, k_\rho F(k_\rho)  \,
\\
   &= \frac{1}{2\pi} \int_{-\infty}^{\infty} \, dk_x \,
                     \int_{-\infty}^{\infty} \, dk_y \,
                     F\left(\sqrt{k_x^2 + k_y^2}\right)
\\
   &= \frac{1}{2\pi} \int_{0}^{\Gamma_x} \, dk_x \,
                     \int_{0}^{\Gamma_y} \, dk_y \,
                     \overline{F}(k_x, k_y)
\end{align*}
$$ \overline{F}(k_x, k_y)
   \equiv \sum_{n_x,n_y=-\infty}^\infty
          F\left(\sqrt{ (k_x + n_x\Gamma_x)^2 + (k_y + n_y\Gamma_y)^2}\right)
$$

\end{document}
