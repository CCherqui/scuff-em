%%%%%%%%%%%%%%%%%%%%%%%%%%%%%%%%%%%%%%%%%%%%%%%%%%%%%%%%%%%%%%%%%%%%%%
%%%%%%%%%%%%%%%%%%%%%%%%%%%%%%%%%%%%%%%%%%%%%%%%%%%%%%%%%%%%%%%%%%%%%%
%%%%%%%%%%%%%%%%%%%%%%%%%%%%%%%%%%%%%%%%%%%%%%%%%%%%%%%%%%%%%%%%%%%%%%
\newpage
\section{Energy/momentum-transfer PFT (MTPFT)}

In the EMTPFT approach, we compute the power, force and torque
on a body $\mc B$ by considering the transfer of energy and
momentum from the total fields to the equivalent surface currents:
%====================================================================%
\begin{subequations}
\begin{align}
 P\sups{abs}&=\frac{1}{2}\text{Re }\int \bmc C^* \cdot \bmc F \, dV 
\\
 F_i&=\frac{1}{2\omega}\text{Im }\int \bmc C^* \cdot \partial_i \bmc F \, dV
\\
 \mc T_i
 &=\frac{1}{2\omega}\text{Im }\int
 \underbrace{
 \Big[   \bmc C^* \times \bmc F 
       + \bmc C^* \cdot \partial_{\theta_i} \vb F\Big]
            }_{\bmc C^* \cdot \wt \partial_i \bmc F}dV
\end{align}
\label{EMTPFT}%
\end{subequations}%
%====================================================================%
Equation (\ref{EMTPFT}a) is just the usual Joule heating
$P=\frac{1}{2}\text{Re } \big(\vb J^*\cdot \vb E + \vb M^* \cdot \vb H\big)$,
while Equations (\ref{EMTPFT}b,c) follow from considering the time-average
Lorentz force 
$d\vb F=\frac{1}{2}\text{Re }
      (\rho\subt{E}^* \vb E + \mu\, \vb J^* \! \times \! \vb H
      +\rho\subt{M}^* \vb H - \epsilon\, \vb M^* \! \times \! \vb E
      )dV
$
and torque $\vb r\times d\vb F$ on the charges and currents
in an infinitesimal volume $dV$; integrating over the
volume and using integration by parts and Maxwell's equations
yields (\ref{EMTPFT}b,c). [The symbol $\partial_{\theta_i}$
in (\ref{EMTPFT}c) denotes differentiation with 
respect to infinitesimal rotation about the $i$th coordinate 
axis. The symbol $\wt{\partial_i}$ is shorthand for the 
operation (cross product plus angular derivative)
involved in (\ref{EMTPFT}c).]

In what follows it will be convenient to express equations
(\ref{EMTPFT}) in terms of the following shorthand operator notation:
%====================================================================%
\numeq{EMTPFTShorthand}
{
 Q=\frac{1}{2}\int_{\partial \mc B}\bmc C^* \bmc Q \bmc F \, dA
}
%====================================================================%
where $Q=\{P\sups{abs}, F_i, \mc T_i\}$ and
he operator $\bmc Q$ correspondingly operates on $\bmc F$ as
in equations (\ref{EMTPFT}a,b,c).

Equations (\ref{EMTPFT}) involve the total fields at the
body surface. Because SIE formulations allow surface fields
to be computed in either of two distinct ways---namely, as the
limiting values of the bulk fields as one approaches the
surface from the exterior or the interior of the body---the
EMTPFT strategy in fact bifurcates into two strategies, which we 
discuss separately below.

\subsection{Exterior EMTPFT}

In the exterior region, the total fields receive
incident and scattered contributions,
$\bmc F = \bmc F\sups{inc} + \bmc F\sups{scat}$,
We identify the corresponding contributions to the
PFT quantities (\ref{EMTPFT}) as respectively the
extinction PFT and the (negative of the) scattered PFT,
%====================================================================%
\begin{subequations}
\begin{align}
 Q &= Q\sups{ext} - Q\sups{scat}
\\
 Q\sups{ext}&=\frac{1}{2}\int_{\partial \mc B}\bmc C^* \bmc Q \bmc F\sups{inc},
\\
 Q\sups{scat}&=-\frac{1}{2}\int_{\partial \mc B}\bmc C^* \bmc Q \bmc F\sups{scat}
\end{align}
\label{ExtScat}%
\end{subequations}
%====================================================================%

\subsubsection*{Extinction contributions to exterior EMTPFT}

The discretized form of (\ref{ExtScat}b) for the extinction reads
%====================================================================%
$$ Q\sups{ext}=
   \frac{1}{2}\sum_{\alpha} c_\alpha
   \int_{\sup \bmc B_\alpha} \bmc B_\alpha \bmc Q \bmc F\sups{inc} \, dA
$$
%====================================================================%
The integrals here are non-singular two-dimensional integrals over
the basis-function supports, with the integrand involving values 
and derivatives of the incident fields. These are evaluated 
in {\sc scuff-em} by simple low-order numerical cubature.

\subsubsection*{Scattered contributions to exterior EMTPFT}

The scattered-field contributions to the EMTPFT power and force 
follow from the discretized form of (\ref{ExtScat}c) and read
%====================================================================%
\begin{subequations}
\begin{align}
 P\sups{scat}&=-\frac{1}{2}\text{Re} \sum_{ab}
 \left(\begin{array}{c} k_a \\ n_a\end{array}\right)^\dagger
 \left(\begin{array}{cc}i\omega \mu \vb G_{ab} & \wh{\vb C}_{ab} \\
                        -\wh{\vb C}_{ab} & i\omega\epsilon \vb G_{ab}
 \end{array}\right)
 \left(\begin{array}{c}k_b \\ n_b\end{array}\right)
\\
 F_i\sups{scat}&=-\frac{1}{2\omega}\text{Im} \sum_{ab}
 \left(\begin{array}{c} k_a \\ n_a\end{array}\right)^\dagger
 \left(\begin{array}{cc}i\omega \mu \partial_i \vb G_{ab} & \partial_i \wh{\vb C}_{ab} \\
                        -\partial_i \wh{\vb C}_{ab} & i\omega\epsilon \partial_i \vb G_{ab}
 \end{array}\right)
 \left(\begin{array}{c}k_b \\ n_b\end{array}\right)
\end{align}
\label{PFExteriorPFT1}%
\end{subequations}
%====================================================================%
(The torque is similar to the force with $\partial_i \to \wt{\partial_i}$.)
In these equations, 
\begin{itemize}
\item
$a$ runs over all interior edges (RWG functions) 
on the surface bounding the region on which we are 
computing the power or force;
\item
$b$ runs over all interior edges on all surfaces bounding 
the region exterior to the region on which we are 
computing the power or force;
\item
$\epsilon=\epsilon_0 \epsilon^r$ and 
$\mu=\mu_0 \mu^r$ are the material properties of the 
exterior region
\item
 $\vb G_{ab}=\vmv{\vb b_a}{\mb G}{\vb b_b}$
 is the matrix element of the $\mb G$ kernel for
 the exterior medium between RWG basis functions.
\item
 $\wh{\vb C}_{ab}=\vmv{\vb b_a}{ik\mb C}{\vb b_b}.$
\end{itemize}

Upon using the relations
%====================================================================%
$$\{\vb G_{ba}, \vb C_{ba}\} = \{\vb G_{ab}, \vb C_{ab}\}, 
  \qquad
  \partial_i\{\vb G_{ba}, \vb C_{ba}\} = -\partial_i\{\vb G_{ab}, \vb C_{ab}\}, 
$$
%====================================================================%
equations (\ref{PFExteriorPFT1}) may be simplified to read
%====================================================================%
\begin{align*}
 P\sups{scat} &= +\frac{1}{2}
  \sum_{ab}\bigg\{
           \omega \mu_0
           \Big[\text{Re }k_a^* k_b\Big]
           \Big[\text{Im }\mu^r \vb G_{ab}\Big]
          +\omega \epsilon_0
           \Big[\text{Re }n_a^* n_b\Big]
           \Big[\text{Im }\epsilon^r \vb G_{ab}\Big]
\\
&\hspace{1in} 
  +\Big[\text{Im }k_a^* n_b - n_a^* k_b\Big]
           \Big[\text{Im } \wh{\vb C}_{ab}\Big]\bigg\}
\\
 F_i\sups{scat} &= +\frac{1}{2\omega}
  \sum_{ab}\bigg\{
           \omega \mu_0
           \Big[\text{Im }k_a^* k_b\Big]
           \Big[\text{Im }\mu^r \partial_i \vb G_{ab}\Big]
          +\omega \epsilon_0
           \Big[\text{Im }n_a^* n_b\Big]
           \Big[\text{Im }\epsilon^r \partial_i \vb G_{ab}\Big]
\\
&\hspace{1in} 
  -\Big[\text{Re }k_a^* n_b - n_a^* k_b\Big]
           \Big[\text{Im } \partial_i \wh{\vb C}_{ab}\Big]\bigg\}
\end{align*}
%====================================================================%

\subsubsection*{EMTPFT matrix elements}

%====================================================================%
\begin{align*}
   \partial_i \vb G_{ab}
&= \partial_i \vb G^1_{ab} +\partial_i \vb G^2_{ab}
\\
\partial_i \vb G^1_{ab}
&=
 \int_{\sup \vb b_a}\, d\vb x_a \, 
 \int_{\sup \vb b_b}\, d\vb x_b \, 
 \left(\vb b_a \cdot \vb b_b\right) \partial_i G_0(\vb x_a-\vb x_b)
\\
\partial_i \vb G^2_{ab}
&=
 -\frac{1}{k^2}
  \int_{\sup \vb b_a}\, d\vb x_a \,
  \int_{\sup \vb b_b}\, d\vb x_b \,
   (\nabla \cdot \vb b_a)(\nabla \cdot \vb b_b)
   \partial_i G_0(\vb x_a-\vb x_b)
\end{align*}
%====================================================================%

%====================================================================%
\begin{align*}
 \text{Im } \partial_i \wh{\mb C}_{jk}
&= \text{Im }
   \partial_i\left[ \frac{e^{ikr}}{4\pi r^3}(ikr-1)\varepsilon_{jk\ell}r_\ell\right]
\\
\end{align*}
%====================================================================%

%====================================================================%
\begin{align*}
\partial_i G_0(\vb r)
   &= r_i(ikr-1)\frac{e^{ikr}}{4\pi r^3}
   &= -\frac{r_i}{4\pi r^3} + \Big[\partial_i G_0\Big]\supt{DS}
\end{align*}
%====================================================================%

%====================================================================%
\begin{align*}
 F_i\supt{S}
 &\frac{\mu_0}{2} \text{Im }{k_a^* k_b}\Big[\text{Im }
\end{align*}
%====================================================================%
