%%%%%%%%%%%%%%%%%%%%%%%%%%%%%%%%%%%%%%%%%%%%%%%%%%%%%%%%%%%%%%%%%%%%%%
%%%%%%%%%%%%%%%%%%%%%%%%%%%%%%%%%%%%%%%%%%%%%%%%%%%%%%%%%%%%%%%%%%%%%%
%%%%%%%%%%%%%%%%%%%%%%%%%%%%%%%%%%%%%%%%%%%%%%%%%%%%%%%%%%%%%%%%%%%%%%
\newpage
\section{Cartesian Multipole Moments}

\subsection{Electric Dipole Moment of an RWG Electric Current}
The electric dipole moment of an electric current distribution $\vb J(\vb x)$ is 
$$\vb p= \int_{\sup \vb J}  \vb x\, \rho(\vb x) d\vb x $$
where $\rho(\vb x)\equiv \frac{1}{i\omega}\nabla \cdot \vb J$ is the charge 
density associated with $\vb J$. For an RWG function populated with unit 
strength we have 
\begin{align} 
 \rho_\alpha(\vb x) 
 &=\frac{1}{i\omega}\nabla \cdot \vb f_\alpha(\vb x) \nn
 &=\frac{1}{i\omega}
   \begin{cases} 
    +\displaystyle{\frac{l_\alpha}{A^+_\alpha}}, 
     \qquad &\vb x \in \mathcal{P}_\alpha^+ \\\\
    -\displaystyle{\frac{l_\alpha}{A^-_\alpha}}, 
      \qquad &\vb x \in \mathcal{P}_\alpha^-.
   \end{cases} \label{rhoRWG}
\end{align} 
Then the electric dipole moment of an electric current distribution 
described by an RWG function populated with unit strength is 
\begin{align*}
  \vb p_\alpha 
&= \frac{2l_\alpha}{i\omega}
   \left\{ 
   \frac{1}{2A^+_\alpha} \int_{\pan_\alpha^+} \vb x \,d\vb x  
   \,\,- \,\,
   \frac{1}{2A^-_\alpha} \int_{\pan_\alpha^-} \vb x \, d\vb x 
   \right\}  \\
&= \frac{2l_\alpha}{i\omega}
   \left\{ 
            \int_0^1 \,du\,\int_0^u \, dv 
            \Big[ \vb Q^+ + u(\vb V_1-\vb Q^+) + v(\vb V_2 - \vb V_1)  \Big]
   \right. \\
&  \qquad \qquad- 
   \left. 
            \int_0^1 \,du\,\int_0^u \, dv 
            \Big[ \vb Q^- + u(\vb V_1-\vb Q^-) + v(\vb V_2 - \vb V_1)  \Big]
   \right\} \\
&= \frac{l_\alpha}{3 i\omega}\Big(\vb Q^+_\alpha - \vb Q^-_\alpha\Big).
\end{align*}

\subsection{Magnetic Dipole Moment of an RWG Electric Current}

The magnetic dipole moment of an electric current distribution $\vb J(\vb x)$ is 
$$\vb m= \frac{1}{2} \int_{\sup \vb J}  \vb x \times \vb J(\vb x) \,d\vb x. $$
Then the magnetic dipole moment of an electric current distribution 
described by an RWG function populated with unit strength is 
\begin{align*}
  \vb m_\alpha 
&= \frac{l_\alpha}{2}
   \left\{ 
   \frac{1}{2A^+_\alpha} \int_{\pan_\alpha^+} 
                         \vb x \times \Big(\vb x-\vb Q^+\Big) 
                         \,d\vb x  
   \,\,- \,\,
   \frac{1}{2A^-_\alpha} \int_{\pan_\alpha^-} 
                         \vb x \times \Big(\vb x-\vb Q^-\Big) 
                         \,d\vb x  
   \right\}  \\
&= \frac{l_\alpha}{2}
   \left\{ 
   \frac{1}{2A^+_\alpha} 
     \left[\vb Q^+ \times \int_{\pan_\alpha^+} \vb x \,d\vb x  \right]
   \,\,- \,\,
   \frac{1}{2A^-_\alpha} 
     \left[ \vb Q^- \times \int_{\pan_\alpha^-} \vb x \,d\vb x \right]
   \right\}\\
&= \frac{l_\alpha}{2}
   \left\{ 
            \int_0^1 \,du\,\int_0^u \, dv 
            \Big[  u(\vb Q^+ \times \vb V_1) 
                  +v(\vb Q^+ \times (\vb V_2 - \vb V_1)) 
            \Big]
   \right. \\
&  \qquad \qquad- 
   \left. 
            \int_0^1 \,du\,\int_0^u \, dv 
            \Big[  u(\vb Q^- \times \vb V_1) 
                  +v(\vb Q^- \times (\vb V_2 - \vb V_1)) 
            \Big]
   \right\} \\
&= \frac{l_\alpha}{12}
   \Big(\vb Q^+_\alpha - \vb Q^-_\alpha\Big) \times 
   \Big(\vb V_{1\alpha} + \vb V_{2\alpha}\Big).
\end{align*}

\subsection{Dipole Moments of RWG Magnetic Currents}

In a non-PEC scattering problem we must account for the 
contributions of the magnetic currents to the electric
and magnetic dipole moments. To ferret out what these
are, we need to recall how dipole moments are defined
in the first place. We begin by considering a 
vector-valued function $\vb K(\vb x^\prime)$ 
confined to a region near the origin of spatial extent 
much smaller than the distance to some evaluation point 
$\vb x$. In this case it is just a mathematical fact that 
the convolutions of $\vb K$ with the $\vb G$ and $\vb C$
kernels approach 

\newcommand{\moment}[2]{#1^{\text{\tiny{($\mathbf{#2}$)}}}}
\begin{align*}
\oint \vb G(\vb x, \vb x^\prime) \cdot \vb K(\vb x^\prime) d\vb x^\prime
 &\quad\to\quad   A_{ij}(\vb x) \moment{p_j}{K}
                + B_{ij}(\vb x) \moment{m_j}{K}
\\
\oint \vb C(\vb x, \vb x^\prime) \cdot \vb K(\vb x^\prime) d\vb x^\prime
 &\quad\to\quad  -B_{ij}(\vb x) \moment{p_j}{K}
                + A_{ij}(\vb x) \moment{m_j}{K}
\end{align*}
where $\{A,B\}_{ij}(\vb x)$ are certain functions whose precise
form we will not need, and where 
$\moment{\vb p}{K}$
and 
$\moment{\vb m}{K}$
are certain constants associated with the $\vb K$ 
distribution:
$$ \moment{\vb p}{K}
    =\frac{1}{i\omega} \int \Big(\nabla \cdot \vb K\Big)d\vb x, 
   \qquad
   \moment{\vb m}{K}
    =\frac{1}{2} \int \vb x \times \vb K(\vb x) d\vb x.
$$ 
Using these, the fields at distant points are
%====================================================================%
\begin{align*}
 \vb E\big[\vb K; \vb x\big]
 &= ikZ \oint \vb G(\vb x, \vb x^\prime) \cdot \vb K(\vb x^\prime) d\vb x^\prime
\\
 &\to ikZ \Big[   A_{ij} \moment{p}{K} + B_{ij} \moment{m}{K} \Big]
\\
 \vb H\big[\vb K; \vb x\big]
 &=  -ik\oint \vb C(\vb x, \vb x^\prime) \cdot \vb K(\vb x^\prime) d\vb x^\prime
\\
 &\to -ik\Big[  -B_{ij} \moment{p}{K}
               + A_{ij} \moment{m}{K}
         \Big]
\end{align*}
%====================================================================%
Now we consider the effect of adding magnetic currents 
$\vb N$ into the mix. The fields become
%====================================================================%
\begin{align*}
 \vb E\big[\vb K, \vb N; \vb x\big]
 &=  ikZ \oint \vb G(\vb x, \vb x^\prime) \cdot \vb K(\vb x^\prime) d\vb x^\prime
     + ik\oint \vb C(\vb x, \vb x^\prime) \cdot \vb N(\vb x^\prime) d\vb x^\prime
\\[5pt]
 &\to ikZ \Big[   A_{ij} \moment{p_j}{K} + B_{ij} \moment{m_j}{K} \Big] 
       +ik\Big[  -B_{ij} \moment{p_j}{N} + A_{ij} \moment{m_j}{N} \Big]
\\[5pt]
 &\to ikZ \bigg[   A_{ij} 
                  \underbrace{\Big\{ \moment{p_j}{K} + \frac{1}{Z}\moment{m_j}{N} \Big\}}
                             _{\moment{p_j}{K,N}}
                  +B_{ij} 
                  \underbrace{\Big\{ \moment{m_j}{K} - \frac{1}{Z}\moment{p_j}{N} \Big\}}
                             _{\moment{m_j}{K,N}}
          \bigg]
\\[10pt]
 \vb H\big[\vb K, \vb N; \vb x\big]
 &=  -ik\oint \vb C(\vb x, \vb x^\prime) \cdot \vb K(\vb x^\prime) d\vb x^\prime
   +\frac{ik}{Z} \oint \vb G(\vb x, \vb x^\prime) \cdot \vb N(\vb x^\prime) d\vb x^\prime
\\[5pt]
 &\to       -ik\Big[  -B_{ij} \moment{p_j}{K} + A_{ij} \moment{m_j}{K} \Big]
  +\frac{ik}{Z}\Big[   A_{ij} \moment{p_j}{N} + B_{ij} \moment{m_j}{N} \Big].
\\[5pt]
 &\to      -ik\bigg[  -B_{ij} 
                      \underbrace{\Big\{\moment{p_j}{K} + \frac{1}{Z}\moment{m_j}{N}\Big\}}
                                _{\moment{p_j}{K,N}}
                      +A_{ij} 
                      \underbrace{\Big\{\moment{m_j}{K} - \frac{1}{Z}\moment{m_j}{N}\Big\}}
                                _{\moment{m_j}{K,N}}
              \bigg].
\end{align*}
So the effective electric and magnetic dipole moments of a combined
distribution of electric and magnetic currents are 
$$ \moment{\vb p}{K,N} \equiv \moment{\vb p}{K} + \frac{1}{Z}\moment{\vb m}{N},
   \qquad
   \moment{\vb m}{K,N} \equiv \moment{\vb m}{K} - \frac{1}{Z}\moment{\vb p}{N}.
$$
Note that these depend on $Z$.
