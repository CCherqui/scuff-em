%%%%%%%%%%%%%%%%%%%%%%%%%%%%%%%%%%%%%%%%%%%%%%%%%%%%%%%%%%%%%%%%%%%%%%
%%%%%%%%%%%%%%%%%%%%%%%%%%%%%%%%%%%%%%%%%%%%%%%%%%%%%%%%%%%%%%%%%%%%%%
%%%%%%%%%%%%%%%%%%%%%%%%%%%%%%%%%%%%%%%%%%%%%%%%%%%%%%%%%%%%%%%%%%%%%%
\newpage
\section{Periodic Boundary Conditions}

\subsection{Periodic DGFs}

To avoid confusion with the free-space photon wavenumber, 
I use the symbol $\vb p$ to denote a two-dimensional 
Bloch wavevector.

I use an overbar notation to denote the 
\textit{periodic Green's function}, which contains contributions
from all lattice cells appropriately weighted by Bloch phases.
$$
  \GBar(\vb p; k; \vb r, \vb r^\prime)
= \sum_{\vb L} e^{i \vb p \cdot \vb L} G(k; \vb r, \vb r^\prime+\vb L)
\qquad 
 \bigg( G(k; \vb r, \vb r^\prime) 
 = \frac{e^{ik|\vb r-\vb r^\prime|}}{4\pi|\vb r-\vb r^\prime|}
 \bigg)
$$

I will also need to define a version of this function that 
excludes the contributions of the innermost 9 lattice cells.
I will call this version 
$\GBar_{\texttt{AB9}}$
%$\GBar_{\text{\scriptsize{AB9}}}$, 
where the subscript stands for 
stands for ``all but 9.''

\subsection{Assembling the BEM matrix}

\begin{align*}
M_{\alpha\beta}
\end{align*}
